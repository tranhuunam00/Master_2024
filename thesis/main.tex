\documentclass[12pt,a4paper,oneside]{book} % twoside for draf

%\usepackage{babel}
\usepackage[utf8]{vietnam}
%\usepackage{times}
%\usepackage{graphicx}

\usepackage{mathptmx}	% same Time New Roma

%\renewcommand{\rmdefault}{phv} % Arial
%\renewcommand{\sfdefault}{phv} % Arial
% \usepackage[fontsize=13pt]{scrextend}

\usepackage[acronym]{glossaries}

\usepackage{fancyhdr}
% \usepackage[utf8]{inputenc}
\usepackage[vietnamese]{babel}
\usepackage{titlesec}
\usepackage{titletoc}
\usepackage{listings}
\usepackage[bookmarks=true]{hyperref}
\usepackage[left=3cm,right=2cm,top=2.5cm,bottom=3cm]{geometry}
\usepackage{graphicx}
\usepackage{hyperref}
\usepackage{tikz}
\usepackage{varwidth}
\usepackage{float}
\usepackage{color}
\usepackage{multirow}
\usepackage{booktabs}
\usepackage[linesnumbered,lined,ruled,resetcount, algochapter]{algorithm2e}
\usepackage{svg}
\usepackage{tabularx}
\usepackage{nomencl}
\usepackage{scrfontsizes}
\usepackage{longtable}
\usepackage{multicol}
\usepackage[toc,page]{appendix}
\usepackage{adjustbox}

\usepackage{makecell}
\usepackage{longtable}
\usepackage{multirow}
% \usepackage{algpseudocode}
\usepackage{setting/bkthesis}
\usepackage[toc]{appendix}

\usepackage{amsmath}

\newtheorem{definition}{Định nghĩa}

\newcommand\sbullet[1][.5]{\mathbin{\vcenter{\hbox{\scalebox{#1}{$\bullet$}}}}}

%\counterwithin{figure}{chapter}
\usepackage[
backend=biber,
bibstyle=authoryear,
sorting=none,
citestyle=numeric-comp
]{biblatex}
\usepackage{xpatch}

\makeatletter
\input{numeric.bbx}
\makeatother

\xpatchbibmacro{date+extrayear}{%
  \printtext[parens]%
}{%
  \setunit{\addperiod\space}%
  \printtext%
}{}{}

\usepackage[backend=biber]{biblatex} % ✅ Đúng backend
\addbibresource{refs.bib}      

% \renewcommand\labelitemi{--}

\setlength{\parskip}{6pt}

\usetikzlibrary{calc}
\setlength{\parindent}{10mm}
\renewcommand{\baselinestretch}{1.3}
\graphicspath{{images/}}
\renewcommand{\listalgorithmcfname}{Danh sách thuật toán}
\renewcommand{\nomname}{Danh sách ký hiệu, viết tắt}
\renewcommand\appendixtocname{Phụ lục}

\SetKwRepeat{Do}{do}{while}%

% \algdef{SE}[DOWHILE]{Do}{doWhile}{\algorithmicdo}[1]{\algorithmicwhile\ #1}%
% \usepackage{setting/uetthesis}

%%% The following lines add Chapter or Appendix in front of the number
\titlecontents{chapter}%
[0pt]%
{\vspace{1ex}}%
{\bfseries Chương \thecontentslabel\quad}%
{\bfseries}%
{\bfseries\hfill\contentspage}
%%% Initially, for the main part of the document, set the label to "Chapter"
\let\chapappname\chaptername

% \thesislayout
% hyper setup
\definecolor{dkgreen}{rgb}{0,0.6,0}
\definecolor{gray}{rgb}{0.5,0.5,0.5}
\definecolor{mauve}{rgb}{0.58,0,0.82}

% setup code area as listings
\lstset{frame=tb,
  language=Java,
  xleftmargin=10pt,xrightmargin=10pt,
%   framesep=2cm,
%   aboveskip=2cm,
%   belowskip=2cm,
  showstringspaces=false,
  columns=flexible,
  basicstyle={\small\ttfamily},
  numbers=left,
  numberstyle=\tiny\color{gray},
  keywordstyle=\color{blue},
  commentstyle=\color{dkgreen},
  stringstyle=\color{mauve},
  breaklines=true,
  breakatwhitespace=true,
  tabsize=2,
  escapeinside={(*@}{@*)}
}

\hypersetup{
	bookmarks=true,
	pdftitle={Công cụ hỗ trợ kiểm tra sự tuân thủ các quy chuẩn viết mã nguồn Java},
	pdfauthor={Nguyễn Việt Hòa}, % author
	pdfsubject={TeX and LaTeX},
	pdfkeywords={TeX, LaTeX, graphics, images}, % list of keywords
	colorlinks=false,       % false: boxed links; true: colored links
	linkcolor=black,       % color of internal links
	citecolor=black,       % color of links to bibliography
	filecolor=black,        % color of file links
	urlcolor=black,        % color of external links
	linktoc=page            % only page is linked
}

\lstdefinelanguage{JavaScript}{
  morekeywords=[1]{break, continue, delete, else, for, function, if, in,
    new, return, this, typeof, var, void, while, with},
  % Literals, primitive types, and reference types.
  morekeywords=[2]{false, null, true, boolean, number, undefined,
    Array, Boolean, Date, Math, Number, String, Object},
  % Built-ins.
  morekeywords=[3]{eval, parseInt, parseFloat, escape, unescape},
  sensitive,
  morecomment=[s]{/*}{*/},
  morecomment=[l]//,
  morecomment=[s]{/**}{*/}, % JavaDoc style comments
  morestring=[b]',
  morestring=[b]"
}[keywords, comments, strings]  
\lstalias[]{ES6}[ECMAScript2015]{JavaScript}

\renewcommand{\lstlistingname}{Mã nguồn}
\renewcommand\thechapter{\arabic{chapter}}
\renewcommand\thesection{\thechapter.\arabic{section}.}
\renewcommand\thesubsection{\thesection\arabic{subsection}.}
\renewcommand{\thetable}{\thechapter.\arabic{table}}
\renewcommand{\thefigure}{\thechapter.\arabic{figure}}
\renewcommand{\thealgocf}{\thechapter.\arabic{algocf}}


\newglossaryentry{latex}
{
    name=latex,
    description={Is a markup language specially suited 
    for scientific documents}
}

\newglossaryentry{maths}
{
    name=mathematics,
    description={Mathematics is what mathematicians do}
}

\newglossaryentry{callee}
{
    name=\textit{callee},
    description={Xét trong mã nguồn một phương thức, những phương thức khác được gọi trong thân hàm gọi là \textit{callee}}
}

\newglossaryentry{caller}
{
    name=\textit{caller},
    description={Xét trong mã nguồn một phương thức có chứa lời gọi tới những hàm, phương thức khác, phương thức đó gọi là \textit{caller}}
}

\newglossaryentry{extension}
{
    name=extension,
    description={Tiện ích có thể tích hợp với các IDE để hỗ trợ trực tiếp lập trình viên trong quá trình phát triển mã nguồn. Dựa trên phương pháp đề xuất, luận văn xây dựng một tiện ích tích hợp với IDE phổ biến Visual Code}
}

\newglossaryentry{framework}
{
    name=framework,
    description={Framework là các đoạn code đã được viết sẵn, cấu thành nên một bộ khung và các thư viện lập trình được đóng gói. Chúng cung cấp các tính năng có sẵn như mô hình, API và các yếu tố khác để tối giản cho việc phát triển các ứng dụng web phong phú, năng động}
}

\newglossaryentry{API}
{
    name=API,
    description={API là các phương thức, giao thức kết nối với các thư viện và ứng dụng khác nhau. Nó là viết tắt của Application Programming Interface – giao diện lập trình ứng dụng. API cung cấp khả năng truy xuất đến một tập các tài nguyên trong một hệ thống, từ đó có thể trao đổi dữ liệu giữa các ứng dụng}
}

\newglossaryentry{mocking}
{
    name=\textit{mock},
    description={Cơ chế áp dụng cho các hàm/phương thức thực thi một hành vi mới thay thế hành vi cũ khi thực thi kiểm thử}
}

\newglossaryentry{mockdata}
{
    name=\textit{mock data},
    description={Hành vi mới được thiết lập cho các hàm/phương thức truy cập tới dịch vụ bên ngoài dự án}
}

\newglossaryentry{mocks}
{
    name=mocks,
    description={mocks}
}

\newglossaryentry{httpRequest}
{
    name=HTTP Request,
    description={mocks}
}

\newglossaryentry{mocked_method}
{
    name=mocked method,
    description={Phương thức gọi đến dịch vụ bên ngoài, cần được thiết lập hành vi mới khi thực thi kiểm thử để đảm bảo quá trình kiểm thử là độc lập}
}

\newglossaryentry{rest}
{
    name=REST,
    description={Tiêu chuẩn thiết kế và quản lý các API trong ứng dụng Web}
}

\newglossaryentry{URL}
{
    name=URL,
    description={Uniform Resource Locator - Định vị tài nguyên thống nhất, sử dụng để định vị các tài nguyên trên mạng Internet}
}

\newglossaryentry{OOP}
{
    name=OOP,
    description={Object-oriented Programming - Lập trình hướng đối tượng}
}

\newglossaryentry{HTTP}
{
    name=HTTP,
    description={HyperText Transfer Protocol - Giao thức truyền siêu văn bản}
}

\newglossaryentry{HTTPS}
{
    name=HTTPS,
    description={HyperText Transfer Protocol Secure - Giao thức truyền siêu văn bản có cơ chế mã hóa bảo mật}
}

\newglossaryentry{JSON}
{
    name=JSON,
    description={JavaScript Object Notation - Ký pháp đối tượng JavaScript}
}

\newglossaryentry{CFG}
{
    name=CFG,
    description={Control Flow Graph - Đồ thị dòng điều khiển, mô hình mô tả kịch bản thực thi của phương thức}
}

\newglossaryentry{CCFG}
{
    name=CCFG,
    description={Connected Control Flow Graph - Đồ thị dòng điều khiển liên thông, bao gồm một danh sách các CFG liên kết với nhau, mô tả kịch bản thực thi của phương thức, bao gồm cả các phương thức được gọi}
}

\newglossaryentry{AST}
{
    name=AST,
    description={Abstract Syntax Tree - Cây cú pháp trừu tượng}
}

\newglossaryentry{pattern}
{
    name=\textit{pattern},
    description={Từ đại diện cho một khuôn mẫu mô tả cách sử dụng biến trong biểu thức điều kiện}
}

\newglossaryentry{OSA}
{
    name=OSA,
    description={Obstructive Sleep Apnea - Hội chứng ngưng thở khi ngủ do tắc nghẽn}
}

\newglossaryentry{AHI}
{
    name=AHI,
    description={Apnea-Hypopnea Index - Chỉ số ngưng thở giảm thở }
}

\newglossaryentry{PSG}
{
    name=PSG,
    description={Polysomnography - Đa ký giấc ngủ }
}

\newglossaryentry{BLE}
{
    name=BLE,
    description={Bluetooth Low Energy - Bluetooth năng lượng thấp }
}

\newglossaryentry{GATT}
{
    name=GATT,
    description={Generic ATTribute Profile- Đặc điểm cấu hình chung }
}




\makeglossaries



\begin{document}
\renewcommand{\thelstlisting}{\thechapter.\arabic{lstlisting}}

\renewcommand{\thedefinition}{\thechapter.\arabic{definition}}



% \makeatletter 
% \renewcommand\thealgorithm{\thechapter.\arabic{algorithm}} 
% \makeatother

\pagestyle{plain}
\frontmatter


%-------TITLE PAGE------%
\begin{titlepage}
	\center
	\begin{tikzpicture}[overlay,remember picture]
		\draw [line width=3pt,rounded corners=0pt,]
		($ (current page.north west) + (25mm,-25mm) $)
		rectangle
		($ (current page.south east) + (-15mm,25mm) $);
		\draw [line width=1pt,rounded corners=0pt]
		($ (current page.north west) + (26.5mm,-26.5mm) $)
		rectangle
		($ (current page.south east) + (-16.5mm,26.5mm) $);
	\end{tikzpicture}

	{\large \bfseries ĐẠI HỌC QUỐC GIA HÀ NỘI\\ TRƯỜNG ĐẠI HỌC CÔNG NGHỆ}\\[1cm]
	\includegraphics[width=0.25\linewidth]{images/Logo_UET.png}\\[1cm]
	{\Large  \bfseries Trần Hữu Nam}\\[1cm]
	% 	{ \LARGE \bfseries  XÂY DỰNG CÔNG CỤ KIỂM THỬ TỰ ĐỘNG }\\[0.05cm]
	%     {\LARGE \bfseries CHO CÁC ỨNG DỤNG SỬ DỤNG NGÔN NGỮ}\\[0.2cm]
	%      { \LARGE \bfseries   TYPESCRIPT}\\[0.5cm]
	% 	\hfill\\[2cm]
	{ \LARGE \bfseries
	NGHIÊN CỨU, PHÁT TRIỂN MÔ HÌNH HỌC MÁY TẠI BIÊN NHẰM PHÂN LOẠI TƯ THẾ NGỦ}\\[0.05cm]
	\hfill\\[1cm]
	{\large \bfseries LUẬN VĂN THẠC SĨ}\\
	% {\large \bfseries Ngành: Kĩ thuật điện tử}	
	\hfill\\[5.3cm]
	{\large \bfseries HÀ NỘI - 2025}\\
	\vfill
\end{titlepage}

%-------TITLE PAGE+6hbk,------%
\begin{titlepage}
	\center
	\begin{tikzpicture}[overlay,remember picture]
		\draw [line width=3pt,rounded corners=0pt,]
		($ (current page.north west) + (25mm,-25mm) $)
		rectangle
		($ (current page.south east) + (-15mm,25mm) $);
		\draw [line width=1pt,rounded corners=0pt]
		($ (current page.north west) + (26.5mm,-26.5mm) $)
		rectangle
		($ (current page.south east) + (-16.5mm,26.5mm) $);
	\end{tikzpicture}

	{\large \bfseries ĐẠI HỌC QUỐC GIA HÀ NỘI\\ TRƯỜNG ĐẠI HỌC CÔNG NGHỆ}\\[2cm]
	% 	\includegraphics[width=0.25\linewidth]{images/Logo_UET.png}\\[1cm]

	{\Large  \bfseries Trần Hữu Nam}\\[2cm]
	{ \LARGE \bfseries  NGHIÊN CỨU, PHÁT TRIỂN MÔ HÌNH HỌC MÁY TẠI BIÊN NHẰM PHÂN LOẠI TƯ THẾ NGỦ
	}\\[0.05cm]
	\hfill\\[1cm]
	\begin{flushleft}
		{\large \bfseries Ngành: Điện tử viễn thông}\\
		{\large \bfseries Chuyên ngành: Kĩ thuật điện tử}\\
		{\large \bfseries Mã số học viên: 23025029}\\

	\end{flushleft}
	\hfill\\[1cm]
	{\large \bfseries LUẬN VĂN THẠC SĨ}\\
	\hfill\\[1cm]

	{\large \bfseries NGƯỜI HƯỚNG DẪN KHOA HỌC: PGS.TS. Mai Anh Tuấn}\\

	\begin{flushleft}
		% 	{\large \bfseries Cán bộ đồng hướng dẫn: CN. Bùi Quang Cường}\\	
	\end{flushleft}
	\hfill\\[3cm]
	{\large \bfseries HÀ NỘI - 2025}\\
	\vfill
\end{titlepage}

\changefontsizes[16pt]{14pt}
\addtocontents{toc}{\vspace{-1cm}}
\addcontentsline{toc}{chapter}{Lời cam đoan}
\begin{center}
	\textbf{LỜI CAM ĐOAN}
\end{center}
Tôi xin cam đoan: luận văn thạc sĩ
“Nghiên cứu, phát triển mô hình học máy tại biên nhằm phân loại tư thế ngủ”
là công trình nghiên cứu của tôi dưới sự hướng dẫn của
thầy \textbf{PGS. TS. Mai Anh Tuấn} và thầy \textbf{ThS. Trần Ngọc Thái}
cùng với sự tham khảo từ những tài liệu đã liệt kê trong mục Tài liệu tham khảo. Tôi không sao chép công trình nghiên cứu của cá nhân khác dưới bất kỳ hình thức nào. Nếu có tôi xin hoàn toàn chịu trách nhiệm.

\begin{flushright}
	\begin{varwidth}{\linewidth}\centering
		Hà Nội, ngày \space\space\space\space tháng  \space\space\space\space năm 2025\\
		Học viên\\[2cm]
		Trần Hữu Nam
	\end{varwidth}
\end{flushright}

\newpage

\addcontentsline{toc}{chapter}{Lời cảm ơn}
\begin{center}
	\textbf{LỜI CẢM ƠN}
\end{center}

Lời đầu tiên, tôi xin gửi lời cảm ơn đến thầy PGS.TS. Mai Anh Tuấn và thầy ThS.
Trần Ngọc Thái vì đã tận tình hướng dẫn, truyền đạt kiến thức cho tôi trong
suốt quá trình học tập và thực hiện luận văn. Tôi xin cảm ơn tập thể thầy, cô
khoa “Điện tử Viễn thông”, Trường Đại học Công nghệ - ĐHQGHN, đã giảng dạy tôi
trong quá trình tôi học tập tại trường. Tôi cũng xin cảm ơn các anh chị ở Bộ
môn Công nghệ Vi cơ Điện tử và Kỹ thuật Máy tính đã tạo điều kiện giúp đỡ, chỉ
bảo tôi trong thời gian làm luận văn. Cuối cùng, tôi xin cảm ơn bố mẹ, gia đình
cũng như bạn bè, tập thể lớp K30 đã luôn đồng hành, chia sẻ và động viên tôi
suốt thời gian qua.

\newpage
\addcontentsline{toc}{chapter}{Tóm tắt}
\begin{center}
	\textbf{TÓM TẮT}
\end{center}
% \changefontsizes[16pt]{12pt}
\textit{\textbf{Tóm tắt: }}
Ngưng thở tắc nghẽn khi ngủ (Obstructive Sleep Apnea – OSA) là một rối loạn
hô hấp khi ngủ do sự hẹp hoặc tắc nghẽn một phần hay toàn bộ đường hô hấp
trên, bao gồm vùng mũi họng, hầu họng hoặc cả hai. Tình trạng này được đặc
trưng bởi các cơn ngưng thở hoặc giảm thở ngắn, lặp đi lặp lại trong khi ngủ,
gây gián đoạn giấc ngủ do thức giấc thường xuyên và dẫn đến giảm oxy máu
từng đợt\cite{Epstein2009}. Nghiên
cứu\cite{Salari2025} đã phân tích dữ liệu từ 15 nghiên cứu với tổng cộng 42.924
người. Kết quả cho thấy tỷ lệ mắc chứng buồn ngủ ban ngày quá mức (EDS) ở bệnh
nhân ngưng thở tắc nghẽn khi ngủ (OSA) trên toàn cầu là 39,9\% (khoảng tin cậy
95\%: 34,4\%–45,7\%).

Tư thế ngủ đóng vai trò quan trọng trong việc khởi phát và làm trầm trọng thêm
các triệu chứng của OSA. Đặc biệt, tư thế nằm ngửa thường làm tăng mức độ tắc
nghẽn đường hô hấp so với các tư thế khác, dẫn đến dạng OSA phụ thuộc tư thế
(positional OSA). Trong bối cảnh đó, việc theo dõi chính xác tư thế ngủ trong
thời gian thực có thể cung cấp thông tin quan trọng phục vụ chẩn đoán sớm, đánh
giá nguy cơ và cải thiện tình trạng OSA bằng việc thay đổi tư thế ngủ.

Trong bối cảnh công nghệ chế tạo ngày càng phát triển mạnh mẽ, việc thu nhỏ và
tối ưu hiệu suất vi điều khiển, cảm biến, pin đã trở thành một yếu tố then chốt
trong quá trình tích hợp chúng vào các thiết bị điện tử có kích thước nhỏ gọn.
Không chỉ góp phần nâng cao độ chính xác trong việc đo lường các thông số sinh
lý quan trọng, nó còn giúp giảm kích thước thiết bị, tăng tính di động và mở
rộng khả năng ứng dụng trong nhiều lĩnh vực, đặc biệt nổi bật trong y học cá
thể hóa.

Bên cạnh đó, sự phát triển vượt bậc của trí tuệ nhân tạo (AI) đóng vai trò quan
trọng trong việc nâng cao hiệu quả khai thác dữ liệu cảm biến. AI không chỉ tối
ưu hóa quy trình xử lý và phân tích dữ liệu mà còn thúc đẩy khả năng phân loại,
phân cụm, dự đoán và đưa ra quyết định. Trong bài toán chẩn đoán hội chứng
ngưng thở khi ngủ, sự phối hợp giữa dữ liệu từ cảm biến và thuật toán học máy
không chỉ đảm bảo độ chính xác cao trong thu thập dữ liệu mà còn mở ra khả năng
phân tích chuyên sâu về các yếu tố sinh lý, phục vụ quá trình đánh giá toàn
diện. Đặc biệt, sự nổi lên của lĩnh vực học máy triển khai trực tiếp trên các
vi điều khiển hoặc thiết bị biên có tài nguyên hạn chế (Tiny Machine Learning -
TinyML) đã đánh dấu bước tiến quan trọng trong việc hiện thực hóa các hệ thống
giám sát sức khỏe thuận tiện, tiết kiệm năng lượng, chi phí thấp và có khả năng
hoạt động độc lập không phụ thuộc vào kết nối mạng hoặc nền tảng đám mây.

Trên cơ sở đó, luân văn này tập trung nghiên cứu, phát triển hệ thống thu thập,
mô hình học máy biên nhằm phân loại tư thế ngủ, sử dụng đơn dữ liệu cảm biến
gia tốc. Luận văn được thực hiện qua hai giai đoạn chính. Giai đoạn thứ nhất
tập trung vào việc nghiên cứu và phát triển hệ thống thu thập, xử lý, lưu trữ
dữ liệu cảm biến; đồng thời tiến hành phân tích, trích xuất đặc trưng và huấn
luyện, đánh giá một số mô hình học máy phù hợp cho bài toán phân loại tư thế
ngủ. Giai đoạn thứ hai tập trung vào nghiên cứu và phát triển phần cứng riêng,
cũng như triển khai mô hình học máy đã lựa chọn lên nền tảng phần cứng đó.

\vspace{-0.5cm}
\begin{flushleft}
	\textit{\textbf{Từ khóa: } cảm biến gia tốc, học máy, ngưng thở tắc nghẽn khi ngủ,  Tiny ML}
\end{flushleft}
% \changefontsizes[16pt]{13pt}

\changefontsizes[16pt]{13pt}


% \changefontsizes[16pt]{14pt}
% \input{chapters/0.2.start}

\tableofcontents

% \pagestyle{fancy}

\addcontentsline{toc}{chapter}{\listfigurename}
\listoffigures


\listoftables
\addcontentsline{toc}{chapter}{\listtablename}

\printglossaries

% \newpage
% \chapter*{Thuật ngữ}

% \addcontentsline{toc}{chapter}{Thuật ngữ}
% \begin{table}[h!]
%     \centering
%     \begin{adjustbox}{width=1\textwidth}
%     \small
%     \begin{tabular}{|c|c|c|}
%          \hline
%          \textbf{Từ viết tắt} & \textbf{Từ đầy đủ} &\textbf{Ý nghĩa}\\
%          \hline
%         API   & Application Programming Interface & Giao diện lập trình ứng dụng\\
%         \hline
%         AST   & Abstract Syntax Tree & Cây cú pháp trừu tượng\\
%         \hline
%          BDD   & Behaviour Driven Development & Mô hình phát triển phần mềm định hướng hành vi\\
%         \hline
%         CFG   & Control Flow Graph & Đồ thị dòng điều khiển\\
%          \hline
%         CFT   & Control Flow Testing & Kiểm thử dòng điều khiển\\
%          \hline
%         DFT  & Data Flow Testing & Kiểm thử dòng dữ liệu\\
%         \hline
%         HTML   & Hyper Text Markup Language & Ngôn ngữ đánh dấu siêu văn bản\\
%         \hline
%         JSON   & JavaScript Object Notation & Cú pháp lưu trữ và trao đổi dữ liệu\\
%         \hline
%         MC/DC   & Modified Condition/Decision Coverage & Bao phủ quyết định sửa đổi\\
%         \hline
%         OOP   & Object-oriented Programming & Lập trình hướng đối tượng\\
%          \hline
%         SE   & Symbolic Execution & Thực thi tượng trưng\\
%         \hline
%         SMT   & Satisfiability Modulo Theories & Lý thuyết mô-đun thỏa mãn\\
%         \hline
%         TDD   & Test Driven Development & Mô hình phát triển phần mềm định hướng kiểm thử\\
%         \hline
%         XML   & Extensible Markup Language & Ngôn ngữ đánh dấu có thể mở rộng\\
%         \hline
%         \end{tabular}
%     \end{adjustbox}
%     % \caption{Bảng thuật ngữ được sử dụng trong khóa luận}
%     \label{table:1}
% \end{table}

\mainmatter



\changefontsizes[16pt]{13pt}
\chapter*{Đặt vấn đề}
\addcontentsline{toc}{chapter}{Đặt vấn đề}
\thispagestyle{empty}
Ngưng thở tắc nghẽn khi ngủ (Obstructive Sleep Apnea – OSA) là một rối loạn hô hấp phổ biến trong giấc ngủ, được đặc trưng bởi các đợt ngưng thở hoặc giảm thông khí tắc nghẽn lặp đi lặp lại trong lúc ngủ, dẫn đến việc gián đoạn giấc ngủ do vi thức giấc và giảm oxy trong máu.  Tỷ lệ hiện mắc OSA tại Việt Nam ước tính khoảng 8,5\% \cite{nguoimacOSA_VN}. OSA hiện được công nhận là một yếu tố có nguy cơ độc lập đối với nhiều bệnh lý liên quan đến tim mạch, đặc biệt là tăng huyết áp. Ngoài ra, hội chứng này còn có mối liên hệ đáng kể với các nguy cơ như tai nạn giao thông, tai nạn lao động dẫn đến làm gia tăng gánh nặng kinh tế xã hội\cite{osa_bike}\cite{Marin2005}\cite{drive_osa}. Đáng chú ý, tình trạng ngưng thở khi ngủ kéo dài và không được phát hiện, điều trị có thể ảnh hưởng nghiêm trọng đến sức khỏe gây ra rối loạn nhịp tim và một trong những nguyên nhân gây đột tử \cite{sumarry_osa}. Theo PSG.TS Nguyễn Thy Khuê, Hội Y học Giấc ngủ Việt Nam, ngưng thở khi ngủ còn là một trong nhưng dấu hiệu rõ ràng của bệnh đái tháo đường, bệnh thận \cite{bsThyKhue}. OSA được phát hiện ở hơn 20\% người bệnh đái tháo đường và làm trầm trọng thêm các rối loạn chuyển hóa, đặc biệt là đái tháo đường type 2. Trong một nghiên cứu tiêu biểu tại Việt Nam, GS.TS. Dương Quý Sỹ và cộng sự đã khảo sát 524 trẻ em mắc rối loạn tăng động giảm chú ý (Attention Deficit Hyperactivity Disorder - ADHD) tại Bệnh viện Nhi Trung ương Việt Nam. Kết quả cho thấy tỷ lệ mắc (OSA) ở nhóm này là 23.3\%, trong đó chủ yếu ở mức độ trung bình đến nặng \cite{ThaySUCHildren}. Nghiên cứu cũng đồng thời xác định mối tương quan đáng kể giữa mức độ nghiêm trọng của OSA và các triệu chứng mất tập trung, tăng động, rối loạn hành vi, lo âu và trầm cảm. Phát hiện nhấn mạnh sự cần thiết của việc sàng lọc OSA trong quá trình điều trị toàn diện ADHD ở trẻ em. 

Một dạng đặc biệt của OSA được ghi nhận là ngưng thở khi ngủ do tư thế (Positional Obstructive Sleep Apnea - pOSA). Bệnh nhân được chẩn đoán mắc pOSA có chỉ số AHI lớn hơn 5, và giá trị AHI ở tư thế ngửa cao gấp ít nhất hai lần so với nằm ở tư thế khác \cite{heinzer2018}. Các nghiên cứu gần đây chỉ ra rằng tỷ lệ mắc pOSA lên tới 50\% bệnh nhân OSA \cite{sabil2020}. Điều này cho thấy tư thế ngủ có ảnh hưởng sinh lý rõ rệt đến sự sụp đổ đường thở trên, đặc biệt ở tư thế nằm ngữa. Lúc nằm ngửa, trọng lực làm xẹp các cơ vùng họng dẫn đến làm hẹp khoang khí.

Việc chẩn đoán hội chứng ngưng thở tắc nghẽn khi ngủ (OSA) hiện nay chủ yếu được thực hiện thông qua hai phương pháp: đa ký giấc ngủ (Polysomnography – PSG) và thiết bị kiểm tra giấc ngủ tại nhà (Home Sleep Test – HST). Trong đó, PSG được xem là tiêu chuẩn vàng trong việc đánh giá OSA. Do phương pháp này cho phép thu thập đồng thời nhiều thông số sinh lý quan trọng bao gồm: luồng khí hô hấp qua mũi và/hoặc miệng, cử động thành ngực và bụng, tiếng ngáy, điện não đồ (Electroencephalography – EEG), điện tâm đồ (Electrocardiography – ECG), điện cơ đồ (Electromyography – EMG), và độ bão hòa oxy trong máu (SpO$_2$). Quá trình đo PSG phải được thực hiện trong môi trường có kiểm soát tại các cơ sở y tế chuyên khoa, dưới sự giám sát trực tiếp của bác sỹ chuyên ngành giấc ngủ hoặc kĩ thuật viên có chuyên môn\cite{psg_paper}\cite{kushida2005psg}. 

Mặc dù PSG vẫn giữ vai trò là phương pháp tham chiếu trong chẩn đoán và theo dõi chất lượng giấc ngủ cũng như các rối loạn liên quan, nhưng việc triển khai kỹ thuật này thường đòi hỏi chi phí cao, trang thiết bị chuyên dụng và điều kiện thực hiện tại các cơ sở y tế chuyên khoa. Một thách thức khác của PSG là người bệnh thường cảm thấy bất tiện và cảm giác khó chịu do mang nhiều cảm biến gắn trên cơ thể trong suốt đêm, dẫn đến nguy cơ gián đoạn hoặc sai lệnh dữ liệu trong quá trình ghi nhận. Vì vậy, các hệ thống HST đang ngày càng thu hút sự quan tâm từ cộng đồng khoa học toàn cầu\cite{e3hst}\cite{hstSurvey}\cite{hst_paper}. Các thiết bị HST hiện đại dùng các cảm biến không xâm lấn nhằm ghi nhận và phân tích một số tín hiệu sinh lý cơ bản như luồng khí hô hấp, tư thế ngủ, áp suất mũi, độ bão hòa oxy và nhịp tim. Việc cải thiện chất lượng, kéo dài thời lượng sử dụng và tăng độ chính xác và cải thiện mức độ thoải mái vẫn là những thách thức lớn đối với giới nghiên cứu và các nhà sản xuất thiết bị y tế. Tuy nhiên, với những tiến bộ công nghệ đang diễn ra nhanh chóng, HST có tiềm năng trở thành một công cụ chẩn đoán quan trọng và được ứng dụng rộng rãi trong lâm sàng. Điều này không chỉ mang lại sự thuận tiện và chấp nhận cao hơn từ phía người bệnh, mà còn góp phần làm giảm gánh nặng chi phí và áp lực cho hệ thống chăm sóc sức khỏe.

Trong những năm gần đây, nhiều nhóm nghiên đang chú trọng nghiên cứu, phát triển hệ thống HST nhằm mục đích thay thế hoặc hỗ trợ cho đa ký giấc ngủ. Năm 2011, Collop và cộng sự đã phát triển một hệ thống phân loại SCOPER (Sleep, Cardiovacular, Oximetry, Position, Effort, and Respiration) để đánh giá các tín hiệu sinh lý thu nhận trong việc chẩn đoán OSA\cite{hst_6p_paper}. Tác giả Morillo D và cộng sự đề xuất một phương pháp sàng lọc ngưng thở tắc nghẽn khi ngủ dựa trên cảm biến gia tốc gắn tại vị trí hõm ức, cho phép trích xuất các tín hiệu hô hấp, tim mạch và tiếng ngáy bằng kỹ thuật xử lý tín hiệu số, từ đó chứng minh tính khả thi của thiết bị di động đơn giản và chi phí thấp trong hỗ trợ chẩn đoán hội chứng ngưng thở – giảm thở khi ngủ\cite{morillo2010accelerometer}. Trong nhóm các thiết bị đeo, A.H. Yüzer và cộng sự đã phát triển thiết bị đeo tay sử dụng cảm biến gia tốc ADXL345 để phát hiện và phát tín hiệu rung khi cảnh báo. \cite{hst_wear_paper}. Tương tự, nhóm Yunyoung Nam và cộng sự cũng đã tích hợp hệ thống thu thập, phân tích sử dụng một cảm biến gia tốc ba trục và một cảm biến áp suất để giám sát chất lượng giấc ngủ tư thế ngủ, trạng thái ngủ, giai đoạn ngủ (REM và chu kỳ giai đoạn ngủ không REM) \cite{hst_pressure_paper}. Tại Việt Nam, nhóm nghiên cứu của  Giáo sư Lê Tiến Thường, trường đại học Bách Khoa TP Hồ Chí Minh đã sử dụng cảm biến gia tốc MPU6050 cùng với vi xử lý ESP32 nhằm ghi nhận hơi thở và nhịp tim của bệnh nhân OSA thông qua rung động, và dòng chảy của động mạch và tĩnh mạch ở cổ \cite{thuong_wear_paper}. Gần đây, Domingues và cộng sự (2024) xây dựng một mô hình mạng nơ-ron nhân tạo dựa trên dữ liệu từ máy đo (SpO$_2$), cấm biến gia tốc và ghi âm tiếng ngáy của hệ thống Biologix, nhằm dự đoán chính xác trạng thái ngủ. Kết quả cho thấy mô hình này có khả nâng cao độ chính xác trong chẩn đoán ngưng thở khi ngủ tại nhà, tiệm cận với tiêu chuẩn của đa ký giấc ngủ truyền thống \cite{domingues2024sleep}. Một hướng nghiên cứu khác là cảm biến đặt dưới nệm giường. Tác giả Andrei Boiko và cộng sự đánh giá hệ thống phát sử dụng cảm biến gia tốc đặt dưới đệm giường để ghi dao động do cử động ngực khi thở. Kết quả cho thấy thuật toán phát hiện ngưng thở đạt độ chính xác, độ đặc hiệu và độ nhạy lần lượt là 94.6\%, 95.3\% và 93.7\% \cite{Boiko2023}. 

Nhìn chung, việc chẩn đoán hội chứng ngưng thở khi ngủ (OSA) và đặc biệt là dạng phụ thuộc tư thế (pOSA), 
đòi hỏi một hệ thống giám sát có khả năng thu thập liên tục dữ liệu sinh lý và 
đưa ra quyết định chính xác trong thời gian thực. 
Trong bối cảnh đó, các mô hình học máy đang trở thành 
công cụ đắc lực để phân loại mức độ nghiêm trọng của OSA 
thông qua chỉ số AHI hoặc nhận diện tư thế ngủ dựa trên tín hiệu cảm biến. 
Đây là hướng tiếp cận liên ngành giữa y học giấc ngủ và trí tuệ nhân tạo ứng dụng \cite{osa_sanchez2025}.
Trong số các thuật toán học máy truyền được sử dụng phổ biến, 
Random Forest (RF) \cite{genuer2020random} nổi bật nhờ khả năng kháng chống lại quá khớp và độ chính xác cao. 
Trong nghiên cứu \cite{wang2023ml_wearable}, Wang và cộng sự đã ứng dụng RF để phân loại các trường hợp ngưng thở khi ngủ, đạt độ chính xác 93.88\%, độ nhạy 89.93\% và độ đặc hiệu 91.8\%. 
Một nghiên cứu khác \cite{yeo2022respiratory}, Yeo và cộng sự sử dụng RF cho nhiệm vụ phân loại sự kiện hô hấp, thu được độ chính xác 83\%, độ nhạy 99\% và F1-score 81\%. 
Mặc dù có sự khác biệt về nguồn dữ liệu và phương pháp trích chọn đặc trưng, RF vẫn cho thấy hiệu quả vượt trội khi 
so sánh với các thuật toán khác như SVM, LDA hay QDA \cite{wang2023ml_wearable}, \cite{yeo2022respiratory}, \cite{parbat2024multiscale}.

Bên cạnh đó, thuật toán SVM \cite{cortes1995svm} cũng đã được áp dụng nhằm xác định siêu phẳng tối ưu để phân loại các nhóm trong không gian đặc trưng. 
Trong nghiên cứu \cite{wang2023ml_wearable}, Wang cũng sử dụng thêm mô hình SVM và đạt độ chính xác 88,28\%, độ đặc hiệu 91,69\% và độ nhạy 83,94\%, 
cho thấy hiệu quả cao trong phát hiện ngưng thở khi ngủ, dù mô hình Random Forest thường có kết quả cao hơn.
Ở nghiên cứu \cite{yeo2022respiratory}, SVM đạt độ chính xác 83\% và hệ số Cohen’s kappa 0,53 trong phân loại sự kiện hô hấp theo từng phút. 
Trong \cite{parbat2024multiscale}, SVM được huấn luyện trên tín hiệu ECG một kênh, đạt độ chính xác 69,13\%, góp phần cải thiện hiệu suất của hệ thống phân loại khi tích hợp trong mô hình tổ hợp. 
Những kết quả này cho thấy SVM vẫn là một phương pháp có giá trị trong ứng dụng học máy cho chẩn đoán ngưng thở khi ngủ.

K-Nearest Neighbors (KNN) \cite{cunningham2007knn} là một thuật toán khác 
cũng thường xuyên được áp dụng trong các nghiên cứu 
về phát hiện ngưng thở khi ngủ \cite{wang2023ml_wearable}, \cite{jeon2020realtime}. 
Dựa trên nguyên lý đo độ tương đồng trong không gian đặc trưng, 
KNN phân loại một điểm dữ liệu mới dựa trên nhãn của các điểm lân cận gần nhất. 
Wang và cộng sự đã dùng mô hình KNN xử lý tín hiệu quang học PPG và đạt độ chính xác 85.06\%, với độ đặc hiệu 86.11\% và độ nhạy 83.72\% \cite{wang2023ml_wearable}. 
Trong khi đó, nghiên cứu \cite{jeon2020realtime} báo cáo hiệu quả vượt trội hơn với accuracy lên đến 95\%, đồng thời vẫn đảm bảo thời gian thực thi đáp ứng yêu cầu hệ thống. 
Thành công này được cho là đến từ khả năng đo lường chính xác độ tương đồng giữa dữ liệu quan sát và dữ liệu đã học, 
giúp mô hình KNN đưa ra dự đoán phù hợp với mức độ nghiêm trọng của OSA.

Bên cạnh các thuật toán truyền thống, mô hình XGBoost \cite{chen2016xgboost} cũng được đưa vào thử nghiệm trong nghiên cứu \cite{wang2023ml_wearable}
nhằm đánh giá khả năng phân loại các mức độ ngưng thở khi ngủ. 
Là một biến thể của thuật toán boosting, XGBoost được thiết kế tối ưu cho hiệu suất tính toán 
và có khả năng xử lý hiệu quả cả bài toán hồi quy và phân loại. 
Kết quả cho thấy XGBoost đạt độ chính xác 82.05\%, độ đặc hiệu 84.91\% và độ nhạy 78.42\%, 
cho thấy tiềm năng lớn của mô hình này trong ứng dụng lâm sàng, 
đặc biệt trong các hệ thống đòi hỏi cân bằng giữa độ chính xác và tốc độ huấn luyện.
Trong nghiên cứu \cite{yeo2022respiratory}, 
thuật toán Linear Discriminant Analysis (LDA) \cite{tharwat2017lda} được đánh giá là một phương pháp quan trọng. 
LDA sử dụng trung bình và ma trận hiệp phương sai của từng lớp để xác định ranh giới quyết định tối ưu, 
nhằm tối đa hóa sự phân biệt giữa các lớp và giảm thiểu phương sai nội bộ. 
Trong bối cảnh nghiên cứu, LDA cho thấy hiệu quả vượt trội trong phát hiện sự kiện hô hấp với độ chính xác 
81\%, độ nhạy 88\%, độ đặc hiệu 79\% và điểm F1 đạt 81\%.




Luận văn này nhằm nghiên cứu và phát triển một hệ thống 
giám sát tư thế ngủ, ứng dụng cảm biến gia tốc kết hợp 
với thuật toán học máy gọn nhẹ (TinyML) và phần mềm di động. 
Thiết bị được thiết kế theo hướng đeo được (wearable), 
cho phép ghi nhận và phân loại chính xác các tư thế 
ngủ phổ biến như nằm ngửa, nằm sấp, nằm nghiêng trái và phải. 
Hệ thống tích hợp khả năng thu thập – xử lý – phân loại tín 
hiệu ngay trên vi điều khiển, hướng tới độ chính xác cao, 
độ trễ thấp và khả năng triển khai tại nhà với chi phí hợp lý. 
Tuy chưa thực hiện chức năng sàng lọc hội chứng ngưng thở khi ngủ (Obstructive Sleep Apnea – OSA), 
hệ thống được xây dựng như một nền tảng kỹ thuật tiềm năng, phục vụ cho các nghiên cứu ứng dụng trong tương lai, 
đặc biệt là hỗ trợ đánh giá nguy cơ pOSA dựa trên tư thế ngủ.
Luận văn trình bày tổng quan về hội chứng ngưng thở khi ngủ (Obstructive Sleep Apnea – OSA), 
nhấn mạnh tầm quan trọng của tư thế ngủ trong việc đánh giá nguy cơ mắc OSA theo tư thế (positional OSA – pOSA). 
Bên cạnh đó, các xu hướng công nghệ hiện đại ứng dụng trí tuệ nhân tạo (AI), 
cảm biến đeo được (wearable sensors) và học máy gọn nhẹ (TinyML) 
trong giám sát giấc ngủ cũng được hệ thống hóa nhằm làm 
cơ sở cho thiết kế hệ thống.

Trên cơ sở đó, đề tài tập trung xây dựng một hệ thống phần mềm ứng dụng cảm biến gia tốc 3 trục, 
sử dụng phần cứng có sẵn, để thu thập – xử lý và phân loại tư thế ngủ theo thời gian thực. 
Thiết bị kết nối với ứng dụng di động qua Bluetooth, giúp đồng bộ dữ liệu, 
hỗ trợ huấn luyện và triển khai mô hình học máy ngay trên thiết bị 
vi điều khiển. Hệ thống hướng tới khả năng ứng dụng tại nhà, 
hỗ trợ theo dõi tư thế ngủ – một yếu tố quan trọng trong đánh giá nguy cơ pOSA. Mục tiêu cụ thể của khóa luận gồm: 
01) nâng cao độ chính xác trong phân loại tư thế ngủ; 
02) tối ưu quy trình xử lý và phân tích dữ liệu; 
03) hỗ trợ thu thập, lưu trữ và xử lý dữ liệu theo thời gian thực; 
và 04) triển khai mô hình TinyML trên thiết bị biên.
Luận văn được thực hiện thông qua các phương pháp chính: 01) khảo sát và tổng hợp tài liệu liên quan đến OSA, cảm biến và học máy; 
02) xây dựng phần mềm thu thập – xử lý dữ liệu và thiết lập quy trình phân loại tư thế ngủ; 
và 03) thực nghiệm thu thập dữ liệu từ người dùng, huấn luyện và đánh giá hiệu năng mô hình học máy trên thiết bị nhúng.


Cấu trúc luận văn được trình bày trong ba chương chính như sau:

\noindent\textbf{Chương 1:} Tổng quan về hội chứng ngưng thở khi ngủ và các giải pháp công nghệ trong giám sát tư thế ngủ.

\noindent\textbf{Chương 2:} Xây dựng hệ thống thu thập dữ liệu, huấn luyện mô hình và chuẩn bị triển khai trên thiết bị biên.

\noindent\textbf{Chương 3:} Thử nghiệm mô hình phân loại tư thế ngủ và đánh giá triển khai trên nền tảng vi điều khiển.


\chapter{Tổng quan về OSA và các thiết bị theo dõi \label{background_section}}

\section{Hội chứng ngưng thở khi ngủ}

Trong bối cảnh nghiên cứu các rối loạn hô hấp liên quan đến giấc ngủ, 
việc định nghĩa rõ ràng các kiểu sự kiện hô hấp là 
cần thiết nhằm phục vụ cho mục đích chẩn đoán, 
phân tầng nguy cơ và điều trị. 
Ba sự kiện hô hấp quan trọng bao gồm: ngưng thở (apnea), giảm thở (hypopnea), và hiện tượng kích hoạt liên quan đến nỗ lực hô hấp
(Respiratory Effort–Related Arousal – RERA) \cite{berry2012scoring}.

Ngưng thở (Apnea) được Hiệp hội Y học Giấc ngủ Hoa Kỳ (AASM) định nghĩa là 
sự ngưng luồng khí hô hấp qua mũi và miệng trong thời gian tối thiểu 10 giây. 
Các sự kiện ngưng thở có thể kéo dài đến 30 giây hoặc hơn trong những trường hợp nặng.
Có ba dạng chính của hội chứng ngưng thở khi ngủ \cite{ThaySYOSA}: ngưng thở tắc nghẽn, ngưng thở trung ương, Ngưng thở hỗn hợp. Trong đó: 
01)Ngưng thở khi ngủ do tắc nghẽn (Obstructive Sleep Apnea – OSA) là dạng phổ biến nhất, xảy ra khi các cơ vùng họng giãn ra và làm tắc đường thở, cản trở không khí đi vào phổi \cite{osa_summary}; 
02)Ngưng thở khi ngủ do trung ương (Central Sleep Apnea – CSA) là tình trạng não không gửi tín hiệu đúng đến các cơ kiểm soát hô hấp \cite{eckert2007csa}. Mặc dù ít gặp hơn OSA, CSA vẫn có thể gây ra mệt mỏi kéo dài và đau đầu vào buổi sáng.
03)Ngưng thở hỗn hợp (Mixed Apnea) là sự kết hợp của cả hai yếu tố: giai đoạn đầu của sự kiện không có nỗ lực hô hấp (giống CSA), sau đó xuất hiện nỗ lực hô hấp (giống OSA). 
Dạng này thường xuất hiện ở những bệnh nhân OSA nặng và được phân loại vào nhóm ngưng thở tắc nghẽn trong chỉ số AHI

Giảm thở (Hypopnea) với hai mức tiêu chuẩn đánh giá: 
01) Tiêu chuẩn khuyến nghị: một sự kiện được xác định 
là hypopnea nếu thỏa mãn đồng thời ba điều kiện: 
(i) biên độ tín hiệu luồng khí giảm $\geq 30$\% so với nền trước sự kiện, 
đo bằng cảm biến áp lực mũi hoặc thiết bị CPAP; 
(ii) thời gian giảm tín hiệu kéo dài $\geq 10$ giây; 
và (iii) kèm theo giảm độ bão hòa oxy $\geq 3$\% và/hoặc gây kích hoạt điện não (arousal); 
02) Tiêu chuẩn chấp nhận được (Acceptable): tương tự như trên, tuy nhiên yêu cầu giảm độ bão hòa 
oxy phải đạt từ 4\% trở lên.

RERA là sự kiện gia tăng nỗ lực hô hấp kéo dài $\geq 10$ giây, 
gây đánh thức khỏi giấc ngủ nhưng không đủ tiêu chí của apnea hoặc hypopnea. 
01) Phương pháp tiêu chuẩn để đo là đo áp lực thực quản, tuy nhiên khó áp dụng do gây khó chịu cho bệnh nhân. 
02) Phương án thay thế đáng tin cậy là dùng ống thông mũi kết hợp cảm biến áp lực, 
cho kết quả tương đương về mặt lâm sàng. 
03) RERA được tính vào chỉ số rối loạn hô hấp (Respiratory Disturbance Index - RDI); 
RDI >5 là bất thường, >15 là có ý nghĩa lâm sàng.

Trong số các rối loạn hô hấp liên quan đến giấc ngủ đã đề cập, 
hội chứng ngưng thở khi ngủ do tắc nghẽn (Obstructive Sleep Apnea – OSA) 
là dạng phổ biến nhất và có tác động sâu rộng đến sức khỏe cộng đồng. 
Đáng chú ý, một phân nhóm quan trọng của OSA là ngưng thở khi ngủ 
có liên quan đến tư thế (positional OSA – pOSA), 
trong đó tần suất các sự kiện ngưng thở tăng rõ rệt khi người bệnh 
nằm ngửa so với các tư thế khác. 
Do vậy, luận văn này tập trung phân tích chuyên sâu về OSA và đặc biệt là pOSA, 
làm rõ cơ chế bệnh sinh, tiêu chí chẩn đoán và các chỉ số phân tầng mức độ nặng. 
Trên cơ sở đó, nghiên cứu đề xuất và phát triển một hệ thống giám sát tư thế ngủ tích hợp cảm biến gia tốc và 
mô hình học máy gọn nhẹ (TinyML), hướng tới khả năng sàng lọc OSA tại nhà bằng thiết bị đeo thông minh hoạt động độc lập trên nền tảng vi điều khiển.

Các mức độ của hội chứng ngưng thở khi ngủ do tắc nghẽn 
(OSA) được đánh giá dựa trên chỉ số ngưng thở giảm thở 
(Apnea–Hypopnea Index - AHI) bằng cách chia tổng số 
lần ngưng thở và hẹp thở cho tổng số giờ đã ngủ, 
với mỗi sự kiện phải kéo dài ít nhất 10 giây Bảng~\ref{ahi} \cite{osa_summary}. 
Hội chứng ngưng thở khi ngủ tắc nghẽn tư thế (positional Obstructive Sleep Apnea – pOSA) 
là một dạng đặc biệt của OSA, trong đó mức độ nghiêm trọng 
của hội chứng ngưng thở chịu ảnh hưởng đáng kể từ tư thế nằm của bệnh nhân. 
Cụ thể, pOSA được xác định khi chỉ số AHI (Apnea–Hypopnea Index) ở tư thế nằm ngửa cao hơn đáng kể so với các tư thế khác, 
và thường gặp nhất ở bệnh nhân OSA mức độ nhẹ đến trung bình \cite{heinzer2018,aloweidat2023positional}. 
Người mắc pOSA thường có các đặc điểm như trẻ tuổi hơn, chỉ số khối cơ thể (BMI) thấp hơn, và mức độ OSA tổng thể nhẹ hơn so với nhóm bệnh nhân không thuộc dạng tư thế (non-positional OSA – NpOSA).

Trong nhiều năm qua, các nhà nghiên cứu đã đề xuất nhiều tiêu chí 
khác nhau nhằm chẩn đoán pOSA, từ đơn giản đến phức tạp. 
Định nghĩa cổ điển nhất được giới thiệu bởi Cartwright, 
theo đó bệnh nhân được coi là pOSA nếu AHI ở tư thế nằm ngửa lớn 
hơn ít nhất hai lần so với AHI ở tư thế không nằm ngửa \cite{cartwright1984position}. 
Mador sau đó kế thừa định nghĩa này và bổ sung tiêu chí rằng 
AHI ở tư thế không nằm ngửa phải nhỏ hơn 5 lần/giờ, 
nhằm tăng tính đặc hiệu trong chẩn đoán \cite{mador2005prevalence}. 
Song song đó, Levendowski đề xuất một cách tiếp cận theo tỷ lệ, 
trong đó pOSA được xác định khi AHI toàn bộ lớn hơn hoặc bằng 1.5 lần 
AHI ở tư thế không nằm ngửa \cite{levendowski2015neck}. 
Các tiêu chí này có ưu điểm là đơn giản và dễ áp dụng trong lâm sàng 
cũng như trên thiết bị theo dõi tại nhà, 
nhưng có thể bỏ sót những trường hợp ranh giới hoặc đa yếu tố.

Một hệ thống phân loại toàn diện hơn là Amsterdam Positional Obstructive Sleep Apnea Classification (APOC), 
được thiết kế để phản ánh chính xác hơn ảnh hưởng của tư thế đến mức độ nghiêm trọng của OSA \cite{frank2014positional}. 
Tiêu chí APOC xác định pOSA khi bệnh nhân có AHI toàn bộ lớn hơn 5 lần/giờ, 
đồng thời tổng thời gian ngủ (Total Sleep Time – TST) ở tư thế tốt nhất (Best Sleeping Position – BSP) và tư thế gây ra chỉ số AHI cao nhất 
(Worst Sleeping Position – WSP) đều chiếm tối thiểu 10\% TST. 
Ngoài ra, bệnh nhân cần thỏa mãn ít nhất một trong ba điều kiện sau: 
01) AHI ở BSP nhỏ hơn 5; 02) AHI ở BSP thấp hơn AHI toàn bộ; 
03) AHI ở BSP thấp hơn tối thiểu 25\% so với AHI toàn bộ trong trường 
hợp AHI toàn bộ vượt quá 40. Hơn nữa, APOC còn cho phép phân nhóm bệnh nhân thành 
ba mức độ đáp ứng điều trị: nhóm APOC-I bao gồm bệnh nhân có thể khỏi hoàn toàn nhờ PT; 
nhóm APOC-II và APOC-III bao gồm các trường hợp có cải thiện một phần như giảm phân loại OSA 
hoặc giảm chỉ số AHI sau can thiệp tư thế.

\begin{table}[h!]
\caption{\texorpdfstring{Phân loại mức độ OSA dựa trên chỉ số AHI}{Phân loại OSA}}
\label{ahi}
\vspace{-3mm}
\begin{center}
\begin{tabular}{|c|c|}
\hline
AHI & Cấp độ \\
\hline
<5 & Không mắc \\
5 đến 10 & Nhẹ \\
15 đến 30 & Trung bình \\
>30 & Nặng \\
\hline
\end{tabular}
\label{tab1}
\end{center}
\end{table}



Ngưng thở tắc nghẽn khi ngủ thường xảy ra ở người lớn tuổi và những người thừa cân béo phì. 
Yếu tố gây ra có thể liên quan đến cấu trúc hoặc phi cấu trúc, bao gồm cả yếu tố di truyền. 
Tỷ lệ ngưng thở tắc nghẽn là từ 2\% đến 9\% ở người lớn. 
Ngưng thở tắc nghẽn khi ngủ có thể tăng gấp 4 lần ở nam giới và gấp 7 lần hơn ở những người béo phì (ví dụ chỉ số khối cơ thể (Body mass Index - BMI) > 30). 
OSA nặng (AHI > 30/h) làm tăng nguy cơ tử vong ở nam giới trung niên.
Nguyên nhân chủ yếu là do diện tích vòng họng hoặc khoang mũi bị thu hẹp như viêm xoang, 
các khối u, bệnh phì đại tuyến lưỡi, amydal, phì đại tuyến mỡ (đặc biệt ở trẻ em), 
béo phì hoặc đến từ các bệnh lý: tiểu đường, huyết áp cao, các bệnh tim mạch v.v \cite{wright1997health}. 
Ngoài ra, có thể đến từ thói quen không lành mạnh của con người như là sử dụng các chất kích thích, hút thuốc, ngáy khi ngủ \cite{reason_osa}\cite{reasonOsa}. 
Bên cạnh đó, các yếu tố không giải phẫu như hoạt động kém của cơ giãn họng, ngưỡng thức giấc thấp và sự điều hòa hô hấp không ổn định cũng góp phần quan trọng vào cơ chế bệnh sinh. 
Sự tương tác giữa các yếu tố này tạo nên tính đa dạng trong biểu hiện và mức độ nặng của OSA.



Phần lớn bệnh nhân mắc hội chứng ngưng thở khi ngủ tắc nghẽn 
(\gls{OSA}) không tự nhận thức được các rối loạn hô hấp xảy ra trong lúc ngủ. 
Điều này đặc biệt đúng với những người sống hoặc ngủ một mình, 
do thiếu sự quan sát từ bên ngoài. Đáng lưu ý, hơn 80\% các 
trường hợp OSA được phát hiện ở những bệnh nhân mắc các 
bệnh lý liên quan đến béo phì như tiểu đường, bệnh thận, rối loạn lipid máu, v.v.~\cite{wright1997health}.
Hội chứng OSA ảnh hưởng nghiêm trọng đến chất lượng cuộc sống. 
Những hệ lụy thường gặp bao gồm: suy giảm trí nhớ, giảm tỉnh táo, 
dễ cáu gắt, trầm cảm, đau đầu và giảm khả năng tập trung~\cite{flemons1997quality}. Các tác động này làm giảm hiệu suất làm việc, gây rối loạn trong các mối quan hệ xã hội và làm tăng nguy cơ tai nạn giao thông.
Một nghiên cứu bởi Mooe và cộng sự~\cite{mooe1996sleep} 
thực hiện trên nam giới bị bệnh mạch vành (Coronary Artery Disease – CAD) 
cho thấy có tới 37\% bệnh nhân có chỉ số AHI vượt quá 10. 
Trong khi đó, nghiên cứu của Young và cộng sự (1997) đã phát hiện rằng chỉ số 
AHI tăng tỉ lệ thuận với huyết áp tâm thu và tâm trương, với mức ý nghĩa lần lượt 
là $p=0.003$ và $p=0.01$~\cite{young1997population}.

Trong điều kiện hiện tại, đa số bệnh nhân nghi ngờ mắc 
hội chứng ngưng thở tắc nghẽn khi ngủ được khám bới 
bs chuyên khoa Tai Mũi Họng và bác sĩ chuyên gia về ngủ ngáy. 
Khám tổng quát kết hợp khai thác bệnh sử liên quan, sử dụng các thang điểm đánh 
giá buồn ngủ và nguy cơ ngưng thở khi ngủ, như Epworth Sleepiness Scale, STOP-BANG (đã được dịch sang tiếng Việt) được chấp thuận tại Việt Nam như một phương án sáng lọc bệnh nhân OSA. 
hoặc có thể khám nội soi Tai Mũi Họng để tìm nguyên nhân. 
Vì đa số các trường hợp ngáy, ngưng thở khi ngủ là do tắc nghẽn 
ngoại biên, nguyên nhân từ Mũi – Họng – Màn hầu , VA và amidan, 
và những bất thường về hàm mặt khác. 
Việc đánh giá ngưng thở khi ngủ bắt đầu thường bắt đầu bằng 
một khảo sát giấc ngủ toàn diện, bao gồm khai thác bệnh sử liên quan đến 
các triệu chứng lâm sàng đặc trưng, 
sau đó tiến hành đánh giá khách quan thông qua đa ký giấc ngủ (PSG) \cite{diagnosis_osa}\cite{medical2006polysomnography}.
Phương pháp do dùng đa ký giấc ngủ (Polysomnography) với sự giám sát của các bác sĩ chuyên môn được coi là tiêu chuẩn vàng trong chẩn đoán chứng ngưng thở khi ngủ. 

Polysomnography là một phương pháp ghi đa kênh liên tục trong suốt một đêm, 
bao gồm nhiều thông số sinh lý nhằm đánh giá toàn diện hoạt động hô hấp và 
thần kinh khi ngủ. Các thành phần chính trong một đánh giá polysomnography 
bao gồm: điện não đồ (EEG) để ghi lại hoạt động điện của não; 
điện cơ ký (EMG) nhằm đo trương lực cơ, 
đặc biệt là ở cằm và chân; điện động mắt (EOG) để theo dõi chuyển động của nhãn cầu, 
giúp xác định các giai đoạn của giấc ngủ; và điện tâm đồ (ECG) để theo dõi hoạt động điện của tim. 
Bên cạnh đó, quá trình đo cũng bao gồm theo dõi độ bão hòa oxy trong máu (SpO$_2$), 
đo lưu lượng khí thở qua mũi và miệng, đánh giá nỗ lực hô hấp thông qua chuyển động của ngực và bụng, 
đo áp lực khí thở qua mũi, và ghi nhận cường độ tiếng ngáy. 
Tư thế ngủ là một tín hiệu quan trọng trong polysomnography (PSG), 
đặc biệt có giá trị trong chẩn đoán và phân loại hội chứng ngưng thở khi 
ngủ tắc nghẽn phụ thuộc tư thế (positional OSA – pOSA). 
Trong quá trình ghi đa ký giấc ngủ, việc theo dõi liên tục tư thế cơ thể giúp xác định mối liên hệ giữa tư thế nằm 
(như nằm ngửa, nằm nghiêng hoặc nằm sấp) với tần suất và mức độ nghiêm trọng của các rối loạn hô hấp. 
Tập hợp các thông số này cho phép bác sĩ chẩn đoán chính xác hội chứng ngưng thở khi ngủ tắc nghẽn (OSA).

Một trong những hạn chế của phương pháp đánh giá sử dụng (\gls{PSG}) là sự bất tiện, 
chi phí cao và khả năng phổ biến thấp, nhất là đối với phần lớn người bệnh có thu nhập thấp. 
Việc yêu cầu bệnh nhân phải lưu trú qua đêm tại cơ sở y tế, cùng với việc gắn nhiều thiết bị 
theo dõi sinh lý lên cơ thể, không chỉ gây cảm giác khó chịu mà còn tiềm ẩn nguy 
cơ ảnh hưởng đến chất lượng và tính chính xác của dữ liệu thu thập được. 
Chính những bất cập này đã thúc đẩy sự phát triển của các thiết bị theo dõi giấc 
ngủ ngoài trung tâm (Out-of-Center devices) hay còn gọi là thiết bị kiểm tra giấc ngủ tại nhà (Home Sleep Test – HST). 
Những thiết bị này thường được thiết kế với số lượng cảm biến tối giản hơn so với 
PSG truyền thống, đồng thời tích hợp các thuật toán phân tích tự động – được xử lý trực tiếp trên thiết bị 
hoặc thông qua phần mềm chuyên dụng – nhằm hỗ trợ chẩn đoán ngưng thở khi ngủ do tắc nghẽn (OSA) 
một cách thuận tiện và tiết kiệm hơn. Những thông số SCOPERA được coi là cơ sở để xây dựng thiết bị HST trong đó giấc ngủ (Sleep - S), tim mạch (Cardiovascular - C), oxi trong máu (Oximetry - O), cố gắng thở (Effort - E), 
luồng không khí lưu thông (Respiratory - R), âm thở (Audio - A).

Thiết bị đeo hỗ trợ theo dõi giấc ngủ (wearable Health Sleep Technology – HST) 
đang trở thành một xu hướng chủ đạo trong nghiên cứu và ứng dụng lâm 
sàng nhờ khả năng thu thập liên tục dữ liệu sinh lý một cách không xâm 
lấn, thuận tiện và có thể triển khai tại nhà. 
Dựa trên đặc điểm hình thái và vị trí gắn trên cơ thể, 
các thiết bị này có thể được phân thành các nhóm: 
vòng tay (bracelet), đai ngực (chest band), miếng dán (adhesive patch), 
tai nghe (headset), nhẫn thông minh (ring), v.v. 
Các thiết bị này có thể là sản phẩm thương mại sẵn có hoặc được thiết kế riêng cho mục đích nghiên cứu.

\begin{table}[htbp]
    \centering
    \caption{Phân loại thiết bị đeo trong phát hiện OSA và tài liệu tham khảo liên quan}
    \label{tab:wearable_types}
    \begin{tabular}{|p{5.5cm}|p{7.5cm}|}
        \hline
        \textbf{Loại thiết bị đeo} & \textbf{Tài liệu tham khảo} \\
        \hline
        Vòng tay (Bracelet) & \cite{jeon2020realtime}, \cite{shen2022mtcnn} \cite{e3hst} \cite{osa_sanchez2025} \\
        \hline
        Đai ngực (Chest band) & \cite{svmHSt2017}, \cite{chen2024hdc} \cite{e3hst} \cite{osa_sanchez2025} \\
        \hline
        Miếng dán (Adhesive patch) & \cite{Vu2025SleepPosition}, \cite{p_3} \cite{osa_sanchez2025}\\
        \hline
        Dạng khác & \cite{Sleep_Posture_Detection}, \cite{hst_wear_paper} \cite{osa_sanchez2025} \cite{hstSurvey} \cite{hst_paper} \cite{hst_wear_paper}\\
        \hline
    \end{tabular}
\end{table}



Nghiên cứu\cite{svmHSt2017} chứng minh rằng tín hiệu chuyển động ngực (Thoracic movement signal - THO) và bụng (Abdominal movement signal - ABD), 
thu từ các dải piezoelectric đeo được, có thể được sử dụng hiệu quả để phân loại các dạng rối loạn thở khi ngủ thông qua mô hình 
thuật toán SVM. Kết quả cho thấy khi kết hợp cả hai tín hiệu, độ chính xác phân loại đạt trung bình 81.8\%, 
khẳng định tiềm năng ứng dụng của phương pháp này trong sàng lọc và theo dõi OSA tại nhà hoặc trong lâm sàng.
Theo tìm hiểu của tác giả, thiết bị HST vẫn có cấu hình khá phức tạp với nhiều cảm biến và thao thác 
cũng chưa hoàn toàn đơn giản, thêm vào đó giá trung bình của thiết bị vào khoảng 2300 USD làm hạn chế khả năng tiếp 
cận của người sử dụng, đặc biệt là những người có thu nhập trung bình hoặc thấp \cite{hst_review}. 

Xu hướng nghiên cứu hiện nay tập trung vào việc tối giản phần cứng bằng 
cách giảm số lượng cảm biến, đồng thời tích hợp các mô hình học máy 
hoặc trí tuệ nhân tạo nhằm nâng cao độ chính xác và khả năng lặp lại của hệ thống. 
Trong phạm vi luận văn này, tác giả lựa chọn hướng tiếp cận ban đầu là sử dụng cảm biến 
gia tốc để phát hiện các tư thế ngủ có liên quan đến OSA, đồng thời xây dựng mô hình học 
máy hướng tới việc phát triển một thiết bị IoT có khả năng ước lượng mức độ nghiêm trọng 
của OSA thông qua chỉ số AHI.

\section{Ứng dụng cảm biến gia tốc trong đánh giá tư thế ngủ của người mắc OSA tại nhà}

Việc theo dõi tư thế cơ thể đặc biệt hữu ích trong phát hiện 
và điều trị hội chứng ngưng thở khi ngủ phụ thuộc tư thế (positional OSA).
Hiểu được mối quan hệ giữa tư thế ngủ và rối loạn hô hấp sẽ mở ra hướng điều trị cá thể hóa, 
chẳng hạn như liệu pháp định hướng tư thế. 
Việc tích hợp cảm biến đeo được như miếng dán, vòng tay hoặc 
nhẫn thông minh không chỉ tăng sự tiện lợi khi theo dõi tại nhà mà còn cung cấp góc 
nhìn toàn diện về hướng cơ thể và chuyển động hô hấp trong suốt thời gian ngủ. 
Các đặc trưng như mức độ chuyển động khí quản và chu kỳ nỗ lực hô hấp 
có thể được sử dụng để ước lượng mức độ nghiêm trọng của OSA, 
từ đó hướng tới phát hiện chính xác hơn và quản lý cá nhân hóa cho từng bệnh nhân.

Nhiều phương pháp kỹ thuật đã được phát triển nhằm ghi nhận và phân tích tư thế 
ngủ một cách chính xác. Các hệ thống ghi hình sử dụng camera hồng ngoại có khả năng 
thu thập toàn bộ quá trình ngủ trong điều kiện ánh sáng thấp, 
trong khi đó, các cảm biến gia tốc ba trục (triaxial accelerometers) 
cho phép nhận diện sự thay đổi tư thế dựa trên dao động và gia tốc của cơ thể. 
Bên cạnh đó, các thiết bị tích hợp cảm biến áp suất cung cấp thông tin về 
sự phân bố trọng lực và áp lực tiếp xúc, từ đó suy luận tư thế ngủ một cách gián tiếp 
nhưng hiệu quả. Những hệ thống tiên tiến hơn còn kết hợp đa cảm biến và 
tích hợp nhiều phương pháp đồng thời, nhằm nâng cao độ tin cậy, 
tính định lượng và khả năng ứng dụng trong cả môi trường lâm sàng lẫn tại nhà. 
Theo S. Akbarian và đồng các tác giả đã đề cập đến phương pháp giám sát tư 
thế ngủ bằng máy ảnh hồng ngoại kết hợp với công nghệ học sâu (Deep learning) 
\cite{Akbarian_osa} có kết quả tốt nhưng có khó khăn để xác định các góc các khác nhau của khuôn mặt. 
Còn theo A. Channa, M. Yousuf và N. Popescu đã sử dụng cảm biến áp suất được gắn dưới đệm để theo dõi 
thư thế ngủ với 2048 điểm cảm biến \cite{Channa_osa}. 
Trong đó cảm biến gia tốc 3 trục được đánh giá là phương pháp hiệu quả, 
tiết kiệm chi phí và độ chính xác cao. Jeng PY và đồng nghiệp đã thực hiện phát 
triển chế tạo thiết bị đeo tay sử dụng cảm biến gia tốc kết hợp với 
thiết bị ở ngực để lấy nhãn cho dữ liệu. Các phương pháp học máy truyền thống 
được sử dụng với độ chính xác đánh giá 4 tư thế khi ngủ trên 85\% \cite{Jeng_osa}. 

Cách đặt và vị trí đặt cảm biến ảnh hướng rất lớn đến chất lượng, độ chính xác của phép đo. 
Theo đó vị trí xương ức cổ được xem là vị trí có thể thu được tín hiệu chính xác để đặt 
đơn cảm biến \cite{Zhang_osa}. Việc sử dụng nhiều cảm biến ở những vị trí khác như cổ tay, trán, … 
sẽ có thêm nhiều dữ liệu hữu ích để phân tích, đánh giá. 
Ngoài ra, hiện nay với sự phát triển vượt bậc của điện thoại di động, việc tận dụng cảm biến gia tốc 
ở ngay trên chính chiếc điện thoại cũng là giải pháp hữu hiệu. 
Nhóm tác giả trong \cite{Ferrer_osa} đã báo cáo nghiên cứu đánh giá tư thế ngủ 
của bệnh nhân sử dụng thiết bị di động đeo ở xương ức kết hợp với 
phần mềm trên nền tảng Android để thu thập lại dữ liệu kể cả khi tắt màn hình. 
Trong một nghiên cứu tiêu biểu, Natale và cộng sự đã khai thác các cảm biến tích 
hợp sẵn trên điện thoại iPhone để ước lượng các thông số liên 
quan đến chất lượng giấc ngủ, bao gồm tổng thời gian ngủ (Total Sleep Time – TST), 
độ trễ vào giấc (Sleep Onset Latency – SOL) và hiệu quả giấc ngủ (Sleep Efficiency – SE). 
Phương pháp tiếp cận này cho thấy tiềm năng trong việc sử dụng thiết bị di động 
như một công cụ theo dõi giấc ngủ tiện lợi và dễ tiếp cận, đặc biệt trong các nghiên 
cứu cộng đồng và ứng dụng tại nhà\cite{Natale_osa}. Đặc điểm của sử dụng tích hợp cảm biến gia tốc 
trên điện thoại là rất tiện lợi, sử dụng trực tiếp mà không cần phát triển phần cứng. 
Tuy nhiên, việc tiếp xúc điện thoại trực tiếp với cơ thể trực tiếp trong thời gian 
lâu cũng có gây những ảnh hưởng nhất định đến người dùng.

\begin{figure}[!ht]
		\centering
% 		\setlength{\abovecaptionskip}{1pt plus 3pt minus 2pt}
 		\includegraphics[width=\textwidth]{images/vị trí đặt cảm biến.png}
 		\vspace*{-7mm}
		\caption{Vị trí tối ưu để đặt cảm biến gia tốc}
		\label{position_sensor}
\end{figure}

Trong khuôn khổ luận văn, tác giả đề xuất thiết kế một thiết bị đeo 
tiếp xúc sử dụng cảm biến gia tốc được đặt tại vị trí xương ức cổ 
nhằm theo dõi và phân tích tư thế ngủ của người dùng. 
Vị trí này được lựa chọn không chỉ do tính ổn định trong 
quá trình ngủ mà còn thuận lợi để tích hợp thêm các cảm biến 
khác như cảm biến âm thanh và cảm biến nhiệt độ – phục vụ cho các mục tiêu nghiên 
cứu mở rộng của nhóm. Tín hiệu từ cảm biến gia tốc sẽ được 
thu thập dưới dạng ba trục không gian (x, y, z), phản ánh chuyển động và 
hướng trọng lực tương ứng với tư thế cơ thể trong suốt thời gian ngủ. 
Sau quá trình thu thập, dữ liệu gia tốc sẽ được xử lý sơ cấp bao gồm hiệu chỉnh, 
lọc nhiễu, và chuẩn hóa nhằm đảm bảo tính chính xác và đồng nhất giữa các mẫu đo. 
Tiếp theo, các đặc trưng định lượng (features) trong miền thời gian 
sẽ được trích xuất để phục vụ cho bài toán phân loại tư thế ngủ (ngửa, nghiêng trái, nghiêng phải, sấp). 
Các đặc trưng này cùng với dữ liệu gốc sẽ được lưu trữ trong hệ thống để phục vụ cho các bước phân tích tiếp theo, 
bao gồm huấn luyện mô hình học máy hoặc tích hợp với các chỉ số sinh lý khác trong đánh giá rối loạn giấc ngủ, 
đặc biệt là hội chứng ngưng thở khi ngủ (OSA) Hình~\ref{position_sensor}.

Cảm biến gia tốc là một thiết bị đo lường có khả năng phát hiện 
và ghi nhận gia tốc – tức là sự thay đổi vận tốc theo thời gian – 
của một vật thể trong không gian ba chiều. 
Với ưu điểm nhỏ gọn, tiêu thụ năng lượng thấp và chi phí hợp lý, 
cảm biến gia tốc được ứng dụng rộng rãi trong nhiều lĩnh vực 
như điện tử tiêu dùng, ô tô, công nghiệp, và đặc biệt là y học, 
trong các thiết bị theo dõi hoạt động và giấc ngủ.

Nguyên lý hoạt động của cảm biến gia tốc dựa trên 
\textbf{Định luật II Newton}:

\begin{equation}
F = ma
\end{equation}

Trong đó, $F$ là lực tác động lên một khối lượng $m$, tạo ra gia tốc $a$. Trong cấu trúc vi cơ điện tử (MEMS) của cảm biến gia tốc, một khối lượng nhỏ được treo bằng các thanh đàn hồi. Khi cảm biến chịu tác động gia tốc, khối lượng này dịch chuyển, gây ra sự thay đổi về đặc tính điện, chẳng hạn như: 01) thay đổi điện dung trong cảm biến kiểu điện dung (capacitive type); 02) thay đổi điện tích do hiệu ứng áp điện trong cảm biến kiểu áp điện (piezoelectric type); và 03) thay đổi điện áp trong các cảm biến điện trở áp (piezoresistive type).

Tín hiệu điện sinh ra từ quá trình này được khuếch đại và số hóa để xử lý trong các ứng dụng khác nhau. Chính khả năng chuyển đổi giữa năng lượng cơ học và điện học giúp cảm biến gia tốc hoạt động hiệu quả trong việc ghi nhận các trạng thái động học của vật thể, bao gồm: 01) dịch chuyển tuyến tính (linear movement), 02) góc nghiêng (tilt), 03) rung động (vibration), và 04) va chạm (shock) hoặc rơi tự do (free fall).

Trong lĩnh vực y sinh, đặc biệt là trong nghiên cứu về giấc ngủ và ngưng thở khi ngủ (OSA), cảm biến gia tốc được sử dụng để: 01) theo dõi tư thế ngủ (supine, prone, lateral); 02) ghi nhận chu kỳ chuyển động hô hấp; và 03) phát hiện dao động vùng khí quản nhằm xác định sự kiện ngưng thở hoặc giảm thở.

Nhờ khả năng tích hợp dễ dàng vào các thiết bị đeo (vòng tay, miếng dán, nhẫn), cảm biến gia tốc trở thành thành phần cốt lõi trong các hệ thống theo dõi không xâm lấn, hỗ trợ hiệu quả cho việc sàng lọc và đánh giá OSA tại nhà hoặc trong môi trường lâm sàng.
\begin{figure}[!ht]
		\centering
% 		\setlength{\abovecaptionskip}{1pt plus 3pt minus 2pt}
 		\includegraphics[width=\textwidth]{images/acce.png}
 		\vspace*{-7mm}
		\caption{Nguyên lý cơ bản của cảm biến gia tốc}
		\label{acce}
\end{figure}

Như minh họa trong Hình~\ref{acce}, khi cảm biến gia tốc chịu tác động từ một chuyển động, khối gia trọng (proof mass) sẽ dịch chuyển, làm lò xo kết nối bị biến dạng. Sự biến dạng này tạo ra một lực đàn hồi theo định luật Hooke, tỷ lệ thuận với độ giãn của lò xo. Áp dụng định luật II Newton, ta có mối quan hệ giữa lực, khối lượng và gia tốc như sau:

\begin{equation} F = m \cdot a \Rightarrow a = \frac{k \cdot \Delta l}{m} \end{equation}

Trong đó: \begin{itemize} \item $F$ là lực đàn hồi tác dụng lên khối gia trọng (N) \item $m$ là khối lượng của khối gia trọng (kg) \item $k$ là hệ số đàn hồi của lò xo (N/m) \item $\Delta l$ là độ biến dạng (thay đổi chiều dài) của lò xo (m) \end{itemize}

Phương trình trên cho thấy gia tốc có thể được tính toán gián tiếp thông qua độ biến dạng của lò xo, từ đó cho phép cảm biến gia tốc chuyển đổi dao động cơ học thành tín hiệu điện phục vụ cho việc đo đạc và phân tích chuyển động. Trong hệ tọa độ của cảm biến gia tốc ba trục, trục z thường nằm theo phương vuông góc với mặt phẳng ngang và sẽ chịu thêm tác dụng của trọng lực. Do đó, ở trạng thái cân bằng (khi thiết bị đứng yên và không có chuyển động nào khác), giá trị gia tốc đo được tại trục z sẽ xấp xỉ bằng gia tốc trọng trường $g$ (khoảng 9.81 m/s²). Đặc điểm này có thể được khai thác trong việc hiệu chuẩn cảm biến cũng như xác định tư thế không gian tương đối của thiết bị.

Trong khuôn khổ luận văn này, tác giả tập trung tìm hiểu và ứng dụng 
cảm biến gia tốc được chế tạo dựa trên công nghệ vi cơ điện tử (Micro-Electro-Mechanical Systems – MEMS). 
Đây là một công nghệ tiên tiến cho phép tích hợp các thành phần phần cứng siêu nhỏ và 
linh kiện điện tử ngay trên cùng một chip bán dẫn, với kích thước cấu trúc có thể dưới 
10 micromet. Một trong những ưu điểm nổi bật của cảm biến gia tốc MEMS là 
khả năng tích hợp trực tiếp lên bo mạch in (Printed Circuit Board – PCB), 
qua đó giảm thiểu thể tích chiếm dụng, tiết kiệm chi phí sản xuất và 
đơn giản hóa thiết kế hệ thống nhúng. Nhờ đó, công nghệ này đặc biệt phù hợp 
cho các ứng dụng trong thiết bị đeo cá nhân, điện thoại di động, và các 
hệ thống theo dõi sức khỏe thế hệ mới.


Dựa trên nguyên lý hoạt động, cảm biến gia tốc MEMS hiện được phân thành ba loại chính, mỗi loại ứng dụng một cơ chế vật lý khác nhau để chuyển đổi dao động cơ học thành tín hiệu điện \cite{Acce}\cite{cambien}.

Hiện nay, cảm biến gia tốc MEMS được chia thành ba loại chính dựa trên nguyên lý hoạt động. 01) Cảm biến gia tốc dựa trên hiệu ứng điện dung (Capacitive accelerometers): 
Đây là loại cảm biến phổ biến nhất trong các thiết bị điện tử tiêu dùng như điện thoại thông minh 
và thiết bị đeo. Nguyên lý hoạt động dựa trên sự thay đổi điện dung giữa 
các bản cực khi khối gia trọng dịch chuyển dưới tác dụng của gia tốc. 
Sự thay đổi này được chuyển đổi thành tín hiệu điện tỷ lệ với mức gia tốc tác động. 
02) Cảm biến gia tốc dựa trên hiệu ứng áp điện trở (Piezoresistive accelerometers): 
Trong loại cảm biến này, ứng suất cơ học tác động lên vật liệu bán dẫn 
sẽ làm thay đổi điện trở của nó – hiện tượng gọi là hiệu ứng áp điện 
trở. Đặc tính tuyến tính giữa lực và điện trở giúp loại cảm biến này 
hoạt động ổn định trong môi trường có điều kiện khắc nghiệt, đặc biệt là 
nhiệt độ cao. 03) Cảm biến gia tốc dựa trên hiệu ứng áp điện 
(Piezoelectric accelerometers): Loại cảm biến này dựa trên khả năng 
sinh điện tích của các tinh thể áp điện khi bị nén hoặc kéo. Điện tích 
sinh ra tỷ lệ với lực tác động, cho phép ghi nhận các dao động cơ học 
có tần số cao. Cảm biến áp điện thường được sử dụng trong các ứng dụng 
yêu cầu đo rung động chính xác.



\begin{figure}[H]
	\centering
	\includegraphics[width=0.7\textwidth]{images/diendung.png}
	\vspace*{-7mm}
	\caption{Cấu trúc cảm biến gia tốc điện dung}
	\label{acce_mems}
\end{figure}




\textbf{Cảm biến gia tốc kiểu điện dung (Capacitive Accelerometers)}

\textit{Nguyên lý hoạt động}: Cảm biến gia tốc kiểu điện dung hoạt động dựa trên nguyên lý biến thiên điện dung giữa các bản cực trong cấu trúc tụ điện khi chịu tác động bởi gia tốc. Cấu hình cơ bản của cảm biến bao gồm một khối lượng vi mô (proof mass) được treo bằng hệ thống lò xo vi cơ (MEMS spring system), trong đó một đầu được cố định và đầu còn lại liên kết với bản cực di động của tụ điện. Khi có gia tốc tác động theo một phương nhất định, khối lượng này sẽ lệch khỏi vị trí cân bằng, làm thay đổi khoảng cách giữa các bản cực và kéo theo sự biến đổi điện dung. Biến thiên này được phát hiện thông qua mạch đo nhạy điện dung, sau đó được chuyển đổi thành tín hiệu điện tử tỷ lệ với độ lớn của gia tốc. 
Quá trình này cho phép cảm biến ghi nhận gia tốc theo thời gian 
thực với độ chính xác cao và độ nhiễu thấp. Hình~\ref{acce_mems} minh họa nguyên lý dịch chuyển khối lượng và sự thay đổi điện dung trong cấu trúc cảm biến gia tốc MEMS kiểu điện dung.


\begin{figure} [!]
		\centering
 		\includegraphics[width=\textwidth]{images/acce_aptro.png}
 		\vspace*{-7mm}
		\caption{Cấu trúc cảm biến áp trở}
		\label{acce_aptro}
  \FloatBarrier
\end{figure}

\textbf{Cảm biến gia tốc kiểu áp điện trở (Piezoresistive Accelerometers)}

\textit{Nguyên lý hoạt động}: Cảm biến gia tốc kiểu áp điện trở hoạt động dựa trên hiện tượng thay đổi điện trở của các phần tử nhạy cảm khi chịu ứng suất cơ học. Trong cấu hình tiêu chuẩn, các phần tử áp điện trở được gắn trực tiếp lên thanh dầm (cantilever) liên kết với một khối gia trọng được treo trong vùng đo. Khi có gia tốc tác động lên hệ thống, lực quán tính gây ra sự biến dạng cơ học của thanh dầm, từ đó làm thay đổi điện trở của các phần tử cảm biến. Để tăng độ chính xác và khuếch đại tín hiệu, cấu hình cảm biến thường được tích hợp trong một mạch cầu Wheatstone, giúp tối đa hóa độ nhạy và cải thiện tỷ số tín hiệu trên nhiễu (Signal-to-Noise Ratio – SNR) của phép đo (xem Hình~\ref{acce_aptro}).

Cảm biến áp điện trở có ưu điểm nổi bật trong việc ghi nhận các biến đổi gia tốc chậm và có thể hoạt động hiệu quả trong một dải đo rộng. Nhờ đó, loại cảm biến này đặc biệt phù hợp cho các ứng dụng cần đo dao động có biên độ hoặc tần số lớn, chẳng hạn như trong thử nghiệm va chạm, đo động học kết cấu, hoặc môi trường công nghiệp có điều kiện phức tạp. Ngoài ra, cảm biến cũng thể hiện khả năng ổn định tốt trước các dao động nhiệt của môi trường xung quanh.

Tuy nhiên, hạn chế chính của cảm biến kiểu áp điện trở nằm ở độ nhạy tương đối thấp khi đo các tín hiệu yếu hoặc biên độ dao động nhỏ. Điều này có thể làm giảm độ chính xác trong các ứng dụng yêu cầu độ phân giải cao. Bên cạnh đó, chi phí chế tạo và triển khai cảm biến áp điện trở thường cao hơn so với các cảm biến điện dung MEMS, khiến việc tích hợp vào các hệ thống nhúng hoặc thiết bị tiêu dùng gặp nhiều giới hạn về mặt kinh tế.

\begin{figure} [!]
		\centering
 		\includegraphics[width=\textwidth]{images/acce_apdien.png}
 		\vspace*{-7mm}
		\caption{Cấu trúc cảm biến áp điện}
		\label{acce_apdien}
  \FloatBarrier
\end{figure}

\textbf{Cảm biến gia tốc kiểu áp điện (Piezoelectric Accelerometers)}

\textit{Nguyên lý hoạt động}:Cảm biến gia tốc kiểu áp điện hoạt động dựa trên hiệu ứng áp điện của một số vật liệu đặc biệt như gốm sứ hoặc thạch anh. Khi các vật liệu này chịu ứng suất cơ học, chúng sẽ bị biến dạng và tạo ra điện thế trên bề mặt. Lượng điện tích sinh ra có độ lớn tỉ lệ thuận với lực tác động lên cảm biến, trong khi cực tính phụ thuộc vào hướng của lực. Một trong những ưu điểm nổi bật của cảm biến áp điện so với các loại cảm biến gia tốc khác là khối lượng nhẹ và khả năng đáp ứng tần số rất cao, có thể đạt đến mức hàng megahertz. Nhờ đó, cảm biến áp điện đặc biệt phù hợp cho các ứng dụng yêu cầu đo dao động nhanh, ngắn hạn trong môi trường khắc nghiệt hoặc cần độ chính xác cao về thời gian và tần số.

Tuy nhiên, cảm biến kiểu áp điện có trở kháng đầu ra rất cao và chỉ tạo ra điện áp nhỏ, điều này khiến tín hiệu dễ bị suy giảm hoặc nhiễu nếu không được xử lý đúng cách. Để đảm bảo chất lượng tín hiệu và giảm thiểu sai số do tải (loading error), hệ thống đo thường được tích hợp thêm các bộ khuếch đại chuyển đổi trở kháng chuyên dụng, chẳng hạn như bộ khuếch đại điện tích (charge amplifier) (xem Hình~\ref{acce_apdien}).



Trong khuôn khổ luận văn này, cảm biến gia tốc MEMS kiểu điện dung được lựa chọn vì có độ nhạy cao với chuyển động chậm và biên độ nhỏ – đặc trưng của thay đổi tư thế khi ngủ. Loại cảm biến này có kích thước nhỏ, tiêu thụ điện năng thấp, dễ tích hợp vào thiết bị đeo và cho tín hiệu ổn định trong theo dõi dài hạn. So với các loại cảm biến khác, điện dung MEMS có chi phí hợp lý, phù hợp với yêu cầu kỹ thuật và tính khả thi triển khai trong hệ thống theo dõi tư thế ngủ.

\section{Ứng dụng trí tuệ nhân tạo trong đánh giá tư thế ngủ và hỗ trợ chẩn đoán hội chứng ngưng thở khi ngủ (OSA)}
Trong những năm gần đây, các mô hình học máy và mạng nơ-ron đã được ứng dụng rộng rãi trong lĩnh vực y sinh, đặc biệt là trong bài toán phân loại tư thế ngủ và hỗ trợ chẩn đoán hội chứng ngưng thở khi ngủ (OSA). Việc lựa chọn mô hình phù hợp có ý nghĩa quan trọng trong việc xử lý dữ liệu cảm biến phức tạp, tối ưu hóa độ chính xác và nâng cao khả năng phát hiện sớm các rối loạn liên quan đến giấc ngủ.

Nhìn chung, việc chẩn đoán hội chứng ngưng thở khi ngủ (OSA) và đặc biệt là dạng phụ thuộc tư thế (pOSA), 
đòi hỏi một hệ thống giám sát có khả năng thu thập liên tục dữ liệu sinh lý và 
đưa ra quyết định chính xác trong thời gian thực. 
Trong bối cảnh đó, các mô hình học máy đang trở thành 
công cụ đắc lực để phân loại mức độ nghiêm trọng của OSA 
thông qua chỉ số AHI hoặc nhận diện tư thế ngủ dựa trên tín hiệu cảm biến. 
Đây là hướng tiếp cận liên ngành giữa y học giấc ngủ và trí tuệ nhân tạo ứng dụng \cite{osa_sanchez2025}.
Trong số các thuật toán học máy truyền được sử dụng phổ biến, 
Random Forest (RF) \cite{genuer2020random} nổi bật nhờ khả năng kháng chống lại quá khớp và độ chính xác cao. 
Trong nghiên cứu \cite{wang2023ml_wearable}, Wang và cộng sự đã ứng dụng RF để phân loại các trường hợp ngưng thở khi ngủ, đạt độ chính xác 93.88\%, độ nhạy 89.93\% và độ đặc hiệu 91.8\%. 
Một nghiên cứu khác \cite{yeo2022respiratory}, Yeo và cộng sự sử dụng RF cho nhiệm vụ phân loại sự kiện hô hấp, thu được độ chính xác 83\%, độ nhạy 99\% và F1-score 81\%. 
Mặc dù có sự khác biệt về nguồn dữ liệu và phương pháp trích chọn đặc trưng, RF vẫn cho thấy hiệu quả vượt trội khi 
so sánh với các thuật toán khác như SVM, LDA hay QDA \cite{wang2023ml_wearable}, \cite{yeo2022respiratory}, \cite{parbat2024multiscale}.

Bên cạnh đó, thuật toán SVM \cite{cortes1995svm} cũng đã được áp dụng nhằm xác định siêu phẳng tối ưu để phân loại các nhóm trong không gian đặc trưng. 
Trong nghiên cứu \cite{wang2023ml_wearable}, Wang cũng sử dụng thêm mô hình SVM và đạt độ chính xác 88,28\%, độ đặc hiệu 91,69\% và độ nhạy 83,94\%, 
cho thấy hiệu quả cao trong phát hiện ngưng thở khi ngủ, dù mô hình Random Forest thường có kết quả cao hơn.
Ở nghiên cứu \cite{yeo2022respiratory}, SVM đạt độ chính xác 83\% và hệ số Cohen’s kappa 0,53 trong phân loại sự kiện hô hấp theo từng phút. 
Trong \cite{parbat2024multiscale}, SVM được huấn luyện trên tín hiệu ECG một kênh, đạt độ chính xác 69,13\%, góp phần cải thiện hiệu suất của hệ thống phân loại khi tích hợp trong mô hình tổ hợp. 
Những kết quả này cho thấy SVM vẫn là một phương pháp có giá trị trong ứng dụng học máy cho chẩn đoán ngưng thở khi ngủ.

K-Nearest Neighbors (KNN) \cite{cunningham2007knn} là một thuật toán khác 
cũng thường xuyên được áp dụng trong các nghiên cứu 
về phát hiện ngưng thở khi ngủ \cite{wang2023ml_wearable}, \cite{jeon2020realtime}. 
Dựa trên nguyên lý đo độ tương đồng trong không gian đặc trưng, 
KNN phân loại một điểm dữ liệu mới dựa trên nhãn của các điểm lân cận gần nhất. 
Wang và cộng sự đã dùng mô hình KNN xử lý tín hiệu quang học PPG và đạt độ chính xác 85.06\%, với độ đặc hiệu 86.11\% và độ nhạy 83.72\% \cite{wang2023ml_wearable}. 
Trong khi đó, nghiên cứu \cite{jeon2020realtime} báo cáo hiệu quả vượt trội hơn với accuracy lên đến 95\%, đồng thời vẫn đảm bảo thời gian thực thi đáp ứng yêu cầu hệ thống. 
Thành công này được cho là đến từ khả năng đo lường chính xác độ tương đồng giữa dữ liệu quan sát và dữ liệu đã học, 
giúp mô hình KNN đưa ra dự đoán phù hợp với mức độ nghiêm trọng của OSA.

Bên cạnh các thuật toán truyền thống, mô hình XGBoost \cite{chen2016xgboost} cũng được đưa vào thử nghiệm trong nghiên cứu \cite{wang2023ml_wearable}
nhằm đánh giá khả năng phân loại các mức độ ngưng thở khi ngủ. 
Là một biến thể của thuật toán boosting, XGBoost được thiết kế tối ưu cho hiệu suất tính toán 
và có khả năng xử lý hiệu quả cả bài toán hồi quy và phân loại. 
Kết quả cho thấy XGBoost đạt độ chính xác 82.05\%, độ đặc hiệu 84.91\% và độ nhạy 78.42\%, 
cho thấy tiềm năng lớn của mô hình này trong ứng dụng lâm sàng, 
đặc biệt trong các hệ thống đòi hỏi cân bằng giữa độ chính xác và tốc độ huấn luyện.
Trong nghiên cứu \cite{yeo2022respiratory}, 
thuật toán Linear Discriminant Analysis (LDA) \cite{tharwat2017lda} được đánh giá là một phương pháp quan trọng. 
LDA sử dụng trung bình và ma trận hiệp phương sai của từng lớp để xác định ranh giới quyết định tối ưu, 
nhằm tối đa hóa sự phân biệt giữa các lớp và giảm thiểu phương sai nội bộ. 
Trong bối cảnh nghiên cứu, LDA cho thấy hiệu quả vượt trội trong phát hiện sự kiện hô hấp với độ chính xác 
81\%, độ nhạy 88\%, độ đặc hiệu 79\% và điểm F1 đạt 81\%.

Tư thế ngủ của con người thường được phân loại thành bốn nhóm chính: nằm ngửa, nghiêng trái, nghiêng phải và nằm sấp (Hình~\ref{4_tuthe}) \cite{4_ngu}. Việc phân biệt rõ ràng các tư thế này giúp nâng cao độ chính xác trong việc phân tích ảnh hưởng của tư thế đến các chỉ số sinh lý trong giấc ngủ.

\begin{figure}
		\centering
 		\includegraphics[width=\textwidth]{images/4ngu.png}
 		\vspace*{-7mm}
		\caption{Các tư thế ngủ cơ bản của con người}
		\label{4_tuthe}
\end{figure}


Sự phát triển nhanh chóng của trí tuệ nhân tạo (AI) trong lĩnh vực y học đã mở ra nhiều hướng tiếp cận mới trong việc đánh giá tư thế ngủ và chẩn đoán OSA. Các hệ thống AI đang dần chứng minh hiệu quả vượt trội trong việc xử lý dữ liệu cảm biến lớn và phức tạp, từ đó cung cấp các phân tích chính xác về hành vi giấc ngủ của bệnh nhân. Các mô hình học máy được huấn luyện trên tập dữ liệu cảm biến từ accelerometer, gyroscope hoặc thiết bị đeo thông minh có thể tự động phân loại các tư thế ngủ theo thời gian thực, với độ chính xác lên đến trên 90\% trong nhiều nghiên cứu gần đây \cite{Sleep_Posture_Detection}\cite{Vu2025SleepPosition}\cite{HOANG2025116309}. 

Nhờ vào khả năng học và tự hiệu chỉnh, các thuật toán này có thể phân biệt hiệu quả giữa các trạng thái nằm nghiêng trái, nghiêng phải, nằm ngửa và nằm sấp – ngay cả khi có sự biến đổi nhẹ về góc độ hoặc cử động cơ thể. Hơn nữa, AI còn cho phép tích hợp thông tin về tư thế ngủ với các chỉ số sinh lý khác như nhịp tim, nhịp thở, SpO$_2$ và dữ liệu âm thanh để xây dựng mô hình chẩn đoán OSA đa chiều. Việc kết hợp các nguồn dữ liệu này giúp phát hiện chính xác các giai đoạn ngưng thở và giảm thở, đồng thời đánh giá được mức độ ảnh hưởng của từng tư thế đến tình trạng hẹp đường thở trong khi ngủ. Đây là một bước tiến quan trọng hướng đến cá nhân hóa chẩn đoán và điều trị OSA – điều mà các phương pháp truyền thống như đa ký giấc ngủ (polysomnography) còn nhiều hạn chế do chi phí cao và điều kiện thực hiện phức tạp. Đặc biệt, các hệ thống AI có thể được triển khai trong các thiết bị đeo thông minh tại nhà, hỗ trợ theo dõi lâu dài và liên tục mà không gây xâm lấn hay gián đoạn giấc ngủ. Nhờ đó, dữ liệu thu thập được phản ánh chính xác hơn về hành vi giấc ngủ trong môi trường tự nhiên của người bệnh, từ đó nâng cao giá trị lâm sàng của các kết quả phân tích. Ngoài ra, sự tích hợp AI trong các thiết bị di động, cùng với công nghệ điện toán biên (edge computing), có thể cho phép xử lý dữ liệu tại chỗ và phản hồi thời gian thực – mở ra tiềm năng to lớn trong việc sàng lọc, theo dõi và cá nhân hóa chiến lược quản lý OSA.

















% \section{Cơ chế giả lập hành vi\label{mock_mechanism}} 
% % Từ khóa sử dụng: mock, mocking, mock data, HTTP Request, caller, callee, API method
% \begin{lstlisting}[float,language=JavaScript,caption={Ví dụ phương thức API ($searchMasterData()$) gọi đến hai phương thức khác, trong đó có một phương thức ($get()$) cần được tạo hàm giả khi thực thi kiểm thử}, label=api_example,captionpos=b]
async get(req: RequestModel): Promise<any> { //Accessing database (*@\label{declare_get_method}@*)
    return await this.httpService
      .post<any>(Utils.join(this.apiUrl, this.config.CORE_GET), req) (*@\label{call_http_service}@*)
      .toPromise(); (*@\label{get_method_toPromise}@*)}
      
@Post('search') //The tested API
async searchMasterData(@Body() body: RequestModel<BaseDto>): Promise<any> { (*@\label{declare_api_method}@*)
    // Hidden code...
    await this.checkRequired(body.condition['type']); (*@\label{lst_exp_check_require}@*)
    let reqCon = body.condition;
    switch (reqCon['type']) { (*@\label{lst_exp_switch_stmt}@*)
        case "MASTER":(*@\label{lst_exp_case_branch}@*)
            if (!(type of reqCon["keyword"] === "string") (*@\label{lst_exp_if_stmt}@*)) {
                (*@\label{lst_exp_keyword_branch}@*)return ResponseModel(RESULT_STATUS.ERROR);}
            const ret = await this.get(reqCon);(*@\label{lst_exp_start_uncover}@*)
            if (ret.status == 200) { (*@\label{lst_exp_200}@*) 
                // Hidden code... (*@\label{lst_exp_hidden_code}@*)
                return  new ResponseModel(RESULT_STATUS.OK, ret); (*@\label{lst_exp_end_uncover}@*)}
            return new ResponseModel(RESULT_STATUS.ERROR);(*@\label{lst_exp_return_error}@*)
        //Other cases....
        default: (*@\label{lst_exp_default_branch}@*)
            return new ResponseModel(RESULT_STATUS.ERROR);}
}

checkRequired(s: string): Promise<any> { //A normal method (*@\label{declare_check_required_method}@*)
    if (s == null (*@\label{check_null_equal}@*)) {
        this.errors.push("error message");}
}
\end{lstlisting}

% \begin{lstlisting}[float,language=JavaScript,caption={An example presents one caller ($search()$) and two callees ($get()$ and $check()$), in which the first callee ($get()$) needs to be mocked.}, label=api_example,captionpos=b]
% @Post('search')
% async search(body: Dto): any { (*@\label{declare_api_method}@*)
%     this.check(body['type']); (*@\label{lst_exp_check_require}@*)
%     switch (body['type']) { (*@\label{lst_exp_switch_stmt}@*)
%         case "MASTER":(*@\label{lst_exp_case_branch}@*)
%             if (body["keyword"] is not a string)(*@\label{lst_exp_if_stmt}@*)) 
%                 return ...
%             const ret = await this.get(reqCon);(*@\label{lst_exp_start_uncover}@*)
%             if (ret.status == 200 (*@\label{lst_exp_200}@*)) 
%                 return ... (*@\label{lst_exp_end_uncover}@*)
%             else  return ...(*@\label{lst_exp_return_error}@*)
%     }
% }
% //A method gets data from database
% async get(req): any (*@\label{declare_get_method}@*){ ...}
% check(s: string): any (*@\label{declare_check_required_method}@*) {  //A normal callee
%     if (s == null (*@\label{check_null_equal}@*))  
%         this.errors.push("error message");}
% \end{lstlisting}
% % return await this.httpService
% %       .post<any>(Utils.join(this.apiUrl, this.config.CORE_GET), req) (*@\label{call_http_service}@*).toPromise(); (*@\label{get_method_toPromise}@*)
% \begin{lstlisting}[float,language=JavaScript,caption=An example of mocking $get()$ method, label=mock_example,captionpos=b]
% it("first test", (done) => {
%     // generated input
%     const body = {...}; (*@\label{input_declaration}@*)
%     // initialize mock data
%     const response: AxiosResponse < any > = {(*@\label{mock_declare_response}@*) status: 200,
%         data: {}, headers: {}, statusText: 'OK', config: {url:'x'}};
%     // set mock data to method accessing database
%     jest.spyOn(controller_class, "get").mockResolvedValue(response); (*@\label{mock_spy}@*)
%     // call the tested API
%     return request(server).post("/search") (*@\label{begin_test_driver}@*)
%     .send(body)  (*@\label{end_test_driver}@*).expect(201).then(res => {...})
% });
% \end{lstlisting}
% Quá trình thực thi kiểm thử cần được đảm bảo là độc lập và nhanh chóng \cite {clean_coder}. Để làm được điều này, chúng ta có thể áp dụng cơ chế giả lập hành vi cho một số hàm hoặc phương thức tương tác với thành phần bên ngoài như cơ sở dữ liệu, dịch vụ bên thứ ba, v.v. Cơ chế giả lập hành vi (\textit{\gls{mocking}}) được sử dụng để cài đặt các hàm hoặc phương thức với các hành vi mới. Thay vì truy cập các tài nguyên từ xa như trang Web hoặc cơ sở dữ liệu, các nhà phát triển hoặc người kiểm thử có thể thay thế hành vi mới của hàm/phương thức bằng cách sử dụng dữ liệu giả (\textit{\gls{mockdata}}). \Gls{mockdata} được cố ý chèn vào một phần trong mã nguồn. Nó thường được sử dụng như là kết quả của các phương thức hoặc hàm. Nó có nghĩa là những phương thức/hàm này được thay đổi hành vi để phù hợp với việc thực thi kiểm thử. Để đơn giản, thuật ngữ \textit {``\gls{mocked_method}"} dùng để chỉ một phương thức kết nối với các tài nguyên từ xa và phải được thay đổi hành vi. Sử dụng cơ chế \gls{mocking} có hai lợi ích chính là quá trình kiểm thử trở nên nhanh chóng và độc lập, điều này cần có trong các nguyên tắc FIRST \cite {clean_coder}. 

% Ưu điểm đầu tiên của việc sử dụng cơ chế \gls{mocking} là nhanh chóng. Kỹ thuật này làm giảm chi phí tính toán để thực thi dữ liệu thử nghiệm. Nếu không áp dụng cách thức này, việc kết nối với các tài nguyên từ xa có thể gặp phải một số sự cố do kết nối mạng kém hoặc không khả dụng. Thay vì đợi phản hồi từ các tài nguyên từ xa, các phương thức có thể trả về nhanh chóng \gls{mockdata}. Một lợi thế khác của việc sử dụng cơ chế \gls{mocking} là đảm bảo tính độc lập. Nó giúp duy trì tính nhất quán của cơ sở dữ liệu khi thực thi dữ liệu thử nghiệm nhiều lần. Khi một ứng dụng Web doanh nghiệp được triển khai, hầu hết các API có thể yêu cầu một số thay đổi tương tác với cơ sở dữ liệu. Tuy nhiên, trong môi trường thử nghiệm, việc thực thi thử nghiệm không được tác động đến cơ sở dữ liệu để duy trì tính nhất quán của dữ liệu gốc. Thay vì thực hiện hành vi thực tế, các phương thức có thể trả về \gls{mockdata} ngay lập tức. Do đó, việc thực thi thử nghiệm sẽ không bao giờ thực hiện bất kỳ cập nhật nào đối với cơ sở dữ liệu.

% Để làm rõ hơn cơ chế \gls{mocking} được sử dụng như thế nào trong thực tế, ví dụ trong Mã nguồn~\ref{api_example} và Mã nguồn~\ref{mock_example} thể hiện một trường hợp cần phải dùng tới cơ chế này. Ví dụ đầu tiên Mã nguồn~\ref{api_example} có API $``/search"$ lấy dữ liệu từ tài nguyên bên ngoài bằng cách gọi phương thức $get()$ (dòng \ref{lst_exp_start_uncover}). Bên cạnh đó, Mã nguồn~\ref{mock_example} trình bày một ca kiểm thử có sử dụng \gls{mocking} cho API $``/search"$  trong Mã nguồn~\ref{api_example}. 
% Để đơn giản hóa, nếu phương thức $m_1$ gọi phương thức $m_2$, $m_1$ và $m_2$ tương ứng được gọi là
% \textit{``caller"} và \textit{``callee"}.

% \begin{lstlisting}[float,language=JavaScript,caption=Ví dụ áp dụng \gls{mocking} đối với mã nguồn dự án sử dụng framework NestJS, label=mock_example,captionpos=b]
it("first test", (done) => {
    const body = {...}; (*@\label{input_declaration}@*)  // input generated by the two methods
    const response: AxiosResponse < any > = {(*@\label{mock_declare_response}@*)// initialize mock data
        data: {}, headers: {}, config: {},
        status: 200, statusText: 'OK'};
    // set mock data to method accessing database
    jest.spyOn(controller_class, "get")
        .mockResolvedValue( of(response).toPromise()); (*@\label{mock_spy}@*)
    // send request data to the tested API
    return request(server)
    .post("/search") (*@\label{begin_test_driver}@*)
    .send(body)  (*@\label{end_test_driver}@*)
    .expect(201).then(res => {
        console.log(res.body);
        expect(res.body).not.toBeNull();
        done();
    })
});
\end{lstlisting}
% Thứ nhất, Mã nguồn~\ref{api_example} thể hiện một ví dụ cần phải sử dụng cơ chế \gls{mocking} khi thực thi kiểm thử. Ví dụ này có một phương thức chính ($searchMasterData()$) và hai phương thức phụ ($get()$ và $checkRequired()$). Chúng là những đại diện điển hình của mã nguồn dự án được sử dụng làm thực nghiệm trong Mục~\ref{experiment_section} Mỗi phương thức đều có những đặc điểm riêng biệt. Đầu tiên, phương thức chính ($searchMasterData()$) được đánh dấu với ký hiệu $@Post("search")$ cung cấp bởi thư viện NestJS (dòng \ref{declare_api_method}). Điều này có nghĩa là phương thức này tương ứng với một API. API này có thể được lựa chọn nằm trong môi trường kiểm thử. Phương thức này gọi tới hai phương thức khác: phương thức $get()$ để lấy kết quả phản hồi từ một cơ sở dữ liệu (dòng \ref{lst_exp_start_uncover}) và phương thức thứ hai $checkRequired()$ để kiểm tra lại tính hợp lệ của dữ liệu đầu vào (dòng \ref{lst_exp_check_require}). 
% Ngoài ra, phương thức đầu tiên ($get()$) kết nối với máy chủ cơ sở dữ liệu từ xa bằng cách sử dụng đối tượng $httpService$ với phương thức $post()$ để gửi POST Request (dòng \ref{call_http_service}). Phương thức này cần được thiết lập hành vi giả trong tệp kiểm thử. Liên quan đến vấn đề này, kiểu trả về của phương thức này có cấu trúc lớp $AxiousResponse$ bao gồm một số thuộc tính như là $data, headers,config, status$, và $statusText$ \footnote{https://github.com/axios/axios\#response-schema}. Những thuộc tính này cần được cung cấp trong \textit{mock data} để thỏa mãn yêu cầu về kiểu trả về của phương thức. Cuối cùng, phương thức $checkRequired()$ là một phương thức bình thường có nhiệm vụ là kiểm tra sự tồn tại của một thuộc tính đặc biệt trong đầu vào. Vì vậy, nó không cần thiết phải áp dụng cơ chế \gls{mocking} trong tệp thực thi kiểm thử.
% % The caller is an API that needs to be tested and it calls the callee which connects to a remote database.
% % As a result, the consistency of data could be impacted during test execution. Therefore, the actual implementation of this callee has to be replaced by using mock data

% Thứ hai, Mã nguồn~\ref{mock_example} trình bày ví dụ một tập lệnh cho một ca kiểm thử cho API $@Post("search")$ trong Mã nguồn~\ref{api_example}. Tập lệnh này có sử dụng cơ chế \gls{mocking} cho phương thức truy cập đến có sở dữ liệu  $get()$. Đây là ví dụ một khối lệnh viết bằng ngôn ngữ Typescript để xây dựng một ca kiểm thử cho một API trong ứng dụng NestJS. API được gọi với đầu là một HTTP Request có chứa dữ liệu kiểm thử ảnh hưởng đến luồng thực thi của chương trình. Khối lệnh này có cấu trúc bao gồm ba phần: Khai báo giá trị đầu vào (dòng \ref{input_declaration}), khai báo \textit{mock data}  (dòng \ref{mock_declare_response}-\ref{mock_spy}), và lời gọi API (dòng \ref{begin_test_driver}-\ref{end_test_driver}).
% Như đã đề cập ở trước đó, phương thức $get()$ kết nối tới máy chủ cơ sở dữ liệu và cần được thiết lập hành vi thay thế. Vì vậy, \textit{mock data} được cung cấp (dòng \ref{mock_declare_response}-\ref{mock_spy}). Ở khía cạnh đầu tiên, nếu cơ chế \gls{mocking} không được áp dụng, phương thức $get()$ sẽ gửi một POST Request tới máy chủ từ xa dẫn đến tốn thêm thời gian để đợi phản hồi. Thêm vào đó, nếu máy chủ đang không khả dụng, giá trị thuộc tính $status$ của biến $ret$ trong Mã nguồn~\ref{api_example} luôn luôn khác $200$. Điều này có nghĩa là biểu thức điều kiện $ret.status == 200$ luôn luôn nhận giá trị $false$ (dòng \ref{lst_exp_200} trong Mã nguồn~\ref{api_example}), khiến một số câu lệnh không thể được thực thi khi chạy kịch bản kiểm thử (dòng \ref{lst_exp_hidden_code},\ref{lst_exp_end_uncover} trong Mã nguồn~\ref{api_example}). 
% Ở khía cạnh khác, cơ chế \gls{mocking} cần được áp dụng để đảm bảo cơ sở dữ liệu không bị ảnh hưởng (dòng \ref{mock_spy}). \Gls{mockdata} của phương thức $get()$  là giá trị của biến $response$ bao gồm tất cả những thuộc tính cần thiết như là $data, headers,config, status$, và $statusText$ (dòng \ref{mock_declare_response}). Những giá trị này sẽ giúp chương trình thực thi nhiều câu lệnh hơn. 

% Trên thực tế, \gls{mockdata} thường được các nhà phát triển hoặc người thử nghiệm sửa đổi dựa trên kinh nghiệm của họ. Bởi vì luồng thực thi có thể phụ thuộc vào cách các phương thức được giả lập, \gls{mockdata} có thể ảnh hưởng đến phạm vi bao phủ. Do đó, các nhà phát triển hoặc người kiểm tra cần phải hiểu rõ ràng về mã nguồn để thiết lập \gls{mockdata} phù hợp nhằm đạt được độ phủ cao hơn.

% \section{Kiểm thử ứng dụng Web}
% Cách đơn giản nhất để kiểm thử một ứng dụng Web là kiểm thử viên sẽ thực hiện các thao tác nhấp chuột thủ công trên giao diện của hệ thống và đánh giá cách ứng dụng phản hồi. Đây là phương pháp kiểm thử hộp đen \cite{black_box_testing} vì kiểm thử viên không cần biết chi tiết nội hàm của chương trình ứng dụng. Họ chỉ cần quan tâm xem với một đầu vào cụ thể, ứng dụng có thực thi hành vi đúng như đặc tả hay không. Phương pháp kiểm thử này dễ dàng thực hiện được vì không cần bất cứ thiết lập nào trước đó. Tuy nhiên, nó có thể bị ảnh hưởng bởi những lỗi liên quan đến người thực hiện kiểm thử. Ngoài ra, quá trình này cũng tốn rất nhiều thời gian và công sức khi mà tổ hợp các kịch bản người dùng thực hiện trên giao diện là một con số rất lớn. Điều này trở nên thách thức hơn khi mã nguồn luôn luôn có sự thay đổi, quá trình kiểm thử hồi quy cần được thực hiện liên tục để kiểm tra lại các thành phần trước đó vẫn hoạt động đúng như ban đầu. Vì vậy, việc kiểm thử đầy đủ nếu chỉ dựa vào kiểm thử thủ công trên giao diện là không hiệu quả.

% Thay vì kiểm thử viên nhấp chuột thủ công để kiểm thử hệ thống, họ có thể viết các dữ liệu kiểm thử sử dụng trình điều khiển Web. Trình điều khiển Web thực hiện lần lượt các bước mô tả trong dữ liệu kiểm thử và kiểm tra hành vi của hệ thống. Quá trình thực thi các dữ liệu kiểm thử này có thể được tự động hóa cho nên nó có thể tiết kiệm thời gian cho kiểm thử viên ở giai đoạn thực thi hệ thống. Những dữ liệu kiểm thử có thể được tái sử dụng cho những lần kiểm thử hồi quy sau này. Tuy nhiên, về bản chất, việc xây dựng và viết mã lệnh cho các dữ liệu kiểm thử vẫn phải thực hiện thủ công. Đây là công việc rất tốn thời gian và nguồn nhân lực, đòi hỏi lập trình viên có kiến thức về trình điều khiển Web. Để kiểm thử đầy đủ cho một ứng dụng Web, số lượng dữ liệu kiểm thử là rất lớn. Vì vậy, trong thực tế, danh sách dữ liệu kiểm thử thường không đầy đủ dẫn đến một số lỗi tiềm ẩn trong chương trình không thể phát hiện và thường bị bỏ qua.

% Với những hạn chế như đã được để cập ở trên, việc sinh dữ liệu kiểm thử cần được tự động hóa sao cho đảm bảo tính hiệu quả để tiết kiệm thời gian và chi phí cho doanh nghiệp phát triển phần mềm. Đây cũng là một trong những mảng nghiên cứu khá là quan trọng, dành được nhiều sự quan tâm của các nhà nghiên cứu trong lĩnh vực công nghệ phần mềm. Việc kiểm thử cho ứng dụng Web hiện nay gặp phải một số thách thức \cite{web_testing_1, web_testing_2}. Hiện tại, quá trình nghiên cứu các phương pháp tự động hóa sinh dữ liệu kiểm thử cho ứng dụng Web cũng đã đạt được một số kết quả nhất định. Có thể kể đến một số dự án nổi bật như là Artemis\cite{artermis}, Crawljax \cite{crawljax}, và SymJS \cite{symjs}. Artemis là một công cụ hỗ trợ kiểm thử tự động cho ứng dụng Javascript, tập trung vào các ứng dụng đơn trang. Công cụ này sử dụng một số hằng số và giá trị ngẫu nhiên để sinh ra đầu vào cho dữ liệu kiểm thử. Tiếp theo, Crawljax là một công cụ thu thập thông tin và kiểm thử cho các ứng dụng Web Ajax. Nó có thể thực hiện trên ứng dụng Web có quy mô lớn nhưng chỉ sinh ra các giá trị ngẫu nhiên cho đầu vào. Cả hai công cụ này mặc dù có thể sinh dữ liệu kiểm thử tự động cho ứng dụng Web nhưng bộ dữ liệu kiểm thử không hiệu quả vì không phân tích sâu đến nội hàm các xử lý lô-gic của thành phần kiểm thử. Khác với hai công cụ trước đó, SymJS là công cụ có phân tích mã nguồn của thành phần kiểm thử trong ứng dụng, thu thập các toán tử có điều kiện và thực thi tượng trưng để sinh ra bộ các đầu vào đi qua các đường đi khác nhau trong chương trình. Tuy nhiên, hiện tại công cụ này mới chỉ hỗ trợ mã nguồn JavaScript phía người dùng. Có thể nói rằng SymJS đã bước đầu nghiên cứu các phương pháp kiểm thử hộp trắng cho các ứng dụng Web. Phương pháp kiểm thử hộp trắng từ lâu đã được ứng dụng để sinh tự động dữ liệu kiểm thử ở nhiều ngôn ngữ lâu đời như Java, C++, C\#, v.v. và đã đạt được một số kết quả khả quan. Bộ dữ liệu kiểm thử được sinh ra bởi phương pháp này có thể đạt độ phủ cao, thực thi qua nhiều thành phần có trong chương trình. Tuy nhiên, phương pháp này cũng vẫn tồn tại mộ số nhược điểm nhất định và chưa thể coi như là một giải pháp tổng thể cho tất cả các mã nguồn dự án phần mềm. Việc áp dụng phương pháp này như thế nào thì phải phụ thuộc vào bài toán cụ thể hoặc đặc trưng của mã nguồn được áp dụng. Thời điểm hiện tại vẫn chưa có nhiều nghiên cứu ứng dụng phương pháp kiểm thử hộp trắng cho ứng dụng Web.

% \section{Kiểm thử hộp trắng}
% Phương pháp được đề xuất trong luận văn này xây dựng dựa trên phương pháp kiểm thử dòng điều khiển. Đây là một trong những phương pháp kiểm thử hộp trắng đảm bảo tất cả các thành phần có trong mã nguồn đều được thực thi. Cụ thể, kiểm thử hộp trắng là phương pháp sinh dữ liệu kiểm thử dựa trên việc phân tích cấu trúc bên trong của mã nguồn \cite{software_testing, white_box_testing_2}. Nếu kiểm thử hộp đen cho phép phát hiện lỗi/khiếm khuyết có thể quan sát được thì kiểm thử hộp trắng có thể phát hiện những lỗi/khiếm khuyết tiềm ẩn bên trong chương trình/đơn vị phần mềm. Các lỗi này thường rất khó có thể phát hiện bằng kiểm thử hộp đen, tuy nhiên điều này không có nghĩa là kiểm thử hộp đen là không quan trọng. Mỗi phương pháp đều có những ưu nhược điểm và mục đích khác nhau, thường xuyên được sử dụng kết hợp với nhau trong quy trình kiểm thử nhằm đảm bảo phần mềm có chất lượng tốt nhất. Kiểm thử hộp trắng có các dữ liệu kiểm thử được sinh ra từ mã nguồn bằng các kỹ thuật phân tích phức tạp. Vì vậy, để có thể áp dụng được phương pháp này, các kỹ thuật viên không chỉ cần nắm rõ giải thuật mà còn cần có các kỹ năng và kiến thức tốt về ngôn ngữ lập trình, hiểu rõ được mã nguồn mới có thể đưa ra những cách giải quyết phù hợp. Do đó, việc áp dụng các phương pháp kiểm thử hộp trắng thường tốn nhiều thời gian và công sức, đặc biệt khi thành phần kiểm thử có kích thước lớn. Với lý do như vậy, các phương pháp kiểm thử hộp trắng thường được áp dụng trong pha kiểm thử đơn vị.

% Hai phương pháp được sử dụng trong kiểm thử hộp trắng là kiểm thử dòng điều khiển (Control Flow Testing - CFT) và kiểm thử dòng dữ liệu (Data Flow Testing - DFT) \cite{whitebox-testing}. Phương pháp kiểm thử dòng điều khiển tập trung kiểm thử tính đúng đắn của các giải thuật sử dụng trong thành phần cần kiểm thử. Phương pháp kiểm thử dòng dữ liệu quan tâm đến tính đúng đắn của việc sử dụng các biến dữ liệu trong thành phần kiểm thử. Luận văn này sử dụng phương pháp kiểm thử dòng điều khiển. Vì vậy, các kiến thức liên quan đến kiểm thử dòng điều khiển được trình bày chi tiết trong phần tiếp theo.
% % cần kiểm thử nhưng vẫn phát hiện được lỗi ngay từ giai đoạn đầu tiên trong quy trình phát triển phần mềm. Nhờ đó, các khiếm khuyết trong thiết kế và code được sửa chữa sớm, giảm thời gian và chi phí hoàn thiện sản phẩm. Đồng thời, hiệu suất phát triển cũng được nâng cao vì thiết kế được cải tiến, code có chất lượng tốt hơn, dễ bảo trì. Ngoài việc kiểm tra tài liệu (code reviews) và đánh giá cú pháp tự động, kiểm thử hộp trắng có thể được ứng dụng trong việc sinh dữ liệu kiểm thử thỏa mãn những tiêu chí đánh giá về độ phủ. Cụ thể trong phạm vi khóa luận này, phương pháp kiểm thử hộp trắng được áp dụng là xây dựng đồ thị luồng điều khiển đại diên cho cấu trúc chương trình và sử dụng tiêu chuẩn bao phủ nhánh của đồ thị để làm căn cứ đánh giá sự hiểu quả của giải pháp. 

% % % \section{Cây cú pháp trừu tượng}
% % % Cây cú pháp trừu tượng (Abstract Syntax Tree - AST) được sử dụng rộng rãi trong các trình biên dịch hoặc IDE. Với đầu vào là mã nguồn, các trình biên dịch/IDE này sẽ xây dựng AST tương ứng. AST là một cách biểu diễn cấu trúc mã nguồn dưới dạng cây. Mỗi một thành phần trong cây tương ứng với một thành phần mã nguồn như câu lệnh gán, khối lệnh điều kiện, biến, phép toán, v.v. Đối với một ngôn ngữ bất kỳ, AST
% % % Mỗi thành phần trong cây đều có các kiểu khác nhau được quy định bởi trình biên dịch. Ví dụ, trong CDT, kiểu IASTDeclSpecifier tương ứng với kiểu trả về của hàm hay kiểu biến. Kiểu IASTBinaryExpression tương ứng với dấu phép toán. Kiểu IASTName đại diện tên biến, tên hàm. IASTReturnStatement chính là câu lệnh return. 
% \section{Đồ thị dòng điều khiển}
% Như đã giới thiệu, phương pháp được sử dụng trong luận văn này là kiểm thử dựa trên dòng điều khiển. Tổng quan của phương pháp này là phân tích mã nguồn, xây dựng đồ thị dòng điều khiển và phân tích các biểu thức điều kiện có trong đồ thị đề sinh ra các giá trị hữu ích. Việc sinh dữ liệu kiểm thử dựa trên phân tích mã nguồn sẽ gặp rất nhiều khó khăn nếu chỉ thao tác với mã nguồn ở dạng văn bản đơn thuần. Vì vậy, chúng ta cần có một cấu trúc dữ liệu khác cũng có thể mô tả mã nguồn nhưng đơn giản để phân tích hơn. Đồ thị dòng điều khiển là một cấu trúc dữ liệu hỗ trợ giải quyết vấn đề này. Đồ thị dòng điều khiển (Control Flow Graph - \gls{CFG}) mô tả kịch bản thực thi của chương trình một cách trực quan, bao gồm các đỉnh là đại diện cho câu lệnh/nhóm câu lệnh và các cạnh là dòng điều khiển giữa các câu lệnh/nhóm câu lệnh đó \cite{CFG_definition}. Tất cả các đồ thị luồng điều khiển đều có đỉnh bắt đầu và đỉnh kết thúc đại diện cho trạng thái bắt đầu và trạng thái kết thúc của chương trình. Các cạnh là các mũi tên có hướng thể hiện thứ tự thực hiện của câu lệnh/nhóm câu lệnh. Cạnh nối hai đỉnh $i$ và $j$ theo hướng từ đỉnh $i$ đến đỉnh $j$ nghĩa là câu lệnh thứ $i$ được thực hiện trước câu lệnh thứ $j$.
% Về cơ bản, CFG bao gồm các thành phần chính là đỉnh bắt đầu, đỉnh xử lý, đỉnh quyết định, đỉnh kết nối và đỉnh kết thúc. 
% \begin{itemize}
%     \item Đỉnh bắt đầu: Đánh dấu thời điểm bắt đầu của chương trình, được thể hiện bằng hình tròn
%     \item Đỉnh xử lý: Đại diện cho các câu lệnh gán, khai báo và khởi tạo, được thể hiện bằng hình chữ nhật
%     \item Đỉnh quyết định: Đại diện cho câu lệnh điều khiển trong khối lệnh điều khiển rẽ nhánh, được thể hiện bằng hình thoi
%     % \item Đỉnh kết nối: Đại diện cho câu lệnh được thực hiện ngay sau các lệnh rẽ nhánh, có nhiều hơn hai đỉnh trỏ đến, được thể hiện bằng hình tròn
%     \item Đỉnh kết thúc: Đánh dấu thời điểm kết thúc của hàm, được thể hiện bằng hình tròn
% \end{itemize} 
% \begin{figure}[!ht]
% 		\centering
% % 		\setlength{\abovecaptionskip}{1pt plus 3pt minus 2pt}
%  		\includegraphics[width=\textwidth]{figures/cfg-control.pdf}
%  		\vspace*{-7mm}
% 		\caption{Các cấu trúc điều khiển phổ biến trong TypeScript}
% 		\label{cau-truc-dieu-khien}
% \end{figure}

% Hình~\ref{cau-truc-dieu-khien} mô tả các cấu trúc điều khiển chính có trong TypeScript được mô phỏng dưới dạng các đỉnh của CFG, bao gồm có cấu trúc điều khiển tuần tự, rẽ nhánh, vòng lặp \textit{for}, vòng lặp \textit{do…while}, vòng lặp \textit{ while…do}.

% \section{Các độ đo kiểm thử}
% Các độ đo kiểm thử thường được xác định là các quy tắc hoặc yêu cầu mà một tập hợp dữ liệu kiểm thử cần đáp ứng \cite{coverage_criteria}. Có một số độ đo phổ biến là bao phủ hàm (function coverage), bao phủ câu lệnh (statement coverage) và bao phủ nhánh (branch coverage). Bao phủ hàm là độ đo dễ đạt được nhất trong ba tiêu chí bao phủ này. Nó được đo bằng tỷ lệ phần trăm các hàm hoặc phương thức đã thực thi trên tổng số các hàm/phương thức có trong mã nguồn thử nghiệm. Bởi vì một hàm được thực thi có thể chứa các đoạn chưa được thực thi như các câu lệnh và các nhánh, một số lỗi bên trong một hàm có thể không được xem xét. Để giải quyết vấn đề này, quá trình kiểm tra phải được thực hiện với cả phạm vi bao phủ của câu lệnh và phạm vi bao phủ nhánh. Liên quan đến bao phủ câu lệnh, nó được biểu thị bằng tỷ lệ phần trăm các câu lệnh được thực thi trong tổng số các câu lệnh thuộc phạm vi kiểm thử. Nếu độ phủ câu lệnh đạt 100\%, thì bao phủ hàm/phương thức cũng đạt đến 100\%. Tuy nhiên, nó không thể xác nhận rằng tất cả các nhánh của điều kiện đều được thực thi. Vì vậy, độ phủ nhánh được đề xuất để đánh giá quá trình thử nghiệm một cách toàn diện hơn. Nó được đo bằng phần trăm các nhánh được thực thi trên tất cả các nhánh thuộc phạm vi kiểm thử. Nếu việc thực thi kiểm thử đạt được bao phủ nhánh tối đa, có thể đảm bảo rằng bao phủ câu lệnh và hàm cũng đạt đến giá trị lớn nhất. Vì vậy, bao phủ nhánh được sử dụng là độ đo cơ bản để đánh giá bộ dữ liệu kiểm thử.

% Công thức tổng quát để tính độ phủ theo các độ đo của $n$ tệp được trình bày trong Công thức \ref{coverage_equation}:
% \begin{equation} \label{coverage_equation}
% \begin{split}
%         e_c &= f(c, tested\ files) \\
%         &= \frac{\sum_{i=1}^n e_{i_c}}
%         {\sum_{i=1}^n t_{i_c}}*100 \\
%  \end{split}
% \end{equation}
% ,trong đó: $c$: loại độ đo bao phủ, bao gồm bao phủ hàm (\textit{function coverage}), bao phủ câu lệnh (\textit{statement coverage}), bao phủ nhánh (\textit{branch coverage})\\
% $e_{i_c}$: số lượng các thành phần được thực thi theo từng độ đo $c$ trong tệp thứ $i$. Thành phần được coi là các hàm, câu lệnh, và nhánh trong mã nguồn kiểm thử. \\
% $t_{i_c}$: số lượng tất cả các thành phần theo độ đo $c$ trong tệp thứ $i$

% Các tiêu chí bao phủ này được sử dụng để đánh giá hiệu quả của phương pháp được đề xuất trong việc tạo dữ liệu thử nghiệm. Nếu độ phủ tăng lên, nhiều thành phần trong mã nguồn được thực thi. Trong trường hợp các độ phủ không đạt 100 \%, các vấn đề sau có thể gặp phải. Thứ nhất, dữ liệu kiểm thử không thực thi toàn bộ các thành phần có trong mã nguồn. Do đó, có thể có một số lỗi tiềm ẩn không được phát hiện. Mặt khác, mã nguồn có thể chứa những câu lệnh không bao giờ có thể thực thi. Các nhà phát triển phải loại bỏ những đoạn mã này để tối ưu kích thước chương trình, tránh thực hiện các hành vi không đúng hoặc đơn giản hóa cấu trúc chương trình.

% % Trong kiểm thử hộp trắng nói chung và kiểm thử dòng điều khiển nói riêng, bài toán kiểm thử là sinh được bộ dữ liệu kiểm thử sao cho thỏa mãn các tiêu chuẩn cho trước. Các tiêu chuẩn này đã được thống nhất và định nghĩa bằng văn bản trong ISO \cite{iso_coverage}. Công thức tính toán độ đo theo các tiêu chuẩn này dựa trên mức độ bao phủ của chương trình với một tập dữ liệu kiểm thử cho trước. Tập dữ liệu kiểm thử có độ phủ cao sẽ đáng tin cậy hơn tập dữ liệu kiểm thử có độ phủ thấp. Mục tiêu là tập dữ liệu kiểm thử có số lượng tối thiểu nhưng đạt được độ phủ tối đa. Hiện nay, có nhiều tiêu chuẩn bao phủ khác nhau được sử dụng. Độ phủ của mỗi tiêu chuẩn đánh giá đều có công thức tính riêng nhưng về cơ bản sẽ được tính bằng tỉ lệ thành phần được kiểm thử trên tổng số các thành phần cần kiểm thử. Thành phần ở đây có thể là câu lệnh, nhánh chương trình, điểm quyết định, điều kiện con hoặc sự kết hợp giữa chúng. Độ đo này giúp các kỹ thuật viên kiểm soát và quản lý quá trình kiểm thử tốt hơn, có thể kiểm tra lại thành phần không được chạy qua hoặc bổ sung thêm dữ liệu kiểm thử trong trường hợp độ phủ thấp. Dưới đây là ba độ đo kiểm thử được sử dụng nhiều trong quy trình kiểm thử phần mềm \cite{Lee03}.
% % \begin{itemize}
% %     \item Độ phủ câu lệnh (statement coverage): mỗi câu lệnh được đi qua ít nhất một lần sau khi chạy bộ dữ liệu kiểm thử.
% %     \item Độ phủ nhánh (branch coverage): nhánh đúng và nhánh sai của mỗi đỉnh điều kiện có trong đồ thị dòng điều khiển được đi qua ít nhất một lần sau khi chạy bộ dữ liệu kiểm thử.
% %     \item Độ phủ điều kiện con (Modified Condition/Decision Coverage - MC/DC): các điều kiện con thuộc các đỉnh điều kiện phức tạp đều được thực hiện cả hai nhánh đúng và nhánh sai ít nhất một lần mỗi nhánh sau khi chạy bộ dữ liệu kiểm thử.
% % \end{itemize}

% % \section{Đường kiểm thử}
% % Bộ dữ liệu kiểm thử sinh ra dành cho một hàm bao gồm nhiều dữ liệu kiểm thử. Mỗi dữ liệu kiểm thử là một bộ giá trị đầu vào của tham số. Với một bộ giá trị đầu vào, chương trình của hàm sẽ chạy qua một số câu lệnh và dừng lại khi tới điểm kết thúc. Tập hợp các câu lệnh theo thứ tự thực hiện tạo thành một đường đi. Những đường đi được chọn để sinh dữ liệu kiểm thử  được gọi là đường kiểm thử. Để thống nhất khái niệm sử dụng trong suốt khóa luận, Định nghĩa 2.4 mô tả tổng quát một đường kiểm thử. Mỗi đường kiểm thử có thể bao gồm đầy đủ các câu lệnh khai báo, gán giá trị, khởi tạo và câu lệnh rẽ nhánh. Các đường đi khác nhau sẽ khác nhau ở số lượng, danh sách và thứ tự thực hiện các câu lệnh. Việc sinh dữ liệu kiểm thử tương ứng với đường kiểm thử chính là tìm kiếm bộ giá trị đầu vào sao cho khi thực thi, các nút điều kiện của đường đi đều được thỏa mãn. Trong thực tế, số lượng đường đi của chương trình có thể rất lớn dẫn đến việc sinh bộ dữ liệu kiểm thử cho tất cả các đường đi là không thể. Vì vậy, một số đường đi được chọn để sinh dữ liệu kiểm thử nhằm đáp ứng tiêu chí về độ phủ được gọi là tập đường kiểm thử.\\
% % \textbf{Định nghĩa 2.1}: Đường kiểm thử là một đường đi từ điểm bắt đầu đến điểm kết thúc của CFG, được biểu diễn bằng tập hợp các đỉnh từ $v_1$  đến $v_n$ sao cho cứ hai đỉnh cạnh nhau thì có cạnh nối theo hướng từ trái qua phải. Nếu cạnh ($v_i$, $v_j$) là nhánh sai thì biểu thức điều kiện tại đỉnh $v_i$ được viết dưới dạng phủ định $!v_i$.

% % Để có thể sinh được bộ dữ liệu kiểm thử thỏa mãn yêu cầu về tiêu chuẩn bao phủ, việc lựa chọn tập kiểm thử là một công đoạn không thể thiếu. Tuy nhiên, có hai vấn đề chúng ta cần phải đối mặt:
% % \vspace{-0.5cm}
% % \begin{itemize}
% %     \item Tính khả thi của đường đi: Một đường kiểm thử gọi là có khả khi nếu tồn tại một dữ liệu kiểm thử sao cho khi thực thi trong môi trường thật, tất cả các đỉnh của đường đi được duyệt qua. Ngược lại, đường kiểm thử gọi là không khả thi.
% %     \item 	Sự bùng nổ đường đi: với một hàm có kích thước lớn, nhiều vòng lặp hoặc các lệnh rẽ nhánh phức tạp, số lượng đường đi của chương trình có thể rất lớn. Việc sinh dữ liệu kiểm thử cho tất cả các đường đi để chắc chắn đạt độ phủ 100\% là không thể.
% % \end{itemize}
% % Mục tiêu của khóa luận này là xây dựng công cụ đầu tiên hỗ trợ sinh dữ liệu kiểm thử cho TypeScript, bắt đầu thử nghiệm với các hàm TypeScript kích thước vừa phải. Vì vậy, tập đường kiểm thử được lựa chọn là tập các đường đi có thể có của chương trình. Trong trường hợp tất cả các đường đi đều khả thi, độ bao phủ nhánh có thể đạt được là 100\%.

% % \section{Thư viện sử dụng}

% % \subsection{Thư viện phân tích mã nguồn TypeScript ``ts-morph''}
% % Phương pháp sinh dữ liệu kiểm thử tự động được đề xuất trong khóa luận này dựa trên việc thao tác với CFG của hàm. Việc xây dựng CFG như thế nào sẽ tùy biến theo từng tình huống bài toán. Đặc biệt đối với một ngôn ngữ mới như TypeScript, hiện tại không có thư viện nào có thể hỗ trợ giải quyết tác vụ này. Vì vậy, quá trình này được thực hiện thủ công dựa trên kỹ thuật phân tích mã nguồn và thiết kế cấu trúc dữ liệu mô hình hóa sao cho phù hợp với bài toán kiểm thử hiện tại. Phân tích mã nguồn thành cây cú pháp trừu tượng (Abstract Syntax Tree - AST) giúp việc xây dựng CFG trở nên đơn giản hơn. AST là một cây đại diện cho cấu trúc cú pháp trừu tượng của mã nguồn. Ngôn ngữ lập trình khác nhau có AST khác nhau.  Mỗi nút của cây biểu thị một cấu trúc có trong mã nguồn. AST thường được xây dựng bởi chính trình phân tích cú pháp của ngôn ngữ tương ứng trong quá trình biên dịch. Đối với các ngôn ngữ lâu đời, việc này có thể được thực hiện bởi một số thư viện khác. Do TypeScript là một ngôn ngữ mới nên chưa có công cụ nào hỗ trợ phân tích mã nguồn thành AST. Vì vậy, khóa luận này sử dụng chính trình biên dịch ngôn ngữ TypeScript của Microsoft để phân tích nội dung hàm. ``ts-morph''\footnote{\url{https://github.com/dsherret/ts-morph}} là một thư viện mở rộng từ trình biên dịch TypeScript, cung cấp các giao diện lập trình ứng dụng (Application Programming Interface - API) hỗ trợ người dùng có một cách dễ dàng hơn để điều hướng chương trình và thao tác với mã TypeScript.
% % Với sự hỗ trợ của ``ts-morph'', người sử dụng có đầy đủ các API để trích xuất thông tin cần thiết từ mã nguồn.  Các khối lệnh của thân hàm và các biểu thức  điều kiện đều có thể dễ dàng có được thông qua việc duyệt các đỉnh, hay gọi là $node$ của AST. Ngoài ra, thư viện cung cấp giao diện\footnote{\url{https://TypeScript-ast-viewer.com/}} để người dùng có thể theo dõi kết quả dưới dạng hình cây rất trực quan và dễ nhìn. Từ đó, việc thao tác với AST trở nên dễ dàng hơn.

% % %  Đối với TypeScript, các cấu trúc này có thể là lớp, hàm, thuộc tính, tham số, câu lệnh, v.v.

% % \subsection{Bộ giải hệ ràng buộc Z3}
% % Công cụ sinh dữ liệu kiểm thử cho mỗi đường kiểm thử bằng cách giải hệ ràng buộc ứng với tập các đỉnh điều kiện. Giải hệ ràng buộc nghĩa là quá trình tìm ra giải pháp cho một tập hợp các ràng buộc áp đặt bởi các phép toán điều kiện mà các biến phải thỏa mãn \cite{ref-constraints}. Do đó, một giải pháp là một tập hợp các giá trị cho các biến thỏa mãn tất cả các ràng buộc, đó là một điểm trong vùng khả thi.
% % Hiện nay, có nhiều thư viện, công cụ hỗ trợ việc giải hệ trong đó nổi bật là bộ giải Z3. Bộ giải Z3 được xây dựng chủ yếu bằng ngôn ngữ C++. Các ràng buộc cần được chuyển sang dạng chuẩn của Z3 để công cụ có thể tính toán và giải nghiệm. Z3 có thể giải hệ ràng buộc của các số nguyên, số thực, mảng và hàm tượng trưng. Đặc biệt trong phiên bản 4.8, bộ giải Z3 hỗ trợ giải một số ràng buộc liên quan đến chuỗi (string) \cite{z3_str_paper}. Điều này giúp việc tìm kiếm dữ liệu kiểm thử có tham số đầu vào kiểu chuỗi trở nên đơn giản hơn. Để có thể sử dụng Z3 giải nghiệm, bộ ràng buộc được lưu trong tệp và khởi chạy tiến trình bằng dòng lệnh:
% % \vspace{0.5cm}
% % \begin{lstlisting}
% % z3 -smt2 <file name>.smt2
% % \end{lstlisting}
% % Mã nguồn~\ref{constraints-file-example} là ví dụ một tệp constraints.smt2 hợp lệ làm đầu vào cho bộ giải Z3. Trong tệp, các biến sử dụng cần được khai báo bằng cú pháp \textit{declare-fun}. Sau đó, lệnh \textit{assert} được sử dụng để thêm các ràng buộc của hệ. Để kiểm tra hệ ràng buộc có nghiệm hay không, lệnh \textit{check-sat} được gọi. Kết quả trả về là \textit{sat} nếu có nghiệm và \textit{unsat} trong trường hợp không có nghiệm. Tập các giá trị của các biến thỏa mãn hệ ràng buộc được hiển thị bằng lệnh \textit{get-model}. Trong ví dụ này, hệ ràng buộc sử dụng ba biến tham số đầu vào là tvw\_s,  tvw\_a, tvw\_b và có ba ràng buộc được thêm vào câu lệnh \textit{assert} trong tệp \textit{constraints.smt2}. Mã nguồn~\ref{z3-result-example} là kết quả tương ứng sau khi giải hệ. Trong đó, các giá trị của các biến tìm được là  tvw\_a = 12,  tvw\_b = 11,  tvw\_s = "\textbackslash x00\textbackslash x00\textbackslash x00\textbackslash x00\textbackslash x00". Như vậy, bộ giá trị (a, b, s) = \{12, 11, "\textbackslash x00\textbackslash x00\textbackslash x00\textbackslash x00\textbackslash x00"\} là một nghiệm của hệ ràng buộc.
% % Nếu áp dụng với một đường kiểm thử cụ thể, các biến được khai báo trong hệ ràng buộc là các biến gọi đến trong các câu lệnh. Các đỉnh điều kiện sẽ được biểu diễn qua những biến này và chuẩn hóa thành những ràng buộc của hệ. Kết quả giải hệ là một dữ liệu kiểm thử thỏa mãn đường đi tương ứng. Trong trường hợp hệ ràng buộc của tất cả các đường đi đều giải được bỏi bộ giải Z3, tập các dữ liệu kiểm thử thu được phủ 100\% đường đi của CFG.

% % \vspace{0.5cm}
% % \begin{lstlisting}[caption=Ví dụ nội dung tệp đầu vào cho bộ giải Z3, label=constraints-file-example,captionpos=b]
% % (set-option :timeout 5000)
% % (declare-fun tvw_s () String)
% % (declare-fun tvw_a () Int)
% % (declare-fun tvw_b () Int)
% % (assert (> a b))
% % (assert (> b 10))
% % (assert (not (> (+ (str.len tvw_s) 1) 10)))
% % (check-sat)
% % (get-model)
% % \end{lstlisting}

% % \begin{lstlisting}[caption=Ví dụ kết quả giải hệ của Z3, label=z3-result-example, captionpos=b]
% % sat
% % (model
% %   (define-fun tvw_s () String
% %     "\x00\x00\x00\x00\x00")
% %   (define-fun tvw_b () Int
% %     11)
% %   (define-fun tvw_a () Int
% %     12)
% % )

% % \end{lstlisting}

% % \subsection{Mocha và Istanbul}
% % Để kiểm tra độ phủ đạt được với bộ dữ liệu kiểm thử đã được sinh tự động, công cụ có sử dụng Mocha trong việc thực thi mã nguồn. Mocha (Mocha Test Framework) là một bộ công cụ hỗ trợ kiểm thử dành cho JavaScript giàu tính năng chạy trên Nodejs và trong trình duyệt, giúp cho việc kiểm tra bất đồng bộ trở nên đơn giản và thú vị \cite{ref-mocha}. Các quy trình thực thi kiểm thử thực hiện bởi Mocha chạy ổn định, cho phép báo cáo linh hoạt và chính xác. Mocha có thể áp dụng với TypeScript thông qua một số bước cài đặt kỹ thuật. Ngoài ra,  Mocha cũng có thể kết hợp với một số thư viện để xuất báo cáo kiểm tra độ phủ (test coverage report). Ngoài việc cung cấp chức năng chạy kiểm thử với thao tác bằng tay, Mocha còn kèm theo bộ API để vận hành các bài kiểm tra tự động. Mocha có rất nhiều tính năng tuyệt vời, trong đó có một số tính năng nổi bật được kể đến là:
% % \begin{itemize}
% %     \item Hỗ trợ bất đồng bộ đơn giản
% %     \item Cung cấp đa dạng báo cáo
% %     \item Có thể chạy trong trình duyệt
% %     \item Tương thích với nhiều thư viện xác nhận (assertion library) Javascript
% %     \item Tương thích với mô hình phát triển phần mềm định hướng hành vi (Behaviour Driven Development - BDD) và mô hình phát triển phần mềm định hướng kiểm thử (Test Driven Development - TDD)
% % \end{itemize}
% % Với sự hỗ trợ của Mocha, các tệp kiểm thử được thực thi một cách nhanh chóng. Kết quả chạy kiểm thử được thống kê trực quan với số lượng dữ liệu kiểm thử thành công hay thất bại kèm theo vị trí cụ thể. Từ đó kỹ thuật viên nhanh chóng xác định được dữ liệu kiểm thử bị sai và dễ dàng sửa chữa.

% % Mocha chỉ hỗ trợ quá trình thực thi tệp kiểm thử và thống kê kết quả số lượng dữ liệu kiểm thử thành công hay thất bại. Tuy nhiên, để có thể biết được thống kê độ phủ mà bộ dữ liệu kiểm thử đạt được, Mocha cần kết hợp thêm một số thư viện bên ngoài. trong đó, được sử dụng nhiều nhất phải kể đến thư viện Istanbul \cite{ref-instanbul}. Đây là thư viện hỗ trợ sinh báo cáo độ phủ của quá trình kiểm thử đơn vị với nhiều định dạng khác nhau như HTML, XML, terminal output, JSON, v.v. Người dùng có thể lựa chọn kiểu báo cáo phù hợp với nhu cầu. Công cụ được phát triển trong khóa luận sử dụng báo cáo thể hiện dưới dạng HTML, bao gồm các thông tin về các độ phủ như số lượng hàm, câu lệnh, nhánh được thực thi. Đồng thời, báo cáo cũng nổi bật các đoạn mã không được chạy qua. Từ đó, kỹ thuật viên có thể phát hiện ra  được các đoạn mã không bao giờ được chạy để làm sạch mã nguồn hoặc bổ sung thêm dữ liệu kiểm thử mới để bộ dữ liệu kiểm thử hoàn thiện hơn.

% \chapter{Phương pháp kiểm thử tự động hàm Typescript sử dụng kỹ thuật kiểm thử tĩnh luồng điểu khiển }
\chapter{LỰA CHỌN VÀ XÂY DỰNG THIẾT BỊ\label{the_proposed_method_section}}
Dựa trên các cơ sở lý thuyết tại chương I, trong chương này, luận văn sẽ trình
bày về phương pháp nghiên cứu và phát triển hệ thống thu thập, xử lý và lưu trữ
dữ liệu cảm biến gia tốc. Từ bộ dữ liệu thu thập, sẽ tiến hành trích xuất đặc
trưng, đề xuất mô hình học máy phù hợp và thiết lập các kịch bản để đánh giá
các mô hình đề xuất đó.
\section{Nghiên cứu, phát triển phần cứng }

Trong phần này sẽ có các nội dung bao gồm: nghiên cứu, lựa chọn vi điều khiển
và cảm biến phù hợp; từ đó tiến hành thiết kế sơ đồ mạch, mô phỏng mạch 3D và
hoàn thiện sản phẩm.

\subsection{Vi xử lý}
\label{subsec:vi_xu_ly}

Với sự phát triển vượt bậc và đa dạng của công nghệ thiết kế và chế tạo, có rất
nhiều cấu hình phần cứng được nhiều nhóm tác giả lựa chọn phù hợp với các mục
đích khác nhau. Trong đó, \cite{p_1} các tác giả đã sử dụng máy tính đơn
Raspberry Pi để phát hiện 4 tư thế ngủ với sự lấy nhãn từ video theo dõi người
bệnh trong suốt quá trình lấy mẫu. Tác giả Kwasnicki và cộng sự đã sử dụng bộ
xử lý công suất thấp TI MSP430 và mô-đun RF Chipcon CC2420 cho truyền thông
không dây kết hợp với cảm biến gia tốc ba trục ADXL330, con quay hồi chuyển đạt
được 99.5\% độ chính xác \cite{kwasnicki2018}. Nhóm của tác giả I.Yun đã phát
triển thiết bị theo dõi tư thế ngủ của trẻ nhỏ sử dụng vi xử lý ATmega328P-PU
kết hợp cảm biến gia tốc ADXL335 được đặt trên bụng \cite{p_3}. Trong nghiên
cứu của Abdulsadig và cộng sự, hệ thống thu thập dữ liệu được xây dựng dựa trên
một bo mạch tùy chỉnh tích hợp vi điều khiển nRF5232. Vi điều khiển này đảm
nhiệm đồng thời cả việc lấy mẫu dữ liệu từ cảm biến gia tốc ba trục LIS2DH12
với tần số 100 Hz và truyền dữ liệu không dây theo thời gian thực \cite{
    abdulsadig2023, Sleep_Posture_Detection}. Trong nghiên cứu của tác giả Vũ Hoàng
Diệu, mô-đun ESP32 được lựa chọn làm đơn vị xử lý trung tâm nhờ tích hợp bộ vi
điều khiển hiệu năng cao, kết nối không dây Wi-Fi đáp ứng tốt yêu cầu của hệ
thống thu thập dữ liệu tư thế ngủ theo thời gian thực \cite{vu2023}. Thiết bị
không chỉ cho phép truyền dữ liệu trực tiếp lên máy chủ hoặc nền tảng đám mây
thông qua Wi-Fi, mà còn hỗ trợ lưu trữ cục bộ trên thẻ nhớ microSD, đảm bảo
tính liên tục trong điều kiện mất kết nối mạng.

Qua phân tích các nghiên cứu trên có thể thấy rằng phần lớn các cấu hình phần
cứng có kích thước vẫn còn lớn hoặc gặp giới hạn trong khả năng tích hợp mô
hình học máy tại thiết bị. Vì vậy, trong khuôn khổ luận văn sẽ sử dụng vi điều
khiển nRF52840 làm bộ điều khiển trung tâm \cite{nrf52840} với các lý do sau:

\begin{figure}[H]
    \centering
    \includegraphics[width=1\textwidth]{images/nrf52840_feature.png}
    \caption{Các tính năng Nordic nRF52840}
    \label{Nornrf52840_featuredic}
\end{figure}

nRF52840 tích hợp bộ giao thức không dây Bluetooth năng lượng thấp hoạt động ở
băng tần 2.4~GHz và bộ xử lý trung tâm Arm Cortex-M4F chạy ở xung nhịp 64~MHz,
kèm bộ xử lý dấu phẩy động (FPU). Vi xử lý này được trang bị bộ nhớ 1~MB Flash
và 256~kb RAM phù hợp để triển khai các mô hình cho bài toán phân loại tư thế
ngủ. Ngoài khả năng hoạt động trong dải điện áp rộng từ +1.7~V đến +5.5~V
(tương thích với nguồn pin và USB), nRF52840 còn cung cấp các giao tiếp ngoại
vi phong phú: hai giao diện I2C, bốn SPI chủ (master), ba SPI tớ (slave), bốn
kênh điều chế xung (PWM) hỗ trợ EasyDMA, cùng với bộ định thời 32-bit, phù hợp
cho các ứng dụng đòi hỏi xử lý thời gian thực chính xác
Hình~\ref{nrf52840_schematic}.

\begin{figure}[H]
    \centering
    \includegraphics[width=0.5\textwidth]{images/nrf52840_schematic.png}
    \caption{Sơ đồ khối Nordic NRF52840}
    \label{nrf52840_schematic}
\end{figure}

Đặc biệt, nRF52840 có
một hệ sinh thái phần mềm đi kèm, bao gồm bộ công cụ phát triển (SDK) của Nordic Semiconductor
và nền tảng TensorFlow Lite for Microcontrollers, giúp rút ngắn thời gian phát triển và triển khai hệ
thống TinyML trực tiếp lên chip \cite{Nordic2021_tinyml}.
Hiện nay, nRF52840 cũng tích hợp trong nhiều bộ kit phát triển thương mại,
như Adafruit Circuit Playground Bluefruit,
SparkFun Pro nRF52840 Mini,
Seeed Studio XIAO.
Các bộ kit này cung cấp tài nguyên phần cứng và thư viện mã nguồn mở,
hỗ trợ kết nối BLE, USB, GPIO, SPI, I²C, cùng khả năng lập trình trực
tiếp qua môi trường Arduino IDE, PlatformIO hoặc Zephyr RTOS.
Nhờ đó, nRF52840 càng trở nên phổ biến trong các nghiên cứu và dự án về IoT,
thiết bị đeo thông minh và học máy tại biên.

\begin{figure}[H]
    \centering
    \begin{minipage}{0.48\textwidth}
        \centering
        \includegraphics[height=4cm]{images/Seeed.jpg}
        \caption*{(a) Seeed Studio XIAO
        }
    \end{minipage}
    \hfill
    \begin{minipage}{0.48\textwidth}
        \centering
        \includegraphics[height=4cm]{images/adafruit.jpg}
        \caption*{(b) Adafruit Circuit Playground Bluefruit}
    \end{minipage}
    \caption{Một số bộ kit phát triển dựa trên vi điều khiển Nordic nRF52840}
    \label{fig:nrf52840_devkits}
\end{figure}

% Kiến trúc ARM có nhiều dòng vi xử lý khác nhau, được phát triển và nâng cấp
% liên tục nhằm đáp ứng nhu cầu đa dạng trong lĩnh vực công nghệ nhúng. Trong đó,
% dòng Cortex-M thuộc kiến trúc ARMv7 đã trở thành nền tảng phổ biến cho các hệ
% thống nhúng sử dụng vi điều khiển nhờ vào hiệu suất cao, khả năng mở rộng và
% mức tiêu thụ năng lượng tối ưu. Dòng Cortex-M bao gồm nhiều phiên bản như
% Cortex-M0, Cortex-M0+, Cortex-M1, Cortex-M3, Cortex-M4 và Cortex-M7, mỗi phiên
% bản được thiết kế để phục vụ cho các mức độ yêu cầu hiệu năng khác nhau
% \cite{arm_cortex_m_comparison}. Các vi xử lý thuộc họ Cortex-M chủ yếu được ứng
% dụng trong các hệ thống nhúng thời gian thực, nơi yêu cầu sự cân bằng giữa hiệu
% suất xử lý, tiêu thụ năng lượng và chi phí. Một số vi xử lý ARM khác, không
% thuộc họ Cortex-M, được sử dụng trong các thiết bị hiệu suất cao như điện thoại
% thông minh và máy tính bảng, vốn yêu cầu cấu hình phần cứng mạnh hơn và khả
% năng xử lý đa tác vụ cao hơn. Theo tài liệu \cite{cortexM4}, vi xử lý Cortex-M4
% là một bộ xử lý 32-bit sử dụng kiến trúc tập lệnh rút gọn (RISC), được xây dựng
% theo kiến trúc Harvard, trong đó bus dữ liệu và bus lệnh được tách biệt nhằm
% tối ưu hiệu suất truy xuất bộ nhớ. Vi xử lý này hỗ trợ đầy đủ cả tập lệnh
% Thumb-1 (16-bit) và Thumb-2 (hỗn hợp 16/32-bit), mang lại sự linh hoạt trong mã
% hóa lệnh và tiết kiệm không gian bộ nhớ chương trình.

% Về hiệu năng, Cortex-M4 đạt từ 1,25 đến 1,95 DMIPS/MHz (Dhrystone Million
% Instructions Per Second per MHz), cho thấy khả năng xử lý hiệu quả trong các
% ứng dụng nhúng yêu cầu độ chính xác và độ phản hồi thời gian thực cao. Bên cạnh
% đó, vi xử lý hỗ trợ tối đa 240 tín hiệu ngắt, bao gồm cả ngắt không thể bị chặn
% (Non-Maskable Interrupts - NMI), cùng khả năng cấu hình từ 8 đến 256 mức ưu
% tiên ngắt, giúp hệ thống hoạt động ổn định trong môi trường có nhiều sự kiện
% cạnh tranh đồng thời. Ngoài ra, hiện nay ứng dụng trí tuệ nhân tạo (AI) tại
% thiết bị biên (Edge AI) đang ngày càng phổ biến, đặc biệt trong các lĩnh vực
% như nhà thông minh, thiết bị đeo, giám sát an ninh và công nghiệp 4.0. Với khả
% năng xử lý tín hiệu số (DSP) và hỗ trợ các mạng nơ-ron nhỏ gọn, các vi xử lý
% Cortex-M, đặc biệt là dòng Cortex-M4, đang được khai thác để triển khai các mô
% hình học sâu nhẹ (tinyML) ngay trên vi điều khiển
% \cite{electronics11162545}\cite{applicationCortexM4}.

% \begin{figure}[htbp]
%     \centering
%     \includegraphics[width=0.8\textwidth]{images/cortexM4.png}
%     \caption{Thành phần chính của vi điều khiển Cortex-M4}
%     \label{cortexM4}
% \end{figure}

% Kết nối bus được mô tả trong Hình~\ref{cortexM4} cho phép truyền dữ liệu đồng
% thời trên nhiều bus khác nhau, đồng thời cung cấp khả năng quản lý truyền dữ
% liệu hiệu quả, chẳng hạn như sử dụng bộ đệm ghi và điều khiển hướng bit hoạt
% động (bit-banding). Hệ thống cũng có thể bao gồm các cầu bus (bus bridges) nhằm
% kết nối nhiều loại bus vào một mạng duy nhất sử dụng chung không gian bộ nhớ.
% Ngoài ra, bộ xử lý được trang bị hệ thống hỗ trợ gỡ lỗi tích hợp, bao gồm khả
% năng kiểm soát gỡ lỗi, thiết lập điểm ngắt (breakpoint) chương trình và điểm
% theo dõi dữ liệu (watchpoint). Khi xảy ra sự kiện gỡ lỗi, hệ thống có thể tạm
% dừng trạng thái hoạt động của lõi xử lý để phục vụ việc phân tích và xử lý lỗi.

% Bên cạnh đó, kiến trúc Cortex-M4 tích hợp Bộ điều khiển ngắt vectored lồng nhau
% (Nested Vectored Interrupt Controller - NVIC) với khả năng hỗ trợ lên đến 240
% tín hiệu yêu cầu ngắt, bao gồm cả ngắt không chắn được (NMI). NVIC hỗ trợ xử lý
% ngắt lồng nhau một cách tự động bằng cách so sánh mức ưu tiên giữa các yêu cầu
% ngắt với mức ưu tiên hiện tại đang được xử lý.

% Đối với các ứng dụng yêu cầu tiết kiệm năng lượng, hệ thống còn được trang bị bộ đánh thức ngắt (Wake-up Interrupt Controller - WIC), cho phép đưa bộ vi điều khiển vào chế độ nghỉ bằng cách tắt hầu hết các thành phần không cần thiết, đồng thời duy trì khả năng đánh thức hệ thống khi phát hiện một yêu cầu ngắt. Ngoài ra, cơ chế bảo vệ bộ nhớ cũng được tích hợp nhằm đảm bảo an toàn cho hệ thống, ví dụ như chỉ cho phép truy cập đọc tại một số vùng bộ nhớ hoặc ngăn người dùng truy cập vào các vùng dữ liệu đặc quyền của hệ điều hành hoặc ứng dụng hệ thống.

\subsection{Cảm biến}
\label{subsec:cam_bien}

Trong quá trình ngủ, các chuyển động thân thể chủ yếu là chuyển động chậm, với
biên độ nhỏ và ít mang tính đột ngột \cite{Wang2018_slidingwindow}. Trong giai
đoạn ngủ REM, cơ thể gần như bất động. Do đó, việc ghi nhận chính xác các thay
đổi tư thế ngủ đòi hỏi cảm biến có độ nhạy cao. Như đã trình bày trong Chương
I, các cảm biến gia tốc MEMS kiểu điện dung hiện đang được ứng dụng rộng rãi
trong giám sát tư thế và chuyển động khi ngủ nhờ vào đặc điểm nổi bật là tần số
lấy mẫu phù hợp và đặc biệt là độ nhạy cao với mọi loại chuyển động.

Sau khi đã khảo sát, cảm biến gia tốc Bosch BMI270 được lựa chọn cho hệ thống
nhận diện tư thế ngủ với các lý do sau:

\begin{figure}[H]
    \centering
    \includegraphics[width=0.3\textwidth, keepaspectratio]{images/BMi270.png}
    \caption{Bosch BMI270}
    \label{Bosch_BMI270}
\end{figure}

Thứ nhất, độ chính xác và dải đo linh hoạt: BMI270 tích hợp cảm biến gia tốc ba
trục độ phân giải cao 16-bit với các dải đo linh hoạt ±2g, ±4g, ±8g và ±16g,
cho phép ghi nhận chính xác cả chuyển động chậm và biên độ nhỏ trong khi ngủ.

Thứ hai, mức tiêu thụ năng lượng thấp và khả năng hoạt động ổn định: cảm biến
tiêu thụ trung bình chỉ khoảng 685 µA khi hoạt động bình thường, đồng thời được
trang bị bộ quản lý năng lượng tích hợp hỗ trợ nhiều chế độ tiết kiệm điện khác
nhau. Điều này cho phép thiết bị hoạt động lâu dài trên nền tảng phần cứng công
suất thấp mà không ảnh hưởng đến hiệu năng đo.

Thứ ba, tốc độ lấy mẫu và thời gian đáp ứng nhanh: BMI270 hỗ trợ tốc độ lấy mẫu
(ODR) từ 0.7~Hz đến~1.6 kHz cho cảm biến gia tốc, phù hợp bài toán phân loại
giấc ngủ yêu cầu tần số lấy mẫu thấp (10~Hz).

\begin{figure}[]
    \centering
    \includegraphics[width=0.75\textwidth, keepaspectratio]{images/bmi270_block.png}
    \caption{Sơ đồ khối Bosch BMI270}
    \label{bmi270_block}
\end{figure}

Thứ tư, khả năng chống nhiễu và hiệu chuẩn tự động: cảm biến có cơ chế bù sai
số và bù độ nhạy theo thời gian thực, giúp giảm sai lệch khi hoạt động lâu dài.
Tính năng bù sai lệch cho cảm biến gia tốc đảm bảo độ ổn định cao của dữ liệu
đo. Ngoài ra, cảm biến còn tích hợp thêm bộ lọc thông thấp
Hình~\ref{bmi270_block}.

Cuối cùng, tích hợp tính năng thông minh được lập trình trong các thanh ghi
riêng được Google chứng nhận tương thích với hệ điều hành Wear OS và hỗ trợ
phát hiện chuyển động tự động: BMI270 có khả năng phát hiện các trạng thái như
phát hiện có chuyển động (any motion), không có chuyển động (no motion) hoặc di
chuyển các thiết bị đeo ở cổ tay (wrist wear wakeup). Các tính năng này có thể
được tận dụng để giảm tải tính toán cho bộ vi điều khiển, chỉ kích hoạt mô hình
học máy khi có thay đổi tư thế đáng kể, qua đó tiết kiệm năng lượng và tăng
tuổi thọ pin.

\begin{figure}[H]
    \centering
    \includegraphics[width=0.4\textwidth, keepaspectratio]{images/bmi270_pin.png}
    \vspace*{-5mm}
    \caption{Sơ đồ chân Bosch BMI270}
    \label{bmi270_pin}
\end{figure}

\subsection{Bluetooth năng lượng thấp}

Các công nghệ truyền thông không dây khác đều bộc lộ những hạn chế đối với
thiết bị phân loại tư thế ngủ. Wi-Fi, có ưu thế về băng thông, lại tiêu thụ
năng lượng cao và phụ thuộc vào internet. Bluetooth cổ điển, vốn được thiết kế
cho các ứng dụng truyền tải dữ liệu dung lượng lớn như âm thanh, cũng đòi hỏi
năng lượng cao hơn so với BLE. ZigBee, hiệu quả trong các mạng cảm biến nhà
thông minh nhờ cấu trúc lưới, lại thiếu sự hỗ trợ trên điện thoại và đòi hỏi
gateway riêng. Với mục tiêu tối ưu hóa năng lượng và đảm bảo khả năng hoạt động
lâu dài cho thiết bị đeo sử dụng pin, BLE được lựa chọn làm chuẩn kết nối không
dây chính trong hệ thống phần cứng. So với các giao thức khác, BLE tỏ ra vượt
trội nhờ mức tiêu thụ năng lượng rất thấp, tốc độ khởi tạo kết nối nhanh và khả
năng tương thích rộng rãi với hầu hết các thiết bị di động hiện nay
\cite{BLE_compare}.

Ngoài ra, BLE còn được hỗ trợ trên vi điều khiển nRF52840 đã chọn ở phần trên.
Nhờ sự kết hợp này, hệ thống có thể duy trì khả năng truyền dữ liệu ổn định,
tiết kiệm năng lượng, và đáp ứng tốt yêu cầu hoạt động liên tục trong quá trình
giám sát tư thế ngủ.

\begin{figure}[H]
    \centering
    \includegraphics[width=0.8\textwidth]{images/ble.png}
    \caption{Các kiểu kết nối không dây trong họ chip nRF52}
    \label{ble}
\end{figure}

\subsubsection*{Các thành phần, chức năng}

BLE là giao thức kết nối không dây được thiết kế chuyên biệt cho các ứng dụng
năng lượng thấp, hoạt động ở băng tần ISM 2.4~GHz, hỗ trợ thông lượng ứng dụng
lên đến 1.4 Mbps \cite{BluetoothSIG2019_Core5}. BLE hiện được hỗ trợ phổ biến
trên hầu hết các hệ điều hành như iOS, Android, macOS, Windows và Linux, cũng
như trong các thiết bị di động hiện đại.

Về mặt bảo mật, BLE tích hợp các cơ chế mã hóa và xác thực nhằm đảm bảo tính bí
mật, toàn vẹn của dữ liệu. Bluetooth 5 là bước phát triển đột phá tiếp theo kể
từ khi BLE được giới thiệu trong chuẩn Bluetooth 4.0, mang đến hàng loạt cải
tiến đáng kể giúp mở rộng phạm vi ứng dụng và nâng cao hiệu suất hệ thống. Quan
trọng hơn, chế độ này còn giúp giảm đáng kể mức tiêu thụ năng lượng \cite{BLE}.
Đặc biệt, chế độ Long Range mở rộng đáng kể phạm vi truyền thông của BLE, cho
phép các thiết bị duy trì kết nối trong toàn bộ không gian trong nhà.

\begin{figure}[htbp]
    \centering
    \includegraphics[width=0.5\textwidth]{images/gatt.drawio.png}
    \caption{Cấu trúc của GATT}
    \label{gatt}
\end{figure}

BLE tổ chức logic giao tiếp dựa trên mô hình hồ sơ thuộc tính tổng quá
\gls{gatt}. GATT quy định cách hai thiết bị BLE trao đổi dữ liệu thông qua các
đơn vị logic: dịch vụ (services) và đặc tính (characteristics). Giao thức nền
tảng là Attribute Protocol (ATT). Mỗi dịch vụ, đặc tính được định danh bằng mã
định danh duy nhất (UUID) 16-bit hoặc 128-bit, với quyền truy cập như chỉ đọc,
chỉ ghi, hoặc hỗ trợ thông báo. Trong mô hình GATT là tính kết nối độc quyền:
tại một thời điểm, thiết bị ngoại vi chỉ có thể duy trì một kết nối duy nhất
với thiết bị trung tâm. Khi kết nối được thiết lập, thiết bị ngừng quảng cáo,
điều này hạn chế khả năng kết nối đồng thời từ nhiều thiết bị.

Ngoài ra, một điểm sáng nữa là Bluetooth Mesh, cho phép thiết lập mạng lưới
nhiều nút. Mỗi nút trong mạng có thể đóng vai trò chuyển tiếp, cho phép dữ liệu
lan truyền đến các vùng rộng hơn theo mô hình phân tán. Điều này phù hợp với
các ứng dụng IoT quy mô đa cảm biến đặt tại nhiều nơi khác nhau.

\subsection{Thiết kế mạch}

Sau khi đã lựa chọn được vi điều khiển, cảm biến của hệ thống phần cứng phân
loại tư thế ngủ. Trong phần này, luận văn sẽ trình bày về thiết kế sơ đồ nguyên
lý cho mạch thông qua những câu hỏi: Các thành phần trong cần thiết để phục vụ
cho bài toán phân loại tư thế ngủ là gì? Chúng được kết nối với nhau như thế
nào? và chi tiết từng thành phần đó là gì?

Để đảm bảo các yêu cầu đặt ra, hệ thống cần
các thành phần sau: 01) Khối điều khiển: có chức năng xử lý toàn bộ logic trong
mạch, kết nối được với các thiết bị khác như điện thoại thông qua BLE và đặc
biệt, đủ hiệu năng để triển khai các mô hình học máy; 02) Khối cảm biến: có
nhiệm vụ thu thập dữ liệu sinh lý bao gồm 2 cảm biến chính là gia tốc (đã được
trình bày các phần trên) và thêm cảm biến âm thanh để phục vụ bài toán xác định
tiếng ngáy của nhóm; 03) Khối nguồn: có nhiệm vụ ổn áp về đúng dải điện áp
tương thích, lọc nhiễu, bảo vệ dòng; 04) Khối nạp và gỡ lỗi: có nhiệm vụ nạp mã
chương trình vào vi điều khiển; 05) Khối hiển thị: có nhiệm vụ đưa ra thông báo
khi có phát hiện bất thường. Kích thước của mạch là hình tròn, có đường kính
không quá 4~cm. Đây chính là những đầu mục tiên quyết để coi là đạt được mục
tiêu trong phần này.

\subsubsection{Khối điều khiển}

Với yêu cầu quan trọng đối điều khiển thì vi điều khiển nRF52840 được chọn hoàn
toàn có thể đáp ứng các tiêu trí đã nêu trong phần~\ref{subsec:vi_xu_ly} để làm
vi điều khiển của mạch. Tuy nhiên để đáp ứng đủ các yêu cầu của khối điều khiển
thì vẫn đang thiếu các thành phần như khối thu phát tín hiệu vô tuyến bao gồm
ăng ten, mạch phối hợp trở khánh, bộ dao động tinh thể và bộ đếm định thời nhằm
duy trì hoạt động ổn định trong các chế độ năng lượng thấp.

Sau khi tìm hiểu, trong khuôn khổ luận văn sẽ sử dụng module U-blox NINA-B306
để đảm nhiệm các nhiệm vụ của khối điều khiển \cite{uBlox2019_NINA_B3}.
\begin{figure}[H]
    \centering
    \includegraphics[width=0.7\textwidth]{images/ninab306_block.png}
    \caption{Sơ đồ khối U-blox NINA-B306}
    \label{ninab306_block}
\end{figure}

Mô-đun NINA-B306 được phát triển dựa trên vi điều khiển Nordic nRF52840, tích
hợp ăng-ten PCB nội bộ được cấp phép từ Abracon ngay trên bảng mạch của mô-đun
Hình~\ref{ninab306_block}. Ngoài ra, NINA-B306 được thiết kế với dải điện áp
hoạt động chuẩn hóa từ 1.7 V đến 3.6 V, thấp hơn so với giới hạn tối đa 5.5 V
của vi điều khiển nRF52840, nhằm tối ưu hiệu suất năng lượng.

\begin{figure}[H]
    \centering
    \includegraphics[width=1\textwidth]{images/ninab306_pin.png}
    \caption{Sơ đồ chân U-blox NINA-B306}
    \label{ninab306_pin}
\end{figure}

Sơ đồ chân thể hiện cấu trúc bố trí tối ưu với hơn 50 chân GPIO đa chức năng,
bao gồm các kênh analog, các đường truyền debug (SWD, TRACE) và các chân nguồn
được phân bố đối xứng, góp phần tăng cường độ ổn định tín hiệu và giảm nhiễu
điện từ Hình~\ref{ninab306_pin}.

\subsubsection{Khối cảm biến}
Như đã đề cập chi tiết ở phần~\ref{subsec:cam_bien}, cảm biến gia tốc Bosch
BMI270 được lựa chọn để thu thập tín hiệu gia tốc cho bài toán phân loại tư thế
ngủ. Ngoài ra, việc hướng tới việc xác định chứng ngưng thở khi ngủ OSA thì
nhóm có bổ sung thêm cảm biến âm thanh STMicroelectronics MP34DT06J. Cảm biến
MP34DT06J là microphone MEMS kỹ thuật số định hướng đa hướng có kích thước siêu
nhỏ gọn và tiêu thụ năng lượng thấp. Ngoài ra, MP34DT06J là cảm biến số có tỷ
lệ tín hiệu trên nhiễu (SNR) 64 dB và độ nhạy -26 dBFS ± 1 dB, đảm bảo chất
lượng thu âm ổn định. Thiết bị được đóng gói theo chuẩn SMD top-port và được
đảm bảo hoạt động ổn định trong dải nhiệt độ mở rộng từ -40 °C đến +85 °C và
kèm thêm điện áp đầu vào 1.8V hoàn toàn phù hợp với các thông số của vi điều
khiển. Tuy nhiên, trong khuôn khổ luận văn này sẽ không đề cập sâu vào cảm biến
này để tập trung tối đa vào dữ liệu cảm biến gia tốc cho bài toán phân loại tư
thế ngủ.

\subsubsection{Khối nguồn, nạp và gỡ lỗi, hiển thị}
Ba khối chức năng này được thiết kế nhằm đảm bảo sự ổn định và khả năng vận
hành tin cậy của toàn bộ hệ thống phần cứng. Đối với khối nguồn, hệ thống được
xác định có hai nguồn cung cấp chính gồm nguồn từ cổng USB và nguồn từ pin
Li-Polymer. Điện áp đầu vào của toàn mạch được xác định ở mức 3.3 V để tương
thích với các linh kiện và vi điều khiển. Ngoài ra, luận văn sử dụng một bộ
chuyển đổi hạ áp (step-down converter) tích hợp mạch điều khiển MP2322GQH để ổn
định nguồn đầu vào tại 3.3 V Hình~\ref{fig:MP2322GQH}

\begin{figure}[H]
    \centering
    \includegraphics[width=1\textwidth]{images/MP2322GQH.png}
    \caption{Sơ đồ khối IC MP2322GQH, dải đầu vào 3~V - 22~V, dòng tải tối đa 1~A, hiệu suất cao, tần số chuyển mạch 1.25~MHz}
    \label{fig:MP2322GQH}
\end{figure}

Bên cạnh đó, nguồn USB còn được bổ sung các thành phần hỗ trợ như diode bảo vệ
ngược dòng, tụ lọc nhiễu cao tần và bù tín hiệu, cùng điện trở xả tĩnh điện,
nhằm đảm bảo an toàn và giảm nhiễu điện từ trong quá trình vận hành.

Đối với khối hiển thị, hệ thống sử dụng bốn đèn LED khác màu bố trí tại rìa của mạch, có nhiệm vụ thông báo kết quả tư thế được phân loại, trạng thái hoạt động của hệ thống như nguồn, kết nối, truyền dữ liệu hoặc lỗi, giúp người dùng và kỹ sư dễ dàng theo dõi tình trạng thiết bị.

Với khối nạp và gỡ lỗi, hai phương thức chính để nạp bao gồm: (1) Qua Serial
Wire Debug (SWD); (2) Cơ chế Bootloader thông qua kết nối USB. Sự kết hợp này
giúp nâng cao tính linh hoạt và khả năng mở rộng trong quá trình phát triển
phần mềm, đồng thời giảm thiểu thời gian thử nghiệm.

\begin{figure}
    \centering
    \includegraphics[width=0.9\textwidth]{images/schematic.png}
    \caption{Sơ đồ mạch}
    \label{fig:schematic}
\end{figure}

\subsubsection{Sơ đồ mạch}

Sau khi xác định các linh kiện, các yêu cầu về điện áp, kiểu dữ liệu đầu vào,
đầu ra của các thành phần kể trên, sơ đồ mạch sẽ được xây dựng thông qua phần
mềm Altium Designer. Mạch được thiết kế hình tròn kích thước đường kính 3~cm,
2~lớp đúng với định hướng của nhóm. Kết quả được thể hiện tại
Hình~\ref{fig:schematic}

\begin{figure}[H]
    \centering
    \begin{subfigure}[b]{0.48\textwidth}
        \centering
        \includegraphics[width=0.5\textwidth]{images/pcd_dong_top.png}
        \label{fig:pcb_dong_top}
        \caption{lớp trên}

    \end{subfigure}
    \hfill
    \begin{subfigure}[b]{0.48\textwidth}
        \centering
        \includegraphics[width=0.5\textwidth]{images/pcb_dong_bottom.png}
        \label{fig:pcb_dong_bottom}
        \caption{lớp dưới}

    \end{subfigure}
    \caption{Bố trí mạch in hai mặt của bo mạch}
    \label{fig:pcb_dong}
\end{figure}

Sau khi hoàn thiện thiết kế, công đoạn tiếp theo là thực hiện hàn, lắp các linh
kiện theo sơ đồ mạch, và tiến hành kiểm tra hoạt động của các khối chức năng
trên bo mạch để đánh giá tính ổn định và độ tin cậy của hệ thống.

\begin{figure}
    \centering
    \begin{subfigure}[b]{0.48\textwidth}
        \centering
        \includegraphics[width=0.5\textwidth]{images/pcb_3d_top.png}
        \caption{lớp trên}
    \end{subfigure}
    \hfill
    \begin{subfigure}[b]{0.48\textwidth}
        \centering
        \includegraphics[width=0.5\textwidth]{images/pcb_3d_bottom.png}
        \caption{lớp dưới}

    \end{subfigure}
    \caption{Mô hình mạch 3D}
    \label{fig:pcb_3d}
\end{figure}

\newpage

\section{Hệ thống phần mềm}
Phần này trình bày tổng quan kiến trúc hệ thống phần mềm bao gồm: lập trình
phần cứng để đọc giá trị cảm biến, cấu hình BLE; phát triển ứng dụng di động
làm cầu nối giữa phần cứng và hệ thống đám mây, cùng với máy chủ và cơ sở dữ
liệu lưu trữ dữ liệu. Nội dung cũng đề cập đến các yêu cầu chức năng, phi chức
năng và thiết kế hệ thống ở mức cao nhằm đảm bảo khả năng triển khai thực tế và
mở rộng trong tương lai.

\subsection{Lập trình phần cứng}

Ở bước này, nhóm ưu tiên lập trình phần cứng bằng cách sử dụng các framework và thư viện do hãng cung cấp.
Cách tiếp cận này cho phép khai thác hầu hết các phương thức tương tác với phần cứng, cộng đồng hỗ trợ lớn đồng thời giúp rút ngắn đáng kể thời gian phát triển và giảm thiểu sai sót trong quá trình lập trình.
Tuy nhiên, để có thể kiểm soát toàn diện các chức năng của hệ thống cũng như tối ưu hóa mức tiêu thụ năng lượng, trong tương lai sẽ chuyển sang phương pháp lập trình ở mức thấp.

\subsubsection{Cấu hình vi xử lý và cảm biến}
Bước đầu tiên, nạp Bootloader của Arduino Nano vào vi điều khiển bằng giao thức
Serial Wire Debug (SWD). Điều này cho phép thiết lập môi trường lập trình ban
đầu và hỗ trợ việc tải chương trình trực tiếp từ máy tính thông qua cổng USB.

Việc đọc các giá trị cảm biến thông qua thư viện SparkFun BMI270 Arduino
Library \cite{SparkFun_BMI270_Library}. Đây là thư viện mã nguồn mở do SparkFun
phát triển đóng vai trò như một lớp trừu tượng, cho phép vi điều khiển giao
tiếp trực tiếp với cảm biến Bosch BMI270. Thư viện cung cấp đầy đủ các hàm để
cấu hình dải đo, tần số lẫy mẫu và chế độ năng lượng tiêu thụ.

\begin{lstlisting}[language=C,
caption={Cấu hình cảm biến qua thư viện \texttt{SparkFun\_BMI270\_Arduino\_Library}},
label={lst:cauhinh_accelerometer},
captionpos=b]

  Wire1.begin();

  if (imu.beginI2C(BMI2_I2C_PRIM_ADDR, Wire1) != BMI2_OK) {
    Serial.println("Error: BMI270 not detected!");
    while (1);
  }

  struct bmi2_sens_config config;
  config.type = BMI2_ACCEL;
  config.cfg.acc.odr = BMI2_ACC_ODR_12_5HZ;
  config.cfg.acc.range = BMI2_ACC_RANGE_2G;
  config.cfg.acc.bwp = BMI2_ACC_NORMAL_AVG4;
  config.cfg.acc.filter_perf = BMI2_POWER_OPT_MODE;

  if (imu.setConfig(config) == BMI2_OK)
    Serial.println("Accelerometer configured successfully!");
  else
    Serial.println("Failed to configure accelerometer!");

  if (imu.getConfig(&config) == BMI2_OK) {
    Serial.println("Current Accelerometer Config:");
    Serial.print("  Range setting = ");
    switch (config.cfg.acc.range) {
      case BMI2_ACC_RANGE_2G:  Serial.println("$\\pm$2g"); break;
      case BMI2_ACC_RANGE_4G:  Serial.println("$\\pm$4g"); break;
      case BMI2_ACC_RANGE_8G:  Serial.println("$\\pm$8g"); break;
      case BMI2_ACC_RANGE_16G: Serial.println("$\\pm$16g"); break;
      default: Serial.println("Unknown"); break;
    }

    Serial.print("  ODR setting = ");
    switch (config.cfg.acc.odr) {
      case BMI2_ACC_ODR_0_78HZ: Serial.println("0.78 Hz"); break;
      case BMI2_ACC_ODR_1_56HZ: Serial.println("1.56 Hz"); break;
      case BMI2_ACC_ODR_3_12HZ: Serial.println("3.12 Hz"); break;
      case BMI2_ACC_ODR_6_25HZ: Serial.println("6.25 Hz"); break;
      case BMI2_ACC_ODR_12_5HZ: Serial.println("12.5 Hz"); break;
      case BMI2_ACC_ODR_25HZ:   Serial.println("25 Hz"); break;
      case BMI2_ACC_ODR_50HZ:   Serial.println("50 Hz"); break;
      case BMI2_ACC_ODR_100HZ:  Serial.println("100 Hz"); break;
      default: Serial.println("Unknown"); break;
    }
  } else {
    Serial.println();
  }

\end{lstlisting}


Phần cấu hình được thể hiện trong mã nguồn~\ref{lst:cauhinh_accelerometer}.
Việc gọi hàm \texttt{imu.beginI2C()} nhằm thiết lập kênh truyền
I\textsuperscript{2}C và xác nhận sự hiện diện của thiết bị. Tần số lấy mẫu
được đặt ở 12.5 Hz và dải đo ở mức ±2g nhằm tối ưu độ phân giải cho các dao
động nhỏ khi ngủ ở người. Việc thiết lập cấu hình qua hàm \texttt{setConfig()}
và xác thực lại bằng \texttt{getConfig()} thể hiện nguyên tắc kiểm chứng hai
chiều nhằm chắc chắn các cấu hình đã được đặt trên phần cứng.

\subsubsection{Cấu hình BLE}

Quá trình khởi tạo, kết nối và truyền lên thiết bị trung tâm bao gồm các bước
sau: tạo bản tin quảng bá (advertising), thiết lập cơ chế kết nối/ngắt kết nối,
cũng như định nghĩa UUID của dịch vụ và đặc tính theo mô hình \gls{gatt}.

\begin{lstlisting}[language=C,
caption={Chương trình thiết lập kết nối, truyền dữ liệu qua BLE},
label={lst:ble_bmi270},
escapeinside={(*@}{@*)}
]
const char* deviceServiceUuid = "19b10000-e8f2-537e-4f6c-d104768a1214";
const char* deviceServiceCharacteristicUuid = "19b10001-e8f2-537e-4f6c-d104768a1214";

BLEService accelerometerService(deviceServiceUuid);
BLECharacteristic accelerometerCharacteristic(
  deviceServiceCharacteristicUuid,
  BLERead | BLEWrite | BLENotify,
  9, 3
);
BMI270 imu;

void setup() {
  Serial.begin(9600);
  Serial.println("Started");
  Wire1.begin();

  if (!BLE.begin()) {
    Serial.println("- Starting Bluetooth Low Energy module failed!");
    while (1);
  }

  BLE.setLocalName("Master_2025_BLE");
  BLE.setDeviceName("Master_2025_BLE");
  BLE.setAdvertisedService(accelerometerService);

  accelerometerService.addCharacteristic(accelerometerCharacteristic);
  BLE.addService(accelerometerService);
  accelerometerCharacteristic.canSubscribe();
  accelerometerCharacteristic.subscribed();

  BLE.advertise();

  Serial.println("Nano 33 BLE (Peripheral Device)");

  **SETUP accelerometer**
}

static float x, y, z;

void loop() {
  BLEDevice central = BLE.central();
  Serial.println("- Discovering central device...");
  delay(500);

  if (central) {
    Serial.println("* Connected to central device!");
    Serial.print("* Device MAC address: ");
    Serial.println(central.address());
    while (central.connected()) {
      imu.getSensorData();
      x = imu.data.accelX;
      y = imu.data.accelY;
      z = imu.data.accelZ;

      Serial.print(x);
      Serial.print('\t');
      Serial.print(y);
      Serial.print('\t');
      Serial.println(z);

      uint8_t accelData[9] = {
        (x >= 0) ? 1 : 0,
        abs((int)x),
        abs((int)(x * 100) % 100),
        (y >= 0) ? 1 : 0,
        abs((int)y),
        abs((int)(y * 100) % 100),
        (z >= 0) ? 1 : 0,
        abs((int)z),
        abs((int)(z * 100) % 100)
      };

      accelerometerCharacteristic.writeValue(accelData, sizeof(accelData)); 
      delay(100); 
    }
  }
}

\end{lstlisting}


\usetikzlibrary{positioning}

\begin{figure}[H]
    \centering
    \begin{tikzpicture}[
            every node/.style={font=\footnotesize, align=center},
            process/.style={rectangle, draw, rounded corners,
                    minimum width=4cm, minimum height=1cm,
                    fill=blue!10, inner sep=5pt},
            arrow/.style={->, thick}
        ]
        \matrix[column sep=0mm, row sep=0.7cm] {
            \node[process] (init) {Khởi tạo BLE \& BMI270};                                                \\
            \node[process] (config) {Cấu hình cảm biến \\ ODR = 12.5 Hz, Range = $\pm$2g}; \\
            \node[process] (connect) {Kết nối thiết bị trung tâm qua BLE};                                 \\
            \node[process] (send) {Đọc \& gửi dữ liệu $x$, $y$, $z$};       \\
            \node[process] (loop) {Lặp chu kỳ 100 ms};                                                     \\
        };

        \foreach \i/\j in {init/config, config/connect, connect/send, send/loop}
        \draw[arrow] (\i) -- (\j);
    \end{tikzpicture}
    \caption{Sơ đồ tóm tắt quy trình hoạt động BLE-BMI270 của đoạn mã nguồn~\ref{lst:ble_bmi270}}
    \label{fig:ble_bmi270_shortflow}
\end{figure}

Thứ nhất, khởi tạo BLE, kích hoạt truyền thông BLE với tên hiển thị là
Master\_2025\_BLE, thiết lập cấu trúc dịch vụ GATT với hai UUID cho service và
characteristic. Trong đó characteristic được phép đọc, ghi và phát tín hiệu.

Thứ hai, khởi tạo cảm biến BMI270 và cấu hình.

Thứ ba, tạo kết nối với thiết bị trung tâm, khi tín hiệu quảng bá được phản
hồi. Sau khi kết nối thành công, phần cứng lấy được thông tin của thiết bị
trung tâm ở đây là ứng dụng điện thoại. Đây một bước xác thực cần thiết để chỉ
cho phép thiết bị trung tâm nào được lập trình sẵn.

Thứ tư, thu nhận dữ liệu cảm biến ba trục (x, y, z) từ BMI270. Các giá trị gia
tốc được đọc và chuẩn hóa trước khi truyền. Việc chuẩn hóa ở đây gồm tách dấu
và phần nguyên - phần thập phân để giảm kích thước gói tin mà vẫn đảm bảo độ
chính xác khi tái lập lại.

Cuối cùng, truyền dữ liệu định dạng byte qua BLE thông qua characteristic đã
định nghĩa. Mỗi gói tin gồm 9 byte chứa đầy đủ thông tin về hướng và biên độ
gia tốc trên ba trục, được gửi chu kỳ 100 ms, bảo đảm tốc độ lấy mẫu ổn định
(10 Hz) phục vụ phân tích tư thế trong thời gian thực.

Việc chuẩn hóa 5 giai đoạn này có ưu điểm có thể bóc tách cho mục đích khác
nhau, tối ưu cho từng giai đoạn, đảm bảo code mạch lạc và dễ bảo trì.

\subsubsection{Lọc nhiễu bằng bộ lọc Kalman}

Trong phần phân tích các bộ lọc cũng như từ thực nghiệm nhận thấy, khi có những
chuyển động đột ngột gia tốc cao có thể gây ra sai lệnh giá trị cảm biến.

\begin{figure}[htbp]
    \centering
    \includegraphics[width=1\textwidth]{images/z_kalman.png}
    \caption{So sánh tín hiệu trục Z trước và sau lọc Kalman}
    \label{fig:z_kalman}
\end{figure}

Hình~\ref{fig:z_kalman} trình bày so sánh tín hiệu gia tốc thu được trên trục Z
giữa hai trạng thái: dữ liệu thô và dữ liệu sau khi qua bộ lọc Kalman. Có thể
quan sát thấy, trong vùng được khoanh tròn, tín hiệu thô (đường màu xanh) xuất
hiện một giá trị đột biến âm với biên độ lớn, thể hiện hiện tượng trôi tín hiệu
tạm thời do nhiễu cảm biến hoặc rung động ngoài mong muốn.

Sau khi áp dụng bộ lọc Kalman, đường tín hiệu (đường màu cam) trở nên mượt hơn
và ổn định, đồng thời loại bỏ được phần lớn sai lệch ngắn hạn. Kết quả này
chứng tỏ bộ lọc Kalman không chỉ thực hiện vai trò giảm nhiễu mà còn có khả
năng ước lượng trạng thái thật của hệ thống thông qua mô hình dự đoán - cập
nhật liên tục. Việc lọc này đặc biệt quan trọng trong bối cảnh dữ liệu thu từ
cảm biến gia tốc dùng cho nhận diện tư thế ngủ, bởi vì sai số đột biến dù chỉ
trong một khoảng ngắn cũng có thể dẫn đến phân loại sai tư thế trong mô hình
học máy.

Ngoài ra, vùng sai lệch này cũng cho thấy hạn chế của tín hiệu thô khi cảm biến
được gắn trên cơ thể người, nơi dao động nhỏ do hô hấp, chuyển động tự nhiên
hoặc thay đổi điểm đặt có thể gây ra nhiễu. Việc sử dụng Kalman giúp hiệu
chỉnh, đảm bảo giá trị đo được bám sát quỹ đạo thật của chuyển động. Một vấn đề
là bộ lọc Kalman có thể làm giảm giá trị biên tại các đỉnh. Tuy nhiên, vấn đề
này được nhận định là không ảnh hưởng nhiều đến toàn bộ bài toán phân loại tư
thế ngủ. Lý do là vì bộ đặc trưng không sử dụng trực tiếp các giá trị đỉnh này
mà là một cửa sổ mẫu và các thống kê trong cửa sổ đó.

\subsection{Phần mềm thu thập, lưu trữ}

Phần mềm trong nghiên cứu này không chỉ đơn thuần là công cụ trực quan hoá dữ
liệu cảm biến, mà còn được thiết kế như một mắt xích trọng yếu trong toàn bộ
quy trình từ thu thập, truyền tải, lưu trữ, cho đến huấn luyện và triển khai mô
hình học máy. Cách tiếp cận này đảm bảo rằng dữ liệu thu nhận từ môi trường
thực tế được xử lý nhất quán, có khả năng tái sử dụng và dễ dàng tích hợp với
các hệ thống khác.

\begin{figure}[H]
    \centering
    \includegraphics[width=1\textwidth]{images/architecture_software.png}
    \caption{Kiến trúc tổng thể của hệ thống}
    \label{fig:architecture_software}
\end{figure}
Hình~\ref{fig:architecture_software} mô tả tổng quan các thành phần cần thiết trong luận văn và kết nối giữa chúng.
Để đưa ra kết quả phân loại tư thế ngủ dữ liệu từ cảm biến có thể có 2 hướng đi. Thứ nhất, dữ liệu từ cảm biến thông qua vi điều khiển đẩy lên ứng dụng di động
sau đó được lưu trữ hoặc được suy luận trên máy chủ đám mây rồi trả về kết quả tại ứng dụng di động.
Thứ hai. dữ liệu từ cảm biến sau khi được đọc từ vi điều khiển thì
được suy luận ngay tại biên và hiển thị kết quả thông qua các chỉ báo đèn led.
Cả 2 cách đều được nghiên cứu và triển khai trong luận văn này.

\subsubsection{Ứng dụng di dộng}
Được coi như một cổng kết nối trong điện toán đám mây
có nhiệm vụ gom các tín hiệu từ cảm biến gia tốc hoặc có thể mở
rộng đa cảm biến khác, hiển thị chúng theo thời gian thực và chuyển
lên máy chủ đám mây.
Dựa trên yêu cầu bài toán, xác định được các yêu cầu của phần mềm như
Bảng~\ref{tab:app_features}. Ngoài ra, các yêu cầu phi chức năng cũng được chú ý như là bảo mật thông tin, thời gian phản hồi hệ thống.
\begin{table}[H]
    \centering
    \caption{Các nhóm chức năng chính của ứng dụng}
    \label{tab:app_features}
    \begin{tabularx}{\textwidth}{|l|l|X|}
        \hline
        \textbf{Nhóm} & \textbf{Chức năng}             & \textbf{Mô tả}                                                                               \\
        \hline

        \multirow{3}{*}{Xác thực}
                      & Đăng ký                        & Người dùng đăng ký thông tin và xác thực qua email, số điện thoại.                           \\ \cline{2-3}
                      & Đăng nhập                      & Sử dụng tài khoản và mật khẩu đã đăng ký để truy cập hệ thống.                               \\ \cline{2-3}
                      & Quên mật khẩu                  & Hỗ trợ lấy lại tài khoản của người dùng.                                                     \\
        \hline

        \multirow{5}{*}{Ứng dụng}
                      & Kết nối Bluetooth              & Kết nối ứng dụng với phần cứng thông qua BLE.                                                \\ \cline{2-3}
                      & Chọn cảm biến                  & Lựa chọn cảm biến như gia tốc hoặc âm thanh.                                                 \\ \cline{2-3}
                      & Câu hỏi STOP-BANG, Epworth     & Trả lời bảng câu hỏi STOP-BANG và Epworth để đánh giá nguy cơ OSA; hiển thị lịch sử trả lời. \\ \cline{2-3}
                      & Theo dõi thông tin từ cảm biến & Hiển thị dữ liệu thời gian thực thông qua biểu đồ.                                           \\ \cline{2-3}
                      & Chatbot y tế                   & Chatbot sử dụng tập câu hỏi y học giấc ngủ (2000 câu) và công nghệ RAG.                      \\
        \hline
    \end{tabularx}
\end{table}

Ứng dụng di động được viết bằng ngôn ngữ DART trên nền tảng Flutter.
Vì sao lại chọn Dart mà không phải là các ngôn ngữ thuần túy như Java hay
Swift? Nói chung, chọn Dart trên nền tảng Flutter lý do chính là có thể
tạo ra hai phiên bản trên Android và IOS trên cùng một bộ mã chương trình.
Ngoài ra, việc khởi tạo dự án nhanh, cú pháp thân thiện với người lập trình
hơn, kèm cộng đồng đông đảo làm giảm đáng kể thời gian phát triển.

Để ứng dụng dễ dàng kiểm thử và nâng cấp,
BLoC được lựa chọn để quản lý các trạng thái của ứng dụng. Nó hoạt động dựa trên nguyên tắc nhận sự kiện đầu vào từ giao diện gồm trạng thái và dữ liệu đầu vào, xử lý trong khối BLOC như gọi lên máy chủ trung tâm
và trả lại dữ liệu. Qua đó, luồng dữ liệu trở nên rõ ràng. Cấu trúc tổng thể của kiến trúc
BLoC gồm ba lớp chính được mô tả trong Hình~\ref{flutter}.

\begin{figure}[htbp]
    \centering
    \includegraphics[width=0.8\textwidth]{images/flutter.png}
    \caption{Cấu trúc kiến trúc BLoC trong ứng dụng Flutter}
    \label{flutter}
\end{figure}

Mã nguồn của toàn bộ ứng dụng di động sẽ để dưới phần mục lục. Còn trong khuôn
khổ luận văn nay sẽ giải thích chức năng chính là kết nối phần cứng, hiển thị
và chuyển dữ liệu lên máy chủ trung tâm.

\begin{figure}[H]
    \centering
    \begin{tikzpicture}[
            every node/.style={font=\footnotesize, align=center},
            process/.style={rectangle, draw, rounded corners,
                    minimum width=4cm, minimum height=1cm,
                    fill=blue!10, inner sep=5pt},
            arrow/.style={->, thick}
        ]
        \matrix[column sep=0mm, row sep=0.7cm] {
            \node[process] (init) {Yêu cầu cấp quyền Bluetooth từ thiết bị thật};                 \\
            \node[process] (finding) {Tìm kiểm các thiết bị BLE ngoại vi};                        \\
            \node[process] (connect) {Thiết lập kết nối với phần cứng qua tên Master\_2025\_BLE}; \\
            \node[process] (choosing) {Chọn đúng dịch vụ và đặc trưng theo UUID};                 \\
            \node[process] (notify) {Bắn tín hiệu theo dõi};
            \node[process] (present) {Hiển thị dữ liệu và đẩy lên máy chủ trung tâm};             \\
        };

        \foreach \i/\j in {init/finding, finding/connect, connect/choosing, choosing/notify, notify/present}
        \draw[arrow] (\i) -- (\j);
    \end{tikzpicture}
    \caption{Sơ đồ kết nối và nhận dữ liệu của ứng dụng di động}
    \label{fig:mobile_ble}
\end{figure}

Sau khi tìm hiểu, thư viện flutter\_blue\_plus được chọn để kết nối BLE từ ứng
dụng di động đến phần cứng \cite{flutter_blue_plus}. Nó hỗ trợ các phương thức
gồm quét, kết nối, đọc/ghi đặc tính và nhận thông báo từ thiết bị BLE trên
Android, iOS, macOS, Linux và web. Ngoài ra nó không có thêm phụ thuộc ngoài
Flutter, và được thiết kế cho vai trò BLE trung tâm (không hỗ trợ Bluetooth cổ
điển như HC-05). Đây là thư viện miễn phí cho cá nhân, tổ chức nhỏ, và phi lợi
nhuận; các công ty trên 50 nhân viên cần mua bản quyền để sử dụng thương mại.

\begin{lstlisting}[language=C,
caption={Mã nguồn hàm quét và kết nối các thiết bị BLE},
label={lst:ble_scan},
captionpos=b]

  Future onScanPressed() async {
    try {
      // `withServices` is required on iOS for privacy purposes, ignored on android.
      var withServices = [Guid("180f")]; // Battery Level Service
      _systemDevices = await FlutterBluePlus.systemDevices(withServices);
    } catch (e) {
      Snackbar.show(ABC.b, prettyException("System Devices Error:", e),
          success: false);
      print(e);
    }
    try {
      await FlutterBluePlus.startScan(
        timeout: const Duration(seconds: 15),
        webOptionalServices: [
         
        ],
      );
    } catch (e) {
      Snackbar.show(ABC.b, prettyException("Start Scan Error:", e),
          success: false);
      print(e);
    }
    if (mounted) {
      setState(() {});
    }
  }

  Future onStopPressed() async {
    try {
      FlutterBluePlus.stopScan();
    } catch (e) {
      Snackbar.show(ABC.b, prettyException("Stop Scan Error:", e),
          success: false);
      print(e);
    }
  }

  void onConnectPressed(BluetoothDevice device) {
    device.connectAndUpdateStream().catchError((e) {
      Snackbar.show(ABC.c, prettyException("Connect Error:", e),
          success: false);
    });
    MaterialPageRoute route = MaterialPageRoute(
        builder: (context) => DeviceScreen(device: device),
        settings: RouteSettings(name: '/DeviceScreen'));
    Navigator.of(context).push(route);
  }

\end{lstlisting}


Mã nguồn~\ref{lst:ble_scan} trình bày quy trình quét và kết nối thiết bị ngoại
vi BLE trong Flutter dựa trên thư viện flutter\_blue\_plus. Hàm onScanPressed()
mục đích quét thiết bị đang khả dụng thông qua dịch vụ mức pin có UUID là 180F.
Khi này BLoc nhận trạng thái quét, thực hiện quét và trả về danh sách mảng các
thiết bị bao gồm tên hiển thị, địa chỉ MAC. Hàm onStopPressed() dừng quá trình
quét, còn onConnectPressed() thiết lập kết nối với thiết bị được chọn và chuyển
đến màn hình điều khiển chi tiết.

\begin{lstlisting}[float,language=Java,caption={Tập lệnh để tìm kiểm dịch vụ cảm biến},label=flutterBle,captionpos=t]
StreamBuilder<List<BluetoothService>>(
  stream: device.services,
  initialData: [],
  builder: (c, snapshot) {
    if (snapshot.data!.length > 0) {
      isService = true;
    }
    BluetoothService serviceAcclerometer;
    if (snapshot.data == null || snapshot.data!.length == 0) {
      return Text("Please contact customer Service");
    }
    for (int i = 0; i < snapshot.data!.length; i++) {
      if (snapshot.data![i].uuid.toString() ==
          Constants.ACCLEROMETER_SERVICE) {
        accelerometerService = snapshot.data![i];
      }
    }
    if (accelerometerService == null) {
      return Text("Please contact customer Service");
    }
    for (int i = 0;
        i < accelerometerService!.characteristics.length;
        i++) {
      print(accelerometerService!.characteristics[i].uuid);
      if (accelerometerService!.characteristics[i].uuid
              .toString() ==
          Constants.ACCLEROMETER_CHARACTION) {
        accelerometerCharactis =
            accelerometerService!.characteristics[i];
      }
    }
  });
\end{lstlisting}

Mã nguồn~\ref{flutterBle} trình bày cách ứng dụng hoạt động khi người dùng chọn
đúng BLE của phần cứng. Cụ thể, StreamBuilder được sử dụng để lắng nghe các
thay đổi của device.services, cho phép hệ thống phản ứng tức thời khi thiết bị
BLE truyền lên các dịch vụ khả dụng. Quá trình so khớp UUID giữa dịch vụ thực
tế đã cài đặt ở phần cứng và hằng số định nghĩa (ACCLEROMETER\_SERVICE,
ACCLEROMETER\_CHARACTION) trên ứng dụng di động cho phép xác thực logic nhằm
đảm bảo rằng ứng dụng chỉ giao tiếp với dịch vụ và đặc tính cảm biến hợp lệ.

Sau khi đã khớp các thành phần BLE, ứng dụng di động có thêm nút ấn để lấy
thông tin. Khi đó, ứng dụng di động sẽ gửi 1 yêu cầu theo dõi đến phần cứng.
Như đoạn mã~\ref{lst:ble_bmi270}, vi điều khiển lắng nghe được hành động theo
dõi sẽ lập tức gửi các gói tin lên lại ứng dụng di dộng theo cấu trúc đã xác
lập.

Đến đây, luận văn đã trình bày \textbf{hoàn thành quy trình kết nối và nhận dữ liệu từ phần cứng}.
Tiếp theo, luận văn sẽ mô tả rõ cách thức dữ liệu được đóng gói và chuyển lên phần máy chủ trung tâm.

Ngoài dữ liệu 3 trục x, y, z, nhóm xác định tập dữ liệu cần lưu thêm thời gian
tạo và giá trị ID của người đang dùng. Việc này đảm bảo việc minh bạch, dễ đối
chiếu và phân tích, mô phỏng lại giá trị cảm biến theo chuỗi thời gian. Mã
nguồn~\ref{format_ble} mô tả việc tối ưu hóa đẩy dữ liệu HTTP theo lô giúp giảm
tải phía máy chủ trung tâm. Điều này tăng tính mở rộng lên hàng trăm thiết bị
hoặc giá trị cảm biến khác đẩy lên cùng 1 thời điểm.

\begin{lstlisting}[float,language=C,caption="Cấu trúc dữ liệu của phần nội dung đẩy lên máy chủ",label=format_ble,captionpos=b]
{
    "value": "0.88%0.66%0.99@2022-01-01/0.88%0.66%0.99@2022-01-01/0.88%0.66%0.99@2022-01-01",
    "customer": "62a5f5672ad9c724ef117d76"
}

\end{lstlisting}


Một điểm đáng lưu ý nữa, nhóm kết hợp cùng GS.TS. Dương Quý Sỹ phát triển thêm
tính năng \textbf{chatbot y học giấc ngủ} dựa trên kỹ thuật
\textbf{Retrieval-Augmented Generation (RAG)}. Chatbot này được xây dựng từ cơ
sở dữ liệu gồm hơn 2000 câu hỏi và câu trả lời chuyên sâu liên quan đến giấc
ngủ được biên tập bởi GS.TS Dương Quý Sỹ, bao gồm cả tài liệu lâm sàng, nghiên
cứu khoa học và các hướng dẫn thực hành. Người dùng có thể đặt câu hỏi tự nhiên
như “Tôi có nên lo nếu ngủ ngáy liên tục?” hoặc “STOP-BANG > 5 có ý nghĩa gì?”,
và chatbot sẽ phản hồi dựa trên kiến thức được truy xuất từ tài liệu nền và
được tổng hợp lại bằng mô hình ngôn ngữ.

Từ những phân tích nêu trên, luận văn đã chuẩn hóa được các bước cơ bản để từ
dữ liệu ở phần cứng đến lưu trữ trên máy chủ đám mây. Đây cũng chính là các
bước của hệ thống điện toán đám mây khi mọi dữ liệu, mô hình tính toán sẽ cùng
một nơi trên 1 hệ thống phần cứng mạnh mẽ.

\subsubsection{Máy chủ đám mây}

Để giảm thời gian phát triển, toàn bộ hệ thống máy chủ được triển
khai trển hạ tầng của Amazon Web Services (AWS) gồm 2 phần chính là máy chủ trung tâm và cơ sở dữ liệu.

\textbf{Máy chủ trung tâm}
có nhiệm vụ mở các kết nối (API) ra bên ngoài, nhận yêu cầu từ ứng dụng di động.
Ngôn ngữ Typescript (TS) trên nền tảng NodeJS được lựa chọn để xây dựng máy chủ trung tâm với các lý do: 1) TS cung cấp khả năng định nghĩ kiểu dữ liệu
cho biến, hàm. Các lỗi đó sẽ được phát hiện trong quá trình biên dịch ngay trước khi đến quá trình chạy (run time); 2) NodeJS có nhiều thư viện
hỗ trợ bảo mật, tương tác với các thành phần khác và khả năng khởi tạo nhanh dự án.
Về mặt cấu trúc mã nguồn, hệ thống được tổ chức theo mô hình phân lớp như thể hiện trong Hình~\ref{be_structure}.
\begin{table}[htbp]
    \centering
    \caption{Mô tả các thư mục chính trong cấu trúc mã nguồn máy chủ trung tâm}
    \label{tab:be_folder_structure}
    \begin{tabular}{p{3.2cm} p{10.5cm}}
        \toprule
        \textbf{Thư mục}      & \textbf{Chức năng và nội dung chính}                                                                         \\
        \midrule
        \texttt{mqtt\_broken} &
        Chứa mã nguồn khởi tạo và quản lý máy chủ MQTT, chịu trách nhiệm giao tiếp theo mô hình publish-subscribe với các thiết bị đầu cuối. \\

        \texttt{src/common}   &
        Bao gồm các mô-đun dùng chung như xác thực người dùng, quản lý kết nối cơ sở dữ liệu, và chuẩn hóa phản hồi lỗi hệ thống.            \\

        \texttt{src/config}   &
        Lưu trữ các thông số cấu hình nội bộ, bao gồm biến môi trường, thông tin kết nối và khóa truy cập nội bộ giữa các dịch vụ.           \\

        \texttt{src/migrate}  &
        Chứa lịch sử các lần thay đổi cấu trúc cơ sở dữ liệu nhằm đảm bảo khả năng truy vết, phục hồi và đồng bộ khi triển khai.             \\

        \texttt{src/modules}  &
        Tập hợp toàn bộ các mô-đun nghiệp vụ, nơi thực hiện các chức năng xử lý chính và điều phối hoạt động của hệ thống.                   \\
        \bottomrule
    \end{tabular}
\end{table}

\begin{figure}[htbp]
    \centering
    \includegraphics[width=0.3\textwidth]{images/be_structure.png}
    \caption{Cấu trúc cây thư mục mã nguồn của máy chủ trung tâm}
    \label{be_structure}
\end{figure}

Kiến trúc trên không chỉ đảm bảo tính tách biệt giữa các tầng xử lý mà còn giúp
nâng cao khả năng bảo trì, kiểm thử và mở rộng trong tương lai.

\textbf{Cơ sở dữ liệu} có nhiệm vụ lưu trữ dữ liệu bao gồm dữ liệu về người dùng, cảm biến.
Sau khi tìm hiểu, nhóm chọn kiến trúc lưu trữ được thiết kế theo hướng lai,
kết hợp cả NoSQL (MongoDB) và SQL (Postgres), nhằm tận dụng thế mạnh riêng của từng loại hình.

Với đặc thù của dữ liệu cảm biến là mang tính phi cấu trúc, có tần suất cập
nhật gần như không và yêu cầu truy xuất theo chuỗi thời gian. Hệ thống lựa chọn
sử dụng \textbf{MongoDB} làm nền tảng lưu trữ chính cho nhóm dữ liệu này.
MongoDB thể hiện ưu thế vượt trội nhờ mô hình lưu trữ dạng tài liệu , cho phép
linh hoạt mở rộng cấu trúc dữ liệu mà không cần định nghĩa cột hàng cố định.
Bên cạnh đó, cơ chế \textit{indexing} tối ưu giúp tăng tốc độ truy vấn đối với
dữ liệu thời gian, đồng thời hỗ trợ các phép lọc, nhóm và tổng hợp hiệu quả
trong môi trường có khối lượng dữ liệu lớn.

\begin{figure}
    \centering
    \includegraphics[width=0.8\textwidth]{images/flow_http2.jpg}
    \caption{Lưu đồ thuật toán lưu trữ dữ liệu cảm biến}
    \label{flow_http}
\end{figure}

Ngược lại, \textbf{PostgreSQL} được triển khai nhằm quản lý các dữ liệu có cấu
trúc ổn định và đòi hỏi tính toàn vẹn quan hệ cao, bao gồm các thông tin định
danh người dùng, mật khẩu được mã hoá, kết quả bảng điểm STOP-BANG, thang điểm
buồn ngủ ban ngày Epworth, chỉ số BMI, cùng tiền sử bệnh nền. Cách tiếp cận này
không chỉ tăng cường khả năng kiểm soát logic nghiệp vụ ở tầng dữ liệu mà còn
là yêu cầu tiên quyết để hệ thống tuân thủ các tiêu chuẩn về \textbf{an toàn và
    bảo mật thông tin y tế}.

Lưu đồ thuật toán lưu trữ dữ liệu được thể hiện trong Hình~\ref{flow_http}, bao
gồm hai nhánh xử lý. \textit{(i)} Khi người dùng không có kết nối mạng, hệ
thống vẫn duy trì liên kết BLE và hiển thị tín hiệu thời gian thực; dữ liệu
được lưu tạm ở phía thiết bị thay vì đẩy lên cloud. \textit{(ii)} Khi người
dùng đã đăng nhập và có Internet, ứng dụng chuyển sang chế độ đồng bộ theo lô:
mỗi 1000 mẫu được gộp thành một gói để gửi lên máy chủ. Cơ chế này sẽ làm giảm
số lần yêu cầu từ ứng dụng lên máy chủ.

Do lý do độ dài trong khuôn khổ luận văn, luận văn không trình bày hết toàn bộ
các nghiệp vụ của máy chủ trung tâm. Phần mã nguồn sẽ được chú thích tại mục
lục.

\subsection{Học máy trong phân loại tư thế ngủ}
\begin{figure}[H]
    \centering
    \includegraphics[width=1\textwidth]{images/hocmaytime.png}
    \caption{Phân bố khối lượng công việc với bài toán học máy phân loại tư thế ngủ trong luận văn}
    \label{hocmay_time}
\end{figure}

Tổng quan về các bước xây dựng hệ thống học máy cho bài toán phân loại tư thế
ngủ đã được trình bày tại Chương~\ref{chapter:1-introduction}. Sau khi đánh
giá, 4 mô hình học máy truyền thống là rừng ngẫu nhiên (Random forest - RF),
máy vec-tơ hỗ trợ (Support vector machine - SVM), hồi quy Logistic (Logistic
regression - LR), thuật toán tăng cường Gradient Boosting (GB) và một mô hình
mạng nơ-ron nhân tạo (Artificial Neural Network - ANN), được lựa chọn để đánh
giá.

Việc lựa chọn này dựa trên hai yếu tố là khả năng khái và độ phức tạp thuật
toán. Về khả năng khái quát, các mô hình được lựa chọn phản ánh những hướng
tiếp cận tiêu biểu trong học máy truyền thống, bao gồm cả mô hình tuyến tính,
phi tuyến và mô hình học sâu cơ bản. Cụ thể, RF và GB là hai đại diện tiêu biểu
của nhóm mô hình ensemble dựa trên cây quyết định, có khả năng mô hình hóa các
quan hệ phi tuyến phức tạp và khắc phục hiện tượng quá khớp cao. Hồi quy
logistic giúp giải thích trực quan mối quan hệ giữa các đặc trưng đầu. Máy
véc-tơ hỗ trợ được lựa chọn vì có khả năng tạo ranh giới phân lớp rõ ràng trong
không gian đặc trưng. Bên cạnh đó, Mạng nơ-ron nhân tạo được đưa vào nhằm khảo
sát khả năng trừu tượng hóa sâu của dữ liệu cảm biến, nơi mối quan hệ giữa đặc
trưng và nhãn phân loại có thể phi tuyến và phức tạp hơn so với các mô hình cổ
điển. Trong đó, mô hình ANN được thiết kế với hai lớp ẩn nhằm đánh giá sơ bộ
hiệu quả của mô hình học sâu trước khi mở rộng sang các kiến trúc phức tạp hơn
như mạng tích chập (Convolutional Neural Network - CNN), mạng hồi tiếp
(Recurrent Neural Network - RNN) hay mạng bộ nhớ dài ngắn hạn (Long Short-Term
Memory - LSTM).

\subsubsection{Hồi quy Logistic}

Đây là thuật toán học máy có giám sát được thiết kế cho các tác vụ phân loại dựa trên các đặc trưng đầu vào.
Về mặt cấu trúc, hồi quy Logistic làm việc dựa trên nguyên tắc của hàm sigmoid - một hàm phi tuyến
chuyển đầu vào của nó thành xác suất thuộc về một trong hai
lớp đối với bài toán phân loại nhị phân:

\begin{equation}
    \sigma(z) = \frac{1}{1 + e^{-z}}, \quad \text{với } z = \mathbf{w}^T \mathbf{x} + b
\end{equation}

Trong đó, $\mathbf{w}$ là vector trọng số, $\mathbf{x}$ là vector đặc trưng đầu
vào, và $b$ là hệ số điều chỉnh (bias). Khi xác suất này vượt quá một ngưỡng
nhất định (thông thường là $0.5$), mô hình sẽ gán nhãn cho $\mathbf{x}$ là
thuộc lớp dương. Ngược lại, nếu nhỏ hơn ngưỡng, mẫu được phân loại vào lớp âm.

Mặc dù đơn giản và dễ triển khai, hồi quy logistic nguyên thủy chỉ phù hợp với
các bài toán phân loại nhị phân. Để mở rộng cho bài toán phân loại đa lớp, có
thể sử dụng biến thể \textbf{Softmax Regression}, trong đó mô hình ước lượng
xác suất đầu ra theo phân phối softmax:

\begin{equation}
    P(y = j \mid \mathbf{x}) = \frac{e^{\mathbf{w}_j^\top \mathbf{x}}}{\sum_{k=1}^{K} e^{\mathbf{w}_k^\top \mathbf{x}}}
\end{equation}

Trong đó, $K$ là tổng số lớp và $\mathbf{w}_j$ là vector trọng số tương ứng với
lớp $j$. Cách tiếp cận này cho phép mô hình tính toán đồng thời xác suất cho
tất cả các lớp và lựa chọn lớp có xác suất cao nhất làm đầu ra.

Ngoài phương pháp softmax, một chiến lược mở rộng khác thường được sử dụng là
\textbf{One-vs-Rest (OvR)}. Trong chiến lược này, mô hình sẽ huấn luyện $K$ bộ
phân loại nhị phân độc lập, mỗi bộ phân biệt một lớp cụ thể với phần còn lại
của tập dữ liệu. Khi dự đoán, mẫu dữ liệu mới được đưa vào cả $K$ mô hình, và
lớp có xác suất dự đoán cao nhất sẽ được chọn làm kết quả cuối cùng.

Trong bài toán phân loại tư thế ngủ, LR được kì vọng sẽ tối ưu về mặt chi phí
tính toán và kích thước mô hình để phù hợp với các bài toán triển khai tại
biên.

\subsubsection{Máy vec-tơ hỗ trợ}

Đây là một thuật
toán học có giám sát, đặc biệt hiệu quả cho các bài toán phân loại nhị
phân với biên ranh giới rõ ràng \cite{cortes1995svm}. Ý tưởng chính của SVM là tìm kiếm một
\textbf{mặt siêu phẳng} trong không gian đặc trưng để phân
chia các điểm dữ liệu thành hai lớp sao cho biên phân cách giữa các lớp
là lớn nhất.

Trong không gian hai chiều, mặt siêu phẳng tương ứng với một đường thẳng; trong
không gian ba chiều, đó là một mặt phẳng. Trong không gian nhiều chiều hơn, nó
là một siêu mặt phẳng tổng quát. SVM chọn mặt siêu phẳng sao cho khoảng cách từ
nó đến các điểm dữ liệu gần nhất của mỗi lớp - gọi là \textbf{support vectors}
là tối đa.

Trong không gian hai chiều, khoảng cách từ một điểm có tọa độ $(x_0, y_0)$ đến
đường thẳng có phương trình:
\[
    w_1 x + w_2 y + b = 0
\]
được xác định bởi công thức:
\begin{equation}
    d = \frac{|w_1 x_0 + w_2 y_0 + b|}{\sqrt{w_1^2 + w_2^2}}
\end{equation}

Tương tự, trong không gian ba chiều, khoảng cách từ một điểm $(x_0, y_0, z_0)$
đến mặt phẳng có phương trình:
\[
    w_1 x + w_2 y + w_3 z + b = 0
\]
được tính bằng:
\begin{equation}
    d = \frac{|w_1 x_0 + w_2 y_0 + w_3 z_0 + b|}{\sqrt{w_1^2 + w_2^2 + w_3^2}}
\end{equation}

Dấu của biểu thức $(w_1 x_0 + w_2 y_0 + w_3 z_0 + b)$ cho biết điểm $(x_0, y_0,
    z_0)$ nằm về phía nào của mặt phẳng đang xét. Những điểm có giá trị dương nằm
về cùng một phía (phía dương), trong khi những điểm có giá trị âm nằm về phía
còn lại (phía âm). Các điểm nằm trên chính mặt phẳng này sẽ làm cho tử số bằng
không, tức là khoảng cách bằng 0.

Khái niệm này có thể được \textbf{tổng quát hóa lên không gian nhiều chiều}.
Giả sử ta có một điểm (vector) $\mathbf{x}_0 \in \mathbb{R}^d$ và một siêu mặt
phẳng có phương trình \cite{MLCoBan2017_SVM}:
\[
    \mathbf{w}^\top \mathbf{x} + b = 0
\]
thì khoảng cách vuông góc từ điểm $\mathbf{x}_0$ đến siêu mặt phẳng này được
cho bởi:
\begin{equation}
    d = \frac{|\mathbf{w}^\top \mathbf{x}_0 + b|}{\|\mathbf{w}\|_2}
\end{equation}

trong đó:
\[
    \|\mathbf{w}\|_2 = \sqrt{\sum_{i=1}^{d} w_i^2}
\]
với vector trọng số $\mathbf{w}$, và $d$ là số chiều của không gian.

Hình~\ref{svm} minh hoạ khái niệm mặt siêu phẳng và các máy hỗ trợ trong không
gian hai chiều.

Với mặt phẳng phân chia như trên, khoảng cách được tính là khoảng cách gần nhất
từ một điểm tới mặt phẳng đó (bất kể điểm thuộc hai lớp nào):

\[
    \text{margin} = \min_{n} \frac{y_n \left( \mathbf{w}^T \mathbf{x}_n + b \right)}{||\mathbf{w}||_2}
\]

Bài toán tối ưu trong SVM chính là bài toán tìm \(\mathbf{w}\) và \(b\) sao cho
\textit{margin} này đạt giá trị lớn nhất:

\[
    (\mathbf{w}, b) = \arg\max_{\mathbf{w},\, b}
    \left\{
    \min_{n}
    \frac{y_n \left( \mathbf{w}^T \mathbf{x}_n + b \right)}{||\mathbf{w}||_2}
    \right\}
\]

\begin{figure}[htbp]
    \centering
    \includegraphics[width=0.6\textwidth]{images/svm.png}
    \caption{Minh họa mặt siêu phẳng phân tách hai lớp trong SVM}
    \label{svm}
\end{figure}

Thêm vào đó, SVM có thể mở rộng cho các bài toán không tuyến tính thông qua
việc sử dụng các hàm kernel, chẳng hạn như \textbf{Gaussian RBF kernel} hoặc
\textbf{polynomial kernel}, giúp ánh xạ dữ liệu vào không gian mới nơi mà việc
phân tách tuyến tính trở nên khả thi.

Để mở rộng cho các bài toán phân loại đa lớp, có thể áp dụng hai kỹ
thuật phổ biến: \textbf{one-vs-one} và \textbf{one-vs-rest}, được minh
hoạ trong Hình~\ref{svm_ovso} và Hình~\ref{ovsr}.

\begin{figure}[htbp]
    \centering
    \includegraphics[width=0.6\linewidth]{images/svm_ovso.png}
    \caption{Chiến lược phân loại đa lớp bằng phương pháp One-vs-One}
    \label{svm_ovso}
\end{figure}

\begin{figure}[htbp]
    \centering
    \includegraphics[width=0.6\linewidth]{images/ovsr.png}
    \caption{Chiến lược phân loại đa lớp bằng phương pháp One-vs-Rest}
    \label{ovsr}
\end{figure}

\textbf{One-vs-One (OvO):} Trong phương pháp này, một mô hình SVM được
huấn luyện cho mỗi cặp lớp. Với $K$ lớp, tổng cộng $\frac{K(K-1)}{2}$
mô hình con được huấn luyện. Mỗi mô hình học cách phân biệt giữa hai
lớp cụ thể và bỏ qua các lớp còn lại. Trong quá trình dự đoán, một cơ
chế bỏ phiếu (voting) được sử dụng để xác định lớp cuối cùng.

\textbf{One-vs-Rest (OvR):} Phương pháp này huấn luyện một mô hình cho
mỗi lớp, trong đó mô hình học cách phân biệt giữa một lớp cụ thể và
phần còn lại. Với $K$ lớp, ta có $K$ mô hình. Trong quá trình suy luận,
mô hình đưa ra xác suất hoặc độ tin cậy, và lớp có giá trị cao nhất
sẽ được chọn.

Cả hai chiến lược OvO và OvR đều giúp mở rộng SVM từ mô hình phân loại nhị phân
thành phân loại đa lớp hiệu quả, nhưng mỗi phương pháp đều có ưu và nhược điểm
riêng về thời gian huấn luyện, độ phức tạp tính toán và hiệu năng phân loại.

\subsubsection{Rừng ngẫu nhiên}

Đây là một mô hình học có
giám sát thuộc nhóm thuật toán tổ hợp (ensemble learning),
được xây dựng dựa trên nền tảng của
\textbf{Cây quyết định (Decision Tree)} \cite{breiman2001random}.
Khác với việc sử dụng một cây quyết định duy nhất
truyền thống, rừng ngẫu nhiên xây dựng một tập hợp gồm nhiều cây
quyết định độc lập, mỗi cây học trên một phần khác nhau của dữ liệu và
không sử dụng toàn bộ tập thuộc tính. Dự đoán cuối cùng của mô hình
được xác định thông qua cơ chế biểu quyết hoặc trung bình hoá
(trong bài toán hồi quy).

Ý tưởng chính của rừng ngẫu nhiên nhằm giảm thiểu hiện tượng
\textbf{quá khớp (overfitting)} thường gặp trong câu quyết định đơn lẻ.
Khi xây dựng một cây quyết định mà không giới hạn độ sâu,
cây có xu hướng học thuộc hoàn toàn dữ liệu huấn luyện,
dẫn đến khả năng tổng quát kém trên tập kiểm thử.
RF khắc phục điều này bằng cách đưa vào hai cơ chế ngẫu nhiên chính:

\vspace{0.5em}
\noindent\textbf{1) Lấy mẫu bootstrap:} Mỗi cây được huấn luyện trên
một tập con của dữ liệu ban đầu, được chọn ngẫu nhiên có lặp lại
(bootstrap sampling). Như vậy, một phần dữ liệu được bỏ qua,
làm tăng tính đa dạng giữa các cây.

\vspace{0.5em}
\noindent\textbf{2) Lựa chọn ngẫu nhiên tập thuộc tính:} Tại mỗi nút
phân chia của cây, chỉ một tập con ngẫu nhiên của các thuộc tính được
xem xét để chọn điểm chia tốt nhất. Điều này làm giảm sự tương quan
giữa các cây trong rừng.

Ngoài ra, việc tổng hợp kết quả của nhiều cây giúp giảm phương sai, cải thiện
khả năng tổng quát hoá. Nhờ đó, RF đạt được sự ổn định kể cả với các bộ dữ liệu
có nhiễu hoặc mất cân bằng.

\subsubsection{Gradient boosting}

Đây là một phương pháp học
có giám sát cũng thuộc nhóm thuật toán tổ hợp (ensemble learning) \cite{chen2016xgboost}.
Khác với RG khi các cây được xây dựng song song và độc lập,
GB xây dựng mô hình theo từng bước lặp, mỗi cây tiếp
theo được huấn luyện để sửa lỗi còn lại từ mô hình trước đó. Cụ thể, tại mỗi
vòng lặp $t$, mô hình hiện tại $F_t(x)$ được cập nhật bằng cách cộng thêm một
cây mới $h_t(x)$ được huấn luyện để xấp xỉ gradient âm của hàm mất mát:

\begin{equation}
    F_{t+1}(x) = F_t(x) + \gamma h_t(x)
\end{equation}

Trong đó, $\gamma$ là hệ số học (learning rate), điều chỉnh mức đóng góp của
cây mới vào tổng thể mô hình.

Một trong những đặc điểm quan trọng của GB là khả năng tối ưu hoá trực tiếp một
hàm mất mát bất kỳ, chẳng hạn như hàm log-loss trong bài toán phân loại, hoặc
hàm bình phương sai số trong bài toán hồi quy. Nhờ đó, GB thường đạt độ chính
xác rất cao, đặc biệt trên các bài toán với dữ liệu có quan hệ phi tuyến và có
nhiều đặc trưng tương tác phức tạp.

Tuy nhiên, GB cũng có những hạn chế rõ rệt. Do các cây được xây dựng tuần tự
phụ thuộc lẫn nhau, GB thường mất nhiều thời gian huấn luyện hơn so với RF. Hơn
nữa, mô hình nhạy cảm với nhiễu và dữ liệu nhiễu sẽ dễ dàng bị mô hình học
theo, dẫn đến hiện tượng quá khớp hơn.

\subsubsection{Mạng nơ-ron nhân tạo}

Đây thuộc lớp
mô hình học sâu mô phỏng cấu trúc hoạt động của hệ thần kinh sinh học,
trong đó các nơ-ron nhân tạo (artificial neurons) được tổ chức thành nhiều
lớp (layers) và kết nối với nhau qua các trọng số (weights) \cite{jain1996}.

Trong nghiên cứu này, kiến trúc đơn giản nhất gồm 3 lớp: lớp đầu vào (input
layer), hai lớp ẩn (hidden layers) và lớp đầu ra (output layer), sẽ được sử
dụng.

Mỗi nơ-ron trong lớp ẩn thực hiện một tổ hợp tuyến tính giữa các đầu vào, sau
đó áp dụng một hàm kích hoạt phi tuyến như hàm ReLU (Rectified Linear Unit):

\begin{equation}
    f(x) = \max(0, x)
\end{equation}

Đầu ra của mạng được tính thông qua lan truyền tiến,
và mô hình được huấn luyện bằng cách tối thiểu hóa một hàm mất mát.

Ưu điểm chính của mạng nơ-ron là khả năng học các quan hệ phi tuyến
phức tạp và tự động trích xuất đặc trưng từ dữ liệu. Khác với các mô
hình tuyến tính như LR hoặc SVM, ANN có thể biểu diễn các ranh giới
phân lớp không tuyến tính và phù hợp với các bài toán đa tín hiệu cảm
biến, đa cấu trúc.

Tuy nhiên, trong khuôn khổ của nghiên cứu này, mục tiêu chính không chỉ là đạt
độ chính xác tối đa mà còn là đánh giá ảnh hưởng của các đặc trưng trên miền
thời gian và miền tần số đối với hiệu suất của mô hình học máy nên nhóm quyết
định chọn LR, SVM, RF, GB, NN để tiến hành thử nghiệm. Còn tương lai, nhóm sẽ
mở rộng với các thuật toán phức tạp hơn để đánh giá toàn diện hơn đối với bài
toán phân loại tư thế ngủ.

Đến đây, luận văn đã trình bày đầy đủ phương pháp xây dựng hệ thống, bao gồm cả phần cứng, phần mềm và các mô hình học máy được lựa chọn phù hợp với bài toán phân loại tư thế ngủ.
Phương pháp đề xuất được đánh giá một cách toàn diện, từ cơ sở lý thuyết đến triển khai thực tế, nhằm đảm bảo tính khả thi và hiệu quả trong ứng dụng.


\chapter{KẾT QUẢ THỰC NGHIỆM VÀ ĐÁNH GIÁ}
Trong phần này, tác giả trình bày các kết quả đạt được từ hai giai đoạn chính: 
(i) thu thập và xử lý dữ liệu cảm biến từ các tư thế ngủ khác nhau; 
(ii) huấn luyện mô hình học máy và triển khai mô hình tối ưu lên vi điều khiển nhằm đánh giá tính khả thi của giải pháp trên thiết bị biên.

\section{Hệ thống thực nghiệm}

\begin{figure}[htbp]
    \centering
    \begin{subfigure}[b]{0.45\linewidth}
        \includegraphics[width=\linewidth,height=6cm,keepaspectratio]{images/sleepdive2.png}
        \caption{}
        \label{fig:device_position}
    \end{subfigure}
    \hfill
    \begin{subfigure}[b]{0.45\linewidth}
        \includegraphics[width=\linewidth,height=6cm,keepaspectratio]{images/thucnghiem.png}
        \caption{}
        \label{fig:practical_test}
    \end{subfigure}
    \caption{Hệ thống thử nghiệm: (a) vị trí đặt thiết bị cảm biến; (b) minh hoạ thực nghiệm thực tế trong tư thế nằm.}
    \label{fig:experiment_system}
\end{figure}

Tác giả đã hoàn thiện việc lập trình firmware cho bộ vi mạch cảm biến 
tích hợp vi điều khiển NRF52840 và cảm biến gia tốc ba trục LIS3DH. 
Firmware được xây dựng sử dụng ngôn ngữ C/C++ trên nền tảng Arduino Core, 
tối ưu hóa để vận hành ổn định trong môi trường năng lượng thấp và 
hỗ trợ giao tiếp không dây chuẩn Bluetooth Low Energy (BLE).

Để đảm bảo khả năng hoạt động liên tục trong suốt một đêm ngủ 
(tối thiểu 8 giờ), hệ thống được thiết kế sử dụng pin cúc áo CR2032, 
với dòng tiêu thụ trung bình được đo đạt dưới 8 mA trong chế độ 
ghi nhận liên tục và truyền dữ liệu định kỳ.

Mã nguồn firmware bao gồm các khối chức năng chính: 
khởi tạo cảm biến, hiệu chỉnh dải đo và tần số lấy mẫu (10~Hz), 
lọc nhiễu đầu vào (bằng kỹ thuật trung bình trượt), 
đóng gói dữ liệu,
và truyền dữ liệu qua BLE đến ứng dụng Android. 
Ngoài ra, tác giả cũng tích hợp cơ chế báo hiệu bằng LED để xác nhận 
trạng thái hoạt động của thiết bị (kết nối, truyền dữ liệu, và lỗi).
Kết quả thử nghiệm thực tế cho thấy hệ thống hoạt động ổn định 
trong suốt thời gian ghi nhận dữ liệu qua đêm, không xảy ra hiện tượng 
rớt kết nối hay tràn bộ đệm dữ liệu. Toàn bộ dữ liệu thu được được đồng 
bộ theo thời gian thực tới ứng dụng di động, phục vụ cho các giai đoạn 
xử lý tín hiệu và huấn luyện mô hình học máy ở các chương sau.


Một trong các nhiệm vụ chính mà tác giả thực hiện là phát triển ứng dụng 
di động phục vụ cho quá trình thu thập, hiển thị và xử lý dữ liệu. 
Dựa trên phản hồi từ nhóm nghiên cứu, tư vấn khoa học của Thầy PGS.TS. Mai Anh Tuấn, tư vấn
y khoa của Thầy GS.TS. Dương Quý Sỹ,
ứng dụng được thiết kế với tiêu 
chí giao diện thân thiện, thao tác đơn giản và tính năng tập trung vào 
mục tiêu thử nghiệm.

Sau khi cài đặt, người dùng có thể đăng nhập hoặc đăng ký tài khoản 
thông qua giao diện như được thể hiện trong Hình~\ref{appAuth}. 
Với người dùng mới, quá trình đăng ký yêu cầu xác thực địa chỉ 
email nhằm đảm bảo bảo mật và hỗ trợ tính năng khôi phục tài khoản.
Hình ~\ref{app_cate} là giao diện khi người dùng đăng nhập thành công, 
bao gồm các tính năng: Kết nối BLE và đọc dữ liệu, chuyển người dùng, 
xem thông tin người dùng v.v.
Hình ~\ref{listble} thể hiện danh sách BLE có thể kết nối và 
dịch vụ kết nối với phần cứng đã được nhắc tới bên trên.


\begin{figure}[htbp]
    \centering
    \includegraphics[width=0.8\linewidth]{images/appAuth.png}
    \caption{Giao diện chức năng đăng ký và đăng nhập}
    \label{appAuth}
\end{figure}

\begin{figure}[h!]
    \centering
    \includegraphics[width=0.8\linewidth]{images/app_cate.png}
    \caption{Giao diện trang chủ}
    \label{app_cate}
\end{figure}


\begin{figure}[htbp]
    \centering
    \includegraphics[width=0.5\linewidth]{images/app_ble.png}
    \caption{Giao diện màn hình danh sách BLE và chi tiết các dịch vụ kết nối với phần cứng}
    \label{listble}
\end{figure}

Hình~\ref{appsleep} minh họa giao diện của các chức năng hỗ trợ người 
dùng trong quá trình sàng lọc nguy cơ ngưng thở khi ngủ và cung cấp 
thông tin về chất lượng giấc ngủ. Giao diện đầu tiên (từ trái sang) 
hiển thị mục "Hỏi – Đáp về Giấc Ngủ", nơi người dùng có thể tra cứu 
các thông tin được tổng hợp từ chuyên gia trong lĩnh vực y học giấc ngủ. 
Giao diện thứ hai trình bày tập hợp các công cụ đo lường phổ biến như 
thang điểm Epworth, STOP-BANG, chỉ số khó có thể duy trì sự tỉnh táo 
(ESS), và các bộ câu hỏi dành riêng cho trẻ em hoặc đánh giá chất lượng 
giấc ngủ theo thang điểm Pittsburgh (PSQI).
Giao diện thứ ba mô tả chi tiết một bộ câu hỏi sàng lọc nguy cơ ngưng 
thở khi ngủ (tên mục: “Tầm soát ngày – Ngưng thở”) bao gồm các câu hỏi 
tổng quát và chuyên biệt nhằm đánh giá các yếu tố liên quan đến OSA 
(Obstructive Sleep Apnea), như tần suất ngáy, triệu chứng ngủ gật ban 
ngày, gián đoạn giấc ngủ, hoặc các đặc điểm nhân trắc học có liên quan.

Giao diện thứ tư là chức năng chatbot – nơi người dùng có thể trao 
đổi trực tiếp với hệ thống trí tuệ nhân tạo được lập trình sẵn để 
phản hồi các câu hỏi về OSA. Chatbot có khả năng nhận diện từ khóa 
và cung cấp phản hồi ngắn gọn dựa trên cơ sở dữ liệu đã huấn luyện. 
Trong ví dụ minh họa, chatbot phản hồi một truy vấn liên quan đến chỉ 
số BMI và nguy cơ mắc OSA, thể hiện vai trò hỗ trợ tư vấn bước đầu 
cho người dùng nghi ngờ có hội chứng ngưng thở khi ngủ.



\begin{figure}[htbp]
    \centering
    \includegraphics[width=0.8\linewidth]{images/appsleep.png}
    \caption{Giao diện chức năng chatbot và bộ câu hỏi tầm soát}
    \label{appsleep}
\end{figure}

\begin{lstlisting}[%
  float,
  language=C,
  caption={Tập lệnh đánh giá tư thế ngủ bằng ngưỡng},
  captionpos=b,
  label={lst:nguong}
]
static Function getPositionSleep = (double x, double y, double z) {
    if ((-6.5 < y && y < 6.5)) {
      if (-7.07 < x && x < 7.07) {
        if (z > 0) {
          return 1; // ngua
        }
        if (z < 0) {
          return 4; //sap
        }
      }
      if (x > 3) return 2; //trai
      if (x < -3) return 3; //phai
  }
    return 6; // khong phai nam
};
\end{lstlisting}


Hình~\ref{appbieudo} minh họa giao diện hiển thị giá trị 
cảm biến theo thời gian thực. Phần đầu hiển thị biểu đồ ba trục 
x, y, z. Phần thứ hai là tổng thời gian theo từng tư thế ngủ 
được tính toán dựa trên tín hiệu nhận dạng. Phần cuối cùng cho biết 
tư thế hiện tại mà hệ thống đang xác định được. 
Tuy nhiên, phương pháp xác định tư thế dựa trên ngưỡng chưa có tính tổng quát cao. 
Do đó, trong các phần tiếp theo, các mô hình học máy sẽ 
được áp dụng để cải thiện độ chính xác và ổn định của hệ 
thống nhận diện tư thế. Việc cập nhật tư thế được thực hiện 
định kỳ mỗi 10 giây để tăng khả năng phản hồi theo thời gian thực.

\begin{figure}[htbp]
    \centering
    \includegraphics[width=0.3\linewidth]{images/appbieudo.png}
    \caption{Giao diện hiển thị dữ liệu gia tốc ba trục}
    \label{appbieudo}
\end{figure}

Về mặt kiến trúc lưu trữ dữ liệu, hệ thống được thiết kế phân tách 
giữa dữ liệu định tính và dữ liệu định lượng. Cụ thể, thông tin người 
dùng (tài khoản, cấu hình cá nhân), các bộ câu hỏi tầm soát 
(ví dụ: STOP-BANG, ESS), cùng với nội dung trả lời và phản hồi của 
chatbot được lưu trữ trong cơ sở dữ liệu quan hệ \textbf{PostgreSQL}. 
Cơ sở dữ liệu này hỗ trợ tính nhất quán cao và dễ dàng cho việc mở 
rộng truy vấn phức tạp trong các bài toán phân tích sau này.
Trong khi đó, dữ liệu cảm biến gia tốc được lưu trữ song song tại cơ 
sở dữ liệu phi quan hệ \textbf{MongoDB}, với định dạng BSON linh hoạt, 
phù hợp cho việc ghi nhận chuỗi thời gian lớn và truy xuất nhanh theo 
timestamp. Ngoài ra, hệ thống được mở rộng với các API cho phép trích 
xuất dữ liệu dưới dạng Excel, nhằm hỗ trợ phân tích và chia sẻ thông 
tin một cách linh hoạt.



\section{Thu thập và gắn nhãn dữ liệu}

\begin{figure}[htbp]
\centerline{\includegraphics[width=0.8\linewidth]{images/4position.png}}
\caption{Mô phỏng thực nghiệm thực tế}
\label{4Position}
\end{figure}

Trong phần này, tác giả trình bày chi tiết phương pháp thu thập dữ liệu, 
các kịch bản thực nghiệm, cũng như quy trình xử lý và trích xuất đặc trưng 
để phục vụ cho việc huấn luyện các mô hình học máy trong bài toán nhận diện tư thế ngủ.

Tổng cộng 25 tình nguyện viên đã được tuyển chọn tham gia vào quá trình 
thu thập dữ liệu, với độ tuổi dao động từ 10 đến 60, trong đó độ tuổi 
phổ biến là 24. Nhóm tình nguyện viên bao gồm cả nam và nữ, được lựa 
chọn với tiêu chí đa dạng về giới tính và độ tuổi nhằm tăng tính đại 
diện và khách quan cho bộ dữ liệu.

Trong kịch bản đầu tiên (gọi là \textbf{thu thập có giám sát}), 
mỗi tình nguyện viên được hướng dẫn gắn thiết bị cảm biến vào vùng 
xương ức (ngay dưới hõm cổ) bằng băng keo y tế hai mặt 3M, 
sau đó đăng nhập vào ứng dụng di động với tài khoản cá nhân đã đăng ký. 
Dưới sự giám sát trực tiếp của tác giả, mỗi người tham gia sẽ lần 
lượt thực hiện các tư thế ngủ cơ bản (nằm ngửa, nằm sấp, nghiêng trái, nghiêng phải) 
trong thời gian tối thiểu 5 phút cho mỗi tư thế. 
Thứ tự thay đổi tư thế được thực hiện ngẫu nhiên nhằm tránh thiên 
lệch theo trình tự. Mỗi tư thế được lặp lại ít nhất hai lần để đảm 
bảo tính lặp lại và ổn định của tín hiệu.
Sau khi xác minh rằng dữ liệu cảm biến đã được lưu trữ đầy đủ 
trên hệ thống (kiểm tra trên MongoDB và giao diện ứng dụng), quá trình 
thu thập dữ liệu từ một tình nguyện viên được xem là hoàn tất.

Bên cạnh đó, để mô phỏng điều kiện thực tế khi sử dụng thiết bị trong 
sinh hoạt ban đêm, tác giả đã tự thực hiện kịch bản thứ hai 
(\textbf{thu thập trong giấc ngủ tự nhiên}). Trong kịch bản này, 
thiết bị được gắn vào cổ trước khi đi ngủ và ghi nhận dữ liệu liên 
tục trong suốt một đêm. Song song đó, một camera cố định được lắp 
đặt phía trên giường để ghi hình toàn bộ quá trình ngủ, từ đó hỗ 
trợ gán nhãn chính xác theo thời gian thực. Dữ liệu trong giai đoạn 
này được xử lý và đồng bộ thủ công giữa tín hiệu cảm biến và video 
để loại bỏ các đoạn có chuyển động hoặc sai lệch nhãn Hình~\ref{4Position}.

Mặc dù phương pháp thu thập trong môi trường tự nhiên sát với điều kiện 
sử dụng thực tế, nhưng đòi hỏi nhiều công sức xử lý hậu kỳ và 
khó kiểm soát chất lượng dữ liệu đầu vào. 
Theo ý kiến tư vấn từ các chuyên gia trong lĩnh vực y học giấc ngủ, 
phương pháp thu thập có giám sát (phương pháp 1) vẫn được ưu tiên do 
khả năng kiểm soát tốt, đảm bảo dữ liệu cân bằng giữa các tư thế, 
đồng thời vẫn duy trì được mức độ tương thích cao với điều kiện 
thực tế khi triển khai ứng dụng theo dõi tại nhà.

Sau quá trình thu thập, bộ dữ liệu huấn luyện bao gồm tổng cộng \textbf{158.750 mẫu} 
hợp lệ sau khi đã lọc nhiễu và loại bỏ các phiên ghi nhận không đạt yêu cầu của 25 tình nguyện viên. 
Dữ liệu kiểm thử sẽ là dữ liệu trong suốt một đêm ngủ tự nhiên của tác giả 
Việc gán nhãn dữ liệu được thực hiện thủ công bằng cách đồng bộ thời gian giữa 
tín hiệu cảm biến và dữ liệu video, sau đó loại bỏ toàn bộ các đoạn có chuyển động 
hoặc tư thế không rõ ràng. Kết quả là bộ dữ liệu kiểm thử gồm \textbf{64.258 mẫu} 
đảm bảo độ chính xác cao về mặt nhãn.

Tất cả dữ liệu thu thập từ các tình nguyện viên và tác giả đều được xuất ra định 
dạng \texttt{CSV}, bao gồm thông tin thời gian (timestamp), 
giá trị cảm biến trên ba trục $x$, $y$, $z$, và nhãn tư thế tương ứng 
(nếu có). Dữ liệu này được sử dụng làm đầu vào cho quá trình trích 
xuất đặc trưng và huấn luyện mô hình học máy.


\section{Phân loại tư thế ngủ bằng học máy}

\subsection{Phân tích dữ liệu}
Tư thế ngủ ban đầu có thể được ước lượng bằng phương pháp dựa trên ngưỡng 
(threshold-based), áp dụng trực tiếp lên dữ liệu cảm biến gia tốc ba 
trục. Trong phương pháp này, các ngưỡng được thiết lập trước cho từng 
trục ($x$, $y$, $z$), và sự thay đổi tư thế được suy đoán khi giá trị 
gia tốc đo được vượt quá ngưỡng tương ứng. Kỹ thuật này có ưu điểm là 
đơn giản, chi phí tính toán thấp, và đặc biệt phù hợp với các hệ thống 
nhúng tiêu thụ năng lượng thấp. 
Mặc dù kỹ thuật dựa trên ngưỡng có ưu điểm đơn giản và phù hợp với 
các hệ thống nhúng có tài nguyên hạn chế, nó tồn tại một số hạn chế 
nhất định. Cụ thể, phương pháp này khó phát hiện các chuyển động nhẹ 
hoặc tư thế trung gian giữa các trạng thái rõ ràng. Ngoài ra, các 
ngưỡng thường cần hiệu chỉnh theo từng cá nhân do sự khác biệt về 
hình thể, kiểu vận động và vị trí gắn cảm biến.


\begin{figure}[htbp]
\centering
\includegraphics[width=0.7\linewidth]{images/threshhold.jpg} 
\caption{Phân bố dữ liệu cảm biến theo ba trục $x$, $y$, $z$ ứng với các tư thế ngủ khác nhau.}
\label{fig:axis_distribution}
\end{figure}
Hình~\ref{fig:axis_distribution} trình bày phân tích chi tiết phân bố 
tín hiệu cảm biến theo ba trục gia tốc ứng với bốn tư thế ngủ cơ bản. 
Ở trục $x$, các phân bố tương đối biệt lập, đặc biệt giữa hai tư thế 
nằm ngửa và nằm sấp, cũng như giữa nghiêng trái và nghiêng phải. 
Điều này cho thấy trục $x$ có khả năng phân biệt tư thế tốt. 
Ngược lại, trục $y$ thể hiện mức độ chồng lấn lớn giữa các tư thế, 
dẫn đến khả năng tách biệt thấp và ít giá trị trong việc xác định tư 
thế ngủ. Đối với trục $z$, có thể quan sát được sự phân tách 
rõ ràng giữa tư thế nằm nghiêng và các tư thế dọc 
(nằm ngửa và nằm sấp), chứng tỏ vai trò quan trọng của trục $z$ 
trong phân loại tư thế.



\begin{figure}[htbp]
\centering
\includegraphics[width=0.6\linewidth]{images/distribution actions.jpg} 
\caption{Phân bố số lượng mẫu trong tập huấn luyện theo từng tư thế.}
\label{fig:countActions}
\end{figure}


Hình~\ref{fig:countActions} minh họa sự phân bố số lượng mẫu trong 
tập huấn luyện theo từng tư thế. Tư thế nằm ngửa (Back) chiếm tỷ 
trọng cao nhất với khoảng 50.000 mẫu, trong khi ba tư thế còn lại 
(nghiêng trái, nghiêng phải và nằm sấp) có số lượng tương đối cân bằng, 
dao động từ 30.000 đến 35.000 mẫu. Phân bố này phản ánh xu hướng phổ 
biến của tư thế nằm ngửa trong giấc ngủ tự nhiên, đồng thời cho thấy 
tầm quan trọng lâm sàng của tư thế này, đặc biệt trong bối cảnh hội 
chứng ngưng thở khi ngủ (OSA), khi tư thế nằm ngửa có thể làm trầm 
trọng tình trạng bệnh.



\subsection{Xử lý và trích xuất đặc trưng}

Dữ liệu cảm biến thu thập được trước tiên được xử lý khử nhiễu bằng 
phương pháp hiệu chỉnh điểm gốc (differential technique), bằng cách 
lấy hiệu giữa giá trị hiện tại và giá trị tham chiếu ban đầu trên ba 
trục $x$, $y$, và $z$. Sau đó, tín hiệu được chia thành các cửa sổ 
thời gian có độ dài 2 giây, với mức chồng lấn 50\% giữa các cửa sổ 
liên tiếp nhằm tăng độ mịn của chuỗi dữ liệu đầu vào.
Chỉ những cửa sổ dữ liệu có nhãn nhất quán trong toàn bộ thời gian 
mới được giữ lại để huấn luyện mô hình. Các cửa sổ chứa nhãn không 
đồng nhất (nhiều hơn một nhãn) hoặc có biểu hiện chuyển động bất 
thường sẽ bị loại bỏ khỏi quá trình xử lý tiếp theo.

\begin{table}[htbp]
\caption{Các đặc trưng thống kê và tín hiệu được sử dụng trong phân loại tư thế ngủ}
\label{tab:features}
\begin{center}
\renewcommand{\arraystretch}{1.5}
\begin{tabular}{|l|p{9.5cm}|}
\hline
\textbf{Đặc trưng} & \textbf{Mô tả / Công thức} \\
\hline
Giá trị trung bình & $\mu_s = \frac{1}{n} \sum_{i=1}^{n} S_i$ \\
\hline
Độ lệch chuẩn & $\sigma_s = \sqrt{\frac{1}{n} \sum_{i=1}^{n} (S_i - \mu_s)^2}$ \\
\hline
Độ lệch tuyệt đối trung bình & $\text{AAD} = \frac{1}{n} \sum_{i=1}^{n} |S_i - \mu_s|$ \\
\hline
Giá trị nhỏ nhất & $\min(s) = \min(S_1, S_2, \ldots, S_n)$ \\
\hline
Giá trị lớn nhất & $\max(s) = \max(S_1, S_2, \ldots, S_n)$ \\
\hline
Hiệu số lớn nhất - nhỏ nhất & $\max(s) - \min(s)$ \\
\hline
Trung vị & $\text{Median}(s) = \text{median}(S_1, S_2, \ldots, S_n)$ \\
\hline
Độ lệch tuyệt đối trung vị & $\text{MAD} = \frac{1}{n} \sum_{i=1}^{n} |S_i - \text{Median}(s)|$ \\
\hline
Khoảng tứ phân vị & $IQR = \text{percentile}(75) - \text{percentile}(25)$ \\
\hline
Số giá trị âm & $\#(S_i < 0)$ \\
\hline
Số giá trị dương & $\#(S_i > 0)$ \\
\hline
Số giá trị lớn hơn trung bình & $\#(S_i > \mu_s)$ \\
\hline
Số đỉnh (local maxima) & Số lượng điểm cực đại cục bộ trong chuỗi tín hiệu \\
\hline
Độ lệch (Skewness) & $\frac{1}{n \sigma_s^3} \sum_{i=1}^{n} (S_i - \mu_s)^3$ \\
\hline
Độ nhọn (Kurtosis) & $\frac{1}{n \sigma_s^4} \sum_{i=1}^{n} (S_i - \mu_s)^4$ \\
\hline
Năng lượng tín hiệu & $\sum_{i=1}^{n} S_i^2$ \\
\hline
Gia tốc tổng hợp trung bình & $\frac{1}{n} \sum_{i=1}^{n} \sqrt{x_i^2 + y_i^2 + z_i^2}$ \\
\hline
Tổng độ lớn tín hiệu (SMA) & $\frac{1}{n} \sum_{i=1}^{n} (|x_i| + |y_i| + |z_i|)$ \\
\hline
\end{tabular}
\end{center}
\end{table}

\subsubsection{Đặc trưng miền thời gian (T1)}\label{AA}

Dữ liệu cảm biến gia tốc vốn là chuỗi thời gian, do đó các đặc trưng miền thời gian đóng vai trò rất quan trọng trong nhận diện tư thế ngủ. Trong nghiên cứu này, tác giả trích xuất tổng cộng 40 đặc trưng thống kê cho mỗi cửa sổ dữ liệu, trên cả ba trục $x$, $y$, $z$. Các đặc trưng bao gồm giá trị trung bình, độ lệch chuẩn, độ lệch tuyệt đối trung bình, giá trị lớn nhất, nhỏ nhất, hiệu số lớn-nhỏ nhất, trung vị, độ lệch tuyệt đối trung vị, khoảng tứ phân vị, số lượng giá trị âm/dương, số lượng giá trị lớn hơn trung bình, số đỉnh tín hiệu, độ lệch, độ nhọn, năng lượng tín hiệu, gia tốc tổng hợp và tổng độ lớn tín hiệu. Các đặc trưng này được lựa chọn dựa trên tính dễ tính toán, hiệu quả phân tách tư thế và khả năng triển khai trên vi điều khiển.

\subsubsection{Đặc trưng miền tần số (F1)}\label{AA}

Để khai thác thông tin trong miền tần số, tác giả sử dụng Biến đổi Fourier Nhanh (FFT) để chuyển đổi dữ liệu từ miền thời gian sang miền tần số. Từ các cửa sổ tín hiệu sau biến đổi, 29 đặc trưng thống kê được tính toán, bao gồm các đặc trưng tương tự như trong miền thời gian: trung bình, độ lệch chuẩn, độ lệch tuyệt đối, giá trị cực đại – cực tiểu, trung vị, khoảng tứ phân vị, số đỉnh, độ lệch, độ nhọn, năng lượng tín hiệu,... Ngoài ra, hai đặc trưng kết hợp là gia tốc tổng hợp trung bình và tổng độ lớn tín hiệu (SMA) cũng được duy trì trong miền tần số để phục vụ so sánh với miền thời gian.

Việc sử dụng đồng thời các đặc trưng từ cả hai miền thời gian và tần số giúp tăng khả năng mô tả đặc trưng cho mô hình học máy, từ đó nâng cao hiệu quả phân loại tư thế ngủ trong các điều kiện khác nhau.

\subsection{Kịch bản kiểm thử và lựa chọn tính năng}

Lựa chọn đặc trưng là một bước quan trọng trong quá trình xây dựng mô hình học máy, giúp giảm chiều dữ liệu, cải thiện hiệu quả huấn luyện, rút ngắn thời gian tính toán và hạn chế hiện tượng quá khớp (overfitting). Nguyên lý chung là các đặc trưng hiệu quả phải có mối tương quan cao với biến mục tiêu (tư thế ngủ), đồng thời có mức tương quan thấp với nhau nhằm tránh dư thừa thông tin.

\textbf{Thứ nhất}, phân tích ma trận tương quan Pearson (Hình~\ref{fig:correlation}) đã cho thấy một số cặp đặc trưng có mức tương quan rất cao, điển hình như $x_{\mathrm{std}}$ và $x_{\mathrm{aad}}$ ($r = 0.98$), hay $y_{\mathrm{std}}$ và $y_{\mathrm{aad}}$ ($r = 0.68$). Điều này gợi ý rằng có thể loại bỏ một phần các đặc trưng trùng lặp nhằm giảm độ phức tạp mô hình mà vẫn giữ được thông tin cốt lõi.

\begin{figure}[htbp]
\centering
\includegraphics[width=1\linewidth]{images/correlation.png} 
\caption{Ma trận tương quan giữa các đặc trưng trích xuất. Cường độ màu thể hiện hệ số tương quan Pearson. Màu đỏ là tương quan dương mạnh, xanh là tương quan âm mạnh, xám là không tương quan.}
\label{fig:correlation}
\end{figure}

\textbf{Thứ hai}, kết quả phân tích SHAP (SHapley Additive exPlanations) ở Hình~\ref{fig:shap} chỉ ra rằng một số đặc trưng – đặc biệt là các đặc trưng miền thời gian trên trục $z$ như trung bình, năng lượng, trung vị – có ảnh hưởng vượt trội đến dự đoán của mô hình. Do đó, việc ưu tiên các đặc trưng này trong kịch bản triển khai nhẹ (TinyML) là hoàn toàn hợp lý về mặt kỹ thuật.

\begin{figure}[htbp]
\centering
\includegraphics[width=0.8\linewidth]{images/shap20.jpg} 
\caption{Phân tích giá trị SHAP nhằm xác định tầm quan trọng của các đặc trưng trong mô hình phân loại tư thế ngủ. Các đặc trưng từ trục $z$ chiếm ưu thế về mức ảnh hưởng đến đầu ra mô hình.}
\label{fig:shap}
\end{figure}

\textbf{Thứ ba}, để hệ thống hóa việc đánh giá vai trò của từng nhóm đặc trưng và cấu hình mô hình, tác giả đã xây dựng tám kịch bản thực nghiệm được trình bày trong Bảng~\ref{tab:scenarios}. Các kịch bản được thiết kế nhằm phản ánh đầy đủ các yếu tố cần đánh giá như loại đặc trưng (miền thời gian, tần số), mức độ tương quan, độ dài cửa sổ tín hiệu và bộ đặc trưng tối ưu hoá bằng SHAP.

\begin{table}[htbp]
\caption{Các kịch bản lựa chọn và sử dụng đặc trưng trong nghiên cứu}
\label{tab:scenarios}
\begin{center}
\renewcommand{\arraystretch}{1.2}
\begin{tabular}{|c|p{6.1cm}|}
\hline
\textbf{Kịch bản} & \textbf{Mô tả} \\
\hline
1 & Sử dụng toàn bộ đặc trưng để đánh giá ảnh hưởng tổng thể đến mô hình. \\
\hline
2 & Áp dụng toàn bộ đặc trưng trong miền thời gian. \\
\hline
3 & Áp dụng toàn bộ đặc trưng trong miền tần số. \\
\hline
4 & Sử dụng đặc trưng miền thời gian, loại bỏ các đặc trưng có tương quan $>$ 95\%. \\
\hline
5 & Chọn ra 11 đặc trưng quan trọng nhất theo giá trị SHAP. \\
\hline
6 & Dùng cửa sổ 3 giây (50\% overlap), với 11 đặc trưng SHAP. \\
\hline
7 & Dùng cửa sổ 1 giây (50\% overlap), với 11 đặc trưng SHAP. \\
\hline
8 & Dùng cửa sổ 2 giây (25\% overlap), với 11 đặc trưng SHAP. \\
\hline
\end{tabular}
\end{center}
\end{table}

Đối với các kịch bản 5 đến 8, việc lựa chọn 11 đặc trưng được thực hiện bằng cách huấn luyện mô hình Random Forest trên toàn bộ tập đặc trưng, sau đó tính giá trị SHAP trung bình cho từng đặc trưng và chọn ra nhóm có ảnh hưởng cao nhất. Việc rút gọn đặc trưng này giúp mô hình nhẹ hơn, nhanh hơn, phù hợp với môi trường nhúng có giới hạn về bộ nhớ và tính toán.

Các kịch bản 6 đến 8 thay đổi độ dài cửa sổ trượt (1 giây, 2 giây, 3 giây) và mức chồng lấn nhằm khảo sát tác động của kích thước đoạn tín hiệu đến hiệu quả mô hình, đồng thời phản ánh điều kiện sử dụng thực tế trong các thiết bị đeo (wearables) hoặc hệ thống biên (edge-AI).

Với thiết kế kịch bản như trên, luận văn không chỉ đánh giá hiệu quả mô hình theo nhiều hướng khác nhau, mà còn hướng tới việc xác định cấu hình tối ưu giữa độ chính xác, chi phí tính toán và khả năng triển khai thực tiễn.

\subsection{Huấn luyện mô hình}

Dựa trên các phương pháp phân loại được đề cập ở các phần trước — 
bao gồm phương pháp ngưỡng, học máy và học sâu — 
tác giả đã lựa chọn một tập hợp đại diện các mô hình để 
tiến hành đánh giá hiệu quả trong bài toán nhận diện tư thế ngủ 
từ dữ liệu cảm biến gia tốc. Cụ thể, bốn mô hình học máy truyền 
thống được lựa chọn từ thư viện \texttt{scikit-learn} gồm: 
\textbf{Random Forest (RF)}, \textbf{Logistic Regression (LR)}, 
\textbf{Support Vector Machine (SVM)}, và \textbf{Gradient Boosting (GB)}. 
Đây đều là các mô hình đã được chứng minh hiệu quả trong việc xử lý dữ 
liệu cảm biến có cấu trúc, đặc biệt trong các bài toán phân loại đa lớp.

Để đảm bảo tính công bằng trong so sánh và khả năng triển khai thực tế trên vi điều khiển, các siêu tham số (hyperparameters) của từng mô hình được lựa chọn dựa trên kinh nghiệm thực tiễn trong các công trình trước và quá trình tinh chỉnh sơ bộ nhằm đạt được sự cân bằng giữa độ chính xác và độ phức tạp tính toán. Chi tiết tham số của từng mô hình được trình bày trong Bảng~\ref{tab:models}.

\begin{table}[htbp]
\caption{Các mô hình học máy và siêu tham số sử dụng trong nghiên cứu}
\label{tab:models}
\centering
\renewcommand{\arraystretch}{1.2}
\begin{tabular}{|l|p{9cm}|}
\hline
\textbf{Mô hình} & \textbf{Tham số cấu hình} \\
\hline
\textbf{Random Forest (RF)} & 
Số cây quyết định: 50; \newline
Độ sâu tối đa: 5; \newline
Số đặc trưng được xét tại mỗi nút: \texttt{log2} \\
\hline
\textbf{Logistic Regression (LR)} & 
Chiến lược đa lớp: \texttt{one-vs-rest}; \newline
Số vòng lặp tối đa: 50; \newline
Hàm tối ưu: \texttt{lbfgs} \\
\hline
\textbf{Support Vector Machine (SVM)} & 
Hàm kernel: \texttt{sigmoid}; \newline
Tham số điều chuẩn $C = 2$; \newline
Chiến lược đa lớp: \texttt{one-vs-rest} \\
\hline
\textbf{Gradient Boosting (GB)} & 
Tốc độ học: 0.01; \newline
Số lượng cây tăng cường: 50; \newline
Độ sâu tối đa: 3; \newline
Số đặc trưng được chọn: \texttt{log2} \\
\hline
\textbf{Mạng nơ-ron (Neural Network, Keras)} & 
Cấu trúc: [50, 15, 4]; \newline
Hàm kích hoạt: \texttt{ReLU}, \texttt{ReLU}, \texttt{Softmax}; \newline
Thuật toán tối ưu: \texttt{RMSprop}; \newline
Hàm mất mát: Mean Squared Error (MSE); \newline
Epochs: 30; Batch size: 1 \\
\hline
\end{tabular}
\end{table}

Ngoài các mô hình học máy, một mạng nơ-ron nhân tạo tuyến tính đơn giản (feedforward neural network) được xây dựng bằng thư viện \texttt{TensorFlow/Keras} để đại diện cho phương pháp học sâu. Mạng bao gồm hai lớp ẩn với số lượng nơ-ron lần lượt là 50 và 15, theo sau là một lớp đầu ra sử dụng hàm kích hoạt \texttt{Softmax} cho bài toán phân loại đa lớp. Các lớp ẩn sử dụng hàm kích hoạt \texttt{ReLU} nhằm mô hình hóa các quan hệ phi tuyến hiệu quả hơn. Mô hình này được tối ưu bằng thuật toán \texttt{RMSprop} và huấn luyện trên dữ liệu thô (raw accelerometer data) mà không cần trích xuất đặc trưng, nhằm kiểm tra khả năng triển khai trực tiếp trong môi trường thiết bị nhúng hạn chế về tài nguyên.

Việc lựa chọn kết hợp các mô hình với mức độ phức tạp khác nhau cho phép đánh giá toàn diện về 
hiệu quả phân loại trong các điều kiện thực tế. 
Từ các mô hình cây đơn giản và dễ diễn giải, đến các mô hình mạnh 
hơn như Gradient Boosting hoặc mạng nơ-ron – nghiên cứu nhằm tìm r
a giải pháp cân bằng tối ưu giữa độ chính xác, kích thước mô hình, 
tốc độ suy luận (inference latency) và mức sử dụng bộ nhớ, 
phục vụ cho các ứng dụng thực tiễn như hệ thống AI biên (Edge-AI) 
hoặc thiết bị đeo thông minh.
\subsection{Đánh giá kết quả}
Để đánh giá hiệu quả của các mô hình học máy và tác động của lựa 
chọn đặc trưng đầu vào, tám kịch bản thực nghiệm đã được thiết kế 
như trình bày ở các phần trước. Các kịch bản này không chỉ cho phép 
phân tích ảnh hưởng của đặc trưng, cửa sổ tín hiệu và trục cảm biến, 
mà còn hướng đến tối ưu hóa trọng số mô hình phục vụ triển khai trên 
thiết bị nhúng.


\begin{table}[htbp]
\caption{Độ chính xác phân loại của các mô hình trong 8 kịch bản}
\label{tab:accuracy}
\centering
\renewcommand{\arraystretch}{1.1}
\scriptsize
\begin{tabular}{|l|c|c|c|c|c|c|c|c|}
\hline
\textbf{Mô hình} & S1 & S2 & S3 & S4 & S5 & S6 & S7 & S8 \\
\hline
LR  & 0.970 & 0.970 & 0.368 & 0.970 & 0.987 & 0.990 & 0.990 & 0.987 \\
RF  & 0.995 & 0.996 & 0.426 & 0.994 & 0.993 & 0.993 & 0.993 & 0.991 \\
SVM & 0.995 & 0.985 & 0.280 & 0.991 & 0.989 & 0.982 & 0.982 & 0.870 \\
GB  & 0.996 & 0.996 & 0.439 & 0.995 & 0.995 & 0.996 & 0.996 & 0.996 \\
NN  &  0.920 &  --   &  --   &  --   &  --   &  --   &  --   & -- \\
\hline
\end{tabular}
\end{table}

Kết quả trong Bảng~\ref{tab:accuracy} cho thấy mô hình 
\textbf{Gradient Boosting (GB)} đạt độ chính xác cao nhất, 
lên đến 0.996 trong hầu hết các kịch bản (S1, S2, S6, S7, S8). 
Mô hình \textbf{Random Forest (RF)} cũng thể hiện hiệu năng ấn tượng 
với độ chính xác trên 0.99, ngoại trừ ở kịch bản S3. 
Tất cả mô hình đều có sự sụt giảm rõ rệt về hiệu suất trong kịch 
bản S3 — nơi chỉ sử dụng đặc trưng miền tần số — 
với độ chính xác của LR giảm còn 0.368 và SVM giảm xuống 0.280. 
Kết quả này phù hợp với phân tích SHAP (Hình~\ref{fig:shap}) 
khi cho thấy các đặc trưng quan trọng nhất chủ yếu đến từ miền 
thời gian, đặc biệt là trục $z$.





\begin{figure}[htbp]
    \centering
    \includegraphics[width=\linewidth]{images/matrix (2).png}
    \caption{Ma trận nhầm lẫn (confusion matrix) của năm mô hình 
    phân loại trong kịch bản S1. GB và RF cho kết quả chính xác 
    cao nhất. Mô hình NN được huấn luyện trực tiếp trên dữ liệu thô.}
    \label{fig:cm_all_models}
\end{figure}

Kết quả này nhấn mạnh vai trò then chốt của lựa chọn đặc trưng đầu vào đối với hiệu quả mô hình. Các kịch bản sử dụng đặc trưng đã rút gọn theo SHAP (như S5–S8) vừa đạt độ chính xác cao, vừa giảm số chiều dữ liệu đầu vào, qua đó hỗ trợ triển khai mô hình nhẹ trong môi trường nhúng. Ngược lại, các kịch bản thiếu chọn lọc như S3 dẫn đến hiệu suất kém.

\begin{table}[htbp]
\caption{Kích thước mô hình (KB) trong 8 kịch bản}
\label{tab:modelsize}
\centering
\renewcommand{\arraystretch}{1.1}
\scriptsize
\begin{tabular}{|l|c|c|c|c|c|c|c|c|}
\hline
\textbf{Mô hình} & S1 & S2 & S3 & S4 & S5 & S6 & S7 & S8 \\
\hline
LR  & 4   & 2   & 2    & 2   & 2   & 2   & 2   & 2   \\
RF  & 187 & 151 & 291  & 176 & 89  & 89  & 141 & 103 \\
SVM & 315 & 232 & 3051 & 294 & 150 & 92  & 183 & 274 \\
GB  & 605 & 602 & 615  & 603 & 587 & 587 & 587 & 589 \\
NN  & 55  & --  & --   & --  & --  & --  & --  & --   \\
\hline
\end{tabular}
\end{table}

Bảng~\ref{tab:modelsize} trình bày kích thước mô hình (tính theo kilobyte) trong từng kịch bản. GB và RF có dung lượng lớn nhất, đặc biệt GB luôn trên 580~KB. RF biến động từ 89~KB đến 291~KB tùy theo số đặc trưng sử dụng. Ngược lại, LR duy trì kích thước rất nhỏ, chỉ từ 2–4~KB ở tất cả kịch bản, phù hợp với hệ thống vi điều khiển có bộ nhớ hạn chế. SVM có kích thước trung bình, nhưng ở kịch bản S3 (3051~KB), dung lượng tăng vọt do số chiều đầu vào lớn. Mô hình NN mặc dù chỉ được đánh giá trong S1 nhưng đạt độ chính xác cao với kích thước chỉ 55~KB — một lựa chọn triển vọng cho các ứng dụng Edge AI.

Nhìn chung, kết quả cho thấy cần đánh đổi giữa hiệu quả dự đoán và kích thước mô hình khi triển khai thực tế trên thiết bị nhúng. Các mô hình ensemble như RF và GB cho kết quả xuất sắc, nhưng LR và NN lại là lựa chọn tối ưu trong môi trường hạn chế tài nguyên.

\section{Triển khai trên chip}

\section{Triển khai trên vi điều khiển nhúng}

Sau khi hoàn tất quá trình huấn luyện và đánh giá trên máy tính, bước tiếp theo của nghiên cứu là kiểm chứng khả năng triển khai mô hình trong môi trường thực tế sử dụng vi điều khiển nhúng. Mục tiêu khoa học ở giai đoạn này không chỉ dừng ở việc “chạy được” mô hình trên phần cứng hạn chế, mà còn nhằm làm sáng tỏ mối quan hệ đánh đổi giữa hiệu năng thuật toán và giới hạn tài nguyên của hệ thống nhúng.  

Cụ thể, nghiên cứu tiến hành triển khai song song hai mô hình: mạng nơ-ron nông (Neural Network – NN) với khả năng biểu diễn phi tuyến mạnh mẽ, và hồi quy logistic (Logistic Regression – LR) với cấu trúc tuyến tính cực kỳ gọn nhẹ. NN được kỳ vọng duy trì độ chính xác cao trong phân loại tư thế ngủ, trong khi LR đóng vai trò như một đối chứng quan trọng, minh chứng cho khả năng đạt được sự cân bằng tối ưu giữa độ chính xác vừa đủ và mức tiêu thụ tài nguyên tối thiểu.  

Qua nhiều lần triển khai thực nghiệm trực tiếp trên vi điều khiển, tác giả rút ra kết luận quan trọng: việc giảm số lượng mẫu huấn luyện và rút gọn tập đặc trưng giúp mô hình suy luận nhẹ hơn, kích thước tệp và mức sử dụng bộ nhớ giảm đáng kể, trong khi độ chính xác chỉ suy giảm ở mức rất nhỏ và hoàn toàn chấp nhận được.  

Điều này cho thấy trong bối cảnh phần cứng hạn chế, giá trị khoa học không nằm ở việc đạt độ chính xác tuyệt đối trong điều kiện lý tưởng, mà ở khả năng thiết kế một mô hình “đủ tốt” nhưng có thể vận hành bền vững trên chip. Chính sự đánh đổi này khẳng định nguyên lý cốt lõi của TinyML: hy sinh một phần nhỏ về độ chính xác để đổi lấy tính khả thi, hiệu quả năng lượng và độ tin cậy trong môi trường thực.  

Kết quả cũng cho thấy sự song hành giữa hai mô hình được lựa chọn. Mạng nơ-ron (NN) duy trì độ chính xác cao nhưng tiêu tốn nhiều tài nguyên, trong khi hồi quy logistic (LR) có dung lượng siêu nhỏ, tốc độ suy luận nhanh, và vẫn giữ mức chính xác tiệm cận. Việc triển khai song song cả NN và LR trên chip vì vậy không chỉ mang ý nghĩa kiểm chứng kỹ thuật, mà còn cung cấp bằng chứng khoa học cho thấy ranh giới cân bằng giữa “độ chính xác tối đa” và “khả năng ứng dụng thực tế” trong hệ thống nhúng y sinh.
\subsection{Lựa chọn mô hình và nền tảng triển khai}

Trong số các mô hình khảo sát, mô hình mạng nơ-ron đạt được sự cân bằng tốt giữa độ chính xác (0.92 trong kịch bản S1) và dung lượng bộ nhớ (khoảng 55~KB). Khác với các mô hình ensemble như Gradient Boosting (GB) hoặc Random Forest (RF) tuy có độ chính xác cao hơn nhưng dung lượng lớn và tính phức tạp tính toán, mạng nơ-ron được thiết kế tối giản và hướng đến triển khai trong các môi trường vi điều khiển tiêu thụ năng lượng thấp.

Vi điều khiển được lựa chọn là \textbf{Arduino Nano 33 BLE Sense}, sử dụng chip \texttt{nRF52840} (ARM Cortex-M4F, 64~MHz), 1~MB flash và 256~KB RAM. Bo mạch này cũng tích hợp sẵn cảm biến gia tốc, rất phù hợp để xây dựng hệ thống nhận diện tư thế ngủ hoàn chỉnh và hoạt động độc lập.

\subsection{Quy trình triển khai mô hình}

Mô hình mạng nơ-ron được huấn luyện bằng TensorFlow/Keras với hai lớp ẩn (50 và 15 nơ-ron), sử dụng hàm kích hoạt \texttt{ReLU} và lớp đầu ra \texttt{Softmax} để phân loại tư thế ngủ thành 4 lớp. Sau huấn luyện, mô hình được chuyển sang định dạng \texttt{TensorFlow Lite (.tflite)} để phục vụ cho suy luận trên thiết bị nhúng.

Vì mô hình đã được thiết kế tối giản từ đầu nên dung lượng ở định dạng TFLite chỉ khoảng 55~KB, không cần thêm bước lượng tử hóa để giảm kích thước. Tuy vậy, để đảm bảo khả năng thực thi ổn định trên nền tảng không có bộ xử lý dấu phẩy động (FPU), mô hình vẫn được chuyển về định dạng số nguyên 8-bit (\texttt{int8}) thông qua kỹ thuật lượng tử hóa sau huấn luyện (post-training quantization). Quá trình này không làm giảm kích thước mô hình đáng kể, nhưng giúp cải thiện tốc độ suy luận và giảm mức tiêu thụ điện năng.

Sau đó, mô hình được tích hợp vào mã nguồn Arduino bằng cách chuyển đổi file `.tflite` thành mảng nhị phân C (`const unsigned char model[]`) và sử dụng thư viện \texttt{TensorFlow Lite for Microcontrollers (TFLM)} để khởi tạo trình suy luận (\texttt{MicroInterpreter}). Tín hiệu đầu vào từ cảm biến gia tốc được xử lý theo đúng chuẩn hóa đầu vào như khi huấn luyện và đưa vào mô hình để thực hiện phân loại tư thế theo thời gian thực.

\subsection{Đánh giá hiệu suất và tài nguyên}

Quá trình triển khai và kiểm thử cho kết quả như sau:

\begin{itemize}
    \item \textbf{Dung lượng mô hình (flash):} khoảng 55~KB;
    \item \textbf{Mức sử dụng RAM trong lúc suy luận:} khoảng 19.6~KB;
    \item \textbf{Thời gian suy luận trung bình:} khoảng 17~ms/mẫu;
    \item \textbf{Tốc độ lấy mẫu cảm biến:} 10~Hz (mỗi 100~ms);
    \item \textbf{Dòng tiêu thụ trung bình:} 6.5~mA (hoạt động liên tục, BLE tắt).
\end{itemize}

Kết quả cho thấy hệ thống đáp ứng tốt yêu cầu suy luận thời gian thực, bộ nhớ sử dụng nằm trong giới hạn an toàn của thiết bị, và có thể hoạt động liên tục nhiều giờ bằng pin cúc áo CR2032.





% \chapter{Nghiên cứu liên quan}
% Hiện tại chưa có nhiều nghiên cứu về bài toán sinh dữ liệu kiểm thử tự động áp dụng cho ứng dụng Web Typescript. Những nghiên cứu liên quan nhất với chủ đề của luận văn có thể kể đến là \cite{evomaster,artermis,symjs,js_based_statement_coverage, jseft}

Andrea Arcuri đề xuất một phương pháp được triển khai trong công cụ EvoMaster \cite{evomaster} để tạo dữ liệu thử nghiệm tự động cho các dịch vụ Web RESTful trong Java. Ý tưởng của phương pháp này là thu thập thông tin hộp trắng từ các dịch vụ Web đang chạy (ví dụ: độ phủ câu lệnh và khoảng cách nhánh), sau đó tạo các ca kiểm thử tiếp theo bằng cách sử dụng một thuật toán tiến hóa. Mặc dù công cụ này có thể phát hiện ra một số lỗi trong các dự án thử nghiệm, nhưng nó có độ phủ thấp do một số vấn đề liên quan đến ràng buộc chuỗi, quyền truy cập vào tài nguyên từ xa.

Artermis \cite {artermis} và SymJS \cite {symjs} là hai framework có thể tạo dữ liệu kiểm thử tự động theo hướng phản hồi (\textit{feedback-directed automated
test data generation}) cho các chương trình của ứng dụng Web phía máy khách trong Javascript. Công cụ đầu tiên Artermis sử dụng kỹ thuật kiểm thử ngẫu nhiên theo hướng phản hồi được đề xuất bởi Pacheco và các cộng sự. \cite{feedback_directed_random_testing}. Kỹ thuật này dựa vào thông tin thực thi của chương trình để định hướng quá trình sinh dữ liệu kiểm thử tạo ra các đầu vào hiệu quả làm tăng độ phủ. Công cụ thứ hai SymJs triển khai một máy ảo thực thi tượng trưng (\textit{symbolic virtual machine}) và xây dựng trình điều khiển tự động với \textit{dynamic taint analysis} \cite{taint_analysis_1, taint_analysis_2} để thu thập thông tin phản hồi để định hướng tạo các đầu vào tiếp theo. Một công cụ liên quan khác là JSEFT \cite{jseft} tạo các ca kiểm thử dựa trên các thuật toán để tối đa hóa bao phủ hàm và giảm thiểu trạng thái của hàm.

% Artermis \cite{artermis} and SymJS \cite{symjs} are two frameworks for feedback-directed automated test data generation for client-side Web programs in Javascript, which language Typescript is based on. The first framework Artermis uses feedback-directed random testing
% technique by Pacheco et al. \cite{feedback_directed_random_testing} to guide test data generation achieving inputs that increase coverage. The second framework SymJs implements a symbolic virtual machine and automatic driver construction with dynamic taint analysis \cite{taint_analysis_1, taint_analysis_2} to collect feedback information for directing the next generated inputs. Another related tool is JSEFT \cite{jseft} which generates test cases based on algorithms to maximize function coverage and minimize function states.

Witthaya Luanghirun và cộng sự \cite {js_based_statement_coverage} giới thiệu một phương pháp tạo các ca kiểm thử cho các hàm Javascript dựa trên độ phủ câu lệnh. Phương pháp này biến đổi các hàm Javascript thành biểu đồ luồng điều khiển, tìm một tập hợp tối thiểu các đường dẫn thực thi, sau đó tạo vectơ đầu vào từ các biểu thức điều kiện của đường dẫn. Phương thức này có thể xử lý các tham số chuỗi, số, boolean của các hàm độc lập.
% Witthaya Luanghirun et al. \cite{js_based_statement_coverage} introduce a method for generating test cases for Javascript functions based on statement coverage. This method transforms Javascript functions into control flow graph, finds a minimal set of execution paths, and then generates input vectors from path predicate expressions. This method can treat string, number, boolean parameters of independent functions.
\chapter{Kết luận}
\chapter*{KẾT LUẬN}

Luận văn đã xây dựng được một nền tảng phần cứng hoàn chỉnh phục vụ nghiên cứu,
trong đó vi điều khiển nRF52840 được tích hợp cùng cảm biến gia tốc và các khối
chức năng cần thiết nhằm bảo đảm khả năng thu nhận và xử lý tín hiệu ổn định.
Song song với đó, hệ thống thu thập - truyền dẫn - lưu trữ dữ liệu dựa trên
giao tiếp BLE cũng đã được phát triển, bao gồm ứng dụng di động, máy chủ và cơ
sở dữ liệu. Cấu trúc này không chỉ đáp ứng yêu cầu của bài toán tư thế ngủ mà
còn cho phép mở rộng linh hoạt đối với các loại cảm biến khác trong tương lai.

Trên cơ sở dữ liệu thu được, luận văn đã khảo sát và đề xuất các mô hình học
máy phù hợp cho phân loại tư thế ngủ, bao gồm Logistic Regression, Random
Forest, Gradient Boosting, Support Vector Machine và mạng nơ-ron nhân tạo. Các
mô hình này cho thấy hiệu năng ấn tượng, với độ chính xác tối đa đạt 99,8\%,
chứng minh tính khả thi của bài toán khi triển khai trên thiết bị có tài nguyên
hạn chế.

Để đánh giá khả năng vận hành thực tế tại biên, luận văn đã thử nghiệm triển khai một mô hình truyền thống (Logistic Regression) và một mô hình học sâu (ANN) trực tiếp trên phần cứng nRF52840.
Kết quả cho thấy thời gian suy luận đạt khoảng  $501\,\mu\text{s}$
, đáp ứng tốt yêu cầu xử lý theo thời gian thực của thiết bị đeo.

Cuối cùng, luận văn cũng mở rộng hướng nghiên cứu sang bài toán nhận biết trạng
thái “nằm” hoặc “không nằm”, tạo tiền đề cho các ứng dụng giám sát giấc ngủ đa
mục tiêu và hỗ trợ sàng lọc sớm các rối loạn hô hấp khi ngủ.

Đến đây, toàn bộ nội dung luận văn thạc sĩ \textit{NGHIÊN CỨU, PHÁT TRIỂN MÔ HÌNH HỌC
  MÁY TẠI BIÊN NHẰM PHÂN LOẠI TƯ THẾ
  NGỦ} đã được trình bày.

Em xin cảm ơn các Thầy/Cô đã đọc và góp ý!

\chapter*{PHỤ LỤC}

Toàn bộ mã nguồn và dữ liệu được công bố tại:
\href{https://github.com/tranhuunam00/Master_2024}{Github Repository –
  Master\_2024}.

Mọi thông tin thắc mắc có liên quan vui lòng liên hệ tác giả thông qua email:
tranhuunam23022000@gmail.com



% %-	Danh mục TL tham khảo
%-	Phụ lục (nếu có)
\textbf{Tiếng Anh}
\begin{thebibliography}{99}
\bibitem{ref-tiobe}
TIOBE 2020
\url{https://www.tiobe.com/tiobe-index/}, [Accessed 6 June 2020]
\bibitem{ref-mocha}
Mocha 2020
\url{https://mochajs.org/}, [Accessed 6 June 2020]
\bibitem{ref-jest}
Jest 2020
\url{https://jestjs.io/}, [Accessed 6 June 2020]
\bibitem{ref-jasmine}
Jasmine 2020
\url{https://jasmine.github.io/}, [Accessed 6 June 2020]
\bibitem{ref-dotup}
TypeScript Test Generator 2020
\url{https://marketplace.visualstudio.com/items?itemName=dotup.dotup-vscode-test-generator}, [Accessed 6 June 2020]

\bibitem{ref-jsTestGen}
JS Test Gen 2020
\url{https://js-test-gen.github.io/}, [Accessed 6 June 2020]

\bibitem{ref-factory}
Factory 2020
\url{https://www.npmjs.com/package/factory.ts}, [Accessed 6 June 2020]

\bibitem{ref-faker}
Faker 2020
\url{https://www.npmjs.com/package/faker}, [Accessed 6 June 2020]

\bibitem{ref-chance}
Chance 2020
\url{https://www.npmjs.com/package/chance}, [Accessed 6 June 2020]

\bibitem{ref-casual}
Casual 2020
\url{https://www.npmjs.com/package/casual}, [Accessed 6 June 2020]

\bibitem{ref-randexp}
Randexp 2020
\url{https://www.npmjs.com/package/randexp}, [Accessed 6 June 2020]

\bibitem{ref-sonarlint}
SonarLint 2020
\url{https://plugins.jetbrains.com/plugin/7973-sonarlint}, [Accessed 6 June 2020]

\bibitem{ref-tslint}
TSLint 2020
\url{https://palantir.github.io/tslint/}, [Accessed 6 June 2020]

\bibitem{ref-eslint}
ESLint 2020
\url{https://eslint.org/}, [Accessed 6 June 2020]

\bibitem{ref-instanbul}
Instanbul 2020
\url{https://istanbul.js.org/}, [Accessed 6 June 2020]
\bibitem{ref-jstool}
Witthaya Luanghirun, Taratip Suwannasart. 2016. Test Cases Generation Tool for JavaScript Based on Statement Coverage Criteria. In Proceedings of the International MultiConference of Engineers and Computer Scientists 2016 Vol I. IMECS 2016, March 16 - 18, 2016, Hong Kong
\bibitem{ref-symjs}
Guodong Li, Esben Andreasen, and Indradeep Ghosh. 2014. SymJS: automatic symbolic testing of JavaScript web applications. In Proceedings of the 22nd ACM SIGSOFT International Symposium on Foundations of Software Engineering (FSE 2014). Association for Computing Machinery, New York, NY, USA, 449–459.
% \bibitem{ref_rank1}
% Emerging Economies 2020
% \url{https://www.timeshighereducation.com/world-university-rankings/2020/emerging-economies-university-rankings}, 
\bibitem{unit-testing}
Adam D., \textit{The hitchhiker's guide to the galaxt}, San Val, 1995
\bibitem{Lee03}
Copeland Lee, \textit{A practitioner's guide to software test design}, Artech House, Inc., Norwood, MA, USA, 2003
\bibitem{ref-branch-coverage}
Karl J. Ottenstein. An algorithmic approach to the detection and prevention of plagiarism. \textit{ACM SIGSCE
Bulletin}, 8(4):30–41, 1976.
\bibitem{ref-cft4cunit}
Duc-Anh Nguyen and Pham Ngoc Hung. 2017. A Test Data Generation Method for C/C++ Projects. In Proceedings of the Eighth International Symposium on Information and Communication Technology (SoICT 2017). Association for Computing Machinery, New York, NY, USA, 431–438.
\bibitem{check3}
Sam Grier. A tool that detects plagiarism in Pascal programs. \textit{ACM SIGSCE Bulletin (Proc. of 12th
SIGSCE Technical Symp.)}, 13(1):15–20, February 1981.
\bibitem{check4}
K. K. Verco and M. J. Wise. Software for detecting suspected plagiarism: Comparing structure and
attribute-counting systems. In John Rosenberg, editor, \textit{Proc. of 1st Australian Conference on Computer
Science Education}, Sydney, July 1996.
\bibitem{yap3}
Michael J. Wise. Detection of similarities in student programs: YAP’ing may be preferable to Plague’ing.
\textit{ACM SIGSCE Bulletin (Proc. of 23rd SIGCSE Technical Symp.)}, 24(1):268–271, March 1992.
\bibitem{moss}
 Alex Aiken. MOSS (Measure Of Software Similarity) plagiarism detection system.
https://theory.stanford.edu/~aiken/moss/ (as of April 2000) and personal communication, 1998. University of Berkeley, CA.
\bibitem{algorithm}
Michael J. Wise. String similarity via greedy string tiling and running Karp-Rabin matching. Dept. of
CS, University of Sydney, ftp://ftp.cs.su.oz.au/michaelw/doc/RKR GST.ps, December 1993.
% \url{https://www.netacad.com/courses/packet-tracer}
% \bibitem{ref_tool2}
% Công cụ hỗ trợ chấm điểm input/output Verwandlung
% \url{https://github.com/hzxie/voj}
% \bibitem{ref_tool}
% Công cụ hỗ trợ chấm điểm input/output Verwandlung
% \url{https://github.com/hzxie/voj}
\bibitem{whitebox-testing}
Phạm Ngọc Hùng, Trương Anh Hoàng và Đặng Văn Hưng, "Giáo trình kiểm thử phần mềm", 2018.
\end{thebibliography}

% \bibliographystyle{plain} % ieeetr


% \appendix
% \addcontentsline{toc}{chapter}{Phụ lục A}
% \chapter*{Phụ lục A}
% \vspace{-1cm}
\setcounter{table}{0}
\renewcommand{\thetable}{A.\arabic{table}}
\setlength\LTleft{0pt}
\setlength\LTright{0pt}
\setlength\LTcapwidth{\linewidth}
\begin{longtable}{ |p{1cm}|p{7cm}|p{6,5cm}|}
% \begin{longtable}{ @{\extracolsep{\fill}}|p{1cm}|p{6,5cm}|p{6,5cm}|@{}}
    % \centering
    \caption{Bảng danh sách các dạng biểu thức được hỗ trợ trong mã nguồn}
    % \vspace{0.5cm}
    % \begin{adjustbox}{width=1\textwidth}
    % \small
    % \begin{tabular}{ |p{1cm}|p{6,5cm}|p{6,5cm}|  }
    \\
         \hline
         \textbf{STT} & \textbf{Kiểu biểu thức} & \textbf{Ví dụ minh họa}\\
         \hline
         \endhead
         \hline
         \endfoot
        1 & Biểu thức sử dụng các phép toán với biến nguyên thủy & \makecell[l]{ a = 1 + 2;\\
         a = 3*4/2;\\
        if (a > 1);\\ if(a > 1 + 2); \\
        if (a + 1 > 3*4)} 
        \\
        \hline
        2 & Biểu thức sử dụng các phép toán với tên biến.   & \makecell[l]{ a = b + c;\\ a = b*c;\\
        if (a > b); \\if (a > b+c); \\
        if (a + b ==  c + d); \\
        if (a*b != c/d)}
        \\
        \hline
        3 & Biểu thức sử dụng thuộc tính length của biến kiểu string & \makecell[l]{ s = “abcdef”;\\
        a = s.length;\\
        if (s.length > 10);\\
        if(s.length > a+b); \\
        if (s1.length < s2.length + a);}
        \\
        \hline
        4 & Biểu thức sử dụng phương thức startsWith(), endsWith(), includes() của biến kiểu string & \makecell[l]{ a = s.startsWith(“ABC”); \\
        a = endsWith(“DEF”);\\
        if (s.includes(“XYZ”))}
        \\
        \hline
        5 & Gán một chuỗi cụ thể cho biến kiểu string & a = “abcdef”;\\
        \hline
        6 & Biểu thức sử dụng các phép toán với thuộc tính của tham số đối tượng   & \makecell[l]{a = person.height + person.age;\\
        if (person.height > 180)\\
        if (person.height/person.age <  8)} 
        \\ 
        \hline
        7 & Biểu thức sử dụng các phép toán với các hàm getter của đối tượng & \makecell[l]{s = person.getName();\\
        if (person.getName().\\startsWith(“hoaithu”))} 
        \\
         \hline
        8 & Biểu thức sử dụng các phép toán với phần tử mảng có index cụ thể &\makecell[l]{ a = arr[0] + b; \\
        a[0] = a + b + a[1];} 
        \\
         \hline
        9 & Biểu thức khai báo/gán giá trị là một mảng & a = [1,2,3,4,5,6];\\
         \hline
        10 &Biểu thức khai báo/gán giá trị là một Json object & \makecell[l]{a= \{height: 180, age: 23, \\school:\{name: “UET”\}\}}\\
         \hline
        11 & Biểu thức điều kiện kép bao gồm biều biểu thức điều kiện đơn thỏa mãn các tính chất 1 -> 8 & If (a > b \&\& s.length > 10 \&\& s.startsWith(“ABC”))
        % \end{tabular}
    % \end{adjustbox}
    \label{table:expressions}
\end{longtable}

\begin{longtable}{ |p{1cm}|p{7cm}|p{6,5cm}|}
% \begin{longtable}{ @{\extracolsep{\fill}}|p{1cm}|p{6,5cm}|p{6,5cm}|@{}}
    % \centering
    \caption{Bảng danh sách các dạng hình thức mã nguồn chưa được hỗ trợ}
    % \vspace{0.5cm}
    % \begin{adjustbox}{width=1\textwidth}
    % \small
    % \begin{tabular}{ |p{1cm}|p{6,5cm}|p{6,5cm}|  }
    \\
         \hline
         \textbf{STT} & \textbf{Hình thức mã nguồn} & \textbf{Ví dụ minh họa}\\
         \hline
         \endhead
         \hline
         \endfoot
        1 & Cấu trúc điều khiển vòng lặp & \makecell[l]{let list = [4, 5, 6];\\
        for (let i in list) \{ console.log(i); \}\\
        for (let i of list) \{ console.log(i); \}\\
        for (var i = 0; i < list.length; i++) \{\\
            var num = list[i];\\
            console.log(num);\\
        \}
        } \\
        \hline
        2 & Cấu trúc điều khiển \textit{switch}   & \makecell[l]{ a = b + c;\\ a = b*c;\\
        if (a > b); \\if (a > b+c); \\
        if (a + b ==  c + d); \\
        if (a*b != c/d)}
        \\
        \hline
        3 & Câu điều kiện rút gọn & \makecell[l]{ s = “abcdef”;\\
        a = s.length;\\
        if (s.length > 10);\\
        if(s.length > a+b); \\
        if (s1.length < s2.length + a);}
        \\
        \hline
        4 & Biểu thức thay đổi giá trị rút gọn & \makecell[l]{ a = s.startsWith(“ABC”); \\
        a = endsWith(“DEF”);\\
        if (s.includes(“XYZ”))}
        \\
        \hline
        5 & Chỉ số truy cập phần tử mảng là biểu thức & a = “abcdef”;\\
        \hline
        6 & Arrow Function  & \makecell[l]{a = person.height + person.age;\\
        if (person.height > 180)\\
        if (person.height/person.age <  8)} 
        \\ 
        \hline
        7 & Gọi hàm bên ngoài trả về kiểu dữ liệu là đối tượng & \makecell[l]{s = person.getName();\\
        if (person.getName().\\startsWith(“hoaithu”))} 
        \\
         \hline
        8 & Biến kiểu dữ liệu enum, any, undefined, never, tuple, v.v. &\makecell[l]{ a = arr[0] + b; \\
        a[0] = a + b + a[1];}  \\
        \hline
        v.v. & v.v. & v.v.
        %  \hline
        % 9 & Biểu thức khai báo/gán giá trị là một mảng & a = [1,2,3,4,5,6];\\
        %  \hline
        % 10 &Biểu thức khai báo/gán giá trị là một Json object & \makecell[l]{a= \{height: 180, age: 23, \\school:\{name: “UET”\}\}}\\
        %  \hline
        % 11 & Biểu thức điều kiện kép bao gồm biều biểu thức điều kiện đơn thỏa mãn các tính chất 1 -> 8 & If (a > b \&\& s.length > 10 \&\& s.startsWith(“ABC”))
        % \end{tabular}
    % \end{adjustbox}
    \label{table:expressions_not_handle}
\end{longtable}



\newpage
\printbibliography[heading=bibintoc, title=Tài liệu tham khảo]
% \printbibliography[heading=bibintoc, title=Tài liệu tham khảo tiếng Việt,keyword=vietnamese]
% \printbibliography[heading=bibintoc, title=Tài liệu tham khảo tiếng Anh,keyword=english]
% \appendix



\end{document}
