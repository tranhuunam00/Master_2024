
Trong chương này, sẽ trình bày tổng quan về hội chứng ngưng thở khi ngủ do tắc
nghẽn, phân tích mối quan hệ của tư thế ngủ đối với mức độ nghiêm trọng của OSA
và dạng OSA ngủ phụ thuộc tư thế. Tiếp đó, chương tập trung làm rõ cơ sở khoa
học cho việc phát triển các thiết bị theo dõi giấc ngủ, phân loại tư thế ngủ
tại nhà. Cuối cùng, đưa ra xu hướng ứng dụng trí tuệ nhân tạo và điện toán biên
trong thu thập, phân tích dữ liệu cảm biến để phân loại tư thế ngủ, qua đó đặt
nền tảng cho các hướng nghiên cứu và triển khai kỹ thuật được trình bày trong
những chương tiếp theo.
\section{Hội chứng ngưng thở khi ngủ}

Trong lĩnh vực nghiên cứu các rối loạn hô hấp liên quan đến giấc ngủ, theo tiêu
chuẩn của Hiệp hội Y học Giấc ngủ Hoa Kỳ (AASM) \cite{berry2012scoring}, ba
hiện tượng hô hấp chính cần được nhận diện bao gồm: ngưng thở, giảm thở, và
hiện tượng kích hoạt liên quan đến nỗ lực hô hấp (Respiratory Effort-Related
Arousal - RERA).

\subsection{Khái niệm}

Ngưng thở được AASM định nghĩa là sự ngưng luồng khí hô hấp qua mũi và miệng
trong thời gian tối thiểu 10 giây gây giảm nồng độ oxy trong máu. Các sự kiện
ngưng thở có thể kéo dài đến 30 giây hoặc hơn trong những trường hợp nặng. Có
ba dạng chính của hội chứng ngưng thở khi ngủ \cite{ThaySYOSA}: ngưng thở tắc
nghẽn, ngưng thở trung ương, ngưng thở hỗn hợp. Trong đó: 01) Ngưng thở do tắc
nghẽn OSA là sự hẹp hoặc tắc nghẽn một phần hay toàn bộ đường hô hấp trên, bao
gồm vùng mũi họng, hầu họng \cite{osa_summary}; 02) Ngưng thở trung ương là
tình trạng não không gửi tín hiệu đúng đến các cơ kiểm soát hô hấp
\cite{eckert2007csa}. 03) Ngưng thở hỗn hợp là sự kết hợp của cả hai yếu tố.
Dạng này thường xuất hiện ở những bệnh nhân OSA nặng.

\begin{figure}[H]
  \centering
  \includegraphics[width=\textwidth]{images/OSA.png}
  \vspace*{-7mm}
  \caption{Sự khác nhau về đường thở của người bình thường và người mắc OSA}
  \label{OSA}
\end{figure}

Giảm thở phản ánh sự giảm một phần của lưu lượng khí qua đường hô hấp trên mà
không dẫn đến ngưng thở hoàn toàn. Theo AASM sự kiện giảm thở được xác định khi
biên độ tín hiệu luồng khí thường đo bằng cảm biến áp lực mũi hoặc tín hiệu lưu
lượng khí của thiết bị CPAP giảm ít nhất 30\% so với giá trị nền trước sự kiện,
kéo dài tối thiểu 10 giây, và đi kèm với hiện tượng giảm độ bão hòa oxy từ 3\%
trở lên hoặc xuất hiện kích hoạt điện não.

RERA theo AASM là sự kiện gia tăng nỗ lực hô hấp kéo dài ít nhất 10 giây nhưng
không đủ tiêu chí của ngưng thở hoặc giảm thở. Phương pháp tiêu chuẩn để xác
định là đo áp lực thực quản, tuy nhiên khó áp dụng do gây khó chịu cho bệnh
nhân. Phương án thay thế đáng tin cậy là dùng ống thông mũi kết hợp cảm biến áp
lực, cho kết quả tương đương về mặt lâm sàng. RERA được tính vào chỉ số rối
loạn hô hấp (Respiratory Disturbance Index - RDI).

Trong số các rối loạn hô hấp liên quan đến giấc ngủ đã đề cập, OSA là dạng phổ
biến nhất và có tác động sâu rộng đến sức khỏe cộng đồng. Mức độ của OSA được
đánh giá dựa trên chỉ số ngưng thở giảm thở (AHI) bằng cách chia tổng số lần
ngưng thở và giảm thở cho tổng số giờ đã ngủ, với mỗi sự kiện phải kéo dài ít
nhất 10 giây Bảng~\ref{ahi} \cite{osa_summary}.
\begin{table}[H]
  \caption{\texorpdfstring{Phân loại mức độ OSA dựa trên chỉ số AHI}{Phân loại OSA}}
  \label{ahi}
  \vspace{-3mm}
  \begin{center}
    \begin{tabular}{|p{3cm}|p{6cm}|}
      \hline
      AHI       & Cấp độ     \\
      \hline
      <5        & Không mắc  \\
      5 đến 15  & Nhẹ        \\
      15 đến 30 & Trung bình \\
      >30       & Nặng       \\
      \hline
    \end{tabular}
    \label{tab1}
  \end{center}
\end{table}

\subsection{Nguyên nhân}

Nguyên nhân của OSA đến từ đa dạng các yếu tố bao gồm cả hình thái giải phẫu và
các chức năng sinh lý của đường hô hấp trên. Các nguyên nhân thường không đồng
nhất giữa các người bệnh. Về mặt giải phẫu, việc hẹp hoặc giảm độ vững của cấu
trúc đường hô hấp trên là yếu tố tiên quyết. Những bất thường tại vòm họng như
lưỡi to, phì đại amidan, vách mũi lệch, xương hàm nhỏ hoặc tụt về sau, cùng với
phân bố mỡ vùng cổ tăng (vòng cổ lớn) đều làm giảm tiết diện đường thở và tăng
khả năng xẹp trong khi hít vào \cite{Young2004_nguyen_nhan}.

Về chức năng thần kinh - cơ, trương lực cơ duy trì độ mở của đường hô hấp trên
có thể suy giảm, làm tăng xu hướng xẹp đường thở. Ngoài ra, các yếu tố nguy cơ
được ghi nhận bao gồm: béo phì, nam giới, lớn tuổi, hút thuốc lá, uống rượu,
mãn kinh ở nữ giới, nghẹt mũi mạn tính, và yếu tố di truyền
\cite{Spicuzza2015_nguyen_nhan, reason_osa, reasonOsa}.

\subsection{Ảnh hưởng của OSA}
Theo nghiên cứu của Benjafield và cộng sự \cite{Benjafield2019}, ước tính có
gần một tỷ người trên toàn cầu mắc OSA. Tại Việt Nam, theo GS.TS.BS. Dương Quý
Sỹ \footnote{Giáo sư, Tiến sĩ, Bác sĩ Dương Quý Sỹ là Chủ tịch Hội Giấc ngủ
  Việt Nam, chuyên gia đầu ngành về rối loạn giấc ngủ và hô hấp.}, tỷ lệ mắc OSA
ở người trưởng thành chiếm khoảng 8,5\% \cite{nguoimacOSA_VN}. OSA được xem là
một rối loạn hô hấp trong giấc ngủ có ảnh hưởng sâu rộng tới sức khỏe thể chất
lẫn tinh thần và là nguyên nhân y khoa hàng đầu gây ra tình trạng buồn ngủ quá
mức vào ban ngày \cite{Salari2025}.

Một phân tích tổng quát được thực hiện bởi Salari và cộng sự \cite{Salari2025}
trên 15 nghiên cứu với tổng cộng 42,924 đối tượng cho thấy tỷ lệ EDS ở bệnh
nhân OSA trên toàn cầu đạt 39.9\%. Tình trạng buồn ngủ quá mức này làm gia tăng
đáng kể nguy cơ tai nạn giao thông, suy giảm năng suất lao động, và rối loạn
chức năng tình dục \cite{flemons1997quality}.

Hơn nữa, tình trạng giảm oxy máu liên tục tái diễn trong khi ngủ cùng với sự
gián đoạn chu kỳ giấc ngủ được chứng minh có liên quan mật thiết đến nhiều bệnh
lý mãn tính như suy tim, bệnh động mạch vành, rối loạn nhịp tim, gan nhiễm mỡ
do rối loạn chuyển hóa, và đột quỵ
\cite{wright1997health,Zinchuk2018,young1997population}. Nghiên cứu quy mô lớn
của Xia Wang và cộng sự với 25,760 người tham gia cho thấy, khi chỉ số AHI tăng
thêm mỗi 10 đơn vị, nguy cơ mắc bệnh tim mạch tăng tương ứng 17\%
\cite{Wang2013_tim}. GS.TS.BS. Dương Quý Sỹ và cộng sự đã khảo sát 524 trẻ em
mắc rối loạn tăng động tại Bệnh viện Nhi Trung ương Việt Nam
\cite{ThaySUCHildren}. Kết quả cho thấy tỷ lệ mắc OSA ở nhóm này là 23.3\%,
trong đó chủ yếu ở mức độ trung bình đến nặng. Nghiên cứu cũng đồng thời xác
định mối tương quan đáng kể giữa mức độ nghiêm trọng của OSA và các triệu chứng
mất tập trung, tăng động, rối loạn hành vi, lo âu và trầm cảm.

\subsection{Tư thế ngủ và mối liên hệ với OSA}
Tư thế ngủ Hình~\ref{4_tuthe_nguoi} giữ vai trò thiết yếu trong việc duy trì
sức khỏe tổng thể và chất lượng giấc ngủ, bên cạnh các yếu tố khác như thời
lượng, môi trường và thói quen ngủ. Nhiều nghiên cứu đã chỉ ra rằng tư thế ngủ
có thể ảnh hưởng trực tiếp đến hoạt động của hệ hô hấp, tim mạch và hệ cơ -
xương, đặc biệt là cột sống \cite{Cary2021_tu_the_ngủ}. Trong phạm vi hội chứng
ngưng thở tắc nghẽn khi ngủ, tư thế ngủ được xem là một yếu tố quan trọng quyết
định mức độ nghiêm trọng của bệnh.

\begin{figure}[H]
  \centering
  \includegraphics[width=\textwidth]{images/4ngu.png}
  \vspace*{-7mm}
  \caption{Các tư thế ngủ cơ bản của con người}
  \label{4_tuthe_nguoi}
\end{figure}

Các nghiên cứu cho thấy, nhiều bệnh nhân OSA có tần suất ngưng thở và giảm thở
cao hơn rõ rệt khi nằm ngửa so với các tư thế khác. Nguyên nhân chủ yếu là do
tác động của trọng lực lên các cấu trúc mô vùng họng gây tăng khả năng bị xẹp
đường thở Hình ~\ref{fig:tuthe_osa}. Hiện tượng này được gọi là OSA phụ thuộc
tư thế \cite{heinzer2018,aloweidat2023positional}.

\begin{figure}[H]
  \centering
  \includegraphics[width=0.3\textwidth]{images/tuthe_osa.png} % giảm xuống 70% chiều rộng trang
  \vspace*{-5mm}
  \caption{Xẹp đường thở ở tư thế ngửa}
  \label{fig:tuthe_osa}
\end{figure}

Để chẩn đoán pOSA, nhiều tiêu chí khác nhau đã được đề xuất, từ các phương pháp đơn
giản đến phức tạp hơn. Định nghĩa kinh điển nhất do Cartwright (1984) đưa ra cho
rằng bệnh nhân được xem là mắc pOSA khi chỉ số AHI
ở tư thế nằm ngửa cao hơn ít nhất hai lần so với AHI ở tư thế không nằm
ngửa \cite{cartwright1984position}. Tiếp đó, Mador (2005) đã mở rộng định
nghĩa này bằng cách bổ sung tiêu chí rằng AHI ở tư thế không nằm ngửa phải
nhỏ hơn 5 nhằm tăng tính đặc hiệu cho chẩn đoán \cite{mador2005prevalence}.
Levendowski (2015) lại đề xuất một cách tiếp cận theo tỷ lệ, trong đó
pOSA được xác định khi AHI toàn bộ tư thế lớn hơn hoặc bằng 1.5 lần so với AHI ở tư thế
không nằm ngửa \cite{levendowski2015neck}.
Ngoài ra, Frank và cộng sự (2014) đã giới thiệu Amsterdam Positional
Obstructive Sleep Apnea Classification (APOC) - một hệ thống phân loại
dành cho các bệnh nhân có chỉ số AHI cao \cite{frank2014positional}.
Theo tiêu chí APOC, bệnh nhân được xác định mắc pOSA khi có AHI > 5, đồng thời tổng thời gian ngủ
ở tư thế có AHI nhỏ nhất và tư thế có AHI lớn nhất
đều chiếm ít nhất 10\% tổng thời gian ngủ.
Hệ thống này chia bệnh nhân thành ba nhóm: APOC-I (có thể khỏi hoàn
toàn nhờ thay đổi tư thế ngủ), APOC-II (không phụ thuộc tư thế),
và APOC-III (phụ thuộc tư thế một phần).

Từ các nghiên cứu trên, có thể thấy việc hiểu rõ mối liên hệ giữa tư thế ngủ và
sự ổn định của đường thở trên không chỉ giúp cải thiện khả năng chẩn đoán mà
còn mở ra hướng điều trị không xâm lấn đầy tiềm năng - đó là liệu pháp thay đổi
tư thế ngủ, nhằm giảm mức độ tắc nghẽn và cải thiện chất lượng giấc ngủ ở bệnh
nhân OSA.

\section{Công nghệ trong chẩn đoán OSA và phân loại tư thế ngủ}

Với sự phát triển vượt bậc của công nghệ thiết kế, chế tạo cảm biến - vi xử lý
và các thuật toán học máy, việc chẩn đoán hội chứng ngưng thở khi ngủ tắc nghẽn
và phân loại tư thế ngủ đã đạt được những bước tiến đáng kể. Các hệ thống hiện
nay có xu hướng thu nhỏ kích thước thiết bị, mở rộng khả năng ứng dụng thực tế
và giảm đáng kể chi phí triển khai, tạo tiền đề cho việc giám sát giấc ngủ cá
nhân hóa ngay tại nhà.
\subsection{Đa ký giấc ngủ}
Phần lớn bệnh nhân OSA không tự nhận biết được các rối loạn hô hấp xảy ra trong
lúc ngủ \cite{Gottlieb2020_psg}. Hiện tượng này đặc biệt phổ biến ở những người
sống hoặc ngủ một mình, do thiếu sự quan sát từ bên ngoài. Trong phạm vi lâm
sàng hiện nay, bệnh nhân nghi ngờ mắc OSA được thăm khám bởi bác sĩ chuyên khoa
Tai Mũi Họng và bác sĩ chuyên sâu về giấc ngủ. Quy trình thăm khám bao gồm khai
thác lịch sử bệnh lý, đánh giá các yếu tố nguy cơ và sử dụng những thang điểm
sàng lọc như Epworth Sleepiness Scale hoặc STOP-BANG \footnote{Epworth
  Sleepiness Scale hoặc STOP-BANG là các công cụ được chấp thuận tại Việt Nam
  nhằm sàng lọc sớm khả năng mắc OSA}. Ngoài ra, bác sĩ có thể tiến hành nội soi
mũi - họng để tìm nguyên nhân gây hẹp đường thở trên, chẳng hạn như phì đại
amidan, hay bất thường vùng hàm mặt. Sau đó tiến hành ghi nhận bằng đa ký giấc
ngủ (Polysomnography - PSG) \cite{diagnosis_osa, medical2006polysomnography}.
Phương pháp PSG sẽ được thực hiện dưới sự giám sát của kĩ thuật chuyên môn. Đây
được xem là tiêu chuẩn vàng trong chẩn đoán hội chứng ngưng thở khi ngủ. Phương
pháp này sẽ ghi lại đa kênh dữ liệu liên tục trong lúc ngủ suốt một đêm. Qua
đó, cho phép theo dõi đồng thời nhiều thông số sinh lý phản ánh toàn diện hoạt
động thần kinh và hô hấp trong giấc ngủ.

Một hệ thống PSG điển hình bao gồm: điện não đồ để ghi và lưu lại hoạt động
điện của não; điện cơ nhằm đo trương lực cơ; điện nhãn đồ để xác định giai đoạn
ngủ thông qua chuyển động mắt; và điện tâm đồ để giám sát hoạt động tim. Bên
cạnh đó, các kênh dữ liệu như độ bão hòa oxy trong máu (SpO\textsubscript{2}),
lưu lượng khí thở qua mũi và miệng, nỗ lực hô hấp thông qua chuyển động ngực -
bụng, cường độ tiếng ngáy sẽ giúp bác sĩ có thêm thông tin để đưa ra quyết định
chính xác \cite{psg_paper, kushida2005psg}. Trong số các tín hiệu này, tư thế
ngủ là một thông số đặc biệt quan trọng, góp phần xác định dạng bệnh pOSA.
Hình~\ref{fig:noxa1} là màn hình phần mềm NoxA1 hiển thị nhiều tín hiệu sinh lý
theo dấu thời gian, cho phép quan sát đồng thời các kênh hô hấp. Hệ thống cũng
cung cấp công cụ đánh dấu tự động các sự kiện bất thường như ngưng thở và giảm
thở.

\begin{figure}[H]
  \centering
  \includegraphics[width=0.7\textwidth]{images/mceclip13.png} % giảm xuống 70% chiều rộng trang
  \caption{Màn hình tổng hợp tín hiệu của PSG}
  \label{fig:noxa1}
\end{figure}

Tuy nhiên, phương pháp PSG vẫn còn một số hạn chế về mặt chi phí bao gồm chi
phí đi lại lưu trú và thăm khám. Việc bệnh nhân bắt buộc gắn nhiều điện cực và
cảm biến trên cơ thể không chỉ gây bất tiện mà còn có thể làm sai lệch hành vi
ngủ tự nhiên. Điều này còn có thể làm tín hiệu bị gián đoạn trong quá trình thu
thập. Theo tìm hiểu của nhóm, bác sĩ có thể đưa ra nhận định dựa trên một tập
nhỏ kênh tín hiệu còn nguyên vẹn. Điều này gợi mở hướng ứng dụng các mô hình
học máy phản ánh đúng đặc trưng sinh lý nhằm tự động hóa quá trình nhận dạng.

\subsection{Thiết bị theo dõi OSA tại nhà}
Những hạn chế về chi phí, sự bất tiện của đa ký giấc ngủ thúc đẩy sự phát triển
của các thiết bị theo dõi giấc ngủ ngoài trung tâm (Out-of-Center Sleep Testing
Devices - OCST), hay còn gọi là thiết bị kiểm tra giấc ngủ tại nhà (Home Sleep
Test - HST). Các thiết bị HST được thiết kế với mục tiêu giảm kích thước, chi
phí phù hợp, thuận tiện sử dụng tại nhà. Để làm được điều đó thì các thiết bị
HST cần phải giảm thiểu số lượng cảm biến nhưng vẫn đảm bảo độ chính xác. Điều
này có thể đạt được thông qua việc tích hợp các thuật toán xử lý tín hiệu và mô
hình học máy có thể được thực thi trực tiếp trên phần cứng nhúng hoặc thông qua
ứng dụng hỗ trợ trên điện thoại di động. Một trong những khung chuẩn được sử
dụng rộng rãi trong quá trình phát triển các hệ thống HST là mô hình SCOPERA,
bao gồm sáu nhóm thông số cốt lõi: \textit{Sleep (S)} - đặc trưng giấc ngủ;
\textit{Cardiovascular (C)} - hoạt động tim mạch; \textit{Oximetry (O)} - độ
bão hòa oxy; \textit{Effort (E)} - nỗ lực hô hấp; \textit{Respiratory flow (R)}
- lưu lượng khí hô hấp; và \textit{Audio (A)} - âm thanh luồng khí thở.

Dựa trên vị trí gắn trên cơ thể, các thiết bị HST có thể được phân thành các
nhóm chính như: vòng tay, đai ngực, miếng dán, tai nghe và nhẫn thông minh
(Bảng~\ref{tab:wearable_types}).

\begin{table}[htbp] \centering \caption{Phân loại thiết bị đeo trong phát hiện OSA. tư thế ngủ} \label{tab:wearable_types}
  \begin{tabular}{|p{5.5cm}|p{7.5cm}|} \hline
    \textbf{Loại thiết bị đeo} & \textbf{Tài liệu tham khảo}                                                           \\
    \hline Vòng tay            & \cite{e3hst,osa_sanchez2025, shen2022mtcnn, jeon2020realtime,}                        \\
    \hline Đai ngực            & \cite{e3hst, osa_sanchez2025, svmHSt2017, chen2024hdc}                                \\
    \hline Miếng dán           & \cite{osa_sanchez2025, yeo2022resnet, yeo2022respiratory, p_3}                        \\
    \hline Dạng khác           & \cite{hstSurvey, hst_paper, hst_wear_paper, osa_sanchez2025, Sleep_Posture_Detection} \\
    \hline\end{tabular}
\end{table}

Nghiên cứu của Jeon và cộng sự \cite{jeon2020realtime} đã sử dụng thiết bị đeo
Sleep Care Kit (Hình~\ref{fig:sleepcare}) gắn ở ngực để thu nhận tín hiệu nhịp
tim, tín hiệu gia tốc ba trục, tín hiệu hô hấp và SpO$_2$. Các mô hình Gaussian
Naive Bayes, ANN và k-Nearest Neighbor được huấn luyện nhằm phân loại trạng
thái hô hấp và phát hiện ngưng thở. Trong đó mô hình KNN đạt độ chính xác 95\%
và thời gian xử lý chỉ 640~$\mu$s - đáp ứng tiêu chí chẩn đoán OSA theo thời
gian thực. Tương tự, Chen và cộng sự \cite{chen2024hdc} phát triển thiết bị
vòng tay ghi tín hiệu phản hồi quang học để phát hiện biến thiên lưu lượng máu
trên 100 đối tượng tình nguyện. Nghiên cứu này hướng tới tối ưu về bộ nhớ, độ
trễ và năng lượng, đồng thời đồng bộ dữ liệu với PSG để đảm bảo độ tin cậy,
hướng tới ứng dụng giám sát dài hạn tại nhà.

\begin{figure}[H]
  \centering
  \includegraphics[width=1\textwidth]{images/sleepcare.png} % giảm xuống 70% chiều rộng trang
  \vspace*{-5mm}
  \caption{Thiết bị Sleep Care Kit}
  \label{fig:sleepcare}
\end{figure}

Yeo và cộng sự \cite{yeo2022resnet, yeo2022respiratory} triển khai thiết bị dán
T-REX TR100A để ghi điện tâm đồ một kênh tại vùng bụng trên. Thiết bị dán trực
tiếp lên da, đảm bảo tiếp xúc ổn định, giúp ghi liên tục trong thời gian dài mà
không gây khó chịu cho người sử dụng. Bên cạnh đó, nghiên cứu \cite{svmHSt2017}
chỉ ra rằng các tín hiệu chuyển động ngực và bụng thu từ dải cảm biến gia tốc
áp điện có thể được khai thác hiệu quả trong phân loại rối loạn hô hấp khi ngủ
bằng mô hình SVM, đạt độ chính xác trung bình 81.8\%. Ngoài ra, Domingues và
cộng sự đề xuất một mô hình mạng nơ-ron nhân tạo sử dụng dữ liệu từ SpO$_2$,
cảm biến gia tốc và âm thanh ngáy. Kết quả đạt được tiệm cận độ chính xác của
PSG truyền thống \cite{domingues2024sleep}.

Nhìn chung, các hệ thống HST thế hệ mới đang chuyển dịch từ kiến trúc đám mây
sang kiến trúc tại biên. Điều này không chỉ nâng cao tính riêng tư và độ tin
cậy, mà còn mở đường cho các thiết bị chẩn đoán OSA cá nhân hóa, nhỏ gọn và có
khả năng triển khai đại trà trong thực hành y học giấc ngủ hiện đại.

\subsection{Kĩ thuật phân loại tư thế ngủ}

Hiện nay, có nhiều nghiên cứu tập trung phát triển các hệ thống theo dõi tư thế
ngủ. Asma Channa và cộng sự \cite{Channa_osa} đã phát triển hệ thống theo dõi
tư thế ngủ sử dụng hai thảm cảm biến áp lực để thu thập dữ liệu từ 13 người
tham gia, với các thuật toán học máy như KNN và SVM cho độ chính xác tới
98.7\%. Tương tự, Xi Xu và cộng sự \cite{xu2024classification} đề xuất mô hình
kết hợp ba thuật toán (XGBoost, SVM, DNDT) đạt độ chính xác 94.48\% trên dữ
liệu áp suất từ nệm khí.

\begin{figure}[H]
  \centering
  \includegraphics[width=0.8\textwidth]{images/sleep_camera.png} % giảm xuống 70% chiều rộng trang
  \vspace*{-5mm}
  \caption{Thiết kế hệ thống nệm áp suất trong nghiên cứu của Xi Xu và cộng sự}
  \label{fig:sleep_camera}
\end{figure}

Các nghiên cứu sử dụng camera RGB hoặc camera hồng ngoại cũng cho kết quả ấn
tượng trong môi trường có kiểm soát. Khan và cộng sự \cite{KHAN20241729_camera}
phát triển một hệ thống kết hợp camera RGB và camera nhiệt. Sau quá trình huấn
luyện, hai mô hình VGG16 và ResNet50 cho độ chính xác lên tới 99\%. Mahvash và
cộng sự \cite{Mahvash_camera} tiếp tục mở rộng hướng nghiên cứu này bằng cách
sử dụng camera hồng ngoại 2D, kết hợp học sâu với mô hình ResNet, cho độ chính
xác 95.1\%, f1-score 94.9\% vượt trội hơn cảm biến PSG truyền thống. Ngoài ra,
một hướng tiếp cận không tiếp xúc khác sử dụng cảm biến radar cũng được đề xuất
để nhận dạng tư thế ngủ \cite{10123404_camera}. Kết quả cho thấy hệ thống đạt
độ chính xác 91\% trong nhận dạng tư thế và 83.7\% trong phát hiện sự kiện
chuyển tư thế.

Ngoài ra, việc tận dụng cảm biến gia tốc ở ngay trên chính chiếc điện thoại
cũng là một giải pháp hữu hiệu \cite{sun2017sleepmonitor}. Nhóm tác giả trong
\cite{Ferrer_osa} đã báo cáo nghiên cứu đánh giá tư thế ngủ của bệnh nhân sử
dụng thiết bị di động đeo ở ngực kết hợp với phần mềm trên nền tảng Android để
thu thập lại dữ liệu kể cả khi tắt màn hình. Trong một nghiên cứu khác, Natale
và cộng sự đã khai thác các cảm biến tích hợp sẵn trên điện thoại IPhone để ước
lượng các thông số liên quan đến chất lượng giấc ngủ, bao gồm tổng thời gian
ngủ, độ trễ vào giấc và hiệu quả giấc ngủ. Phương pháp tiếp cận này cho thấy
tiềm năng trong việc sử dụng thiết bị di động như một công cụ theo dõi giấc ngủ
tiện lợi và dễ tiếp cận tại nhà \cite{Natale_osa}. Tuy nhiên, việc tiếp xúc
điện thoại trực tiếp với cơ thể trong thời gian lâu cũng có thể gây những ảnh
hưởng nhất định đến người dùng.

Thêm một xu hướng nghiên cứu gần đây nữa là tập trung vào việc tối giản phần
cứng, đặc biệt là sử dụng duy nhất một cảm biến gia tốc để nhận dạng tư thế
ngủ. Trong hướng này, Jeng và cộng sự \cite{Jeng_osa} phát triển hệ thống
iSleePost gồm hai cảm biến gia tốc đeo ở cổ tay và ngực để ánh xạ chuyển động
và nhãn tư thế, với độ chính xác trung bình 0.82\%. Zhang và cộng sự
\cite{Zhang_osa} tiến thêm một bước khi chỉ sử dụng một cảm biến gia tốc ba
trục gắn ở ngực, kết hợp bộ phân loại tuyến tính, cho phép nhận dạng bốn tư thế
ngủ cơ bản và phát hiện chuyển động hiệu quả. Một số thiết kế khác như Wearable
Sleep System (WSS) \cite{kwasnicki2018} hoặc thiết bị HST dạng miếng dán cổ
\cite{Sleep_Posture_Detection} cũng dùng cảm biến gia tốc ba trục có độ chính
xác cao (trên 97\%) trong nhận dạng tư thế ngủ.

Nghiên cứu \cite{abdulsadig2023}, tác giả tập trung vào nhận dạng tư thế ngủ tự
động bằng cảm biến gia tốc đơn gắn tại vùng cổ (Hình~\ref{fig:vitri_co}). Ba mô
hình được thử nghiệm gồm Decision Tree (DT), Extra-Trees (ET) và LSTM Neural
Network (LSTM-NN) và cho kết quả rằng mô hình DT có ưu thế về tốc độ dự đoán
(0.8~ms) và dung lượng nhỏ (1.765~kb), phù hợp triển khai mô bình trên biên.
Tác giả Vũ Hoàng Diệu và cộng sự \cite{Vu2025SleepPosition} đã chứng minh khả
năng của một hệ thống đơn cảm biến gia tốc trong nhận dạng đến 12 tư thế ngủ,
sử dụng mô hình AnpoNet kết hợp 1D-CNN và LSTM, đạt độ chính xác 94.67\% và
f1-score 92.94\%. Các kết quả này cho thấy cảm biến gia tốc đơn không chỉ đảm
bảo độ chính xác cao mà còn mang lại sự tiện lợi, chi phí phù hợp và khả năng
triển khai thực tế.

\begin{figure}[H]
  \centering
  \includegraphics[width=0.6\textwidth]{images/vitri_co.png} % giảm xuống 70% chiều rộng trang
  \vspace*{-5mm}
  \caption{Ví trị đặt cảm biến được đề xuất trong nghiên cứu \cite{abdulsadig2023}}
  \label{fig:vitri_co}
\end{figure}

Bám sát định hướng này, luận văn đề xuất một hệ thống sử dụng duy nhất một cảm
biến gia tốc gắn tại hõm ức để phát hiện các tư thế ngủ liên quan đến OSA. Vị
trí đặt thiết bị ở cổ còn là tiền đề để nhóm mở rộng thêm cảm biến như luồng
khí thở với mục đích đánh giá OSA. Dữ liệu thu được sẽ qua các bước tiền xử lý,
trích xuất đặc trưng và huấn luyện mô hình học máy nhẹ, hướng tới việc phát
triển thiết bị IoT nhỏ gọn, tiêu thụ năng lượng thấp, phân loại tư thế ngủ và
có thể có khả năng ước lượng chỉ số AHI được trình bày ở các chương tiếp theo.

\subsection{Cảm biến gia tốc trong đánh giá tư thế ngủ}

Ưu điểm nổi bật của cảm biến gia tốc nằm ở khả năng vận hành độc lập với mức
tiêu thụ năng lượng thấp, đồng thời có thể tích hợp dễ dàng vào các hệ thống
bảng mạch.

\begin{figure}[H]
  \centering
  \includegraphics[width=0.6\textwidth, keepaspectratio]{images/vị trí đặt cảm biến.png}
  \vspace*{-5mm}
  \caption{Vị trí đặt cảm biến gia tốc}
  \label{position_sensor}
\end{figure}

Nhờ đó, cảm biến này đặc biệt thích hợp cho việc phát triển các thiết bị đeo
thông minh có khả năng xử lý tại chỗ, nhỏ gọn, di động, tiêu thụ năng lượng
thấp phù hợp với xu hướng TinyML trong lĩnh vực điện toán biên. Ngoài ra, cảm
biến gia tốc ít phụ thuộc vào các điều kiện môi trường như ánh sáng, nhiệt độ,
sử dụng chăn, và quan trọng hơn, không xâm phạm quyền riêng tư của người dùng
như các hệ thống giám sát bằng hình ảnh.

Về mặt lựa chọn vị trí, các vùng như cổ tay hoặc trán thường chịu ảnh hưởng
mạnh từ các chuyển động ngoại ý và lệch trục cơ thể, làm giảm độ đại diện của
tín hiệu thu được. Vùng cổ được xem là vị trí lý tưởng nhờ tính ổn định hình
học cao, gần trung tâm của cơ thể, và khả năng phản ánh trực tiếp hướng trọng
lực. Đặc biệt, thuận lợi cho việc tích hợp thêm các cảm biến khác như cảm biến
âm thanh hoặc cảm biến nhiệt độ, phục vụ cho các mục tiêu mở rộng trong các
nghiên cứu tiếp theo. Đây cũng chính là các nguyên nhân nghiên cứu chọn cảm
biến gia tốc. Bước tiếp theo là làm rõ cơ sở nguyên lý đo lường của cảm biến,
nhằm lý giải vì sao thiết bị này có thể phản ánh chính xác tư thế cơ thể trong
khi ngủ.

\subsection*{Nguyên lý hoạt động cảm biến gia tốc}

Cảm biến gia tốc là một thiết bị đo lường có khả năng phát hiện và ghi nhận gia
tốc theo thời gian. Với ưu điểm nhỏ gọn, tiêu thụ năng lượng thấp và chi phí
hợp lý, cảm biến gia tốc được ứng dụng rộng rãi trong nhiều lĩnh vực và đặc
biệt trong các thiết bị theo dõi hoạt động và giấc ngủ
\cite{Santoso2015_mems_accelerometer_apply,
  Szermer2025_mems_accelerometer_apply}.

Hiện nay, các loại cảm biến gia tốc phổ biến được chế tạo bằng công nghệ vi cơ
điện tử (Micro-Electro-Mechanical Systems - MEMS) với ba kiểu chính là: kiểu áp
điện trở, kiểu điện dung và kiểu áp điện \cite{Shi2023_mems_accelerometer}.

\subsubsection*{Cảm biến gia tốc kiểu áp điện trở}

\begin{figure}[H]
  \centering
  \includegraphics[width=0.75\textwidth, keepaspectratio]{images/giatoc_ap_dien_tro.png}
  \caption{Cấu trúc cảm biến gia tốc áp điện trở \cite{Song2018_mems_accelerometer}.}
  \label{giatoc_ap_dien_tro}

  \vspace{0.2cm}
  \small{
    (a) Bố trí tổng thể
    \quad
    (b) Mặt cắt qua khối gia trọng và thanh dầm
    \quad
    (c) Cầu Wheatstone đo gia tốc trục X
    \quad
    (d) Cầu Wheatstone đo gia tốc trục Y
    \quad
    (e) Cầu Wheatstone đo gia tốc trục Z
  }
\end{figure}

Đầu tiên, cảm biến kiểu áp điện trở có cấu tạo gồm một
thanh dầm bằng vật liệu có tính chất có thể thay đổi điện trở, kèm theo một khối gia trọng (mass) và
giá đỡ \cite{Song2018_mems_accelerometer}.
Bề mặt trên của khối gia trọng, thanh dầm và giá đỡ nằm trên cùng một mặt phẳng,
đảm bảo tính đối xứng cấu trúc.
Khi cảm biến chịu tác động của gia tốc ngoài, ứng suất cơ học
xuất hiện tại vùng đầu thanh dầm, dẫn đến sự thay đổi điện trở của các
phần tử này. Các điện trở  được bố trí và
liên kết một cách tối ưu để tạo thành ba mạch cầu Wheatstone độc lập,
tương ứng với việc phát hiện các thành phần gia tốc theo ba trục vuông góc X, Y và Z.

Trong mỗi cầu Wheatstone, sự thay đổi điện trở vi sai được chuyển đổi trực tiếp
thành tín hiệu điện áp đầu ra, cho phép cảm biến biến đổi biến dạng cơ học
thành tín hiệu điện tương ứng.

\subsubsection*{Cảm biến gia tốc kiểu áp điện}

Cảm biến gia tốc kiểu áp điện được cấu tạo gồm hai phần chính
\cite{Li2018_mems_accelerometer}: (1) thanh dầm có một đầu cố định liên kết với
khối gia trọng. Khi có gia tốc tác động, khối này tạo ra lực quán tính khiến
thanh dầm bị uốn cong, sinh ra ứng suất cơ học trên bề mặt thanh dầm. Các thông
số hình học cơ bản bao gồm: chiều dài thanh dầm ($L$), chiều rộng ($b$), độ dày
($h$), chiều rộng khối lượng ($d$) và độ dày khối lượng ($t$). (2) Cấu trúc áp
điện gồm ba lớp vật liệu xếp chồng theo thứ tự: lớp điện cực dưới thường bằng
bạch kim/titan, lớp màng áp điện thường bằng Li-doped ZnO, và lớp điện cực trên
thường bằng nhôm.

\begin{figure}[!ht]
  \centering
  \includegraphics[width=0.75\textwidth, keepaspectratio]{images/giatoc_ap_dien.png}
  \caption{Cấu trúc cơ bản của cảm biến MEMS áp điện: (a) mặt trên (b) mặt dưới \cite{Li2018_mems_accelerometer}}
  \label{giatoc_ap_dien}
\end{figure}

Khi chịu tác dụng của gia tốc ngoài $F$, thanh dầm bị biến dạng đàn hồi. Theo
phân tích cơ học thanh dầm và lý thuyết biến dạng nhỏ
\cite{Hao2013_mems_accelerometer}, ứng suất dọc $\sigma_l$ xuất hiện trên bề
mặt thanh dầm theo phương trục $x$ được biểu diễn như sau:

\begin{equation}
  \sigma_l = \sigma_1 = \frac{6 F L}{b h^2}
\end{equation}

Dựa trên hiệu ứng áp điện thuận, hai bề mặt trên và dưới của màng áp điện ZnO
pha tạp Li lần lượt mang điện tích dương và âm theo phương trục $z$. Lượng điện
tích cảm ứng $q_z$ tại phương trục z được xác định bởi công thức:

\begin{equation}
  q_z = d_{31} \sigma_1
\end{equation}

Trong đó, $d_{31}$ là hệ số áp điện, và $\sigma_1$ là ứng suất pháp tuyến theo
phương $x$. Theo định nghĩa, điện dung $C$ giữa hai điện cực trên và dưới của
cấu trúc áp điện được xác định bởi:

\begin{equation}
  C = \frac{q_z}{V}
\end{equation}

\noindent
với $V$ là điện áp đầu ra giữa hai điện cực của màng áp điện.
Từ các quan hệ trên, điện áp đầu ra có thể được biểu diễn theo dạng:

\begin{equation}
  V = \frac{d_{31} \cdot 6 m L}{b h^2 C} \cdot a
\end{equation}

Kết hợp với định nghĩa độ nhạy của cảm biến, ta có biểu thức:

\begin{equation}
  S = \frac{V}{a} = \frac{d_{31} \cdot 6 m L}{b h^2 C}
\end{equation}

\subsubsection*{Cảm biến gia tốc MEMS điện dung}

Cảm biến gia tốc điện dung được gồm ba khối chức năng chính: khối gia trọng, hệ
thanh dầm đàn hồi và cụm điện cực hình răng lược
\cite{Wang2022_mems_accelerometer}. Khối gia trọng là phần trung tâm của cảm
biến, dao động khi có gia tốc tác động. Các thanh dầm có nhiệm vụ đỡ và liên
kết khối lượng này, đồng thời cho phép chuyển động đàn hồi trong giới hạn thiết
kế. Cụm điện cực hình răng lược (assembly combs) được sử dụng để căn chỉnh
trong quá trình chế tạo, trong khi cụm điện cực hình răng lược (drive combs)
tạo dao động cưỡng bức cho khối gia trọng trong các thí nghiệm đo đáp ứng. Cụm
điện cực hình răng lược (sense combs) đảm nhiệm chức năng phát hiện độ dịch
chuyển của khối lượng thông qua sự thay đổi điện dung giữa các ngón điện cực.

\begin{figure}[H]
  \centering
  \includegraphics[width=0.6\textwidth, keepaspectratio]{images/giatoc_diendung.png}
  \caption{Sơ đồ thiết kế cơ bản của cảm biến MEMS điện dung}
  \label{giatoc_diendung}
\end{figure}

Để đo tín hiệu, cảm biến sử dụng hai cụm
(sense combs 1 và sense combs 2) được mắc theo cấu hình vi sai. Hai tín hiệu điện dung biến thiên ngược pha
này giúp tăng hệ số khuếch đại vi sai và loại bỏ nhiễu chế độ chung, nhờ đó cải
thiện đáng kể độ nhạy và độ ổn định của cảm biến điện dung.

\begin{figure}[H]
  \centering
  \[
    \text{Tác động đầu vào}
    \;\xrightarrow[\text{Gây ra dịch chuyển }x]{}
    \;
    \text{Thay đổi điện dung }(\Delta C)
    \;\xrightarrow[\text{Mạch xử lý tín hiệu}]{}
    \;
    \text{Điện áp ra}
  \]
  \caption{Quá trình biến đổi tín hiệu trong cảm biến điện dung.}
  \label{fig:signal_flow}
\end{figure}

Khi có tác động gia tốc sẽ gây ra sự dịch chuyển cơ học $x$ giữa các bản cực
của tụ điện. Sự dịch chuyển này làm thay đổi điện dung $\Delta C$ của cảm biến.
Biến thiên điện dung $\Delta C$ này được mạch xử lý tín hiệu chuyển đổi thành
điện áp đầu ra $V_{\text{out}}$.

\textbf{Nguyên lý kích thích và chuyển đổi C-V trong cảm biến điện dung MEMS}

Dựa trên nguyên lý cảm ứng điện dung vi sai, hai tụ điện cảm biến có điện dung
lần lượt được xác định bởi \cite{Tirupathi2018_mems_accelerometer}:

\begin{equation}
  C_1 = C_0 + \Delta C,
  \qquad
  C_2 = C_0 - \Delta C
\end{equation}

Trong đó, $C_0$ là điện dung xác định khi không có gia tốc tác động, và $\Delta
  C$ là phần biến thiên điện dung phát sinh do dịch chuyển cơ học của khối gia
trọng dưới tác dụng của gia tốc.

Tín hiệu kích thích được áp dụng lên hai tụ có dạng sóng vuông vi sai $\pm
  V_p$, gọi là điện áp kích thích. Sự dịch chuyển cơ học $x$ gây ra biến thiên
điện dung $\Delta C$ được chuyển đổi thành tín hiệu điện thông qua mạch khuếch
đại điện tích hoạt động theo nguyên lý bảo toàn điện tích:

\begin{equation}
  Q = C_f \, V_{\text{out}}
\end{equation}

Trong đó, $Q$ là điện tích tích lũy tại nút khuếch đại, $C_f$ là điện dung phản
hồi của op-amp, và $V_{\text{out}}$ là điện áp đầu ra của mạch.

Sau quá trình giải điều chế trong hai pha $\varphi_1$ và $\varphi_2$, hai nhánh
đầu ra vi sai của mạch được xác định bởi:

\begin{equation}
  V_{\text{out}+} = V_{\text{cm}} + \frac{4 \, \Delta C}{C_f} \, V_p,
  \qquad
  V_{\text{out}-} = V_{\text{cm}} - \frac{4 \, \Delta C}{C_f} \, V_p
\end{equation}

\noindent
với $V_{\text{cm}}$ là điện áp chung,
$V_p$ là biên độ của tín hiệu kích thích.

Khi lấy hiệu hai đầu ra vi sai, điện áp đầu ra của cảm biến được xác định như
sau:

\begin{equation}
  V_{\text{out}} = V_{\text{out}+} - V_{\text{out}-} = \frac{8\,\Delta C}{C_f}\,V_p
  \label{eq:vout}
\end{equation}

Trong đó, $\Delta C$ biểu thị mức độ biến dạng cơ học của phần tử cảm biến,
$C_f$ quyết định độ khuếch đại của mạch (giá trị $C_f$ càng nhỏ thì độ khuếch
đại càng lớn), và $V_p$ là biên độ kích thích tỉ lệ thuận với biên độ điện áp
đầu ra.

Từ công thức~\ref{eq:vout}, có thể thấy rằng điện áp vi sai đầu ra
$V_{\text{out}}$ tỉ lệ tuyến tính với biến thiên điện dung $\Delta C$, tỉ lệ
thuận với điện áp kích thích $V_p$. Dạng tổng quát của mối quan hệ này có thể
được biểu diễn gọn hơn như sau:

\begin{equation}
  V_{\text{out}} = K \cdot \frac{\Delta C}{C_f} \, V_p
\end{equation}

\noindent
với $K = 8$ là hệ số khuếch đại hiệu dụng
phụ thuộc vào cấu hình vi sai.

Trong khuôn khổ nghiên cứu này, cảm biến gia tốc kiểu điện dung được lựa chọn
vì có độ nhạy cao trong vùng gia tốc thấp, phù hợp với các chuyển động nhỏ khi
ngủ. Việc lựa chọn cảm biến kiểu điện dung không đơn thuần là một quyết định kỹ
thuật thuần tuý, mà còn phản ánh một giải pháp có tính cân bằng giữa hiệu quả
đo lường, khả năng tích hợp phần cứng, và kế hoạch mở rộng ứng dụng trong lâm
sàng theo hướng chi phí phù hợp. Đây là một trong những nguyên tắc quan trọng
trong xu hướng đổi mới công nghệ y tế cộng đồng - nơi mà tính khả thi triển
khai và khả năng mở rộng đóng vai trò không kém phần quan trọng so với độ chính
xác kỹ thuật.

\subsection{Học máy trong phân loại OSA và tư thế ngủ}

Cùng với lượng dữ liệu phong phú và liên tục được thu thập từ các cảm biến, các
thuật toán học máy trở thành công cụ để phân tích và khai thác hiệu quả các
dòng dữ liệu này, giúp đưa ra quyết định nhanh chóng và chính xác. Đặc biệt,
trong bối cảnh các thiết bị theo dõi sức khỏe tại nhà yêu cầu kích thước nhỏ
gọn, có khả năng đeo được, không phụ thuộc vào kết nối Internet, thì học máy
tại biên trở thành như một giải pháp tối ưu vừa mang lại khả năng tính toán
mạnh mẽ, vừa dễ dàng triển khai trên các phần cứng có tài nguyên hạn chế.

\begin{table}[H]
  \centering
  \caption{Các bước chính trong bài toán sử dụng học máy trên biên}
  \label{tab:pipeline_steps}
  \small
  \renewcommand{\arraystretch}{1.2}
  \begin{tabular}{|c|p{3.8cm}|p{9.2cm}|}
    \hline
    \textbf{STT} & \textbf{Giai đoạn}             & \textbf{Mô tả tổng quát}                                                                                                                                                                                                  \\
    \hline
    1            & Thu thập tín hiệu              & Ghi nhận tín hiệu từ cảm biến, xử lý  và lưu trữ.                                                                                                                                                                         \\
    \hline
    2            & Tiền xử lý                     & Lọc nhiễu, phân đoạn theo cửa sổ thời gian, thống kê dữ liệu.                                                                                                                                                             \\
    \hline
    3            & Trích xuất đặc trưng           & Tính toán các đặc trưng thời gian, tần số - đại diện cho nội dung sinh lý trong từng đoạn tín hiệu.                                                                                                                       \\
    \hline
    4            & Lựa chọn và huấn luyện mô hình & Lựa chọn thuật toán học máy phù hợp với bài toán.                                                                                                                                                                         \\
    \hline
    5            & Đánh giá hiệu năng             & Sử dụng các chỉ số đánh giá mô hình như độ chính xác (Accuracy), độ chính xác dự đoán dương (Precision), độ bao phủ (Recall), điểm F1 (F1-Score), diện tích dưới đường cong (AUC) và ma trận nhầm lẫn (Confusion Matrix). \\
    \hline
    6            & Tối ưu mô hình                 & Ứng dụng kỹ thuật cắt tỉa (pruning), lượng tử hóa (quantization) để giảm kích thước và độ phức tạp mô hình nhằm phục vụ triển khai biên.                                                                                  \\
    \hline
    7            & Triển khai thực tế             & Triển khai mô hình trên vi điều khiển, đánh giá và tái huấn luyện.                                                                                                                                                        \\
    \hline
  \end{tabular}
\end{table}

Bảng~\ref{tab:pipeline_steps} trình bày tổng quan quy trình triển khai một bài
toán học máy. Quy trình này có thể được điều chỉnh linh hoạt tùy theo loại tín
hiệu đầu vào và mục đích phân tích cụ thể, chẳng hạn như nhận diện tư thế ngủ,
phát hiện ngưng thở hay theo dõi nhịp hô hấp. Tuy nhiên, nguyên tắc vẫn là đảm
bảo tín hiệu đầu vào có chất lượng cao và mô hình học máy đủ nhẹ để triển khai
trong thực tế.

Phần tiếp theo, luận văn sẽ tập trung trình bày quy trình triển khai nêu trên
vào bài toán phân loại tư thế ngủ và đánh giá OSA.

\subsubsection{Thu thập tín hiệu}

Là bước đầu tiên và đóng vai trò nền tảng trong toàn bộ quy trình tiếp đó. Các
tín hiệu được ghi nhận có thể bao gồm tín hiệu gia tốc trên ba trục để phát
hiện chuyển động và tư thế ngủ; có thể là tín hiệu phản hồi quang học dùng để
đo nhịp tim và độ bão hòa oxy trong máu; có thể là tín hiệu ECG phản ánh hoạt
động điện học của tim; hoặc các tín hiệu hô hấp và áp lực từ cảm biến gắn trên
giường, giúp xác định tư thế ngủ thông qua phân bố trọng lực và các tín hiệu bổ
sung khác \cite{Roebuck2014_sleepSignals, Pan2020_sleepMonitoring}. Tùy theo
dạng dữ liệu thu thập và cấu hình thiết bị, các tín hiệu này được lấy mẫu với
tần số phù hợp \cite{Jeng_osa,Ferrer_osa,osa_sanchez2025,
  Sleep_Posture_Detection,Zhang_osa, Sang, }.

Yêu cầu then chốt trong thu thập dữ liệu là đảm bảo chất lượng của dữ liệu. Các
cảm biến cần được gắn cố định tại các vị trí tối ưu nhằm giảm nhiễu chuyển động
và duy trì tiếp xúc ổn định trong suốt quá trình theo dõi. Hệ thống cũng cần
đảm bảo khả năng truyền dữ liệu hiệu quả thông qua các giao thức không dây
\cite{thuong_wear_paper, hst_wear_paper, Sleep_Posture_Detection}.

Để đảm bảo tính toàn vẹn, thì hệ thống thu thập bao gồm phần
cứng, phần mềm và máy chủ lưu trữ. Tuy nhiên, trong nhiều nghiên cứu hiện nay,
kiến trúc phần mềm cho quá trình thu thập và lưu trữ dữ liệu vẫn chưa được mô
tả một cách đầy đủ hoặc thiếu thông tin về các yếu tố như:
\begin{itemize}
  \item Kiến trúc hệ thống: ứng dụng di động, nền tảng web hoặc máy chủ trung tâm;
  \item Giao thức truyền thông: MQTT, HTTP;
  \item Phương thức lưu trữ dữ liệu: cục bộ, đám mây hoặc cơ sở dữ liệu;
  \item Các vấn đề liên quan đến mã hóa, bảo mật dữ liệu và tuân thủ quy định chuyên
        ngành;
\end{itemize}

Việc thiếu chuẩn hóa trong xây dựng hệ thống có thể ảnh hưởng đáng kể đến khả
năng mở rộng, tích hợp và ứng dụng thực tiễn trong lĩnh vực chăm sóc sức khỏe
tại nhà. Chính vì vậy, luận văn này hướng đến xây dựng một quy trình thu thập
và quản lý tín hiệu thống nhất, bảo mật và sẵn sàng cho triển khai thực tế.

\subsubsection{Tiền xử lý}

Phân tích dữ liệu là bước thống kê lại phần dữ liệu đã thu thập bao gồm các yếu
tố như: độ cân bằng, độ lặp, các giá trị bị khuyết, hay các thống kê khác theo
yêu cầu bài toán. Bước này sẽ cho thấy bức tranh toàn diện nhất về bộ dữ liệu
đã thu thập.

Lọc tín hiệu là nhằm lọc, loại bỏ nhiễu khỏi các tín hiệu có ích. Các loại
nhiễu thường gặp bao gồm: nhiễu điện lưới, nhiễu tần số cao và trôi đường cơ sở
\cite{rossi2023sleep, sheta2021osa}. Các kỹ thuật lọc thường được sử dụng bao
gồm: bộ lọc chặn dải (notch filter) - đặc biệt là bộ lọc notch IIR bậc hai để
loại bỏ nhiễu điện lưới; bộ lọc thông dải (bandpass filter) nhằm giữ lại dải
tần với dữ liệu có ích; và các bộ lọc làm mượt như Butterworth bậc ba; bộ lọc
trung vị (median filter) hoặc bộ lọc trung bình trượt (moving average filter)
\cite{Roebuck2014_sleepSignals, Pan2020_sleepMonitoring}. Trong nghiên cứu của
Sheta và cộng sự \cite{sheta2021osa}, tác giả đã sử dụng bộ lọc notch IIR bậc
hai nhằm loại bỏ nhiễu điện lưới 60~Hz khỏi tín hiệu ECG. Đây là loại nhiễu phổ
biến gây khó khăn cho việc phân tích và trích xuất đặc trưng. Kết quả cho thấy
tín hiệu sau lọc cho chất lượng cao hơn đáng kể và cải thiện hiệu năng của các
mô hình học máy trong chẩn đoán OSA. Đối với tín hiệu gia tốc, các bộ lọc cần
đảm bảo vừa loại bỏ nhiễu vừa bảo toàn các đặc trưng chuyển động. Trong đó, bộ
lọc Kalman và bộ lọc biến thiên toàn phần (total variation filter) được sử dụng
rộng rãi để làm mượt tín hiệu theo thời gian \cite{sun2017sleepmonitor, kalman,
  Cuadrado2021_kalman, Abbasi_filter}. Việc lựa chọn bộ lọc phù hợp cần dựa trên
đặc điểm của từng loại tín hiệu và mục tiêu phân tích cụ thể.

\textbf{Nội suy} là một khâu quan trọng trong quá trình
chuẩn hóa chuỗi thời gian,
nhằm đảm bảo tính đồng nhất về tần số lấy mẫu giữa các kênh dữ liệu và
duy trì tính toàn vẹn của đặc trưng tín hiệu.
Các phương pháp nội suy có thể kể đến như spline
bậc ba (cubic spline), nội suy Hermite đa thức từng đoạn và nội suy tuyến tính
(linear interpolation) thường được áp dụng để điều chỉnh chuỗi dữ liệu về cùng tần số chuẩn, từ đó bảo đảm đầu vào đồng bộ cho các mô
hình học máy hoặc mạng nơ-ron, giúp cải thiện khả năng trích xuất
đặc trưng của tín hiệu \cite{Lepot2017_interpolation}.
Bên cạnh đó, đối với các chuỗi tín hiệu rời rạc như nhịp RR từ ECG hoặc
chuỗi $\mathrm{SpO_2}$ thường xuyên bị gián đoạn hay mất mẫu, kỹ thuật nội
suy đóng vai trò quan trọng trong việc phục hồi dữ liệu bị thiếu
(missing/null values imputation) và tái cấu trúc tín hiệu thành
chuỗi liên tục. Cách tiếp cận này không chỉ nâng cao tính toàn vẹn
của dữ liệu mà còn giúp mô hình học sâu khai thác được các mối quan
hệ trong miền thời gian, dẫn đến hiệu năng suy luận tốt hơn
\cite{zou2024mbtcn}.
Trong nghiên cứu của \cite{olsen2024transfer},
cả hai loại tín hiệu gia tốc và phản hồi quang học đều được nội
suy để tạo thành chuỗi thời gian đồng nhất với tần số lấy mẫu.

Chuẩn hóa dữ liệu giúp đưa các đặc trưng đầu vào về cùng một miền giá trị,
tránh hiện tượng các đặc trưng có giá trị số lớn chi phối quá trình huấn luyện
mô hình. Một số phương pháp phổ biến bao gồm: chuẩn hóa về trung bình 0 và độ
lệch chuẩn 1 (Z-score normalization), đưa về khoảng 0,1 (min-max scaling), dựa
trên trung vị và tứ phân vị, phù hợp với dữ liệu có nhiễu hoặc ngoại lai
\cite{Sleep_Posture_Detection, Vu2025SleepPosition, rossi2023sleep}.

Phân đoạn tín hiệu là khâu tiếp theo trong xử lý tín hiệu. Mục tiêu của bước
này là chia chuỗi dữ liệu liên tục thành các đoạn thời gian ngắn cố định gọi là
cửa sổ (window), giúp mô hình học máy nhận diện hiệu quả các đặc trưng biến đổi
theo thời gian. Thời lượng cửa sổ thường phụ thuộc vào loại tín hiệu và mục
tiêu phân tích: 30 giây cho EEG, 60 giây cho ECG và $\mathrm{SpO_2}$, 5 phút
cho HRV, hoặc các cửa sổ trượt ngắn hơn để phát hiện sự kiện diễn ra trong một
thời gian ngắn \cite{osa_sanchez2025, Sleep_Posture_Detection,
  zou2024mbtcn,HOANG2025116309, }. Trong một số trường hợp, việc phân đoạn còn
dựa vào các đặc điểm sinh lý như đỉnh sóng R trong ECG. Riêng đối với dữ liệu
gia tốc để xác định tư thế ngủ thì cửa sổ phân đoạn là từ 0.5Hz đến 15Hz
\cite{Vu2025SleepPosition,abdulsadig2023}.

Tiền xử không chỉ giúp nâng cao chất lượng dữ liệu mà còn đảm bảo tính nhất
quán đầu vào cho hệ thống học máy. Việc lựa chọn kỹ thuật tiền xử lý cần phù
hợp với đặc điểm của từng loại tín hiệu và mục tiêu phân tích cụ thể.

\subsubsection{Trích xuất đặc trưng}

Trích xuất đặc trưng là một khâu quan trọng trọng nhằm chuyển đổi tín hiệu thô
từ cảm biến thành tập hợp các đặc trưng có ý nghĩa, phản ánh đúng các hiện
tượng vật lý. Quá trình này giúp làm nổi bật các thông tin quan trọng từ tín
hiệu, đồng thời loại bỏ các yếu tố dư thừa hoặc nhiễu không mang giá trị chẩn
đoán.

Trong bài toán phát hiện hội chứng ngưng thở khi ngủ
\cite{hstSurvey,Channa_osa,Sang,HOANG2025116309,Uday,RAJESH2021104199_wavelet},
các đặc trưng được sử dụng phổ biến bao gồm: biến thiên nhịp tim, khoảng RR và
biên độ sóng R từ tín hiệu ECG; các chỉ số thống kê, tần số và entropy từ tín
hiệu $\mathrm{SpO_2}$; hoặc các đặc trưng phi tuyến và miền tần số như năng
lượng phổ EEG. Ngoài ra, các hệ số wavelet thu được từ tín hiệu gốc có khả năng
phản ánh rõ ràng các tín hiệu liên quan đến âm thanh, hình ảnh
\cite{RAJESH2021104199_wavelet}.

Đối với bài toán phân loại tư thế ngủ sử dụng cảm biến gia tốc đơn,
các đặc trưng thường được trích xuất trực tiếp từ miền thời gian.
Trong nghiên cứu \textit{Sleep Posture Monitoring Using a Single Neck-Situated Accelerometer}
\cite{abdulsadig2023}, các đặc trưng được tính toán hoàn toàn trong miền thời gian,
bao gồm giá trị trung bình và trung vị của ba trục gia tốc (X, Y, Z)
cùng với các góc nghiêng giữa các mặt phẳng XY, YZ, và XZ, tạo thành tổng cộng 12
đặc trưng cho mỗi cửa sổ dữ liệu.
Trong khi đó, nghiên cứu \cite{Vu2025SleepPosition, vu2023} không trích xuất đặc trưng thủ công  mà đưa trực tiếp tín hiệu thô từ cảm biến gia tốc $(a_x, a_y,
  a_z)$ vào mô hình LSTM. Mạng học sâu này tự động học các đặc trưng với cửa sổ 20 mẫu, nhờ đó mô hình đạt độ chính xác cao mà không
cần bước trích chọn đặc trưng.

Gán nhãn được thực hiện nhằm liên kết các phân đoạn tín hiệu với nhãn sự kiện
hô hấp tương ứng như “Apnea”, “Hypopnea”, hoặc phân mức độ nặng nhẹ của OSA như
“Mild”, “Moderate”, “Severe”, các nhãn liên quan đến tư thế ngủ. Tùy theo mục
đích nghiên cứu, nhãn có thể được gán thủ công dựa trên chuyên gia hoặc tự động
đồng bộ với thiết bị tham chiếu như camera, hệ thống PSG.

\subsubsection{Lựa chọn và huấn luyện mô hình}

Trong khuôn khổ luận văn sẽ không đi sâu vào phân tích chi tiết các mô hình
liên quan đến hội chứng ngưng thở khi ngủ, mà tập trung vào việc xây dựng và
tối ưu mô hình học máy nhằm phục vụ phân loại tư thế ngủ một cách hiệu quả và
khả thi khi triển khai trên thiết bị biên.

\begin{table}[htbp]
  \centering
  \caption{Hiệu quả mô hình học máy và khả năng triển khai biên trong nhận diện tư thế ngủ}
  \label{tab:sleepml}
  \small
  \renewcommand{\arraystretch}{1.2}
  \resizebox{\textwidth}{!}{
    \fontsize{16pt}{16pt}\selectfont % 
    \begin{tabular}{|c|p{3.2cm}|p{5cm}|p{6cm}|p{3cm}|c|c|p{1.2cm}|}
      \hline
      \textbf{Tài liệu}              & \textbf{Dữ liệu} & \textbf{Nhiệm vụ}                     & \textbf{Đặc trưng}                                     & \textbf{Mô hình}               & \textbf{Độ chính xác} & \textbf{Phần mềm}           & \textbf{On-chip} \\
      \hline
      \cite{abdulsadig2023}          & 18 đối tượng     & 4 tư thế ngủ                          & 12 đặc trưng miền thời gian                            & DT, Extra-Trees, LSTM-NN       & $>$98\%               & BLE -> laptop               & Có               \\
      \cite{Jeng_osa}                & NM               & 4 tư thế ngủ  + 2 tư thế ngồi và đứng & 3 đặc trưng chính - trung bình (mean)                  & RF, SVM                        & 80-90\%               & BLE -> điện thoại -> laptop & NM               \\
      \cite{vu2023}                  & 561.859 mẫu      & 4 tư thế ngủ và không phải            & dữ liệu thô                                            & LSTM\_sq                       & $>$99\%               & NM                          & NM               \\
      \cite{kwasnicki2018}           & 16 đối tượng     & 8 tư thế ngủ                          & Trung bình, phương sai của 3 trục                      & Ngưỡng                         & lên tới 99.5\%        & NM                          & NM               \\
      \cite{elnaggar2023}            & 5 đối tượng      & 12 tư thế ngủ                         & động học khớp cổ tay, mô tả hướng quay và biên độ xoay & Ngưỡng                         & $>$99.2\%             & cảm biến -> PC              & Không            \\
      \cite{Sleep_Posture_Detection} & 18 đối tượng     & 4 tư thế ngủ                          & 12 đặc trưng miền thời gian                            & Ngưỡng , Extra-Trees           & $>$95\%               & cảm biến -> PC              & Không            \\
      \cite{Yoon2015_posture}        & 13 đối tượng     & 4 tư thế ngủ và không phải            & trung bình giá trị gia tốc ba trục                     & Ngưỡng                         & $>$96\%               & cảm biến -> PC              & Không            \\
      \cite{Zhang_osa}               & 7 đối tượng      & 4 tư thế ngủ                          & trung bình giá trị gia tốc ba trục                     & Linear Discriminant Analysis   & $>$99\%               & cảm biến -> PC              & Không            \\
      \cite{HE2024e31839_posture}    & 15 đối tượng     & 4 tư thế ngủ                          & 40 đặc trưng thời gian, tần số                         & DT, RF, SVM, XGBoost, AdaBoost & 95.65\%               & cảm biến -> PC              & Không            \\
      \hline
      \multicolumn{8}{l}{\footnotesize NM = Không đề cập}
    \end{tabular}
  }
\end{table}

Dữ liệu trong Bảng~\ref{tab:sleepml} cho thấy sự đa dạng trong cách tiếp cận
bài toán nhận diện tư thế ngủ. Hầu hết các nghiên cứu đạt độ chính xác cao
(trên 95\%), ngay cả khi sử dụng các thuật toán đơn giản như ngưỡng định sẵn
hoặc mô hình học máy truyền thống. Tuy nhiên, một điểm đáng lưu ý là chỉ duy
nhất nghiên cứu của Abdulsadig và cộng sự \cite{abdulsadig2023} thực sự được
triển khai trên phần cứng có tài nguyên hạn chế. Các nghiên cứu còn lại chủ yếu
dừng ở mức mô phỏng phần mềm hoặc chạy trên máy tính.

Nhóm mô hình học sâu LSTM hoặc kết hợp nhiều lớp tuy đạt độ chính xác vượt trội
(trên 99\%) như trong nghiên cứu \cite{vu2023}, nhưng đều chưa tích hợp trực
tiếp lên vi điều khiển do dung lượng bộ nhớ lớn và nhu cầu tính toán cao. Ngược
lại, các mô hình truyền thống như DT, RF hay ngưỡng cố định cho thấy hiệu quả
không hề thua kém khi được khai thác đúng đặc trưng. Đặc biệt, việc sử dụng các
đặc trưng miền thời gian dễ tính toán so với miền tần số, giúp giảm đáng kể độ
phức tạp tính toán \cite{gomes2021, souza2017}. Điều này rất phù hợp với yêu
cầu thiết kế của TinyML: đơn giản, hiệu quả và khả thi khi triển khai thực tế
trên các nền tảng phần cứng hạn chế.

Nhờ sự phát triển của các nền tảng hỗ trợ triển khai mô hình học máy như
TensorFlow Lite for Microcontrollers, Edge Impulse, và TinyML EON Compiler,
việc huấn luyện, chuyển đổi và triển khai các mô hình đã trở nên dễ tiếp cận
hơn nhiều. Điều này mở ra hướng mới trong chẩn đoán y tế cá nhân hóa, thông qua
thiết bị đeo thông minh hoạt động liên tục tại nhà.

Tiếp theo, luận văn được trình bày theo hai giai đoạn: (i) \emph{Đánh giá đặc
  trưng và mô hình trên dữ liệu thu thập} - xây dựng lộ trình triển khai từ tiền
xử lý, phân đoạn theo cửa sổ, trích xuất đặc trưng đại diện (miền thời gian và
tần số) đến so sánh các mô hình học máy. Tiêu chí lựa chọn không chỉ dựa trên
độ chính xác, độ chụm cho từng tư thế, mà còn cân nhắc tới kích thước mô hình,
chi phí tính toán. (ii) \emph{Triển khai on-chip (TinyML) trên vi điều khiển} -
cụ thể triển khai trên vi điều khiển nRF52840, đồng thời kiểm chứng so với giai
đoạn (i).

