\begin{lstlisting}[float,language=JavaScript,caption={Ví dụ phương thức API ($searchMasterData()$) gọi đến hai phương thức khác, trong đó có một phương thức ($get()$) cần được tạo hàm giả khi thực thi kiểm thử}, label=api_example,captionpos=b]
async get(req: RequestModel): Promise<any> { //Accessing database (*@\label{declare_get_method}@*)
    return await this.httpService
      .post<any>(Utils.join(this.apiUrl, this.config.CORE_GET), req) (*@\label{call_http_service}@*)
      .toPromise(); (*@\label{get_method_toPromise}@*)}
      
@Post('search') //The tested API
async searchMasterData(@Body() body: RequestModel<BaseDto>): Promise<any> { (*@\label{declare_api_method}@*)
    // Hidden code...
    await this.checkRequired(body.condition['type']); (*@\label{lst_exp_check_require}@*)
    let reqCon = body.condition;
    switch (reqCon['type']) { (*@\label{lst_exp_switch_stmt}@*)
        case "MASTER":(*@\label{lst_exp_case_branch}@*)
            if (!(type of reqCon["keyword"] === "string") (*@\label{lst_exp_if_stmt}@*)) {
                (*@\label{lst_exp_keyword_branch}@*)return ResponseModel(RESULT_STATUS.ERROR);}
            const ret = await this.get(reqCon);(*@\label{lst_exp_start_uncover}@*)
            if (ret.status == 200) { (*@\label{lst_exp_200}@*) 
                // Hidden code... (*@\label{lst_exp_hidden_code}@*)
                return  new ResponseModel(RESULT_STATUS.OK, ret); (*@\label{lst_exp_end_uncover}@*)}
            return new ResponseModel(RESULT_STATUS.ERROR);(*@\label{lst_exp_return_error}@*)
        //Other cases....
        default: (*@\label{lst_exp_default_branch}@*)
            return new ResponseModel(RESULT_STATUS.ERROR);}
}

checkRequired(s: string): Promise<any> { //A normal method (*@\label{declare_check_required_method}@*)
    if (s == null (*@\label{check_null_equal}@*)) {
        this.errors.push("error message");}
}
\end{lstlisting}

% \begin{lstlisting}[float,language=JavaScript,caption={An example presents one caller ($search()$) and two callees ($get()$ and $check()$), in which the first callee ($get()$) needs to be mocked.}, label=api_example,captionpos=b]
% @Post('search')
% async search(body: Dto): any { (*@\label{declare_api_method}@*)
%     this.check(body['type']); (*@\label{lst_exp_check_require}@*)
%     switch (body['type']) { (*@\label{lst_exp_switch_stmt}@*)
%         case "MASTER":(*@\label{lst_exp_case_branch}@*)
%             if (body["keyword"] is not a string)(*@\label{lst_exp_if_stmt}@*)) 
%                 return ...
%             const ret = await this.get(reqCon);(*@\label{lst_exp_start_uncover}@*)
%             if (ret.status == 200 (*@\label{lst_exp_200}@*)) 
%                 return ... (*@\label{lst_exp_end_uncover}@*)
%             else  return ...(*@\label{lst_exp_return_error}@*)
%     }
% }
% //A method gets data from database
% async get(req): any (*@\label{declare_get_method}@*){ ...}
% check(s: string): any (*@\label{declare_check_required_method}@*) {  //A normal callee
%     if (s == null (*@\label{check_null_equal}@*))  
%         this.errors.push("error message");}
% \end{lstlisting}
% % return await this.httpService
% %       .post<any>(Utils.join(this.apiUrl, this.config.CORE_GET), req) (*@\label{call_http_service}@*).toPromise(); (*@\label{get_method_toPromise}@*)
% \begin{lstlisting}[float,language=JavaScript,caption=An example of mocking $get()$ method, label=mock_example,captionpos=b]
% it("first test", (done) => {
%     // generated input
%     const body = {...}; (*@\label{input_declaration}@*)
%     // initialize mock data
%     const response: AxiosResponse < any > = {(*@\label{mock_declare_response}@*) status: 200,
%         data: {}, headers: {}, statusText: 'OK', config: {url:'x'}};
%     // set mock data to method accessing database
%     jest.spyOn(controller_class, "get").mockResolvedValue(response); (*@\label{mock_spy}@*)
%     // call the tested API
%     return request(server).post("/search") (*@\label{begin_test_driver}@*)
%     .send(body)  (*@\label{end_test_driver}@*).expect(201).then(res => {...})
% });
% \end{lstlisting}