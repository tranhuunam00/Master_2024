%-------TITLE PAGE------%
\begin{titlepage}
	\center
	\begin{tikzpicture}[overlay,remember picture]
		\draw [line width=3pt,rounded corners=0pt,]
		($ (current page.north west) + (25mm,-25mm) $)
		rectangle
		($ (current page.south east) + (-15mm,25mm) $);
		\draw [line width=1pt,rounded corners=0pt]
		($ (current page.north west) + (26.5mm,-26.5mm) $)
		rectangle
		($ (current page.south east) + (-16.5mm,26.5mm) $);
	\end{tikzpicture}

	{\large \bfseries ĐẠI HỌC QUỐC GIA HÀ NỘI\\ TRƯỜNG ĐẠI HỌC CÔNG NGHỆ}\\[1cm]
	\includegraphics[width=0.25\linewidth]{images/Logo_UET.png}\\[1cm]
	{\Large  \bfseries Trần Hữu Nam}\\[1cm]
	% 	{ \LARGE \bfseries  XÂY DỰNG CÔNG CỤ KIỂM THỬ TỰ ĐỘNG }\\[0.05cm]
	%     {\LARGE \bfseries CHO CÁC ỨNG DỤNG SỬ DỤNG NGÔN NGỮ}\\[0.2cm]
	%      { \LARGE \bfseries   TYPESCRIPT}\\[0.5cm]
	% 	\hfill\\[2cm]
	{ \LARGE \bfseries
	NGHIÊN CỨU, PHÁT TRIỂN MÔ HÌNH HỌC MÁY TẠI BIÊN NHẰM PHÂN LOẠI TƯ THẾ NGỦ}\\[0.05cm]
	\hfill\\[1cm]
	{\large \bfseries LUẬN VĂN THẠC SĨ}\\
	% {\large \bfseries Ngành: Kĩ thuật điện tử}	
	\hfill\\[5.3cm]
	{\large \bfseries HÀ NỘI - 2025}\\
	\vfill
\end{titlepage}

%-------TITLE PAGE+6hbk,------%
\begin{titlepage}
	\center
	\begin{tikzpicture}[overlay,remember picture]
		\draw [line width=3pt,rounded corners=0pt,]
		($ (current page.north west) + (25mm,-25mm) $)
		rectangle
		($ (current page.south east) + (-15mm,25mm) $);
		\draw [line width=1pt,rounded corners=0pt]
		($ (current page.north west) + (26.5mm,-26.5mm) $)
		rectangle
		($ (current page.south east) + (-16.5mm,26.5mm) $);
	\end{tikzpicture}

	{\large \bfseries ĐẠI HỌC QUỐC GIA HÀ NỘI\\ TRƯỜNG ĐẠI HỌC CÔNG NGHỆ}\\[2cm]
	% 	\includegraphics[width=0.25\linewidth]{images/Logo_UET.png}\\[1cm]

	{\Large  \bfseries Trần Hữu Nam}\\[2cm]
	{ \LARGE \bfseries  NGHIÊN CỨU, PHÁT TRIỂN MÔ HÌNH HỌC MÁY TẠI BIÊN NHẰM PHÂN LOẠI TƯ THẾ NGỦ
	}\\[0.05cm]
	\hfill\\[1cm]
	\begin{flushleft}
		{\large \bfseries Ngành: Điện tử viễn thông}\\
		{\large \bfseries Chuyên ngành: Kĩ thuật điện tử}\\
		{\large \bfseries Mã số học viên: 23025029}\\

	\end{flushleft}
	\hfill\\[1cm]
	{\large \bfseries LUẬN VĂN THẠC SĨ}\\
	\hfill\\[1cm]

	{\large \bfseries NGƯỜI HƯỚNG DẪN KHOA HỌC: PGS.TS. Mai Anh Tuấn}\\

	\begin{flushleft}
		% 	{\large \bfseries Cán bộ đồng hướng dẫn: CN. Bùi Quang Cường}\\	
	\end{flushleft}
	\hfill\\[3cm]
	{\large \bfseries HÀ NỘI - 2025}\\
	\vfill
\end{titlepage}

\changefontsizes[16pt]{14pt}
\addtocontents{toc}{\vspace{-1cm}}
\addcontentsline{toc}{chapter}{Lời cam đoan}
\begin{center}
	\textbf{LỜI CAM ĐOAN}
\end{center}
Tôi xin cam đoan: luận văn thạc sĩ
“Nghiên cứu, phát triển mô hình học máy tại biên nhằm phân loại tư thế ngủ”
là công trình nghiên cứu của tôi dưới sự hướng dẫn của
thầy \textbf{PGS. TS. Mai Anh Tuấn} và thầy \textbf{ThS. Trần Ngọc Thái}
cùng với sự tham khảo từ những tài liệu đã liệt kê trong mục Tài liệu tham khảo. Tôi không sao chép công trình nghiên cứu của cá nhân khác dưới bất kỳ hình thức nào. Nếu có tôi xin hoàn toàn chịu trách nhiệm.

\begin{flushright}
	\begin{varwidth}{\linewidth}\centering
		Hà Nội, ngày \space\space\space\space tháng  \space\space\space\space năm 2025\\
		Học viên\\[2cm]
		Trần Hữu Nam
	\end{varwidth}
\end{flushright}

\newpage

\addcontentsline{toc}{chapter}{Lời cảm ơn}
\begin{center}
	\textbf{LỜI CẢM ƠN}
\end{center}

Lời đầu tiên, tôi xin gửi lời cảm ơn đến thầy PGS.TS. Mai Anh Tuấn và thầy ThS.
Trần Ngọc Thái vì đã tận tình hướng dẫn, truyền đạt kiến thức cho tôi trong
suốt quá trình học tập và thực hiện luận văn. Tôi xin cảm ơn tập thể thầy, cô
khoa “Điện tử Viễn thông”, Trường Đại học Công nghệ - ĐHQGHN, đã giảng dạy tôi
trong quá trình tôi học tập tại trường. Tôi cũng xin cảm ơn các anh chị ở Bộ
môn Công nghệ Vi cơ Điện tử và Kỹ thuật Máy tính đã tạo điều kiện giúp đỡ, chỉ
bảo tôi trong thời gian làm luận văn. Cuối cùng, tôi xin cảm ơn bố mẹ, gia đình
cũng như bạn bè, tập thể lớp K30 đã luôn đồng hành, chia sẻ và động viên tôi
suốt thời gian qua.

\newpage
\addcontentsline{toc}{chapter}{Tóm tắt}
\begin{center}
	\textbf{TÓM TẮT}
\end{center}
% \changefontsizes[16pt]{12pt}
\textit{\textbf{Tóm tắt: }}
Ngưng thở tắc nghẽn khi ngủ (Obstructive Sleep Apnea – OSA) là một rối loạn
hô hấp khi ngủ do sự hẹp hoặc tắc nghẽn một phần hay toàn bộ đường hô hấp
trên, bao gồm vùng mũi họng, hầu họng hoặc cả hai. Tình trạng này được đặc
trưng bởi các cơn ngưng thở hoặc giảm thở ngắn, lặp đi lặp lại trong khi ngủ,
gây gián đoạn giấc ngủ do thức giấc thường xuyên và dẫn đến giảm oxy máu
từng đợt\cite{Epstein2009}. Nghiên
cứu\cite{Salari2025} đã phân tích dữ liệu từ 15 nghiên cứu với tổng cộng 42.924
người. Kết quả cho thấy tỷ lệ mắc chứng buồn ngủ ban ngày quá mức (EDS) ở bệnh
nhân ngưng thở tắc nghẽn khi ngủ (OSA) trên toàn cầu là 39,9\% (khoảng tin cậy
95\%: 34,4\%–45,7\%).

Tư thế ngủ đóng vai trò quan trọng trong việc khởi phát và làm trầm trọng thêm
các triệu chứng của OSA. Đặc biệt, tư thế nằm ngửa thường làm tăng mức độ tắc
nghẽn đường hô hấp so với các tư thế khác, dẫn đến dạng OSA phụ thuộc tư thế
(positional OSA). Trong bối cảnh đó, việc theo dõi chính xác tư thế ngủ trong
thời gian thực có thể cung cấp thông tin quan trọng phục vụ chẩn đoán sớm, đánh
giá nguy cơ và cải thiện tình trạng OSA bằng việc thay đổi tư thế ngủ.

Trong bối cảnh công nghệ chế tạo ngày càng phát triển mạnh mẽ, việc thu nhỏ và
tối ưu hiệu suất vi điều khiển, cảm biến, pin đã trở thành một yếu tố then chốt
trong quá trình tích hợp chúng vào các thiết bị điện tử có kích thước nhỏ gọn.
Không chỉ góp phần nâng cao độ chính xác trong việc đo lường các thông số sinh
lý quan trọng, nó còn giúp giảm kích thước thiết bị, tăng tính di động và mở
rộng khả năng ứng dụng trong nhiều lĩnh vực, đặc biệt nổi bật trong y học cá
thể hóa.

Bên cạnh đó, sự phát triển vượt bậc của trí tuệ nhân tạo (AI) đóng vai trò quan
trọng trong việc nâng cao hiệu quả khai thác dữ liệu cảm biến. AI không chỉ tối
ưu hóa quy trình xử lý và phân tích dữ liệu mà còn thúc đẩy khả năng phân loại,
phân cụm, dự đoán và đưa ra quyết định. Trong bài toán chẩn đoán hội chứng
ngưng thở khi ngủ, sự phối hợp giữa dữ liệu từ cảm biến và thuật toán học máy
không chỉ đảm bảo độ chính xác cao trong thu thập dữ liệu mà còn mở ra khả năng
phân tích chuyên sâu về các yếu tố sinh lý, phục vụ quá trình đánh giá toàn
diện. Đặc biệt, sự nổi lên của lĩnh vực học máy triển khai trực tiếp trên các
vi điều khiển hoặc thiết bị biên có tài nguyên hạn chế (Tiny Machine Learning -
TinyML) đã đánh dấu bước tiến quan trọng trong việc hiện thực hóa các hệ thống
giám sát sức khỏe thuận tiện, tiết kiệm năng lượng, chi phí thấp và có khả năng
hoạt động độc lập không phụ thuộc vào kết nối mạng hoặc nền tảng đám mây.

Trên cơ sở đó, luân văn này tập trung nghiên cứu, phát triển hệ thống thu thập,
mô hình học máy biên nhằm phân loại tư thế ngủ, sử dụng đơn dữ liệu cảm biến
gia tốc. Luận văn được thực hiện qua hai giai đoạn chính. Giai đoạn thứ nhất
tập trung vào việc nghiên cứu và phát triển hệ thống thu thập, xử lý, lưu trữ
dữ liệu cảm biến; đồng thời tiến hành phân tích, trích xuất đặc trưng và huấn
luyện, đánh giá một số mô hình học máy phù hợp cho bài toán phân loại tư thế
ngủ. Giai đoạn thứ hai tập trung vào nghiên cứu và phát triển phần cứng riêng,
cũng như triển khai mô hình học máy đã lựa chọn lên nền tảng phần cứng đó.

\vspace{-0.5cm}
\begin{flushleft}
	\textit{\textbf{Từ khóa: } cảm biến gia tốc, học máy, ngưng thở tắc nghẽn khi ngủ,  Tiny ML}
\end{flushleft}
% \changefontsizes[16pt]{13pt}

\changefontsizes[16pt]{13pt}
