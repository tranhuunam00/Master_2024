%-------TITLE PAGE------%
\begin{titlepage}
	\center
	\begin{tikzpicture}[overlay,remember picture]
		\draw [line width=3pt,rounded corners=0pt,]
		($ (current page.north west) + (25mm,-25mm) $)
		rectangle
		($ (current page.south east) + (-15mm,25mm) $);
		\draw [line width=1pt,rounded corners=0pt]
		($ (current page.north west) + (26.5mm,-26.5mm) $)
		rectangle
		($ (current page.south east) + (-16.5mm,26.5mm) $);
	\end{tikzpicture}

	{\large \bfseries ĐẠI HỌC QUỐC GIA HÀ NỘI\\ TRƯỜNG ĐẠI HỌC CÔNG NGHỆ}\\[1cm]
	\includegraphics[width=0.25\linewidth]{images/Logo_UET.png}\\[1cm]
	{\Large  \bfseries Trần Hữu Nam}\\[1cm]
	% 	{ \LARGE \bfseries  XÂY DỰNG CÔNG CỤ KIỂM THỬ TỰ ĐỘNG }\\[0.05cm]
	%     {\LARGE \bfseries CHO CÁC ỨNG DỤNG SỬ DỤNG NGÔN NGỮ}\\[0.2cm]
	%      { \LARGE \bfseries   TYPESCRIPT}\\[0.5cm]
	% 	\hfill\\[2cm]
	{ \LARGE \bfseries
	NGHIÊN CỨU, PHÁT TRIỂN MÔ HÌNH HỌC MÁY TẠI BIÊN NHẰM PHÂN LOẠI TƯ THẾ NGỦ}\\[0.05cm]
	\hfill\\[1cm]
	{\large \bfseries LUẬN VĂN THẠC SĨ}\\
	% {\large \bfseries Ngành: Kĩ thuật điện tử}	
	\hfill\\[5.3cm]
	{\large \bfseries HÀ NỘI - 2025}\\
	\vfill
\end{titlepage}

%-------TITLE PAGE+6hbk,------%
\begin{titlepage}
	\center
	\begin{tikzpicture}[overlay,remember picture]
		\draw [line width=3pt,rounded corners=0pt,]
		($ (current page.north west) + (25mm,-25mm) $)
		rectangle
		($ (current page.south east) + (-15mm,25mm) $);
		\draw [line width=1pt,rounded corners=0pt]
		($ (current page.north west) + (26.5mm,-26.5mm) $)
		rectangle
		($ (current page.south east) + (-16.5mm,26.5mm) $);
	\end{tikzpicture}

	{\large \bfseries ĐẠI HỌC QUỐC GIA HÀ NỘI\\ TRƯỜNG ĐẠI HỌC CÔNG NGHỆ}\\[2cm]
	% 	\includegraphics[width=0.25\linewidth]{images/Logo_UET.png}\\[1cm]

	{\Large  \bfseries Trần Hữu Nam}\\[2cm]
	{ \LARGE \bfseries  NGHIÊN CỨU, PHÁT TRIỂN MÔ HÌNH HỌC MÁY TẠI BIÊN NHẰM PHÂN LOẠI TƯ THẾ NGỦ
	}\\[0.05cm]
	\hfill\\[1cm]
	\begin{flushleft}
		{\large \bfseries Ngành: Điện tử viễn thông}\\
		{\large \bfseries Chuyên ngành: Kĩ thuật điện tử}\\
		{\large \bfseries Mã số học viên: 23025029}\\

	\end{flushleft}
	\hfill\\[1cm]
	{\large \bfseries LUẬN VĂN THẠC SĨ}\\
	\hfill\\[1cm]

	{\large \bfseries NGƯỜI HƯỚNG DẪN KHOA HỌC: PGS.TS. Mai Anh Tuấn}\\

	\begin{flushleft}
		% 	{\large \bfseries Cán bộ đồng hướng dẫn: CN. Bùi Quang Cường}\\	
	\end{flushleft}
	\hfill\\[3cm]
	{\large \bfseries HÀ NỘI - 2025}\\
	\vfill
\end{titlepage}

\changefontsizes[16pt]{14pt}
\addtocontents{toc}{\vspace{-1cm}}
\addcontentsline{toc}{chapter}{Lời cam đoan}
\begin{center}
	\textbf{LỜI CAM ĐOAN}
\end{center}
Tôi xin cam đoan: luận văn thạc sĩ
“Nghiên cứu, phát triển mô hình học máy tại biên nhằm phân loại tư thế ngủ”
là công trình nghiên cứu của tôi dưới sự hướng dẫn của
thầy \textbf{PGS. TS. Mai Anh Tuấn} và thầy \textbf{ThS. Trần Ngọc Thái}
cùng với sự tham khảo từ những tài liệu đã liệt kê trong mục Tài liệu tham khảo. Tôi không sao chép công trình nghiên cứu của cá nhân khác dưới bất kỳ hình thức nào. Nếu có tôi xin hoàn toàn chịu trách nhiệm.

\begin{flushright}
	\begin{varwidth}{\linewidth}\centering
		Hà Nội, ngày \space\space\space\space tháng  \space\space\space\space năm 2025\\
		Học viên\\[2cm]
		Trần Hữu Nam
	\end{varwidth}
\end{flushright}

\newpage

\addcontentsline{toc}{chapter}{Lời cảm ơn}
\begin{center}
	\textbf{LỜI CẢM ƠN}
\end{center}

Lời đầu tiên, tôi xin gửi lời cảm ơn đến thầy PGS.TS. Mai Anh Tuấn và thầy ThS.
Trần Ngọc Thái vì đã tận tình hướng dẫn, truyền đạt kiến thức cho tôi trong
suốt quá trình học tập và thực hiện luận văn. Tôi xin cảm ơn tập thể thầy, cô
khoa “Điện tử Viễn thông”, Trường Đại học Công nghệ - ĐHQGHN, đã giảng dạy tôi
trong quá trình tôi học tập tại trường. Tôi cũng xin cảm ơn các anh chị ở Bộ
môn Công nghệ Vi cơ Điện tử và Kỹ thuật Máy tính đã tạo điều kiện giúp đỡ, chỉ
bảo tôi trong thời gian làm luận văn. Cuối cùng, tôi xin cảm ơn bố mẹ, gia đình
cũng như bạn bè, tập thể lớp K30 đã luôn đồng hành, chia sẻ và động viên tôi
suốt thời gian qua.

\newpage
\addcontentsline{toc}{chapter}{Tóm tắt}
\begin{center}
	\textbf{TÓM TẮT}
\end{center}
% \changefontsizes[16pt]{12pt}
\textit{\textbf{Tóm tắt: }}

Hội chứng ngưng thở khi ngủ do tắc nghẽn (Obstructive Sleep Apnea – OSA) được
đặc trưng bởi sự tắc nghẽn một phần hoặc hoàn toàn của đường hô hấp trên kéo
dài ít nhất 10 giây, trong khi các nỗ lực hô hấp vẫn tiếp tục thông qua cử động
của lồng ngực và bụng \cite{Epstein2009}. Trong nghiên cứu
\cite{Benjafield2019}, tác giả chỉ ra có gần 1 tỷ người mắc OSA và theo GS.BS.
Dương Quý Sỹ, Việt Nam có tới 8.5\% người trưởng thành mắc OSA
\cite{nguoimacOSA_VN}. OSA có ảnh hưởng lớn tới sức khỏe cả thể chất lẫn tinh
thần đối với người mắc. Xia Wang trong nghiên cứu của mình có tới 25.760 người
tham gia đã kết luận rằng: mỗi khi chỉ số ngưng–giảm thở ( Apnea-Hypopnea Index
- AHI) tăng thêm 10 đơn vị, nguy cơ mắc bệnh tim mạch tăng 17\%
\cite{Wang2013_tim}. Như trong nghiên cứu\cite{Salari2025}, nhóm tác giả đã
phân tích dữ liệu từ 15 nghiên cứu với tổng cộng 42.924 người và kết quả cho
thấy tỷ lệ mắc chứng buồn ngủ ban ngày quá mức (Excessive Daytime Sleepiness –
EDS) ở bệnh nhân ngưng thở tắc nghẽn khi ngủ trên toàn cầu là 39,9\%.

Hội chứng ngưng thở khi ngủ do tắc nghẽn chủ yếu do béo phì, đặc điểm hình thái
vùng sọ–mặt, và bất thường cấu trúc đường hô hấp trên gây hẹp đường thở khi ngủ
\cite{Young2004_nguyen_nhan}. Tác giả cũng đưa ra thêm các yếu tố nguy cơ bổ
sung gồm hút thuốc, uống rượu, nghẹt mũi ban đêm, thay đổi nội tiết tố sau mãn
kinh, và di truyền, góp phần làm tăng khả năng xuất hiện và mức độ nặng của
OSA.Tư thế ngủ đóng vai trò quan trọng trong việc khởi phát và làm trầm trọng
thêm các triệu chứng của OSA \cite{Menon2013_position}. Hội chứng ngưng thở khi
ngủ do tắc nghẽn tư thế (POSA) được Cartwright mô tả lần đầu tiên vào năm 1984,
ở những người có các đợt ngưng thở và giảm thở xảy ra chủ yếu khi nằm ngửa
\cite{Cartwright1984}.

Để đánh giá chính xác tình trạng OSA, hiện nay đa ký giấc ngủ
(Polysomnography - PSG) là phương pháp ghi đồng thời nhiều tín hiệu sinh lý gồm
EEG, EOG, EMG, ECG, nồng độ oxy trong máu và các thông số hô hấp
nhằm đánh giá toàn diện hoạt động của cơ thể trong suốt
giấc ngủ \cite{Markun2020_psg}. Trong đó, tư thế ngủ luôn là dữ liệu quan trọng
để đưa ra kết luận về tình trạng OSA và phương pháp điều trị hợp lý.
Phương pháp này được xem là tiêu chuẩn vàng trong chẩn đoán các rối loạn giấc ngủ,
đặc biệt là ngưng thở khi ngủ, rối loạn vận động, động kinh và hành vi bất thường
trong giấc ngủ REM. Tuy nhiên, theo tìm hiểu của tác giả, tại Việt Nam,
chi phí cho một lần thực hiện đa ký giấc ngủ (PSG) còn khá cao,
do người bệnh phải lưu trú qua đêm tại phòng đo đạt chuẩn và được theo dõi trực
tiếp bởi kỹ thuật viên.
Bên cạnh đó, thời gian chờ đặt lịch lâu, cùng việc phải gắn nhiều điện cực
trong khi ngủ gây nhiều bất tiện cho người bệnh, đồng thời có thể làm
sai lệch hoặc gián đoạn tín hiệu ghi nhận trong quá trình đo.

Trong bối cảnh công nghệ chế tạo ngày càng phát triển mạnh mẽ, việc thu nhỏ và
tối ưu hiệu suất vi điều khiển, cảm biến, pin. Bên cạnh đó, sự phát triển vượt
bậc của trí tuệ nhân tạo (AI) đóng vai trò quan trọng trong việc nâng cao hiệu
quả khai thác dữ liệu cảm biến. AI không chỉ tối ưu hóa quy trình xử lý và phân
tích dữ liệu mà còn thúc đẩy khả năng phân loại, phân cụm, dự đoán và đưa ra
quyết định. Đặc biệt, học máy tại biên (Edge machine learning) nổi lên để giải
quyết các bài toán cần đưa ra quyết định nhanh, nhưng đòi hỏi độ chính xác và
triển khai trên các thiết bị phần cứng có tài nguyên hạn chế đi kèm mức tiêu
thụ năng lượng thấp.

Trên cơ sở đó, tác giả quyết định lựa chọn đề tài là nghiên cứu, phát triển mô
hình học máy tại biên nhằm phân loại tư thế ngủ với ba mục tiêu chính bao gồm:
1) Nghiên cứu, đề xuất hệ thống phần cứng phục vụ đo lường, thu thập, xử lý tín
hiệu gia tốc kèm với hiệu năng phù hợp cho việc triển khai mô hình học máy tại
biên trong bài toán phân loại tư thế ngủ ở người; 2) Nghiên cứu, đề xuất những
mô hình học máy thích hợp trong việc phân loại tư thế ngủ ở người dựa trên
những đặc trưng của tín hiệu gia tốc được trích xuất từ bộ dữ liệu cảm biến do
nhóm nghiên cứu thu thập. 3) Chuẩn hóa, thực thi, đánh giá mô hình học máy trên
vi điều khiển Nordic nRF52840, nhằm kiểm chứng hoạt động thực tế của hệ thống
nhận dạng tư thế ngủ tại biên

\vspace{-0.5cm}
\begin{flushleft}
	\textit{\textbf{Từ khóa: } cảm biến gia tốc, học máy, ngưng thở tắc nghẽn khi ngủ,  Tiny ML}
\end{flushleft}
% \changefontsizes[16pt]{13pt}

\changefontsizes[16pt]{13pt}
