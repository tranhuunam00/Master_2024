%-------TITLE PAGE------%
\begin{titlepage}
	\center
	\begin{tikzpicture}[overlay,remember picture]
		\draw [line width=3pt,rounded corners=0pt,]
		($ (current page.north west) + (25mm,-25mm) $)
		rectangle
		($ (current page.south east) + (-15mm,25mm) $);
		\draw [line width=1pt,rounded corners=0pt]
		($ (current page.north west) + (26.5mm,-26.5mm) $)
		rectangle
		($ (current page.south east) + (-16.5mm,26.5mm) $);
	\end{tikzpicture}
	
	{\large \bfseries ĐẠI HỌC QUỐC GIA HÀ NỘI\\ TRƯỜNG ĐẠI HỌC CÔNG NGHỆ}\\[1cm]
	\includegraphics[width=0.25\linewidth]{images/Logo_UET.png}\\[1cm]
	{\Large  \bfseries Trần Hữu Nam}\\[1cm]
% 	{ \LARGE \bfseries  XÂY DỰNG CÔNG CỤ KIỂM THỬ TỰ ĐỘNG }\\[0.05cm]
%     {\LARGE \bfseries CHO CÁC ỨNG DỤNG SỬ DỤNG NGÔN NGỮ}\\[0.2cm]
%      { \LARGE \bfseries   TYPESCRIPT}\\[0.5cm]
% 	\hfill\\[2cm]
	{ \LARGE \bfseries  
        NGHIÊN CỨU, PHÁT TRIỂN THIẾT BỊ XÁC ĐỊNH TƯ THẾ NGỦ SỬ DỤNG CẢM BIẾN GIA TỐC}\\[0.05cm]
	\hfill\\[1cm]
	{\large \bfseries LUẬN VĂN THẠC SĨ}\\	
	% {\large \bfseries Ngành: Kĩ thuật điện tử}	
	\hfill\\[5.3cm]	
	{\large \bfseries HÀ NỘI - 2025}\\	
	\vfill
\end{titlepage}

%-------TITLE PAGE+6hbk,------%
\begin{titlepage}
	\center
	\begin{tikzpicture}[overlay,remember picture]
	\draw [line width=3pt,rounded corners=0pt,]
	($ (current page.north west) + (25mm,-25mm) $)
	rectangle
	($ (current page.south east) + (-15mm,25mm) $);
	\draw [line width=1pt,rounded corners=0pt]
	($ (current page.north west) + (26.5mm,-26.5mm) $)
	rectangle
	($ (current page.south east) + (-16.5mm,26.5mm) $);
	\end{tikzpicture}
	
	{\large \bfseries ĐẠI HỌC QUỐC GIA HÀ NỘI\\ TRƯỜNG ĐẠI HỌC CÔNG NGHỆ}\\[2cm]
% 	\includegraphics[width=0.25\linewidth]{images/Logo_UET.png}\\[1cm]
	
	{\Large  \bfseries Trần Hữu Nam}\\[2cm]		
	{ \LARGE \bfseries  NGHIÊN CỨU, PHÁT TRIỂN THIẾT BỊ XÁC ĐỊNH TƯ THẾ NGỦ SỬ DỤNG
                       CẢM BIẾN GIA TỐC
    }\\[0.05cm]
	\hfill\\[1cm]
	\begin{flushleft}
	{\large \bfseries Ngành: Điện tử viễn thông}\\
	{\large \bfseries Chuyên ngành: Kĩ thuật điện tử}\\
	{\large \bfseries Mã số: 23025029}\\
    
	\end{flushleft}
	\hfill\\[1cm]	
	{\large \bfseries LUẬN VĂN THẠC SĨ}\\
	\hfill\\[1cm]
    
	{\large \bfseries NGƯỜI HƯỚNG DẪN KHOA HỌC: PGS.TS. Mai Anh Tuấn}\\	
    
    
	\begin{flushleft}
% 	{\large \bfseries Cán bộ đồng hướng dẫn: CN. Bùi Quang Cường}\\	
	\end{flushleft}
		\hfill\\[3cm]	
	{\large \bfseries HÀ NỘI - 2025}\\		
	\vfill		
\end{titlepage}

\changefontsizes[16pt]{14pt}
\addtocontents{toc}{\vspace{-1cm}}
\addcontentsline{toc}{chapter}{Lời cam đoan}
\begin{center}
    \textbf{LỜI CAM ĐOAN}
\end{center}
Tôi xin cam đoan: đề tài luận văn thạc sĩ “Nghiên cứu, phát triển thiết bị xác định tư thế ngủ sử dụng cảm biến gia tốc” là công trình nghiên cứu của tôi dưới sự hướng dẫn của thầy \textbf{PGS. TS. Mai Anh Tuấn} và thầy \textbf{ThS. Trần Ngọc Thái} cùng với sự tham khảo từ những tài liệu đã liệt kê trong mục Tài liệu tham khảo. Tôi không sao chép công trình nghiên cứu của cá nhân khác dưới bất kỳ hình thức nào. Nếu có tôi xin hoàn toàn chịu trách nhiệm.

\begin{flushright}
	\begin{varwidth}{\linewidth}\centering
		Hà Nội, ngày \space\space\space\space tháng  \space\space\space\space năm 2025\\
		Học viên\\[2cm]
		Trần Hữu Nam
	\end{varwidth}
\end{flushright}

\newpage

\addcontentsline{toc}{chapter}{Lời cảm ơn}
\begin{center}
    \textbf{LỜI CẢM ƠN}
\end{center}

Lời đầu tiên, tôi xin gửi lời cảm ơn đến thầy PGS.TS. Mai Anh Tuấn và thầy ThS. Trần Ngọc Thái vì đã tận tình hướng dẫn, truyền đạt kiến thức cho tôi trong suốt quá trình học tập và thực hiện đề tài. 
Tôi xin cảm ơn tập thể thầy, cô khoa “Điện tử Viễn thông”,  Trường Đại học Công nghệ - ĐHQGHN, đã giảng dạy tôi trong quá trình tôi học tập tại trường. Tôi cũng xin cảm ơn các anh chị ở Bộ môn Công nghệ Vi cơ Điện tử và Kỹ thuật Máy tính đã tạo điều kiện giúp đỡ, chỉ bảo tôi trong thời gian làm khóa luận. 
Cuối cùng, tôi xin cảm ơn bố mẹ, gia đình cũng như bạn bè, tập thể lớp K30 đã luôn đồng hành, chia sẻ và động viên tôi suốt thời gian qua.


\newpage
\addcontentsline{toc}{chapter}{Tóm tắt}
\begin{center}
    \textbf{TÓM TẮT}
\end{center}
% \changefontsizes[16pt]{12pt}
\textit{\textbf{Tóm tắt: }} 
Ngưng thở tắc nghẽn khi ngủ (Obstructive Sleep Apnea - OSA) là tình trạng rối loạn hô hấp khi ngủ thường gặp, được đặc trưng bởi những cơn ngưng/giảm thở ngắn, lặp lại trong khi ngủ, gây gián đoạn giấc ngủ bởi hành vi thức giấc thường xuyên và giảm oxy máu ngắt quãng\cite{Epstein2009}. Nghiên cứu\cite{Salari2025} đã phân tích dữ liệu từ 15 nghiên cứu với tổng cộng 42.924 người. Kết quả cho thấy tỷ lệ mắc chứng buồn ngủ ban ngày quá mức (EDS) ở bệnh nhân ngưng thở tắc nghẽn khi ngủ (OSA) trên toàn cầu là 39,9\% (khoảng tin cậy 95\%: 34,4\%–45,7\%).

Trong bối cảnh công nghệ chế tạo ngày càng phát triển mạnh mẽ, việc thu nhỏ hóa và tối ưu hiệu suất hoạt động của các loại cảm biến, pin đã trở thành một yếu tố then chốt, mang tính quyết định trong quá trình tích hợp chúng vào các thiết bị điện tử có kích thước nhỏ gọn. Không chỉ góp phần nâng cao độ chính xác trong việc đo lường các thông số sinh lý quan trọng, nó còn giúp giảm thiểu kích thước thiết bị, tăng tính di động và mở rộng khả năng ứng dụng trong nhiều lĩnh vực, đặc biệt nổi bật trong y học cá thể hóa.

Bên cạnh đó, sự phát triển vượt bậc của trí tuệ nhân tạo (AI) đóng vai trò như một đòn bẩy quan trọng trong việc nâng cao hiệu quả khai thác dữ liệu cảm biến. AI không chỉ tối ưu hóa quy trình xử lý và phân tích dữ liệu mà còn thúc đẩy khả năng nhận diện mẫu, phân cụm, dự đoán và đưa ra quyết định mang tính hỗ trợ lâm sàng. Trong lĩnh vực chẩn đoán hội chứng ngưng thở khi ngủ, sự phối hợp giữa cảm biến và thuật toán học máy không chỉ đảm bảo độ chính xác cao trong thu thập dữ liệu mà còn mở ra khả năng phân tích chuyên sâu về các yếu tố sinh lý, phục vụ quá trình đánh giá y tế toàn diện.

Trên cơ sở đó, nghiên cứu này tập trung phát triển một thiết bị theo dõi tình trạng bệnh nhân trong khi ngủ, sử dụng cảm biến gia tốc kết hợp với thuật toán học máy nhằm phân tích tư thế ngủ và phát hiện các dấu hiệu liên quan đến hội chứng ngưng thở khi ngủ. Hệ thống được tiến hành nghiên cứu, triển khai trên ứng dụng di động trước rồi được tích hợp trên chip, hướng đến khả năng giám sát liên tục, thu thập dữ liệu thời gian thực và đưa ra đánh giá có độ tin cậy cao, góp phần cải thiện hiệu quả trong phát hiện sớm và hỗ trợ điều trị bệnh lý hô hấp liên quan đến giấc ngủ.


\vspace{-0.5cm}
\begin{flushleft}
  \textit{\textbf{Từ khóa: } cảm biến gia tốc, học máy, ngưng thở khi ngủ}
\end{flushleft}
% \changefontsizes[16pt]{13pt}



\changefontsizes[16pt]{13pt}
