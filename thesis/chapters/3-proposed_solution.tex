
\section{Phần cứng thực nghiệm \label{section_overview_propsed_method}}
\subsection{Cảm biến \label{section_overview_propsed_method}}

Trong quá trình ngủ, các chuyển động thân thể chủ yếu là chuyển động 
chậm, với biên độ nhỏ và không mang tính đột ngột. 
Các chuyển động thân thể chủ yếu mang tính chậm và thường xảy ra 
trong giai đoạn ngủ không chuyển động mắt nhanh (NREM), 
khi cơ thể có khả năng tự do thay đổi tư thế. Ngược lại, 
trong giai đoạn ngủ REM, hiện tượng ức chế trương lực cơ khiến 
cơ thể gần như bất động.
Do đó, việc ghi 
nhận chính xác các thay đổi tư thế ngủ đòi hỏi cảm biến có độ nhạy cao, 
khả năng phân giải tốt và ổn định với nhiễu nền thấp. Như đã trình bày 
trong Chương I, các cảm biến gia tốc MEMS sử dụng nguyên lý điện dung 
hiện đang được ứng dụng rộng rãi trong giám sát tư thế và chuyển động 
khi ngủ nhờ vào đặc điểm nổi bật là kích thước nhỏ gọn, tiêu thụ năng 
lượng thấp, tần số lấy mẫu phù hợp và đặc biệt là độ nhạy cao với 
chuyển động cường độ thấp.

Trong nghiên cứu của Vu và cộng sự (2023), dữ liệu tư thế ngủ được thu 
thập thông qua một thiết bị đeo đặt tại vùng bụng của người tham gia. 
Thiết bị này tích hợp cảm biến gia tốc ba trục ADXL345, 
bộ điều khiển ESP8266 và pin Lithium, tất cả được đóng gói trong 
một hộp nhựa nhỏ gọn \cite{vu2023}. Trong nghiên cứu của Boiko 
và cộng sự, cảm biến gia tốc ba trục ADXL355z 
được sử dụng để thu nhận tín hiệu hô hấp từ vùng ngực và bụng, 
với tần số lấy mẫu 62 Hz. Đây là cảm biến có độ nhiễu thấp, 
độ trôi nhiệt nhỏ và phù hợp với các ứng dụng y sinh. Trước đó, 
dòng cảm biến này cũng đã được ứng dụng thành công trong các phép 
đo tim–phổi \cite{Boiko2023}. Dữ liệu từ ADXL355z được đối chiếu với tín hiệu chuẩn 
thu từ dây đeo hô hấp của hệ thống SOMNO HD eco PSG, nhằm đảm bảo 
độ chính xác trong đánh giá tín hiệu sinh lý trong khi ngủ.
Trong nghiên cứu của Abdulsadig và cộng sự, dữ liệu gia tốc được 
thu thập bằng bo mạch điện tử thiết kế riêng tích hợp cảm biến gia tốc 
ba trục LIS2DH12 (STMicroelectronics) và vi điều khiển nRF5232 
(Nordic Semiconductor) \cite{Sleep_Posture_Detection}.


Dựa trên khảo sát thực tế và đánh giá các tiêu chí kỹ thuật phù hợp 
với đặc thù đo chuyển động chậm khi ngủ, luận văn lựa chọn hai dòng 
cảm biến gia tốc phổ biến là LIS3DH và LSM6DS3, đều do hãng 
STMicroelectronics sản xuất. Cả hai cảm biến đều tích hợp khả năng đo gia tốc ba trục, 
hỗ trợ các chuẩn giao tiếp \texttt{I\textsuperscript{2}C} và \texttt{SPI}, 
đồng thời có thể hoạt động trong chế độ tiêu thụ điện năng siêu thấp (\textit{ultra-low power mode}), 
đáp ứng yêu cầu sử dụng lâu dài trong các thiết bị đeo cá nhân hoạt động liên 
tục suốt đêm.

Trong các ứng dụng theo dõi tư thế ngủ, độ nhạy của cảm biến là một 
trong những tiêu chí quan trọng hàng đầu, bởi nó quyết định khả năng 
phát hiện các chuyển động nhỏ đặc trưng với biên độ thấp và tốc độ 
thay đổi chậm. Cảm biến \textbf{LIS3DH} cho phép lập trình dải đo động 
từ $\pm2g$ đến $\pm16g$, mang lại tính linh hoạt cao trong thiết kế hệ 
thống. Tuy nhiên, khi mở rộng dải đo, độ nhạy và độ phân giải sẽ giảm 
đáng kể. Ví dụ, đối với cảm biến 16-bit, nếu cấu hình ở dải $\pm100g$, 
độ phân giải chỉ đạt khoảng $0{,}003g$/LSB, điều này có thể làm suy giảm 
khả năng nhận biết các chuyển động vi mô trong khi ngủ - một yếu tố 
then chốt trong phân loại chính xác tư thế ngủ.




\begin{figure}[!ht]
		\centering
 		\includegraphics[width=0.8\textwidth]{images/lis.png}
		\caption{Cảm biến gia tốc LIS3DH và sơ đồ chân kết nối}
		\label{lis}
\end{figure}




\subsection{Vi xử lý}

Với sự phát triển vượt bậc và đa dạng của công nghệ chế tạo, 
có rất nhiều cấu hình phần cứng được nhiều nhóm tác giả lựa chọn phù 
hợp với các mục đích khác nhau. Trong đó, \cite{p_1} các tác giả đã 
sử dụng máy tính đơn Raspberry Pi kết hợp các điện trở cảm biến 
lực để phát hiện 4 tư thế ngủ với sự lấy nhãn từ video theo dõi người 
bệnh trong suốt quá trình lấy mẫu. Kwasnicki và cộng sự đã phát triển 
hệ thống ngủ có thể đeo (wearable sleep system) sử dụng bộ xử lý công 
suất thấp TI MSP430 và mô-đun RF Chipcon CC2420 cho truyền thông không 
dây kết hợp với cảm biến gia tốc 3 trục ADXL330, con quay hồi chuyển 
InvenSense ITG-3200, Honeywell HMC5843 để đo từ trường xác định 99.5\% 
chính xác 4 tư thế ngủ \cite{kwasnicki2018}. Tuy nhiên, các thiết bị vẫn 
yêu cầu một nguồn nặng lượng khiến cho tính liên tục bị hạn chế đáng kể. 
I.Yun và cộng sự đã phát triển thiết bị theo dõi tư thế ngủ của trẻ nhỏ 
sử dụng vi xử lý ATmega328P-PU cùng module Bluetooth kết hợp cảm biến gia 
tốc ADXL335 được đặt trên bụng đã nhưng lựa chọn về mặt cấu hình thiết bị 
và chế tạo ra mạch cung cấp năng lượng cho những thành phần cần thiết 
\cite{p_3}. Từ đó, giảm thiếu đáng kể mức tiêu thụ năng lượng và vẫn 
giữ nguyên độ chính xác nhưng khá bất tiện cho trẻ nhỏ. 
Trong nghiên cứu của Abdulsadig và cộng sự, 
hệ thống thu thập dữ liệu được xây dựng dựa trên một bo mạch tùy chỉnh tích 
hợp vi điều khiển nRF5232 (Nordic Semiconductor) – một SoC thuộc dòng 
ARM Cortex-M4F, hỗ trợ truyền thông không dây thông qua giao 
thức Bluetooth Low Energy (BLE). Vi điều khiển này đảm nhiệm đồng 
thời cả việc lấy mẫu dữ liệu từ cảm biến gia tốc ba trục LIS2DH12 
(STMicroelectronics) với tần số 100 Hz và truyền dữ liệu không dây 
theo thời gian thực \cite{Sleep_Posture_Detection, abdulsadig2023}. 
Trong nghiên cứu của Vũ Hoàng Diệu và cộng sự, 
mô-đun ESP32 được lựa chọn làm đơn vị xử lý trung tâm nhờ tích hợp bộ vi điều khiển hiệu năng cao, 
kết nối không dây Wi-Fi và khả năng mở rộng linh hoạt \cite{vu2023}. 
Với thiết kế nhỏ gọn, chi phí hợp lý và mức tiêu thụ điện năng thấp, 
ESP32 đáp ứng tốt yêu cầu của hệ thống thu thập dữ liệu tư thế ngủ theo 
thời gian thực. Thiết bị không chỉ cho phép truyền dữ liệu trực tiếp 
lên máy chủ hoặc nền tảng đám mây thông qua Wi-Fi, mà còn hỗ trợ 
lưu trữ cục bộ trên thẻ nhớ microSD, đảm bảo tính liên tục và 
an toàn dữ liệu trong điều kiện mất kết nối mạng.

Tuy nhiên, qua phân tích các nghiên cứu trên có thể thấy rằng phần 
lớn các cấu hình phần cứng hiện tại hoặc có chi phí triển khai cao, 
hoặc tiêu tốn năng lượng, hoặc gặp giới hạn trong khả năng tích hợp mô 
hình học máy tại thiết bị. Do đó, việc lựa chọn một kiến trúc vi xử lý 
vừa đảm bảo hiệu suất xử lý tín hiệu sinh lý thời gian thực, vừa tối 
ưu năng lượng và có khả năng triển khai mô hình TinyML là cần thiết. 
Trong số các kiến trúc hiện nay, dòng ARM Cortex-M4 nổi bật nhờ tính 
cân bằng giữa hiệu năng, mức tiêu thụ năng lượng thấp và khả năng hỗ 
trợ xử lý tín hiệu số, phù hợp với các hệ thống đeo được trong theo 
dõi tư thế ngủ.


Kiến trúc ARM có nhiều dòng vi xử lý khác nhau, được phát triển và nâng
cấp liên tục nhằm đáp ứng nhu cầu đa dạng trong lĩnh vực công nghệ nhúng. 
Trong đó, dòng Cortex-M thuộc kiến trúc ARMv7 đã trở thành nền tảng phổ 
biến cho các hệ thống nhúng sử dụng vi điều khiển nhờ vào hiệu suất cao, 
khả năng mở rộng và mức tiêu thụ năng lượng tối ưu. Dòng Cortex-M bao 
gồm nhiều phiên bản như Cortex-M0, Cortex-M0+, Cortex-M1, Cortex-M3, 
Cortex-M4 và Cortex-M7, mỗi phiên bản được thiết kế để phục vụ cho các mức 
độ yêu cầu hiệu năng khác nhau \cite{arm_cortex_m_comparison}. 
Các vi xử lý thuộc họ Cortex-M chủ yếu được ứng dụng trong các hệ thống 
nhúng thời gian thực, nơi yêu cầu sự cân bằng giữa hiệu suất xử lý, tiêu 
thụ năng lượng và chi phí. Một số vi xử lý ARM khác, không thuộc họ 
Cortex-M, được sử dụng trong các thiết bị hiệu suất cao như điện thoại 
thông minh và máy tính bảng, vốn yêu cầu cấu hình phần cứng mạnh hơn và 
khả năng xử lý đa tác vụ cao hơn.
Theo tài liệu \cite{cortexM4}, vi xử lý Cortex-M4 là một bộ xử lý 32-bit 
sử dụng kiến trúc tập lệnh rút gọn (RISC), được xây dựng theo kiến trúc 
Harvard, trong đó bus dữ liệu và bus lệnh được tách biệt nhằm tối ưu 
hiệu suất truy xuất bộ nhớ. Vi xử lý này hỗ trợ đầy đủ cả tập lệnh 
Thumb-1 (16-bit) và Thumb-2 (hỗn hợp 16/32-bit), mang lại sự linh hoạt 
trong mã hóa lệnh và tiết kiệm không gian bộ nhớ chương trình.

Về hiệu năng, Cortex-M4 đạt từ 1,25 đến 1,95 DMIPS/MHz (Dhrystone Million Instructions Per Second per MHz), cho thấy khả năng xử lý hiệu quả trong các ứng dụng nhúng yêu cầu độ chính xác và độ phản hồi thời gian thực cao. Bên cạnh đó, vi xử lý hỗ trợ tối đa 240 tín hiệu ngắt, bao gồm cả ngắt không thể bị chặn (Non-Maskable Interrupts – NMI), cùng khả năng cấu hình từ 8 đến 256 mức ưu tiên ngắt, giúp hệ thống hoạt động ổn định trong môi trường có nhiều sự kiện cạnh tranh đồng thời.
Ngoài ra, hiện nay ứng dụng trí tuệ nhân tạo (AI) tại thiết bị biên (Edge AI) đang ngày càng phổ biến, đặc biệt trong các lĩnh vực như nhà thông minh, thiết bị đeo, giám sát an ninh và công nghiệp 4.0. Với khả năng xử lý tín hiệu số (DSP) và hỗ trợ các mạng nơ-ron nhỏ gọn, các vi xử lý Cortex-M, đặc biệt là dòng Cortex-M4, đang được khai thác để triển khai các mô hình học sâu nhẹ (tinyML) ngay trên vi điều khiển \cite{electronics11162545}\cite{applicationCortexM4}.


\begin{figure}[!ht]
		\centering
 		\includegraphics[width=0.8\textwidth]{images/cortexM4.png}
		\caption{Thành phần chính của vi điều khiển Cortex-M4}
		\label{cortexM4}
\end{figure}

Kết nối bus được mô tả trong Hình~\ref{cortexM4} cho phép truyền dữ liệu đồng thời trên nhiều bus khác nhau, đồng thời cung cấp khả năng quản lý truyền dữ liệu hiệu quả, chẳng hạn như sử dụng bộ đệm ghi và điều khiển hướng bit hoạt động (bit-banding). Hệ thống cũng có thể bao gồm các cầu bus (bus bridges) nhằm kết nối nhiều loại bus vào một mạng duy nhất sử dụng chung không gian bộ nhớ. Ngoài ra, bộ xử lý được trang bị hệ thống hỗ trợ gỡ lỗi tích hợp, bao gồm khả năng kiểm soát gỡ lỗi, thiết lập điểm ngắt (breakpoint) chương trình và điểm theo dõi dữ liệu (watchpoint). Khi xảy ra sự kiện gỡ lỗi, hệ thống có thể tạm dừng trạng thái hoạt động của lõi xử lý để phục vụ việc phân tích và xử lý lỗi.

Bên cạnh đó, kiến trúc Cortex-M4 tích hợp Bộ điều khiển ngắt vectored lồng nhau (Nested Vectored Interrupt Controller – NVIC) với khả năng hỗ trợ lên đến 240 tín hiệu yêu cầu ngắt, bao gồm cả ngắt không chắn được (NMI). NVIC hỗ trợ xử lý ngắt lồng nhau một cách tự động bằng cách so sánh mức ưu tiên giữa các yêu cầu ngắt với mức ưu tiên hiện tại đang được xử lý.

Đối với các ứng dụng yêu cầu tiết kiệm năng lượng, hệ thống còn được trang bị bộ đánh thức ngắt (Wake-up Interrupt Controller – WIC), cho phép đưa bộ vi điều khiển vào chế độ nghỉ bằng cách tắt hầu hết các thành phần không cần thiết, đồng thời duy trì khả năng đánh thức hệ thống khi phát hiện một yêu cầu ngắt. Ngoài ra, cơ chế bảo vệ bộ nhớ cũng được tích hợp nhằm đảm bảo an toàn cho hệ thống, ví dụ như chỉ cho phép truy cập đọc tại một số vùng bộ nhớ hoặc ngăn người dùng truy cập vào các vùng dữ liệu đặc quyền của hệ điều hành hoặc ứng dụng hệ thống.


\begin{figure}[!ht]
		\centering
 		\includegraphics[width=0.8\textwidth]{images/NRF52840-QFA_SPL.jpg}
		\caption{Nordic Semiconductor NRF52840}
		\label{lis}
\end{figure}
Sau quá trình khảo sát và so sánh các dòng vi xử lý phổ biến, tác giả 
lựa chọn nRF52840 (Nordic Semiconductor) làm nền tảng phần cứng cho hệ 
thống đề xuất, nhờ vào các ưu điểm nổi bật như kích thước nhỏ, 
tiêu thụ năng lượng thấp và tích hợp sẵn giao tiếp Bluetooth Low Energy 
(BLE). Đây là vi xử lý cao cấp nhất trong dòng nRF52, thuộc loại hệ thống 
trên một vi mạch (System-on-Chip – SoC), được thiết kế chuyên biệt cho 
các ứng dụng không dây tầm ngắn và tiết kiệm năng lượng \cite{nrf52840}.

\textbf{nRF52840} tích hợp bộ thu phát đa giao thức hoạt động ở băng tần 2.4 GHz 
và bộ xử lý trung tâm Arm Cortex-M4F chạy ở xung nhịp 64 MHz, 
kèm bộ xử lý dấu phẩy động (FPU). Vi xử lý này được trang bị bộ nhớ 
1 MB Flash và 256 KB RAM, hỗ trợ chuẩn Bluetooth 5.3 cùng khả năng giao 
tiếp đa giao thức (multiprotocol), cho phép cải thiện tốc độ, phạm vi 
truyền và độ tin cậy của kết nối không dây. Hệ thống bảo mật tích hợp 
đầy đủ, bao gồm các tính năng mã hóa phần cứng, đáp ứng yêu cầu khắt khe 
về bảo vệ dữ liệu. Ngoài khả năng hoạt động trong dải điện áp rộng 
từ +1.7 V đến +5.5 V (tương thích với nguồn pin và USB), nRF52840 còn 
cung cấp các giao tiếp ngoại vi phong phú: tối đa hai giao diện I2C, 
bốn SPI master, ba SPI slave, bốn kênh PWM hỗ trợ EasyDMA, cùng với 
năm bộ định thời 32-bit, phù hợp cho các ứng dụng đòi hỏi xử lý thời 
gian thực chính xác. Tất cả các đặc điểm trên khiến nRF52840 trở thành 
lựa chọn lý tưởng cho các hệ thống nhúng đeo được tích hợp AI nhẹ và 
kết nối không dây thông minh.

Ngoài ra, nRF52840 hỗ trợ một hệ sinh thái phần mềm mạnh mẽ, bao gồm SDK 
của Nordic và nền tảng TensorFlow Lite for Microcontrollers, giúp rút 
ngắn thời gian phát triển và triển khai hệ thống TinyML. Thiết bị còn 
sở hữu khả năng quản lý năng lượng linh hoạt, tương thích tốt với nguồn pin hoặc USB. 





\subsection{Bluetooth năng lượng thấp}

Với mục tiêu tối ưu hóa năng lượng và đảm bảo khả năng hoạt động lâu dài 
cho thiết bị đeo sử dụng pin, Bluetooth Low Energy (BLE) được lựa chọn 
làm chuẩn kết nối không dây chính trong hệ thống phần cứng.

\begin{figure}[!ht]
		\centering
 		\includegraphics[width=0.8\textwidth]{images/ble.png}
		\caption{Các kiểu kết nối không dây trong họ chip nRF52}
		\label{ble}
\end{figure}

BLE là giao thức kết nối không dây được thiết kế chuyên biệt cho 
các ứng dụng năng lượng thấp, hoạt động ở băng tần ISM 2.4 GHz, 
hỗ trợ thông lượng ứng dụng lên đến 1.4 Mbps. Với ưu thế tiêu thụ năng 
lượng tối thiểu nhưng vẫn đảm bảo tốc độ truyền phù hợp, BLE đặc biệt 
thích hợp cho các thiết bị y sinh hoạt động liên tục bằng pin có dung 
lượng hạn chế. BLE hiện được hỗ trợ phổ biến trên hầu hết các hệ điều 
hành như iOS, Android, macOS, Windows 10 và Linux, cũng như trong các 
thiết bị di động hiện đại.

Về mặt bảo mật, BLE tích hợp các cơ chế mã hóa và xác thực nhằm đảm bảo 
tính bí mật, toàn vẹn và riêng tư của dữ liệu truyền qua mạng. 
Công nghệ này đã trở thành một phần tiêu chuẩn trong hầu hết các 
thiết bị di động hiện đại như smartphone, máy tính bảng, và laptop, 
đồng thời được hỗ trợ đầy đủ trên các hệ điều hành phổ biến bao gồm iOS, 
Android, macOS, Windows 10 và Linux. Bluetooth 5 là bước phát triển 
đột phá tiếp theo kể từ khi BLE được giới thiệu trong chuẩn Bluetooth 
4.0, mang đến hàng loạt cải tiến đáng kể giúp mở rộng phạm vi ứng dụng 
và nâng cao hiệu suất hệ thống. Một trong những cải tiến nổi bật là 
chế độ 2 Mbps, cho phép tăng gấp đôi tốc độ truyền lý thuyết, tương 
ứng với thông lượng thực tế lên đến 1.4 Mbps. Quan trọng hơn, chế độ 
này còn giúp giảm đáng kể mức tiêu thụ năng lượng – cụ thể là giảm một 
nửa năng lượng tiêu thụ trên mỗi bit dữ liệu – từ đó kéo dài thời gian 
hoạt động của thiết bị hoặc cho phép sử dụng các nguồn năng lượng nhỏ 
và chi phí thấp hơn \cite{BLE}. 

Bên cạnh đó, tính năng Advertising Extensions (mở rộng quảng cáo) đã 
cách mạng hóa cơ chế phát sóng của BLE. Các gói quảng cáo giờ đây có 
thể chứa lượng dữ liệu gấp 8 lần so với phiên bản trước, cho phép 
truyền tải các khối dữ liệu lớn hơn mà không cần thiết lập kết nối 
ngay lập tức. Đồng thời, các gói quảng cáo có thể được xâu chuỗi 
để tạo thành các tập tin quảng cáo phức hợp. Tính năng lựa chọn kênh 
được tối ưu hóa giúp tăng cường độ ổn định và khả năng chống nhiễu 
trong các môi trường có mật độ thiết bị cao. Đặc biệt, chế độ Long 
Range mở rộng đáng kể phạm vi truyền thông của BLE, cho phép các thiết 
bị duy trì kết nối trong toàn bộ không gian của một ngôi nhà thông minh 
hoặc trong các ứng dụng IoT công nghiệp quy mô vừa và nhỏ.




\begin{figure}[!ht]
	\centering
 	\includegraphics[width=0.5\textwidth]{images/gatt.drawio.png}
	\caption{Cấu trúc của GATT}
	\label{gatt}
\end{figure}

BLE tổ chức logic giao tiếp dựa trên mô hình GATT 
(Generic Attribute Profile). GATT quy định cách hai thiết bị BLE 
trao đổi dữ liệu thông qua các đơn vị logic: dịch vụ (services) 
và đặc tính (characteristics). Giao thức nền tảng là Attribute Protocol 
(ATT) – nơi mỗi đặc tính được định danh bằng UUID 16-bit hoặc 128-bit, 
với quyền truy cập như chỉ đọc, chỉ ghi, hoặc hỗ trợ thông báo (notify).

Một điểm quan trọng trong mô hình GATT là tính kết nối độc quyền: 
tại một thời điểm, thiết bị ngoại vi chỉ có thể duy trì một kết nối 
duy nhất với thiết bị trung tâm. Khi kết nối được thiết lập, thiết bị 
ngừng quảng cáo, điều này hạn chế khả năng kết nối đồng thời từ 
nhiều thiết bị.

Ngoài ra, vi xử lý nRF52840 còn hỗ trợ Bluetooth Mesh, cho phép thiết 
lập mạng lưới nhiều-nút (many-to-many), sử dụng BLE làm lớp truyền tải 
vật lý. Mỗi nút trong mạng có thể đóng vai trò chuyển tiếp (relay), 
cho phép dữ liệu lan truyền đến các vùng rộng hơn theo mô hình phân 
tán – phù hợp với các ứng dụng IoT quy mô lớn như nhà thông minh, 
chiếu sáng công nghiệp hoặc giám sát phân tán. Trong mạng Mesh, 
các gói dữ liệu có thể được đóng gói qua advertising packet hoặc 
qua các giao tiếp GATT tùy tình huống sử dụng.

Các profile BLE là tập hợp các dịch vụ được chuẩn hóa bởi Bluetooth 
SIG hoặc định nghĩa tùy chỉnh, ví dụ như dịch vụ UART tùy chỉnh gồm 
hai đặc tính RX và TX, tương ứng với kênh nhận và truyền.

\subsection{Thiết bị thực nghiệm}
Trong khuôn khổ của khóa luận, nhằm đảm bảo tiến độ triển khai và tính an 
toàn trong giai đoạn thử nghiệm, tác giả lựa chọn sử dụng bộ kit thương 
mại Adafruit Playground để tiến hành thực nghiệm sơ bộ. Bộ kit này tích 
hợp sẵn cảm biến gia tốc MEMS LIS3DH được gắn tại vị trí trung tâm, 
cho phép đo gia tốc theo ba trục không gian X, Y và Z với độ chính xác cao. 
Theo tài liệu từ nhà sản xuất, chi phí cho mỗi bộ kit Adafruit vào khoảng 
25 USD \cite{ada_overview}. Trong bộ kit, cảm biến LIS3DH được kết nối với 
vi điều khiển thông qua giao thức SPI, với chân chọn thiết bị (CS) được 
gán tại chân số 8 và đầu ra ngắt tùy chọn (IRQ) tại chân số 7 (IRQ \#4). 
Theo sơ đồ bố trí tiêu chuẩn của kit, trục X định hướng theo chiều giắc 
USB, trục Y hướng sang bên trái, và trục Z vuông góc theo hướng mặt trên 
của thiết bị.

Bên cạnh đó, để mở rộng khả năng nghiên cứu và đánh giá tính khả thi khi 
tích hợp học máy nhẹ (TinyML) cũng như kết nối không dây, nhóm nghiên 
cứu sử dụng thêm nền tảng Arduino Nano 33 BLE Sense. Đây là vi điều khiển 
hiện đại tích hợp vi xử lý nRF52840 (ARM Cortex-M4F), hỗ trợ Bluetooth 
Low Energy (BLE) và nhiều cảm biến tích hợp (IMU, microphone, nhiệt độ, 
độ ẩm, v.v.), đồng thời tương thích với nền tảng TensorFlow Lite for 
Microcontrollers \cite{nano33ble}.

Đáng chú ý, bên cạnh việc sử dụng các bộ kit sẵn có, một thành viên khác 
trong nhóm đang tiến hành phát triển và xây dựng bản mạch phần cứng 
tùy chỉnh dựa trên các thông số kỹ thuật đã được phân tích ở các phần 
trước. Hướng tiếp cận này không chỉ giúp nhóm triển khai nhanh chóng hệ 
thống thử nghiệm trong giai đoạn đầu, mà còn mở ra khả năng thiết kế một 
thiết bị nhúng chuyên dụng, tối ưu hơn về chi phí, hiệu năng và khả năng 
tích hợp trong các ứng dụng thực tiễn.


\begin{figure}[!ht]
		\centering
 		\includegraphics[width=0.8\textwidth]{images/detail_ada.png}
		\caption{Cấu trúc các thành phần trên Circuit PlayGround}
		\label{detail_ada}
\end{figure}





\section{Hệ thống thu thập, xử lý, lưu trữ dữ liệu}
Phần này trình bày tổng quan kiến trúc hệ thống bao gồm: 
lập trình firmware trên vi điều khiển để thu thập dữ liệu cảm biến, 
thiết kế ứng dụng di động làm cầu nối giữa phần cứng và hệ thống đám mây, 
cùng với backend và cơ sở dữ liệu lưu trữ phục vụ huấn luyện mô hình. 
Nội dung cũng đề cập đến các yêu cầu chức năng, phi chức năng và 
thiết kế hệ thống ở mức cao nhằm đảm bảo khả năng triển khai thực 
tế và mở rộng trong tương lai.


\subsection{Lập trình vi xử lý}

Thiết bị được lập trình trên nền tảng Arduino IDE, sử dụng thư viện 
\texttt{Adafruit Circuit Playground}. Trong hàm \texttt{setup()}, 
thiết bị khởi tạo các bản tin quảng cáo (advertising), 
cấu hình kết nối/ngắt kết nối, và thiết lập cấu trúc dịch vụ theo 
giao thức \gls{GATT} của BLE, như được minh họa trong Hình~\ref{flowBLE}.

\begin{figure}[htbp]
    \centering
    \includegraphics[width=0.5\textwidth]{images/flowBLE.png}
    \caption{Lưu đồ hoạt động của thiết bị BLE}
    \label{flowBLE}
\end{figure}

Trong đoạn mã~\ref{arduinoBLE}, hàm \texttt{startAdv()} đảm nhiệm cấu 
hình quảng bá BLE cho thiết bị. Quá trình này bao gồm: thiết lập cờ 
kết nối tổng quát, chèn thông tin công suất truyền (Tx Power), 
thêm UUID của dịch vụ tư thế (\texttt{positionService}) và tên thiết 
bị vào gói quảng bá. Các thông số quảng bá được cấu hình theo khuyến 
nghị của Apple nhằm đảm bảo khả năng tương thích với thiết bị iOS: 
chế độ nhanh với chu kỳ 20ms, chế độ chậm 152.5ms, và thời gian chuyển 
chế độ sau 30 giây. Thiết bị sẽ tiếp tục phát tín hiệu quảng bá cho 
đến khi có kết nối được thiết lập.

Trong cấu trúc dịch vụ, tác giả định nghĩa một dịch vụ chính với UUID 
là \texttt{0x1821}, kèm theo hai đặc tính cảm biến: gia tốc 
(\texttt{UUID 0x2713}, đơn vị \texttt{m/s\textsuperscript{2}}) và 
gia tốc góc (\texttt{UUID 0x2744}, 
đơn vị \texttt{rad/s\textsuperscript{2}}). 
Tuy hệ thống hỗ trợ cả hai loại dữ liệu, trong khuôn khổ khoá luận này, 
tác giả chỉ tập trung vào giá trị gia tốc thu được từ cảm biến để phục 
vụ bài toán phân loại tư thế ngủ.

\begin{lstlisting}[float,language=C,caption=Tập lệnh khởi tạo và kết nối Bluetooth từ thư viện của AdaFruit, label=arduinoBLE,captionpos=b]
void startAdv(void)
{
  // Advertising packet
  Bluefruit.Advertising.addFlags(BLE_GAP_ADV_FLAGS_LE_ONLY_GENERAL_DISC_MODE);
  Bluefruit.Advertising.addTxPower();

  // Include HRM Service UUID
  Bluefruit.Advertising.addService(positionService);

  // Include Name
  Bluefruit.Advertising.addName();
  
  /* Start Advertising
   * - Enable auto advertising if disconnected
   * - Interval:  fast mode = 20 ms, slow mode = 152.5 ms
   * - Timeout for fast mode is 30 seconds
   * - Start(timeout) with timeout = 0 will advertise forever (until connected)
   * 
   * For recommended advertising interval
   * https://developer.apple.com/library/content/qa/qa1931/_index.html   
   */
  Bluefruit.Advertising.restartOnDisconnect(true);
  Bluefruit.Advertising.setInterval(32, 244);    // in unit of 0.625 ms
  Bluefruit.Advertising.setFastTimeout(30);      // number of seconds in fast mode
  Bluefruit.Advertising.start(0);                // 0 = Don't stop advertising after n seconds  
}
\end{lstlisting}

\input{chapters/examples/senDataBle}
Ngoài các thao tác khởi tạo dịch vụ, thư viện BLE của Adafruit còn cung cấp các phương thức cấu hình đặc tính (\textit{characteristics}) nhằm kiểm soát hành vi và bảo mật của kết nối BLE.

Cụ thể, phương thức \texttt{setProperties} cho phép cấu hình quyền truy cập của đặc tính, với các lựa chọn phổ biến như:

\begin{description}
    \item[\texttt{CHR\_PROPS\_BROADCAST}] phát sóng đặc tính (bit 0)
    \item[\texttt{CHR\_PROPS\_READ}] cho phép thiết bị đọc (bit 1)
    \item[\texttt{CHR\_PROPS\_WRITE\_WO\_RESP}] ghi không cần phản hồi (bit 2)
    \item[\texttt{CHR\_PROPS\_WRITE}] ghi với phản hồi (bit 3)
    \item[\texttt{CHR\_PROPS\_NOTIFY}] gửi thông báo không xác nhận (bit 4)
    \item[\texttt{CHR\_PROPS\_INDICATE}] gửi thông báo có xác nhận (bit 5)
\end{description}

Ngoài ra, một số phương thức bổ trợ khác bao gồm:

\begin{description}
    \item[\texttt{setPermission}] thiết lập quyền truy cập và mức độ bảo mật (ví dụ: không cần xác thực, cần mã hoá, v.v.)
    \item[\texttt{setFixedLen}] xác định độ dài cố định của dữ liệu truyền
\end{description}

\begin{figure}[htbp]
    \centering
    \includegraphics[width=\textwidth]{images/sendBleFlow.png}
    \caption{Lưu đồ luồng gửi thông tin BLE}
    \label{sendBleFlow}
\end{figure}


Luồng xử lý dữ liệu BLE được minh hoạ tại Hình~\ref{sendBleFlow}. 
Sau khi thu nhận dữ liệu cảm biến, thiết bị kiểm tra trạng thái 
kết nối BLE. Nếu kết nối hợp lệ, nó sẽ tiếp tục lắng nghe hành động đọc 
từ phía thiết bị trung tâm. Dữ liệu sau đó được làm tròn đến hai chữ số 
thập phân và mã hoá thành ba byte: byte đầu tiên lưu dấu, byte thứ hai 
chứa hai chữ số đầu, và byte cuối là hai chữ số cuối của giá trị gia tốc. 
Chuỗi dữ liệu này được gửi qua BLE theo đặc tính đã định nghĩa trước đó.














\subsection{Hiệu chuẩn cảm biến}
Việc thu nhận và tiền xử lý dữ liệu là bước quan trọng trong các hệ đo 
lường. Mặc dù cảm biến thường được hiệu chuẩn từ nhà sản xuất, nhưng 
vẫn cần được hiệu chuẩn lại trong môi trường đo thực tế để cải thiện 
hiệu năng và giảm thiểu sai số. Các sai số này được chia thành hai 
loại chính: (i) sai số hệ thống (mặc định) và (ii) sai số ngẫu nhiên.

\textbf{Hiệu chuẩn sai số hệ thống.} Tác giả sử dụng gia tốc trọng trường 
để hiệu chuẩn cảm biến theo hướng tĩnh. Khi xoay cảm biến sao cho một 
trục hướng lên vuông góc với mặt phẳng nằm ngang, giá trị đo được 
là $-1g$; khi hướng xuống dưới, giá trị là $+1g$. Bằng cách xoay cảm 
biến lần lượt qua sáu vị trí tĩnh tương ứng với các hướng trục chính, 
có thể xác định được các điểm chuẩn, từ đó nội suy để xác định giá 
trị $0g$ một cách chính xác và đáng tin cậy.

\begin{figure}[htbp]
    \centering
    \includegraphics[width=\textwidth]{images/allan.png}
    \caption{Minh hoạ kết quả phân tích đường cong Allan}
    \label{allan}
\end{figure}

\textbf{Phân tích sai số ngẫu nhiên.} Tác giả sử dụng phương sai Allan 
để phân tích các thành phần nhiễu trong dữ liệu cảm biến \cite{allan}. 
Đây là phương pháp phân tích miền thời gian phổ biến nhằm đánh giá độ 
ổn định tần số và định lượng các loại nhiễu khác nhau như nhiễu trắng, 
trôi ngẫu nhiên, và nhiễu lượng tử. Biểu đồ Allan log-log cho phép nhận 
diện các thành phần nhiễu thông qua độ dốc của từng đoạn đường cong.

Trong thử nghiệm, cảm biến được đặt cố định trong phòng ở điều kiện 
nhiệt độ ổn định, với tần số lấy mẫu 10 Hz, thu được tổng cộng 1.211.210 
mẫu. Kết quả biểu diễn trong Hình~\ref{allan_real} cho thấy nhiễu chiếm 
ưu thế là nhiễu lượng tử (quantization noise), đặc trưng bởi hệ số góc 
tương ứng trong đồ thị.



\begin{figure}[htbp]
    \centering
    \includegraphics[width=\textwidth]{images/allan_real.png}
    \caption{Biểu đồ phương sai Allan của trục X}
    \label{allan_real}
\end{figure}

\textbf{Lọc nhiễu bằng bộ lọc Kalman.} Để xử lý nhiễu, đặc biệt là nhiễu 
lượng tử, tác giả sử dụng bộ lọc Kalman \cite{kalman}. Đây là một bộ lọc 
đệ quy có khả năng ước lượng trạng thái tối ưu của hệ thống từ các chuỗi 
đo lường bị nhiễu. Bộ lọc Kalman không chỉ phù hợp cho hệ thống tuyến 
tính mà còn có thể áp dụng cho hệ thống phi tuyến thông qua tuyến tính 
hoá cục bộ.

Trong hệ thống đề xuất, tín hiệu sau khi được cảm biến thu nhận sẽ 
được lọc trực tiếp tại vi điều khiển trước khi truyền đến ứng dụng 
để hiển thị và lưu trữ. Kết quả sau lọc được minh hoạ trong 
Hình~\ref{kalman}, cho thấy sự cải thiện đáng kể về độ mượt và ổn 
định của tín hiệu.


\begin{figure}[htbp]
    \centering
    \includegraphics[width=\textwidth]{images/kalman.png}
    \caption{Kết quả bộ lọc Kalman cho dữ liệu trục X của cảm biến gia tốc}
    \label{kalman}
\end{figure}





\subsection{Xây dựng phần mềm ứng dụng}

Phần mềm ứng dụng được xây dựng với mục tiêu hỗ trợ người dùng trong việc 
kết nối với thiết bị phần cứng và trực quan hoá dữ liệu cảm biến. 
Ứng dụng đảm nhiệm vai trò là cầu nối giữa người dùng và hệ thống nhúng, 
đồng thời cung cấp các chức năng tương tác, cấu hình và theo dõi dữ liệu 
theo thời gian thực.

Các công nghệ và thành phần sử dụng được tóm tắt như sau:

\begin{flushleft}
\textbf{01)} Ngôn ngữ lập trình: \texttt{Dart} \\
\textbf{02)} Framework: \texttt{Flutter} \\
\textbf{03)} Nền tảng triển khai: \texttt{Android} \\
\textbf{04)} Giao tiếp phần cứng: Bluetooth Low Energy (BLE) \\
\textbf{05)} Chức năng chính: kết nối thiết bị, nhận dữ liệu, hiển thị, lưu trữ và cá nhân hóa trải nghiệm người dùng
\end{flushleft}

\begin{figure}[htbp]
    \centering
    \includegraphics[width=\textwidth]{images/app_flow.png}
    \caption{Các nhóm chức năng chính của ứng dụng}
    \label{app_flow}
\end{figure}

Ứng dụng được thiết kế xoay quanh ba nhóm chức năng chính như minh họa 
trong Hình~\ref{app_flow}:

\begin{flushleft}
\textbf{01)} Nhóm bảo mật: đăng nhập, xác thực và khôi phục tài khoản. \\
\textbf{02)} Nhóm chức năng chung: kết nối với thiết bị phần cứng, thu thập và hiển thị dữ liệu cảm biến. \\
\textbf{03)} Nhóm cá nhân hoá: theo dõi chỉ số sức khỏe, khai báo STOP-BANG, lưu hồ sơ người dùng.
\end{flushleft}

\subsubsection*{Kiến trúc phần mềm}
Ứng dụng sử dụng mô hình \textbf{BLoC (Business Logic Component)} để tách biệt giao diện 
người dùng và logic xử lý. BLoC hoạt động dựa trên nguyên tắc nhận sự 
kiện đầu vào và trả về trạng thái phù hợp, giúp quản lý luồng dữ liệu 
hiệu quả. Cấu trúc tổng thể của kiến trúc BLoC gồm ba lớp chính được mô 
tả trong Hình~\ref{flutter}.

\begin{figure}[htbp]
    \centering
    \includegraphics[width=0.8\textwidth]{images/flutter.png}
    \caption{Cấu trúc kiến trúc BLoC trong ứng dụng Flutter}
    \label{flutter}
\end{figure}


\begin{lstlisting}[float,language=Java,caption={Tập lệnh để tìm kiểm dịch vụ cảm biến},label=flutterBle,captionpos=t]
StreamBuilder<List<BluetoothService>>(
  stream: device.services,
  initialData: [],
  builder: (c, snapshot) {
    if (snapshot.data!.length > 0) {
      isService = true;
    }
    BluetoothService serviceAcclerometer;
    if (snapshot.data == null || snapshot.data!.length == 0) {
      return Text("Please contact customer Service");
    }
    for (int i = 0; i < snapshot.data!.length; i++) {
      if (snapshot.data![i].uuid.toString() ==
          Constants.ACCLEROMETER_SERVICE) {
        accelerometerService = snapshot.data![i];
      }
    }
    if (accelerometerService == null) {
      return Text("Please contact customer Service");
    }
    for (int i = 0;
        i < accelerometerService!.characteristics.length;
        i++) {
      print(accelerometerService!.characteristics[i].uuid);
      if (accelerometerService!.characteristics[i].uuid
              .toString() ==
          Constants.ACCLEROMETER_CHARACTION) {
        accelerometerCharactis =
            accelerometerService!.characteristics[i];
      }
    }
  });
\end{lstlisting}

\begin{lstlisting}[float,language=C,caption="Cấu trúc dữ liệu của phần nội dung đẩy lên máy chủ",label=format_ble,captionpos=b]
{
    "value": "0.88%0.66%0.99@2022-01-01/0.88%0.66%0.99@2022-01-01/0.88%0.66%0.99@2022-01-01",
    "customer": "62a5f5672ad9c724ef117d76"
}

\end{lstlisting}


Sau khi kết nối BLE được thiết lập thành công, ứng dụng truy xuất đối tượng đặc 
tính cảm biến (characteristic instance) và liên tục gửi yêu cầu đọc (\texttt{read}) 
đến vi điều khiển. Thiết bị phản hồi bằng cách trả về dữ liệu cảm biến 
dưới dạng mảng \texttt{UInt8}. Các giá trị này được ứng dụng giải mã, 
chuyển đổi sang dạng số thực tương ứng với gia tốc trên ba trục (X, Y, Z), 
và gắn nhãn thời gian thực.

Quá trình xử lý này được thực hiện trong một vòng 
lặp có kiểm soát độ trễ ngắn nhằm đảm bảo khả năng cập nhật liên tục 
nhưng vẫn tối ưu hiệu suất hệ thống.

Mã~\ref{flutterBle} minh hoạ toàn bộ quy trình xử lý: 
từ kết nối BLE, truy xuất đặc tính gia tốc, đọc giá trị nhị phân 
thô từ thiết bị, đến việc chuẩn hoá và gửi dữ liệu lên backend. 
Trong đoạn mã này, dữ liệu dạng \texttt{Uint8List} nhận từ cảm biến được tách và chuyển đổi 
thành ba thành phần tương ứng với ba trục gia tốc. Dữ liệu sau khi được xử lý sẽ được 
đóng gói theo định dạng \texttt{JSON} và gửi đến máy chủ thông qua phương thức 
POST, sử dụng thư viện \texttt{http} trong Flutter.

Định dạng dữ liệu BLE được chuẩn hoá như trong Mã~\ref{format_ble}, 
với trường \texttt{"value"} là chuỗi liên tục các 
giá trị cảm biến (phân tách bằng ký tự đặc biệt) và trường 
\texttt{"customer"} để định danh người dùng.

Việc tối ưu hóa cả quá trình đọc BLE và đẩy dữ liệu HTTP theo lô 
như vậy giúp giảm độ trễ, tránh tình trạng nghẽn băng thông, đồng thời 
vẫn đảm bảo độ chính xác và toàn vẹn của dữ liệu cảm biến.


Ngoài các chức năng thu thập và truyền dữ liệu cảm biến, ứng dụng còn tích hợp 
các công cụ hỗ trợ đánh giá y học lâm sàng ban đầu nhằm phục vụ cho việc sàng 
lọc và phân loại nguy cơ mắc hội chứng ngưng thở khi ngủ (OSA). Trong đó, 
ba thành phần quan trọng được triển khai bao gồm:

\noindent\textbf{01)} Bộ câu hỏi \textbf{STOP-BANG}: Đây là một bảng sàng lọc lâm sàng được sử dụng phổ biến trong y học giấc ngủ để đánh giá nguy cơ mắc OSA. Dữ liệu từ bảng này được lưu trữ cùng với dữ liệu cảm biến và đóng vai trò như đầu vào bổ sung cho các mô hình học máy dự đoán chỉ số AHI (Apnea–Hypopnea Index).

\vspace{0.5em}
\noindent\textbf{02)} Thang điểm \textbf{Epworth Sleepiness Scale (ESS)}: Tác giả triển khai thêm bảng câu hỏi ESS nhằm đánh giá mức độ buồn ngủ ban ngày của người dùng. Thang điểm này giúp phát hiện tình trạng buồn ngủ quá mức và có thể hỗ trợ phân tầng nguy cơ trong mô hình phân loại rối loạn giấc ngủ.

\vspace{0.5em}
\noindent\textbf{03)} Đánh giá \textbf{BMI (Body Mass Index)}: BMI được tự động tính toán dựa trên chiều cao và cân nặng người dùng nhập vào. Chỉ số này đóng vai trò là một trong các yếu tố nguy cơ chính trong chẩn đoán OSA, đặc biệt khi kết hợp cùng STOP-BANG.


Ngoài ra, nhằm cải thiện trải nghiệm người dùng và hỗ trợ trả lời câu hỏi liên quan 
đến giấc ngủ, tác giả 
phát triển thêm tính năng \textbf{chatbot y học giấc ngủ} dựa trên kỹ thuật 
\textbf{Retrieval-Augmented Generation (RAG)}. Chatbot này được xây dựng từ cơ 
sở dữ liệu gồm hơn 2000 câu hỏi và 
câu trả lời chuyên sâu liên quan đến giấc ngủ được biên tập bởi GS.TS Dương Quý Sỹ, 
bao gồm cả tài liệu lâm sàng, nghiên cứu khoa học và các hướng dẫn thực hành. 
Người dùng có thể đặt câu hỏi tự nhiên như “Tôi có nên lo nếu ngủ ngáy liên tục?” 
hoặc “STOP-BANG > 5 có ý nghĩa gì?”, và chatbot sẽ phản hồi dựa trên kiến 
thức được truy xuất từ tài liệu nền và được tổng hợp lại bằng mô hình ngôn ngữ.

Hệ thống RAG kết hợp khả năng truy vấn ngữ nghĩa từ tập văn bản lớn 
(document retrieval) và khả năng sinh văn bản linh hoạt từ mô hình ngôn ngữ lớn 
(LLM), từ đó cung cấp các câu trả lời chính xác, có căn cứ và dễ hiểu cho 
người dùng không chuyên.

\textbf{Tính năng quản lý người dùng} cũng được mở rộng. Người dùng có thể 
tạo tài khoản một lần và sử dụng lại trong các lần đăng nhập sau. Cơ chế này 
giúp rút ngắn thao tác, đồng thời vẫn đảm bảo tính bảo mật và khả năng khôi 
phục dữ liệu khi quên tài khoản hoặc mật khẩu. Dữ liệu người dùng 
(câu hỏi, chỉ số BMI, lịch sử cảm biến) được liên kết thống nhất qua một 
ID định danh duy nhất, hỗ trợ tốt cho việc phân tích, theo dõi tiến triển 
và huấn luyện mô hình học máy cá nhân hoá trong tương lai.



\subsection{Thiết kế và xây dựng hệ thống lưu trữ}
\begin{figure}[htbp]
    \centering
    \includegraphics[width=0.8\textwidth]{images/cloud.png}
    \caption{Mô hình tích hợp giữa mạng cảm biến và cấu trúc dữ liệu đám mây}
    \label{cloud}
\end{figure}

Trong hệ thống đề xuất, dữ liệu cảm biến đóng vai trò trung tâm trong việc 
huấn luyện và triển khai các mô hình trí tuệ nhân tạo (AI). Tuy nhiên, bộ nhớ 
của vi điều khiển và thiết bị đầu cuối thường bị giới hạn, do đó giải pháp lưu 
trữ dữ liệu trên nền tảng đám mây là lựa chọn phù hợp và linh hoạt. Việc triển 
khai dữ liệu lên cloud không chỉ giúp loại bỏ rào cản về địa lý, mà còn hỗ 
trợ truy cập, phân tích và chia sẻ dữ liệu từ bất kỳ đâu miễn có kết nối 
Internet. Đồng thời, hệ thống hỗ trợ xuất dữ liệu dưới dạng văn bản (text), 
CSV hoặc JSON, phục vụ nhu cầu chia sẻ giữa các nhóm nghiên cứu.

Về dài hạn, mục tiêu của hệ thống là tích luỹ một tập dữ liệu lớn và đa dạng 
nhằm huấn luyện các mô hình học máy hỗ trợ chẩn đoán và ra quyết định trong 
sàng lọc hội chứng ngưng thở khi ngủ (\gls{OSA}).



\textbf{Cơ sở dữ liệu sử dụng là MongoDB Atlas} với các đặc điểm kỹ thuật nổi bật như sau:
01) Hỗ trợ lưu trữ hiệu quả dữ liệu lớn, phân tán trên nhiều cụm máy chủ, 
cho phép mở rộng theo chiều ngang.
02) Tối ưu hoá truy vấn theo thời gian thực với dữ liệu dạng \texttt{timestamp}.
03) Cơ chế đánh chỉ mục linh hoạt giúp tăng tốc độ truy vấn và giảm dung lượng lưu trữ.
04) Hỗ trợ tự động xoá dữ liệu cũ dựa trên TTL (Time To Live index), 
đồng thời tích hợp trực tiếp với nền tảng MongoDB Atlas.

\begin{figure}[htbp]
\centering
\includegraphics[width=0.9\textwidth]{images/flow_http.png}
\caption{Lưu đồ thuật toán lưu trữ dữ liệu cảm biến}
\label{flow_http}
\end{figure}
Phía máy chủ của hệ thống được xây dựng bằng nền tảng Node.js và triển khai 
trên Amazon Web Services (AWS), cho phép triển khai nhanh, dễ mở rộng và tối 
ưu chi phí trong giai đoạn thử nghiệm. MongoDB Atlas được lựa chọn là hệ quản 
trị cơ sở dữ liệu chính, hỗ trợ gói miễn phí dung lượng 500MB – phù hợp cho 
việc thu thập và đánh giá dữ liệu ở quy mô ban đầu.

Để tránh tình trạng quá tải server khi có nhiều yêu cầu truy cập đồng thời, 
ứng dụng không thực hiện gửi từng mẫu riêng lẻ. Thay vào đó, dữ liệu cảm biến 
sẽ được tích luỹ theo từng lô gồm 1000 mẫu, sau đó mới được gửi lên backend. 
Mỗi mẫu bao gồm ba thành phần gia tốc (\texttt{x, y, z}) và thời gian ghi 
nhận tương ứng, bảo đảm tính toàn vẹn và khả năng truy xuất ngược theo dòng 
thời gian.

Lưu đồ thuật toán lưu trữ dữ liệu được thể hiện trong Hình~\ref{flow_http}, 
gồm hai trường hợp chính:

\vspace{0.5em}
\noindent\textbf{1)} Khi người dùng không có kết nối mạng, hệ thống vẫn cho phép kết nối BLE và hiển thị dữ liệu cảm biến theo thời gian thực, tuy nhiên sẽ không tiến hành lưu trữ lên cloud.

\vspace{0.5em}
\noindent\textbf{2)} Khi người dùng đã đăng nhập và có kết nối Internet, ứng dụng sẽ tự động lưu trữ dữ liệu sau mỗi 1000 mẫu thu thập. Trong trường hợp thao tác gửi dữ liệu thất bại liên tục quá 10 lần, hệ thống sẽ thông báo lỗi và ngừng tiến trình lưu trữ để đảm bảo độ tin cậy.


Ngoài dữ liệu cảm biến thời gian thực được lưu trữ trên MongoDB Atlas, 
hệ thống còn sử dụng cơ sở dữ liệu quan hệ MySQL để quản lý các dữ liệu định 
danh và nghiệp vụ quan trọng khác. Cụ thể:

\vspace{0.5em}
\noindent\textbf{1)} Thông tin người dùng như tài khoản đăng nhập, mật khẩu mã hoá (hash), email, số điện thoại, lịch sử đăng nhập và phân quyền được lưu trữ trong hệ quản trị cơ sở dữ liệu MySQL. Cấu trúc dữ liệu dạng bảng (table) của MySQL giúp đảm bảo tính toàn vẹn quan hệ và dễ dàng thực hiện các truy vấn xác thực người dùng nhanh chóng, an toàn.

\vspace{0.5em}
\noindent\textbf{2)} Các dữ liệu khảo sát lâm sàng như bảng điểm STOP-BANG, thang điểm Epworth, chỉ số BMI, tiền sử bệnh nền và lịch sử đánh giá lặp lại theo từng thời điểm cũng được lưu trong MySQL nhằm đảm bảo tính liên kết logic giữa các thực thể (người dùng – biểu mẫu – kết quả – thời gian).

\vspace{0.5em}
\noindent\textbf{3)} Việc phân chia lưu trữ theo đặc thù dữ liệu (NoSQL cho dữ liệu cảm biến lớn và động, SQL cho dữ liệu người dùng có cấu trúc ổn định) giúp tối ưu hoá hiệu suất truy xuất, tính mở rộng và khả năng bảo trì hệ thống trong dài hạn.

Sự kết hợp giữa MongoDB (dành cho dữ liệu cảm biến, thời gian thực) 
và MySQL (dành cho thông tin người dùng và nghiệp vụ) tạo thành một kiến trúc 
lưu trữ lai (hybrid storage architecture) đáp ứng linh hoạt cả hai loại dữ 
liệu – phi cấu trúc và có cấu trúc – vốn là đặc trưng phổ biến trong các 
hệ thống y tế ứng dụng trí tuệ nhân tạo hiện đại.

\subsection{Tìm hiểu, ứng dụng phân loại tư thế ngủ bằng học máy  }

Tác giả cũng đã tìm hiểu nhiều mô hình, phương pháp để phân loại các tư thế ngủ, tư thế cơ bản của con người và đánh giá chỉ số AHI dự trên các tín hiệu cảm biến thu được. Các bước cơ bản để tiến hành dự án học máy liên quan đến các tín hiệu cảm biến:


\begin{figure}[b!]
		\centering
 		\includegraphics[width=1\textwidth]{images/hocmay_time.png}
		\caption{Phân bố thời gian sử dụng đối với dự án học máy}
		\label{hocmay_time}
\end{figure}

\begin{itemize}
    \item Thu thập dự liệu (bao gồm thu thập và gắn nhãn cho dữ liệu)
    
    \item Khám phá dữ liệu (đánh giá cân bằng dữ liệu, tỉ lệ dữ liệu có ý nghĩa)
    
    \item Chuẩn bị dữ liệu (làm sạch dữ liệu, tạo ra các đặc tính trên miền thời gian và miền tần số)
    
    \item Mô hình hoá dữ liệu (lựa chọn ra các mô hình phù hợp)
    
    \item Lựa chọn tính năng (lựa chọn ra các tính năng có ý nghĩa cao đối với mô hình)

    \item Tinh chỉnh mô hình
\end{itemize}

Jeng PY và cộng sự đã đề xuất chế tạo 2 thiết bị đeo ở cổ và ở cổ tay để đánh giá tư thế ngủ ở người \cite{Jeng}. Trong dự án này, tín hiệu thu được ở thiết bị đeo ở tay được chia thành những cửa sổ 1 giây rồi trích xuất các tính năng trên cửa sổ đó. Cảm biến đeo ở cổ sẽ được sử dụng để lấy nhãn tín hiệu theo phương pháp lấy đa số của tín hiệu trong cửa sổ. Nhóm tác giả đã sử dụng mô hình SVM và RF để đánh giá và đạt được kết quả có độ chính xác lần lượt là 82\% và 72\%. Nhóm của Saha S., Kabir M và cộng sự đã tiến hành nghiên cứu 1 thiết bị đeo được sử dụng bao gồm cảm biến gia tốc, cảm biến âm trên 31 đối tượng thử nghiệm. Sau đó họ tiến hành loại bỏ các bộ dữ liệu có độ dài dưới 2 giờ và cuối cùng sử dụng so sánh ngưỡng để xác định chứng OSA bằng việc phân chia các cửa sổ 10s với độ lặp 80\% \cite{Saha}. Trong khi đó, nhóm của Syeda Zuriat-e-Zehra Ali và cộng sử đã nghiên cứu và thử nghiệm thiết bị gối ngủ để tự điều chỉnh hoặc báo hiệu khi có chứng ngưng thở khi ngủ dựa trên các tín hiệu thô như nồng độ Oxi trong máu, nhịp tim [27]. Jarvis L, Moninger S và cộng sử đã trình bày hệ thống phát hiện đánh giá 5 tư thế gồm nằm, nằm tựa, ngồi thẳng, đứng, đi bộ với tập dữ liệu đươc lấy từ 2 cảm biến gắn ở cổ và đùi \cite{Syeda}. Dữ liệu được lấy mẫu với tần số 25 Hz sau đó được lưu vào bộ nhớ cục bộ trên điện thoại rồi gửi bản csv qua mail. Mô hình học máy gồm hồi quy logistic, SVM, DT với độ chính xác cao > 96\% đã được sử dụng đánh giá tập dữ liệu gồm 6 hành động thường ngày của con người như đứng, ngồi, đi bộ, lên cầu thang, xuống cầu thang và nằm \cite{Uday}. Nhóm nghiên cứu của Gomes E, Bertini L và cộng sự đã nghiên cứu, xây dựng, đánh giá giữa 3 mô hình: K-Nearest-Neighbor (KNN), cây quyết định (Decision tree) và SVM. Trong các bước tiền xử lý các tác giả đã phân đoạn dữ liệu theo cửa sổ 2.5s không che phủ sau đó phân tích, chuẩn hoá dữ liệu và đã có độ chính xác > 97\% đối với việc phát hiện tư thế. Ở Việt Nam, nhóm tác gủa Vũ Ngọc Thanh Sang và Nguyễn Đức Thắng đã phát triển thiết bị thu thâp dữ liệu từ điện thoại sau đó qua các bước xử lý dữ liệu, trích xuất tính năng và phân loại bằng các mô hình K-hàng xóm gần nhất (KNN) với độ chính xác là 100\% với toàn bộ tư thế ngoại trừ lái xe là 80\% \cite{Sang}. Qua tổng quan tài liệu tác giả nhận thấy các phương pháp học máy cổ điển đang chiến ưu thế hơn so với các phương pháp học sâu vì phát triển nhanh và dễ dàng và phù hợp với tính chất của bài toán đánh giá các tư thế của con người sử dụng cảm biến gia tốc. Trong đó, nổi bật lên là mô hình SVM, hồi quy Logistic và Random Forest. Từ đó, tác giả sẽ tập trung tìm hiểu và hướng tới áp dụng cho tập dữ liệu của tác giả.


\textbf{Hồi quy logistic - LR}: Đây là phương thức tốt nhất cho các vấn đề phân loại nhị phân (vấn đề với hai lớp giá trị). Hồi quy logistic giống như hồi quy tuyến tính với mục đích là để tìm ra các giá trị cho các hệ số mà trọng lượng mỗi biến đầu vào. Không giống như hồi quy tuyến tính, dự đoán đầu ra được chuyển đổi bằng cách sử dụng một hàm không tuyến tính được gọi là hàm logistic. Hàm logistic trông giống như một chữ S lớn và sẽ biến đổi bất kỳ giá trị nào thành 0-1. Tuy nhiên, nhược điểm của nó là chỉ giải quyết được bài toán phân loại 2 lớp. Để giải quyết được những bài toán đa lớp chúng ta có thể sử dụng mô hình Softmax Logistic là dạng tổng quát của hồi quy Logistic.

\textbf{Máy vec tơ hỗ trợ (Support vector machines - SVM)}: xây dựng một mặt siêu phẳng được sử dụng để phân chia không gian biến đầu vào. Trong SVM, một mặt siêu phẳng được chọn để phân tách tốt nhất các điểm trong không gian các biến đầu vào theo lớp của chúng, hoặc là lớp 0 hoặc lớp 1. Trong không gian hai chiều, có thể hình dung nó như một đường thẳng và giả sử rằng tất cả các biến đầu vào có thể được tách hoàn toàn bằng đường thẳng này. Thuật toán SVM tìm ra các hệ số dẫn đến sự phân tách tốt nhất của các lớp theo mặt siêu phẳng Hình ~\ref{svm}.


\begin{figure}
    \centering
    \includegraphics[width=1\linewidth]{images/svm.png}
    \caption{Tối ưu siêu phẳng sử dụng thuật toán SVM}
    \label{svm}
\end{figure}
Khoảng cách giữa mặt siêu phẳng và điểm dữ liệu gần nhất được gọi là biên. Mặt siêu phẳng tốt nhất hoặc tối ưu có thể tách riêng hai lớp là dòng có biên lớn nhất. Chỉ những điểm này có liên quan đến việc xác định hyperplane và trong việc xây dựng các điểm phân loại. Những điểm này được gọi là các vector hỗ trợ. Chúng hỗ trợ hoặc xác định hyperplane. Trong thực tế, một thuật toán tối ưu được sử dụng để tìm các giá trị cho các hệ số tối đa hóa biên. SVM có thể là một trong những phương pháp phân loại hàng đầu mạnh mẽ nhất và đáng thử trên tập dữ liệu. Cũng như hồi quy Logistic thì SVM cũng chỉ sử dụng để phân loại nhị phân. Để giải quyết vấn đề này thì có 2 phương pháp:

\begin{figure}
    \centering
    \includegraphics[width=0.6\linewidth]{images/svm_ovso.png}
    \caption{Thuật toán một với một}
    \label{svm_ovso}
\end{figure}



\begin{figure}
    \centering
    \includegraphics[width=0.6\linewidth]{images/ovsr.png}
    \caption{Thuật toán một với nhiều}
    \label{ovsr}
\end{figure}

\begin{itemize}
    \item Một với một (one vs one): Một mặt siêu phẳng được thiết lập để phân tách giữa hai lớp, bỏ qua các điểm của lớp thứ ba. Điều này có nghĩa là sự phân tách chỉ tính đến điểm của hai lớp trong sự phân tách hiện tại. Ví dụ: đường màu đỏ-xanh dương sẽ tách tối đa khoảng cách chỉ giữa các điểm màu xanh lam và màu đỏ Hình ~\ref{svm_ovso}.
    
    \item Một với nhiều (one vs rest): Cần một mặt siêu phẳng để tách biệt giữa một lớp và tất cả các lớp khác cùng một lúc. Điều này có nghĩa là sự tách biệt có tính đến tất cả các điểm, chia chúng thành hai nhóm; một nhóm cho các điểm của lớp và một nhóm cho tất cả các điểm khác. Ví dụ: đường màu sẽ tách tối đa hóa khoảng cách giữa các điểm màu lục và tất cả các điểm khác cùng một lúc Hình ~\ref{ovsr}.
\end{itemize}






\textbf{Rừng ngẫu nhiên (Random Forest - RF)}: được xây dựng trên cơ sở thuật toán Decision Tree (cây quyết định). Mỗi cây quyết định sẽ khác nhau (có yếu tố ngẫu nhiên khác nhau). Sau đó kết quả dự đoán được tổng hợp từ các cây quyết định. Trong thuật toán Decision Tree, khi xây dựng cây quyết định nếu để độ sâu tùy ý thì cây sẽ phân loại đúng hết các dữ liệu trong tập training dẫn đến mô hình có thể dự đoán tệ trên tập validation/test, khi đó mô hình bị quá khớp (overfitting).
Thuật toán Random Forest gồm nhiều cây quyết định, mỗi cây quyết định đều có những yếu tố ngẫu nhiên:
Lấy ngẫu nhiên dữ liệu để xây dựng cây quyết định.
Lấy ngẫu nhiên các thuộc tính để xây dựng cây quyết định.
Do mỗi cây quyết định trong thuật toán Random Forest không dùng tất cả dữ liệu training, cũng như không dùng tất cả các thuộc tính của dữ liệu để xây dựng cây nên mỗi cây có thể sẽ dự đoán không tốt, khi đó mỗi mô hình cây quyết định không bị overfitting mà có thế bị underfitting, hay nói cách khác là mô hình có high bias. Tuy nhiên, kết quả cuối cùng của thuật toán Random Forest lại tổng hợp từ nhiều cây quyết định, thế nên thông tin từ các cây sẽ bổ sung thông tin cho nhau, dẫn đến mô hình có low bias và low variance, hay mô hình có kết quả dự đoán tốt.
	Những kiến thức cơ bản về học máy sẽ được ứng dụng sâu trong những nghiên cứu tới đây của tác giả.

