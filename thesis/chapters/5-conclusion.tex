\chapter*{KẾT LUẬN}

Luận văn đã xây dựng được một nền tảng phần cứng hoàn chỉnh phục vụ nghiên cứu,
trong đó vi điều khiển nRF52840 được tích hợp cùng cảm biến gia tốc và các khối
chức năng cần thiết nhằm bảo đảm khả năng thu nhận và xử lý tín hiệu ổn định.
Song song với đó, hệ thống thu thập - truyền dẫn - lưu trữ dữ liệu dựa trên
giao tiếp BLE cũng đã được phát triển, bao gồm ứng dụng di động, máy chủ và cơ
sở dữ liệu. Cấu trúc này không chỉ đáp ứng yêu cầu của bài toán tư thế ngủ mà
còn cho phép mở rộng linh hoạt đối với các loại cảm biến khác trong tương lai.

Trên cơ sở dữ liệu thu được, luận văn đã khảo sát và đề xuất các mô hình học
máy phù hợp cho phân loại tư thế ngủ, bao gồm Logistic Regression, Random
Forest, Gradient Boosting, Support Vector Machine và mạng nơ-ron nhân tạo. Các
mô hình này cho thấy hiệu năng ấn tượng, với độ chính xác tối đa đạt 99,8\%,
chứng minh tính khả thi của bài toán khi triển khai trên thiết bị có tài nguyên
hạn chế.

Để đánh giá khả năng vận hành thực tế tại biên, luận văn đã thử nghiệm triển khai một mô hình truyền thống (Logistic Regression) và một mô hình học sâu (ANN) trực tiếp trên phần cứng nRF52840.
Kết quả cho thấy thời gian suy luận đạt khoảng  $501\,\mu\text{s}$
, đáp ứng tốt yêu cầu xử lý theo thời gian thực của thiết bị đeo.

Cuối cùng, luận văn cũng mở rộng hướng nghiên cứu sang bài toán nhận biết trạng
thái “nằm” hoặc “không nằm”, tạo tiền đề cho các ứng dụng giám sát giấc ngủ đa
mục tiêu và hỗ trợ sàng lọc sớm các rối loạn hô hấp khi ngủ.

Đến đây, toàn bộ nội dung luận văn thạc sĩ \textit{NGHIÊN CỨU, PHÁT TRIỂN MÔ HÌNH HỌC
  MÁY TẠI BIÊN NHẰM PHÂN LOẠI TƯ THẾ
  NGỦ} đã được trình bày.

Em xin cảm ơn các Thầy/Cô đã đọc và góp ý!

\chapter*{PHỤ LỤC}

Toàn bộ mã nguồn và dữ liệu được công bố tại:
\href{https://github.com/tranhuunam00/Master_2024}{Github Repository –
  Master\_2024}.

Mọi thông tin thắc mắc có liên quan vui lòng liên hệ tác giả thông qua email:
tranhuunam23022000@gmail.com
