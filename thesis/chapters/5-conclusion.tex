\chapter*{KẾT LUẬN}

Các mô hình học máy LR, RF, SVM và GB, ANN đã được áp dụng để phân loại tư thế
ngủ với độ chính xác cao. Kèm với hai mô hình đã chạy ổn định trên thiết bị
phần cứng của nhóm thiết kế, tác giả tự đánh giá các kết quả này đã hoàn thành
mục tiêu của đề tại đề ra từ trước.

Trong tương lai, nhóm sẽ mở rộng hơn các mô hình học sâu để đánh giá đâu là mô
hình phù hợp nhất trong phân loại tư thế ngủ sử dụng dữ liệu từ cảm biến gia
tốc. Tiếp nữa, nhóm sẽ sử dụng thêm dữ liệu từ các cảm biến khác, nâng cấp hệ
thống thu thập - lưu trữ dữ liệu, nhằm phục vụ các bài toán liên quan đến sàng
lọc OSA.

Đến đây, toàn bộ nội dung luận văn thạc sĩ \textit{NGHIÊN CỨU, PHÁT TRIỂN MÔ HÌNH HỌC
  MÁY TẠI BIÊN NHẰM PHÂN LOẠI TƯ THẾ
  NGỦ} đã được trình bày.

Em xin cảm ơn các Thầy/Cô đã đọc và góp ý!

\chapter*{PHỤ LỤC}

Toàn bộ mã nguồn và dữ liệu được công bố tại:
\href{https://github.com/tranhuunam00/Master_2024}{Github Repository –
  Master\_2024}.

Mọi thông tin thắc mắc có liên quan vui lòng liên hệ tác giả thông qua email:
tranhuunam23022000@gmail.com
