\chapter*{KẾT LUẬN}

Trong giai đoạn nghiên cứu trước, các mô hình học máy như Logistic Regression,
Random Forest, Support Vector Machine và Gradient Boosting đã được áp dụng để
phân loại tư thế ngủ với độ chính xác cao. Đồng thời, một mô hình mạng nơ-ron
(ANN) cũng đã được triển khai thành công trên phần cứng của nhóm phát triển,
chứng minh tính khả thi của việc chạy mô hình trực tiếp trên phần cứng nhúng.

Kèm với hai mô hình đã chạy ổn định trên thiết bị phần cứng của nhóm thiết kế,
tác giả tự đánh giá các kết quả này đã hoàn thành mục tiêu của đề tại đề ra từ
trước.

Trong tương lai, tác giả đánh giá trên các mô hình học máy khác với bài toán tư
thế ngủ để xác định được mô hình phù hợp nhất. Ngoài ra, việc các đặc trưng kết
hợp cũng là một định hướng của nhóm. Không dừng lại ở đó, nhóm sẽ tối ưu thêm
phần cứng, thêm các cảm biến để hướng tới bài toán phân loại OSA. Phần mềm sẽ
được bổ sung các tính năng y học để trở thành công cụ cá nhân hóa.

