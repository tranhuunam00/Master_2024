Nhiều nghiên cứu đã tập trung phát triển các hệ thống theo dõi tư thế ngủ dựa
trên đa dạng công nghệ cảm biến. Các phương pháp này trải rộng từ cảm biến tiếp
xúc trực tiếp đến hệ thống đo không tiếp xúc, cụ thể gồm: Tấm dẫn điện với các
điện cực ECG sắp xếp tại giường để nhận tín hiệu sinh lý và áp lực cơ thể
\cite{Lee2013_12ecg} Cảm biến áp suất dưới nệm đo phân bố lực tiếp xúc toàn
thân \cite{Nam2016_mattre, Enayati2018_mattre}. Camera hồng ngoại ghi hình ban
đêm không cần chiếu sáng, nhận dạng tư thế bằng thị giác máy
\cite{Mohammadi2019_camera, Tam2021_camera}.

Các thiết bị đeo sử dụng tín hiệu đo lường quán tính (Inertial Measurement Unit
- IMU) được đặt đa vị trí trên cơ thể nổi lên như cách tiếp cận nhỏ gọn, thuận
tiện. Trong nghiên cứu \cite{Jeng_osa}, tác giả đã đề xuất hệ thống iSleePost
sử dụng cảm biến gia tốc đeo ở cổ tay có độ chính xác lên tới 85\% trong việc
nhận dạng các tư thế ngủ bằng các phương pháp học máy. Một thiết bị đeo khác có
tên MORFEA được gắn phía trên mũi đã được đề xuất trong một nghiên cứu, với mục
tiêu chính là sàng lọc tại nhà các rối loạn hô hấp liên quan đến giấc ngủ
\cite{Manoni2020_posture}. Thiết bị này tích hợp cảm biến gia tốc ba trục để
phân tích chuyển động, trong đó tư thế ngủ được ước lượng dựa trên góc xoay và
độ nghiêng của đầu, so sánh với bảng điều kiện tham chiếu định sẵn. Tuy nhiên,
tư thế nằm sấp không được xem xét do vị trí gắn thiết bị trên mũi. Một nghiên
cứu gần đây giới thiệu hệ thống NAPPA - tã thông minh được thiết kế để theo dõi
nhịp thở và tư thế của trẻ sơ sinh trong khi ngủ trưa, tích hợp cảm biến gia
tốc và con quay hồi chuyển để ước lượng tư thế ngủ \cite{Ranta2021_ta}. Tuy
nhiên, phương pháp và kết quả phát hiện tư thế ngủ chi tiết không được công bố

Các nghiên cứu sử dụng duy nhất một cảm biến gia tốc cũng cho thấy tiềm năng
trong việc phân loại tư thế ngủ. Nghiên cứu sử dụng cảm biến gia tốc dạng miếng
dán (patch-type accelerometer) cho thấy độ chính xác tổng thể đạt 99,16\% khi
cảm biến được gắn ở bên trái ngực, với các điều kiện định sẵn theo ba trục X,
Y, Z \cite{Yoon2015_posture}. Tuy nhiên, không có tư thế nằm sấp được ghi nhận
trong nghiên cứu này. Nhóm tác giả Vũ Hoàng Diệu đã nghiên cứu phát triển thiết
bị đeo sử dụng một cảm biến gia tốc duy nhất, áp dụng mô hình học sâu AnpoNet
(1D-CNN + LSTM) để phân loại 12 tư thế ngủ với độ chính xác 94,67\%, hỗ trợ
theo dõi giấc ngủ tại nhà cho bệnh nhân trào ngược dạ dày (GERD)
\cite{Vu2025SleepPosition}. Đối với nghiên cứu \cite{abdulsadig2023}, tác giả
sử dụng một cảm biến gia tốc đặt ở cổ để phát hiện tư thế ngủ tự động, đánh giá
ba mô hình (DT, ET, LSTM-NN) và cho thấy rằng với tần số lấy mẫu chỉ 5Hz và cửa
sổ 1 giây, các mô hình vẫn đạt độ chính xác > 98\% (F1-score trung bình
0,945–0,975), trong đó cây quyết định (Decision Tree) có hiệu năng và mức tiêu
thụ bộ nhớ tối ưu nhất. Nhóm tác giả trong nghiên cứu \cite{Ferrer_osa} phát
triển ứng dụng SleepPos với độ chính xác tổng thể 98,2\%, nhưng độ nhạy với tư
thế nằm sấp chỉ đạt 38,9\%, khiến hệ thống không phù hợp cho nhóm người có nguy
cơ cao với tư thế này; thêm vào đó, thiết kế phần cứng cồng kềnh làm giảm tính
thoải mái và độ tin cậy khi sử dụng hằng ngày. Nghiên cứu trên 89 bệnh nhân sử
dụng cảm biến Clebre gắn tại hõm ức tích hợp gia tốc kế 3 trục cho thấy độ
chính xác 96,9–98,6\% trong phát hiện tư thế ngủ, khẳng định hõm ức là vị trí
tối ưu cho cảm biến đeo cổ \cite{Kukwa2022_xuong_uc_co}.

Hầu hết các nghiên cứu về đánh giá tư thế ngủ đều đưa ra những phương pháp toàn
diện sử dụng các mô hình học máy, kết hợp nhiều loại cảm biến khác nhau như cảm
biến quán tính, cảm biến sinh lý, hoặc cảm biến áp suất nhằm thu thập dữ liệu
đa chiều về trạng thái của cơ thể. Một số nghiên cứu đã chứng minh tính khả thi
của việc sử dụng duy nhất một cảm biến gia tốc trong nhận dạng tư thế ngủ, đạt
được độ chính xác cao khi kết hợp với các mô hình học máy. Các kết quả này cho
thấy hướng tiếp cận đơn cảm biến hoàn toàn có thể thay thế cho hệ thống đa cảm
biến truyền thống trong các bài toán phân loại tư thế ngủ.

Tuy nhiên, các công trình hiện có chưa xem xét đầy đủ việc phân biệt trạng thái
“đã nằm” hay “chưa nằm” – một yếu tố nền tảng để xác định chính xác giai đoạn
bắt đầu giấc ngủ hoặc phát hiện sớm thay đổi tư thế. Đồng thời, việc lựa chọn
tập đặc trưng và mô hình học máy phù hợp cho dữ liệu từ một cảm biến gia tốc
duy nhất đặt tại vùng cổ – vị trí có khả năng phản ánh biến thiên tư thế của
thân trên vẫn chưa được nghiên cứu hệ thống.

Đặc biệt, tại Việt Nam cho đến nay chưa có công bố khoa học nào tập
trung vào bài toán nhận dạng tư thế ngủ dựa hoàn toàn trên dữ liệu
cảm biến gia tốc và được triển khai trực tiếp trên phần cứng
hạn chế tài nguyên, như các vi điều khiển sử dụng trong ứng dụng
Edge AI hoặc TinyML. Khoảng trống này đặt ra yêu cầu cần thiết cho
việc phát triển một hệ thống giám sát tư thế ngủ đơn giản, chi phí
thấp, có khả năng xử lý cục bộ, hướng tới ứng dụng trong sàng lọc và
hỗ trợ chẩn đoán sớm hội chứng ngưng thở tắc nghẽn khi ngủ (OSA).

Qua đó, luận văn này tập trung làm rõ các đặc trưng của dữ liệu cảm biến gia
tốc khi được đặt tại vùng cổ, đồng thời đề xuất và đánh giá các mô hình học máy
phù hợp nhằm tối ưu cho việc triển khai trên các thiết bị tính toán biên (edge
devices). Trên cơ sở đó, luận văn xây dựng một hệ thống hoàn chỉnh gồm phần
cứng đeo được kích thước nhỏ gọn, chi phí thấp; phần mềm thu thập và lưu trữ dữ
liệu; cùng với mô-đun xử lý và phân loại tư thế ngủ tích hợp trực tiếp trên vi
điều khiển.

Hệ thống được thiết kế hướng tới độ chính xác cao, độ trễ thấp và khả năng hoạt
động độc lập, cho phép người dùng có thể ghi nhận và phân loại tư thế ngủ tại
nhà một cách thuận tiện. Mặc dù chưa tích hợp chức năng sàng lọc hội chứng
ngưng thở tắc nghẽn khi ngủ (OSA), kết quả nghiên cứu này tạo nền tảng kỹ thuật
tiềm năng cho các ứng dụng trong tương lai, đặc biệt trong việc hỗ trợ đánh giá
nguy cơ pOSA (positional OSA) dựa trên tư thế ngủ và xu hướng thay đổi tư thế
trong suốt giấc ngủ.

Mục tiêu cụ thể của khóa luận gồm: 01) Nghiên cứu, đề xuất hệ thống phần cứng
phục vụ đo lường, thu thập, xử lý tín hiệu gia tốc kèm với hiệu năng phù hợp
cho việc triển khai mô hình học máy tại biên trong bài toán phân loại tư thế
ngủ ở người; 02) Nghiên cứu, đề xuất những mô hình học máy thích hợp trong việc
phân loại tư thế ngủ ở người dựa trên những đặc trưng của tín hiệu gia tốc được
trích xuất từ bộ dữ liệu cảm biến do nhóm nghiên cứu thu thập; 03) Chuẩn
hóa,thực thi, đánh giá mô hình học máy trên vi điều khiển Nordic nRF52840, nhằm
kiểm chứng hoạt động thực tế của hệ thống nhận dạng tư thế ngủ tại biên;

Cấu trúc luận văn được trình bày trong bốn chương chính như sau:

\noindent\textbf{Chương 1:} Tổng quan OSA, tư thế ngủ và công nghệ liên quan

\noindent\textbf{Chương 2:} Hệ thống thu thập và xử lý tín hiệu cảm biến.

\noindent\textbf{Chương 3:} Trích xuất đặc trưng và mô hình học máy.

\noindent\textbf{Chương 4:} Kết quả và đánh giá.