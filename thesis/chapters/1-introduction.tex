Tư thế ngủ nắm vai trò đặc biệt quan trọng trong việc duy trì sức khỏe tổng thể
và chất lượng giấc ngủ, bên cạnh các yếu tố khác như thời lượng, môi trường và
thói quen ngủ. Nhiều nghiên cứu đã chỉ ra rằng tư thế nằm có thể ảnh hưởng trực
tiếp đến hoạt động của hệ hô hấp, tim mạch và hệ cơ - xương, đặc biệt là cột
sống \cite{Cary2021_tu_the_ngủ}. Trong hội chứng ngưng thở tắc nghẽn khi ngủ ,
tư thế ngủ được xem là một yếu tố quyết định mức độ nghiêm trọng của bệnh
\cite{cartwright1984position}. Vì vậy, phân loại đúng tư thế ngủ giúp góp phần
đánh giá đúng dạng, tình trạng OSA và đưa ra những liệu trình điều trị phù hợp.

Các nghiên cứu hiện nay đã đang tập trung phát triển các hệ thống theo dõi tư
thế ngủ dựa trên đa dạng công nghệ cảm biến. Các phương pháp này trải rộng giữa
dạng tiếp xúc và không tiếp xúc, cụ thể gồm: tấm dẫn điện với các điện cực ECG
đặt tại giường thu thập tín hiệu nhịp tim để phân loại tư thế ngủ
\cite{Lee2013_12ecg}; cảm biến áp suất dưới nệm đo phân bố lực tiếp xúc toàn
thân \cite{Nam2016_mattre, Enayati2018_mattre}; camera hồng ngoại ghi hình ban
đêm không cần chiếu sáng, nhận dạng tư thế bằng thị giác máy
\cite{Mohammadi2019_camera, Tam2021_camera}.

Các thiết bị đeo sử dụng tín hiệu đo lường quán tính (Inertial Measurement Unit
- IMU) được đặt tại nhiều vị trí khác nhau trên cơ thể nổi lên như cách tiếp
cận nhỏ gọn, thuận tiện. Trong nghiên cứu \cite{Jeng_osa}, tác giả đã đề xuất
hệ thống iSleePost sử dụng cảm biến gia tốc đeo ở cổ tay có độ chính xác lên
tới 85\% trong việc nhận dạng các tư thế ngủ bằng các phương pháp học máy. Một
thiết bị đeo khác có tên MORFEA được gắn phía trên mũi
\cite{Manoni2020_posture}. Thiết bị này tích hợp cảm biến gia tốc ba trục để
phân tích chuyển động, trong đó tư thế ngủ được ước lượng dựa trên góc xoay và
độ nghiêng của đầu và tham chiếu tới bảng điều kiện định sẵn. Tuy nhiên, tư thế
nằm sấp không được xem xét do vị trí gắn thiết bị trên mũi. Một nghiên cứu gần
đây giới thiệu hệ thống NAPPA - tã thông minh được thiết kế để theo dõi nhịp
thở và tư thế của trẻ sơ sinh trong khi ngủ \cite{Ranta2021_ta}. Thiết bị tích
hợp cảm biến gia tốc và con quay hồi chuyển. Tuy nhiên, trong nghiên cứu này,
phương pháp và kết quả phát hiện tư thế ngủ chi tiết không được công bố

Các nghiên cứu sử dụng duy nhất một cảm biến gia tốc trong việc phân loại tư
thế ngủ cũng cho thấy tiềm năng rất lớn. Nghiên cứu sử dụng cảm biến gia tốc
dạng miếng dán cho thấy độ chính xác tổng thể đạt 99,16\% khi cảm biến được gắn
ở bên trái ngực, với các điều kiện định sẵn theo ba trục X, Y, Z
\cite{Yoon2015_posture}. Tuy nhiên, không có tư thế nằm sấp được ghi nhận trong
nghiên cứu này. Nhóm tác giả Vũ Hoàng Diệu đã nghiên cứu phát triển thiết bị
đeo sử dụng một cảm biến gia tốc duy nhất, áp dụng mô hình học sâu AnpoNet
(1D-CNN + LSTM) để phân loại 12 tư thế ngủ với độ chính xác 94,67\%, hỗ trợ
theo dõi giấc ngủ tại nhà cho bệnh nhân trào ngược dạ dày
\cite{Vu2025SleepPosition}. Đối với nghiên cứu \cite{abdulsadig2023}, tác giả
sử dụng một cảm biến gia tốc đặt ở cổ, đánh giá qua ba mô hình (DT, ET, LSTM)
dựa trên tần số lấy mẫu 5Hz và cửa sổ 1 giây, đạt độ chính xác > 98\% (F1-score
trung bình 0,945-0,975), trong đó DT có hiệu năng và mức tiêu thụ bộ nhớ tối ưu
nhất. Nhóm tác giả trong nghiên cứu \cite{Ferrer_osa} phát triển ứng dụng
SleepPos với độ chính xác tổng thể 98,2\%, nhưng độ nhạy với tư thế nằm sấp chỉ
đạt 38,9\%. Thêm vào đó, thiết kế phần cứng còn khá lớn làm giảm tính thoải
mái. Nghiên cứu trên 89 bệnh nhân sử dụng cảm biến gắn tại hõm ức tích hợp gia
tốc kế 3 trục cho thấy độ chính xác 96.9-98.6\% trong phát hiện tư thế ngủ,
khẳng định hõm ức là vị trí tối ưu cho cảm biến đeo cổ
\cite{Kukwa2022_xuong_uc_co}. Các kết quả này cho thấy hướng tiếp cận đơn cảm
biến hoàn toàn có thể thay thế cho hệ thống đa cảm biến trong bài toán phân
loại tư thế ngủ.

Tuy nhiên, các công trình hiện có chưa xem xét đầy đủ việc phân biệt trạng thái
“đã nằm” hay “chưa nằm” - một yếu tố nền tảng để xác định chính xác giai đoạn
bắt đầu giấc ngủ hoặc phát hiện sớm thay đổi tư thế. Đồng thời, việc lựa chọn
tập đặc trưng và mô hình học máy phù hợp cho dữ liệu từ một cảm biến gia tốc
duy nhất đặt tại vùng cổ - vị trí có khả năng phản ánh biến thiên tư thế tốt
nhất vẫn chưa được nghiên cứu cụ thể. Tại Việt Nam cho đến nay chưa có công bố
khoa học nào tập trung vào bài toán nhận dạng tư thế ngủ dựa hoàn toàn trên dữ
liệu cảm biến gia tốc và được triển khai mô hình học máy trực tiếp trên phần
cứng hạn chế tài nguyên. Khoảng trống này đặt ra yêu cầu cần thiết cho việc
phát triển một hệ thống giám sát tư thế ngủ đơn giản, chi phí thấp, có khả năng
xử lý cục bộ, hướng tới ứng dụng trong sàng lọc và hỗ trợ chẩn đoán sớm hội
chứng ngưng thở tắc nghẽn khi ngủ.

Vì vậy, luận văn này tập trung làm rõ các đặc trưng của dữ liệu cảm biến gia
tốc khi được đặt tại vùng cổ, đồng thời đi sâu tìm hiểu các đặc trưng dữ liệu
cảm biến gia tốc và đánh giá các mô hình học máy phù hợp nhằm tối ưu cho việc
triển khai trên các thiết bị tính toán biên. Để đạt được điều đó, luận văn sẽ
xây dựng một hệ thống hoàn chỉnh gồm phần cứng đeo được kích thước nhỏ gọn, chi
phí thấp; phần mềm thu thập; máy chủ lưu trữ dữ liệu; cùng với mô-đun xử lý và
phân loại tư thế ngủ tích hợp trực tiếp trên vi điều khiển. Mặc dù chưa tích
hợp chức năng sàng lọc hội chứng ngưng thở tắc nghẽn khi ngủ, kết quả nghiên
cứu này tạo nền tảng kỹ thuật tiềm năng cho các ứng dụng trong tương lai, đặc
biệt trong việc hỗ trợ đánh giá nguy cơ tư thế của OSA.

Mục tiêu cụ thể của luận văn gồm: 01) Nghiên cứu, đề xuất hệ thống phần cứng
phục vụ đo lường, thu thập, xử lý tín hiệu gia tốc kèm với hiệu năng phù hợp
cho việc triển khai mô hình học máy tại biên trong bài toán phân loại tư thế
ngủ ở người; 02) Nghiên cứu, đề xuất những mô hình học máy thích hợp trong việc
phân loại tư thế ngủ ở người dựa trên những đặc trưng của tín hiệu gia tốc được
trích xuất từ bộ dữ liệu cảm biến do nhóm nghiên cứu thu thập; 03) Chuẩn hóa,
thực thi, đánh giá mô hình học máy trên vi điều khiển Nordic nRF52840, nhằm
kiểm chứng hoạt động thực tế của hệ thống nhận dạng tư thế ngủ tại biên;

Cấu trúc luận văn được trình bày trong bốn chương chính như sau:

\noindent\textbf{Chương 1:} Tổng quan OSA, mối liên hệ với tư thế ngủ và kĩ thuật sàng lọc, phân loại.

\noindent\textbf{Chương 2:} Hệ thống thu thập, xử lý tín hiệu cảm biến và đánh giá bằng mô hình học máy.

\noindent\textbf{Chương 3:} Kết quả và đánh giá.