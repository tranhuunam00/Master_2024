Nhiều nghiên cứu đã tập trung phát triển các hệ thống theo dõi tư thế ngủ dựa
trên đa dạng công nghệ cảm biến. Các phương pháp này trải rộng từ cảm biến tiếp
xúc trực tiếp đến hệ thống đo không tiếp xúc, cụ thể gồm: Tấm dẫn điện với các
điện cực ECG sắp xếp tại giường để nhận tín hiệu sinh lý và áp lực cơ thể
\cite{Lee2013_12ecg} Cảm biến áp suất dưới nệm đo phân bố lực tiếp xúc toàn
thân \cite{Nam2016_mattre, Enayati2018_mattre}. Camera hồng ngoại ghi hình ban
đêm không cần chiếu sáng, nhận dạng tư thế bằng thị giác máy
\cite{Mohammadi2019_camera, Tam2021_camera}.

Các thiết bị đeo sử dụng tín hiệu đo lường quán tính (Inertial Measurement Unit
- IMU) được đặt đa vị trí trên cơ thể nổi lên như cách tiếp cận nhỏ gọn, thuận
tiện. Trong nghiên cứu \cite{Jeng_osa}, tác giả đã đề xuất hệ thống iSleePost
sử dụng cảm biến gia tốc đeo ở cổ tay có độ chính xác lên tới 85\% trong việc
nhận dạng các tư thế ngủ bằng các phương pháp học máy. Một thiết bị đeo khác có
tên MORFEA được gắn phía trên mũi đã được đề xuất trong một nghiên cứu, với mục
tiêu chính là sàng lọc tại nhà các rối loạn hô hấp liên quan đến giấc ngủ
\cite{Manoni2020_posture}. Thiết bị này tích hợp cảm biến gia tốc ba trục để
phân tích chuyển động, trong đó tư thế ngủ được ước lượng dựa trên góc xoay và
độ nghiêng của đầu, so sánh với bảng điều kiện tham chiếu định sẵn. Tuy nhiên,
tư thế nằm sấp không được xem xét do vị trí gắn thiết bị trên mũi. Một nghiên
cứu gần đây giới thiệu hệ thống NAPPA - tã thông minh được thiết kế để theo dõi
nhịp thở và tư thế của trẻ sơ sinh trong khi ngủ trưa, tích hợp cảm biến gia
tốc và con quay hồi chuyển để ước lượng tư thế ngủ \cite{Ranta2021_ta}. Tuy
nhiên, phương pháp và kết quả phát hiện tư thế ngủ chi tiết không được công bố

Các nghiên cứu sử dụng duy nhất một cảm biến gia tốc cũng cho thấy tiềm năng
trong việc phân loại tư thế ngủ. Nghiên cứu sử dụng cảm biến gia tốc dạng miếng
dán (patch-type accelerometer) cho thấy độ chính xác tổng thể đạt 99,16\% khi
cảm biến được gắn ở bên trái ngực, với các điều kiện định sẵn theo ba trục X,
Y, Z \cite{Yoon2015_posture}. Tuy nhiên, không có tư thế nằm sấp được ghi nhận
trong nghiên cứu này. Nhóm tác giả Vũ Hoàng Diệu đã nghiên cứu phát triển thiết
bị đeo sử dụng một cảm biến gia tốc duy nhất, áp dụng mô hình học sâu AnpoNet
(1D-CNN + LSTM) để phân loại 12 tư thế ngủ với độ chính xác 94,67\%, hỗ trợ
theo dõi giấc ngủ tại nhà cho bệnh nhân trào ngược dạ dày (GERD)
\cite{Vu2025SleepPosition}. Đối với nghiên cứu \cite{abdulsadig2023}, tác giả
sử dụng một cảm biến gia tốc đặt ở cổ để phát hiện tư thế ngủ tự động, đánh giá
ba mô hình (DT, ET, LSTM-NN) và cho thấy rằng với tần số lấy mẫu chỉ 5Hz và cửa
sổ 1 giây, các mô hình vẫn đạt độ chính xác > 98\% (F1-score trung bình
0,945–0,975), trong đó cây quyết định (Decision Tree) có hiệu năng và mức tiêu
thụ bộ nhớ tối ưu nhất. Nhóm tác giả trong nghiên cứu \cite{Ferrer_osa} phát
triển ứng dụng SleepPos với độ chính xác tổng thể 98,2\%, nhưng độ nhạy với tư
thế nằm sấp chỉ đạt 38,9\%, khiến hệ thống không phù hợp cho nhóm người có nguy
cơ cao với tư thế này; thêm vào đó, thiết kế phần cứng cồng kềnh làm giảm tính
thoải mái và độ tin cậy khi sử dụng hằng ngày. Nghiên cứu trên 89 bệnh nhân sử
dụng cảm biến Clebre gắn tại hõm ức tích hợp gia tốc kế 3 trục cho thấy độ
chính xác 96,9–98,6\% trong phát hiện tư thế ngủ, khẳng định hõm ức là vị trí
tối ưu cho cảm biến đeo cổ \cite{Kukwa2022_xuong_uc_co}.

Hầu hết các nghiên cứu về đánh giá tư thế ngủ đều đưa ra những phương pháp toàn
diện sử dụng các mô hình học máy, kết hợp nhiều loại cảm biến khác nhau như cảm
biến quán tính, cảm biến sinh lý, hoặc cảm biến áp suất nhằm thu thập dữ liệu
đa chiều về trạng thái của cơ thể. Một số nghiên cứu đã chứng minh tính khả thi
của việc sử dụng duy nhất một cảm biến gia tốc trong nhận dạng tư thế ngủ, đạt
được độ chính xác cao khi kết hợp với các mô hình học máy. Các kết quả này cho
thấy hướng tiếp cận đơn cảm biến hoàn toàn có thể thay thế cho hệ thống đa cảm
biến truyền thống trong các bài toán phân loại tư thế ngủ.

Tuy nhiên, các công trình hiện có chưa xem xét đầy đủ việc phân biệt trạng thái
“đã nằm” hay “chưa nằm” – một yếu tố nền tảng để xác định chính xác giai đoạn
bắt đầu giấc ngủ hoặc phát hiện sớm thay đổi tư thế. Đồng thời, việc lựa chọn
tập đặc trưng và mô hình học máy phù hợp cho dữ liệu từ một cảm biến gia tốc
duy nhất đặt tại vùng cổ – vị trí có khả năng phản ánh biến thiên tư thế của
thân trên vẫn chưa được nghiên cứu hệ thống.

Đặc biệt, tại Việt Nam cho đến nay chưa có công bố khoa học nào tập
trung vào bài toán nhận dạng tư thế ngủ dựa hoàn toàn trên dữ liệu
cảm biến gia tốc và được triển khai trực tiếp trên phần cứng
hạn chế tài nguyên, như các vi điều khiển sử dụng trong ứng dụng
Edge AI hoặc TinyML. Khoảng trống này đặt ra yêu cầu cần thiết cho
việc phát triển một hệ thống giám sát tư thế ngủ đơn giản, chi phí
thấp, có khả năng xử lý cục bộ, hướng tới ứng dụng trong sàng lọc và
hỗ trợ chẩn đoán sớm hội chứng ngưng thở tắc nghẽn khi ngủ (OSA).

Qua đó, luận văn này tập trung làm rõ các đặc trưng của dữ liệu cảm biến gia
tốc khi được đặt tại vùng cổ, đồng thời đề xuất và đánh giá các mô hình học máy
phù hợp nhằm tối ưu cho việc triển khai trên các thiết bị tính toán biên (edge
devices). Trên cơ sở đó, luận văn xây dựng một hệ thống hoàn chỉnh gồm phần
cứng đeo được kích thước nhỏ gọn, chi phí thấp; phần mềm thu thập và lưu trữ dữ
liệu; cùng với mô-đun xử lý và phân loại tư thế ngủ tích hợp trực tiếp trên vi
điều khiển.

Hệ thống được thiết kế hướng tới độ chính xác cao, độ trễ thấp và khả năng hoạt
động độc lập, cho phép người dùng có thể ghi nhận và phân loại tư thế ngủ tại
nhà một cách thuận tiện. Mặc dù chưa tích hợp chức năng sàng lọc hội chứng
ngưng thở tắc nghẽn khi ngủ (OSA), kết quả nghiên cứu này tạo nền tảng kỹ thuật
tiềm năng cho các ứng dụng trong tương lai, đặc biệt trong việc hỗ trợ đánh giá
nguy cơ pOSA (positional OSA) dựa trên tư thế ngủ và xu hướng thay đổi tư thế
trong suốt giấc ngủ.

Mục tiêu cụ thể của khóa luận gồm: 01) Nghiên cứu, đề xuất hệ thống phần cứng
phục vụ đo lường, thu thập, xử lý tín hiệu gia tốc kèm với hiệu năng phù hợp
cho việc triển khai mô hình học máy tại biên trong bài toán phân loại tư thế
ngủ ở người; 02) Nghiên cứu, đề xuất những mô hình học máy thích hợp trong việc
phân loại tư thế ngủ ở người dựa trên những đặc trưng của tín hiệu gia tốc được
trích xuất từ bộ dữ liệu cảm biến do nhóm nghiên cứu thu thập; 03) Chuẩn
hóa,thực thi, đánh giá mô hình học máy trên vi điều khiển Nordic nRF52840, nhằm
kiểm chứng hoạt động thực tế của hệ thống nhận dạng tư thế ngủ tại biên;

Cấu trúc luận văn được trình bày trong bốn chương chính như sau:

\noindent\textbf{Chương 1:} Tổng quan OSA, tư thế ngủ và công nghệ liên quan

\noindent\textbf{Chương 2:} Hệ thống thu thập và xử lý tín hiệu cảm biến.

\noindent\textbf{Chương 3:} Trích xuất đặc trưng và mô hình học máy.

\noindent\textbf{Chương 4:} Kết quả và đánh giá.

-----------------------

Tư thế ngủ được xem là một trong những yếu tố quan trọng có thể làm trầm trọng
thêm tình trạng ngưng thở khi ngủ, đặc biệt là ở tư thế nằm ngửa. Dạng đặc biệt
của hội chứng này được gọi là ngưng thở khi ngủ do tư thế (Positional
Obstructive Sleep Apnea – pOSA). Theo tiêu chuẩn chẩn đoán, bệnh nhân được xác
định mắc pOSA khi chỉ số Apnea–Hypopnea Index (AHI) lớn hơn 5, đồng thời giá
trị AHI ở tư thế nằm ngửa cao gấp ít nhất hai lần so với khi nằm ở các tư thế
khác \cite{heinzer2018}. Các nghiên cứu gần đây cho thấy tỷ lệ hiện mắc pOSA
chiếm tới khoảng 50\% trong tổng số bệnh nhân OSA \cite{sabil2020}. Phát hiện
này nhấn mạnh rằng tư thế ngủ có ảnh hưởng sinh lý rõ rệt đến mức độ xẹp của
đường hô hấp trên, đặc biệt ở tư thế nằm ngửa, làm gia tăng nguy cơ tắc nghẽn
đường thở.

Mặc dù PSG vẫn giữ vai trò là phương pháp tham chiếu trong chẩn đoán và theo
dõi chất lượng giấc ngủ cũng như các rối loạn liên quan, nhưng việc triển khai
kỹ thuật này thường đòi hỏi chi phí cao, trang thiết bị chuyên dụng và điều
kiện thực hiện tại các cơ sở y tế chuyên khoa. Một thách thức khác của PSG là
người bệnh thường cảm thấy bất tiện và cảm giác khó chịu do mang nhiều cảm biến
gắn trên cơ thể trong suốt đêm, dẫn đến nguy cơ gián đoạn hoặc sai lệnh dữ liệu
trong quá trình ghi nhận. Vì vậy, các hệ thống HST đang ngày càng thu hút sự
quan tâm từ cộng đồng khoa học toàn
cầu\cite{e3hst}\cite{hstSurvey}\cite{hst_paper}. Các thiết bị HST hiện đại dùng
các cảm biến không xâm lấn nhằm ghi nhận và phân tích một số tín hiệu sinh lý
cơ bản như luồng khí hô hấp, tư thế ngủ, áp suất mũi, độ bão hòa oxy và nhịp
tim. Việc cải thiện chất lượng, kéo dài thời lượng sử dụng và tăng độ chính xác
và cải thiện mức độ thoải mái vẫn là những thách thức lớn đối với giới nghiên
cứu và các nhà sản xuất thiết bị y tế. Tuy nhiên, với những tiến bộ công nghệ
đang diễn ra nhanh chóng, HST có tiềm năng trở thành một công cụ chẩn đoán quan
trọng và được ứng dụng rộng rãi trong lâm sàng. Điều này không chỉ mang lại sự
thuận tiện và chấp nhận cao hơn từ phía người bệnh, mà còn góp phần làm giảm
gánh nặng chi phí và áp lực cho hệ thống chăm sóc sức khỏe.

Trong những năm gần đây, nhiều nhóm nghiên đang chú trọng nghiên cứu, phát
triển hệ thống HST nhằm mục đích thay thế hoặc hỗ trợ cho đa ký giấc ngủ. Năm
2011, Collop và cộng sự đã phát triển một hệ thống phân loại SCOPER (Sleep,
Cardiovacular, Oximetry, Position, Effort, and Respiration) để đánh giá các tín
hiệu sinh lý thu nhận trong việc chẩn đoán OSA\cite{hst_6p_paper}. Tác giả
Morillo D và cộng sự đề xuất một phương pháp sàng lọc ngưng thở tắc nghẽn khi
ngủ dựa trên cảm biến gia tốc gắn tại vị trí hõm ức, cho phép trích xuất các
tín hiệu hô hấp, tim mạch và tiếng ngáy bằng kỹ thuật xử lý tín hiệu số, từ đó
chứng minh tính khả thi của thiết bị di động đơn giản và chi phí thấp trong hỗ
trợ chẩn đoán hội chứng ngưng thở – giảm thở khi
ngủ\cite{morillo2010accelerometer}. Trong nhóm các thiết bị đeo, A.H. Yüzer và
cộng sự đã phát triển thiết bị đeo tay sử dụng cảm biến gia tốc ADXL345 để phát
hiện và phát tín hiệu rung khi cảnh báo. \cite{hst_wear_paper}. Tương tự, nhóm
Yunyoung Nam và cộng sự cũng đã tích hợp hệ thống thu thập, phân tích sử dụng
một cảm biến gia tốc ba trục và một cảm biến áp suất để giám sát chất lượng
giấc ngủ tư thế ngủ, trạng thái ngủ, giai đoạn ngủ (REM và chu kỳ giai đoạn ngủ
không REM) \cite{hst_pressure_paper}. Tại Việt Nam, nhóm nghiên cứu của Giáo sư
Lê Tiến Thường, trường đại học Bách Khoa TP Hồ Chí Minh đã sử dụng cảm biến gia
tốc MPU6050 cùng với vi xử lý ESP32 nhằm ghi nhận hơi thở và nhịp tim của bệnh
nhân OSA thông qua rung động, và dòng chảy của động mạch và tĩnh mạch ở cổ
\cite{thuong_wear_paper}.

Trong các phương pháp tiếp cận hiện nay, phần lớn đều ứng dụng các mô hình học
máy nhằm khai thác hiệu quả tín hiệu sinh lý (như ECG, PPG và accelerometer) để
phát hiện các sự kiện hô hấp bất thường cũng như phân loại mức độ nghiêm trọng
của OSA một cách tự động, nhanh chóng và chính xác \cite{osa_sanchez2025}.
Trong số các thuật toán học máy truyền thống, Random Forest (RF) đã được chứng
minh là một trong những phương pháp hiệu quả nhất nhờ khả năng kháng quá khớp
và độ chính xác cao, với độ chính xác trên 93\% trong nhiều nghiên cứu
\cite{genuer2020random,wang2023ml_wearable}. SVM, KNN và LDA cũng được sử dụng
phổ biến nhờ khả năng phân loại mạnh mẽ và khả năng thích ứng với nhiều loại dữ
liệu sinh lý \cite{cortes1995svm, cunningham2007knn, tharwat2017lda}. Bên cạnh
đó, các thuật toán tiên tiến như XGBoost đã được triển khai để tăng tốc huấn
luyện và cải thiện hiệu suất trong các hệ thống học máy gọn nhẹ (TinyML) phục
vụ chẩn đoán lâm sàng \cite{chen2016xgboost}.

Bên cạnh các thuật toán học máy truyền thống, các mô hình học sâu (deep
learning) như mạng nơ-ron nhân tạo (ANN), mạng nơ-ron tích chập (CNN), và mạng
nơ-ron sâu (DNN) đã cho thấy hiệu quả vượt trội trong việc xử lý tín hiệu sinh
lý phức tạp và nhận diện sự kiện hô hấp liên quan đến OSA. Nhiều nghiên cứu đã
ứng dụng các kiến trúc CNN một chiều (1D-CNN), mạng kết hợp CNN–RNN, và mạng
tích chập đa tầng để phát hiện nhịp thở bất thường và ước lượng chỉ số AHI với
độ chính xác có thể đạt trên 90\%
\cite{HOANG2025116309,Sleep_Posture_Detection}. Các mô hình này tận dụng khả
năng tự động trích chọn đặc trưng từ dữ liệu cảm biến thô, hỗ trợ phát hiện OSA
theo thời gian thực và tối ưu hóa hiệu năng. Những kết quả này khẳng định tiềm
năng lớn của học sâu trong việc xây dựng các hệ thống giám sát giấc ngủ thế hệ
mới, đặc biệt là trên các nền tảng nhúng và thiết bị đeo thông minh.

Còn đối với bài toán phân loại tư thế ngủ ở người, tác giả Xi Xu và cộng sự đã
đề xuất một mô hình học máy stacking cải tiến, kết hợp ba thuật toán mạnh
(XGBoost, SVM, DNDT) và tối ưu bằng Bayesian optimization cùng phương pháp
entropy weighting để nhận dạng tư thế ngủ dựa trên dữ liệu áp suất từ nệm khí
\cite{xu2024classification}. Kết quả thực nghiệm cho thấy mô hình đạt độ chính
xác 94.48\%, vượt trội hơn các mô hình đơn lẻ và có tiềm năng ứng dụng trong
các hệ thống giám sát giấc ngủ thông minh. Trong một nghiên cứu khác, tác giả
giới thiệu một thiết bị đeo di động sử dụng cảm biến gia tốc đơn để nhận diện
mười hai tư thế ngủ, hỗ trợ bệnh nhân GERD cải thiện thói quen và chất lượng
giấc ngủ tại nhà \cite{Vu2025SleepPosition}. Mô hình AnpoNet kết hợp 1D-CNN và
LSTM, đạt độ chính xác 94,67\% và F1-score 92,94\%, cho thấy tiềm năng ứng dụng
thực tiễn trong giám sát tư thế ngủ không xâm lấn

Với mục tiêu triển khai hệ thống trên thiết bị đeo có tài nguyên tính toán hạn
chế, việc áp dụng các thuật toán học máy nhẹ (TinyML) đóng vai trò then chốt.
Khác với các mô hình học sâu truyền thống yêu cầu phần cứng mạnh và tiêu tốn
năng lượng, TinyML cho phép thực thi mô hình trực tiếp trên vi điều khiển có
dung lượng bộ nhớ và tốc độ xử lý thấp, đồng thời đảm bảo độ trễ tối thiểu và
khả năng hoạt động ngoại tuyến. Đặc điểm này đặc biệt phù hợp cho các ứng dụng
y sinh tại nhà, nơi thiết bị cần hoạt động liên tục, chi phí thấp và không phụ
thuộc vào kết nối mạng.

Tiny machine learning tập trung vào việc giảm thiểu số lượng tham số mô hình và
độ phức tạp tính toán, nhờ đó trở thành một hướng tiếp cận đầy hứa hẹn trong
các ứng dụng chăm sóc sức khỏe thông minh
\cite{ray2021tinyml,diab2022embedded}. Tuy nhiên, vẫn tồn tại sự đánh đổi giữa
việc đơn giản hóa mô hình và duy trì độ chính xác trong phân loại. Bài toán đặt
ra là làm sao tối ưu hóa đồng thời kích thước mô hình, tốc độ xử lý và hiệu
suất trong việc phân loại tình trạng OSA, nhận diện tư thế, đảm bảo hệ thống
vừa đủ nhẹ để chạy trên phần cứng nhúng, vừa đủ chính xác để có giá trị trong
giám sát y tế.

Các nghiên cứu trước đây đã chứng minh hiệu quả của học máy trong phân loại tư
thế ngủ và phát hiện OSA, tuy nhiên phần lớn các hệ thống vẫn phụ thuộc vào dữ
liệu đa kênh phức tạp, mô hình tính toán nặng, hoặc cần xử lý tập trung trên
máy chủ. Điều này gây hạn chế khi triển khai thực tế tại nhà hoặc trên thiết bị
đeo. Hơn nữa, vẫn còn thiếu các giải pháp tích hợp đầy đủ từ thu thập, xử lý,
đến phân loại tín hiệu ngay trên vi điều khiển với độ trễ thấp và khả năng hoạt
động độc lập không cần kết nối mạng.
