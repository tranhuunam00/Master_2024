Ngưng thở tắc nghẽn khi ngủ (Obstructive Sleep Apnoea – OSA) là một rối loạn hô hấp phổ biến trong giấc ngủ, được đặc trưng bởi các đợt ngưng thở hoặc giảm thông khí tắc nghẽn lặp đi lặp lại trong lúc ngủ, dẫn đến việc gián đoạn giấc ngủ do vi thức giấc và giảm oxy trong máu.  Tỷ lệ hiện mắc OSA tại Việt Nam ước tính khoảng 8,5\% \cite{nguoimacOSA_VN}. OSA hiện được công nhận là một yếu tố có nguy cơ độc lập đối với nhiều bệnh lý liên quan đến tim mạch, đặc biệt là tăng huyết áp. Ngoài ra, hội chứng này còn có mối liên hệ đáng kể với các nguy cơ như tai nạn giao thông, tai nạn lao động dẫn đến làm gia tăng gánh nặng kinh tế xã hội\cite{osa_bike}\cite{Marin2005}\cite{drive_osa}. Đáng chú ý, tình trạng ngưng thở khi ngủ kéo dài và không được phát hiện, điều trị có thể ảnh hưởng nghiêm trọng đến sức khỏe gây ra rối loạn nhịp tim và một trong những nguyên nhân gây đột tử \cite{sumarry_osa}. Theo PSG.TS Nguyễn Thy Khuê, Hội Y học Giấc ngủ Việt Nam, ngưng thở khi ngủ còn là một trong nhưng dấu hiệu rõ ràng của bệnh đái tháo đường, bệnh thận \cite{bsThyKhue}. OSA được phát hiện ở hơn 20\% người bệnh đái tháo đường và làm trầm trọng thêm các rối loạn chuyển hóa, đặc biệt là đái tháo đường type 2. Trong một nghiên cứu tiêu biểu tại Việt Nam, GS.TS. Dương Quý Sỹ và cộng sự đã khảo sát 524 trẻ em mắc rối loạn tăng động giảm chú ý (Attention Deficit Hyperactivity Disorder - ADHD) tại Bệnh viện Nhi Trung ương Việt Nam. Kết quả cho thấy tỷ lệ mắc (OSA) ở nhóm này là 23.3\%, trong đó chủ yếu ở mức độ trung bình đến nặng \cite{ThaySUCHildren}. Nghiên cứu cũng đồng thời xác định mối tương quan đáng kể giữa mức độ nghiêm trọng của OSA và các triệu chứng mất tập trung, tăng động, rối loạn hành vi, lo âu và trầm cảm. Phát hiện nhấn mạnh sự cần thiết của việc sàng lọc OSA trong quá trình điều trị toàn diện ADHD ở trẻ em. 

Một dạng đặc biệt của OSA được ghi nhận là ngưng thở khi ngủ do tư thế (Positional Obstructive Sleep Apnea - pOSA). Bệnh nhân được chẩn đoán mắc pOSA có chỉ số AHI lớn hơn 5, và giá trị AHI ở tư thế ngửa cao gấp ít nhất hai lần so với nằm ở tư thế khác \cite{heinzer2018}. Các nghiên cứu gần đây chỉ ra rằng tỷ lệ mắc pOSA lên tới 50\% bệnh nhân OSA \cite{sabil2020}. Điều này cho thấy tư thế ngủ có ảnh hưởng sinh lý rõ rệt đến sự sụp đổ đường thở trên, đặc biệt ở tư thế nằm ngữa. Lúc nằm ngửa, trọng lực làm xẹp các cơ vùng họng dẫn đến làm hẹp khoang khí.

Việc chẩn đoán hội chứng ngưng thở tắc nghẽn khi ngủ (OSA) hiện nay chủ yếu được thực hiện thông qua hai phương pháp: đa ký giấc ngủ (Polysomnography – PSG) và thiết bị kiểm tra giấc ngủ tại nhà (Home Sleep Test – HST). Trong đó, PSG được xem là tiêu chuẩn vàng trong việc đánh giá OSA. DO phương pháp này cho phép thu thập đồng thời nhiều thông số sinh lý quan trọng bao gồm: luồng khí hô hấp qua mũi và/hoặc miệng, cử động thành ngực và bụng, tiếng ngáy, điện não đồ (Electroencephalography – EEG), điện tâm đồ (Electrocardiography – ECG), điện cơ đồ (Electromyography – EMG), và độ bão hòa oxy trong máu (SpO₂). Quá trình do PSG phải được thực hiện trong môi trường có kiểm soát tại các cơ sở y tế chuyên khoa, dưới sự giám sát trực tiếp của bác sỹ chuyên ngành giấc ngủ hoặc kĩ thuật viên có chuyên môn\cite{psg_paper}\cite{kushida2005psg}. 

Mặc dù PSG vẫn giữ vai trò là phương pháp tham chiếu trong chẩn đoán và theo dõi chất lượng giấc ngủ cũng như các rối loạn liên quan, nhưng việc triển khai kỹ thuật này thường đòi hỏi chi phí cao, trang thiết bị chuyên dụng và điều kiện thực hiện tại các cơ sở y tế chuyên khoa. Một thách thức khác của PSG là người bệnh thường cảm thấy bất tiện và cảm giác khó chịu do mang nhiều cảm biến gắn trên cơ thể trong suốt đêm, dẫn đến nguy cơ gián đoạn hoặc sai lệnh dữ liệu trong quá trình ghi nhận. Vì vậy, các hệ thống HST đang ngày càng thu hút sự quan tâm từ cộng đồng khoa học toàn cầu\cite{e3hst}\cite{hstSurvey}\cite{hst_paper}. Các thiết bị HST hiện đại dùng các cảm biến không xâm lấn nhằm ghi nhận và phân tích một số tín hiệu sinh lý cơ bản như luồng khí hô hấp, tư thế ngủ, áp suất mũi, độ bão hòa oxy và nhịp tim. Việc cải thiện chất lượng, kéo dài thời lượng sử dụng và tăng độ chính xác và cải thiện mức độ thoải mái vẫn là những thách thức lớn đối với giới nghiên cứu và các nhà sản xuất thiết bị y tế. Tuy nhiên, với những tiến bộ công nghệ đang diễn ra nhanh chóng, HST có tiềm năng trở thành một công cụ chẩn đoán quan trọng và được ứng dụng rộng rãi trong lâm sàng. Điều này không chỉ mang lại sự thuận tiện và chấp nhận cao hơn từ phía người bệnh, mà còn góp phần làm giảm gánh nặng chi phí và áp lực cho hệ thống chăm sóc sức khỏe.

Trong những năm gần đây, nhiều nhóm nghiên đang chú trọng nghiên cứu, phát triển hệ thống HST nhằm mục đích thay thế hoặc hỗ trợ cho đa ký giấc ngủ. Năm 2011, Collop và cộng sự đã phát triển một hệ thống phân loại SCOPER (Sleep, Cardiovacular, Oximetry, Position, Effort, and Respiration) để đánh giá các tín hiệu sinh lý thu nhận trong việc chẩn đoán OSA\cite{hst_6p_paper}. Tác giả Morillo D và cộng sự đề xuất một phương pháp sàng lọc ngưng thở tắc nghẽn khi ngủ dựa trên cảm biến gia tốc gắn tại vị trí hõm ức, cho phép trích xuất các tín hiệu hô hấp, tim mạch và tiếng ngáy bằng kỹ thuật xử lý tín hiệu số, từ đó chứng minh tính khả thi của thiết bị di động đơn giản và chi phí thấp trong hỗ trợ chẩn đoán hội chứng ngưng thở – giảm thở khi ngủ\cite{morillo2010accelerometer}. Trong nhóm các thiết bị đeo, A.H. Yüzer và cộng sự đã phát triển thiết bị đeo tay sử dụng cảm biến gia tốc ADXL345 để phát hiện và phát tín hiệu rung khi cảnh báo. \cite{hst_wear_paper}. Tương tự, nhóm Yunyoung Nam và cộng sự cũng đã tích hợp hệ thống thu thập, phân tích sử dụng một cảm biến gia tốc ba trục và một cảm biến áp suất để giám sát chất lượng giấc ngủ tư thế ngủ, trạng thái ngủ, giai đoạn ngủ (REM và chu kỳ giai đoạn ngủ không REM) \cite{hst_pressure_paper}. Tại Việt Nam, nhóm nghiên cứu của  Giáo sư Lê Tiến Thường, trường đại học Bách Khoa TP Hồ Chí Minh đã sử dụng cảm biến gia tốc MPU6050 cùng với vi xử lý ESP32 nhằm ghi nhận hơi thở và nhịp tim của bệnh nhân OSA thông qua rung động, và dòng chảy của động mạch và tĩnh mạch ở cổ \cite{thuong_wear_paper}. Gần đây, Domingues và cộng sự (2024) xây dựng một mô hình mạng nơ-ron nhân tạo dựa trên dữ liệu từ máy đo SPO2, cấm biến gia tốc và ghi âm tiếng ngáy của hệ thống Biologix, nhằm dự đoán chính xác trạng thái ngủ. Kết quả cho thấy mô hình này có khả nâng cao độ chính xác trong chẩn đoán ngưng thở khi ngủ tại nhà, tiệm cận với tiêu chuẩn của đa ký giấc ngủ truyền thống \cite{domingues2024sleep}. Một hướng nghiên cứu khác là cảm biến đặt dưới nệm giường. Tác giả Andrei Boiko và cộng sự đánh giá hệ thống phát sử dụng cảm biến gia tốc đặt dưới đệm giường để ghi dao động do cử động ngực khi thở. Kết quả cho thấy thuật toán phát hiện ngưng thở đạt độ chính xác, độ đặc hiệu và độ nhạy lần lượt là 94.6\%, 95.3\% và 93.7\% \cite{Boiko2023}. 

Nhìn chung, việc chẩn đoán OSA hay pOSA đòi hỏi hệ thống cần có khả năng thu thập liên tục, và đưa ra quyết định chính xác dựa trên những dữ liệu đó. Do đó, việc xây dựng các mô hình học máy để phân loại AHI, hay các tư thế ngủ đang là 1 hướng nghiên cứu phổ biến hợp hợp cả hai lĩnh vực y tế giấc ngủ và công nghệ dữ liệu. Các nghiên cứu tiêu biểu sử dụng mô hình như Decision Tree, Ramdom Forest, SVM và các mô hình Deep Learning cho độ chính xác từ 



Khóa luận văn có mục đích nghiên cứu, ứng dụng cảm biến gia tốc kết hợp với các bộ lọc tín hiệu, trí tuệ nhân tạo, ứng dụng phần mềm để tạo ra một sản phẩm có độ chính xác, tính ứng dụng cao trong cuộc sống. Khóa luận đưa ra cái nhìn tổng quan về chứng ngưng thở khi ngủ và thiết bị chẩn đoán sử dụng cảm biến gia tốc. Kết quả có tiềm năng ứng dụng thực tế phục vụ cho mục đích chuẩn đoán, đánh giá và phân tích về chứng ngưng thở khi ngủ. Khóa luận này trình bày thiết kế, chế tạo và thử nghiệm một thiết bị đeo được tích hợp cảm biến gia tốc 3 trục cho phép tiến hành thực nghiệm chế tạo và thiết lập hệ thống đo đạc cho cảm biến đã đề xuất với những mục tiêu cụ thể: 01) Tăng độ chính xác của hệ thống, 02) Thu nhỏ thiết kế mạch, 03) Tăng trường khả năng lưu trữ dữ liệu và 04) Hỗ trợ người sử dụng thông qua phần mềm ứng dụng. Khóa luận được thực hiện thông qua 01) tham khảo tài liệu, 02) Phương pháp nghiên cứu lý thuyết, 03) phương pháp thực nghiệm. Nội dung nghiên cứu được trình bày trong 3 chương như sau:

    Chương 1: Tổng quan về OSA và các thiết bị theo dõi.

    Chương 2: Quá trình lựa chọn và xây dựng thiết bị.

    Chương 3: Kết quả thực nghiệm, đánh giá và kế hoạch nghiên cứu trong thời gian tới.
