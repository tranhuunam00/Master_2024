\vspace{-1cm}
\setcounter{table}{0}
\renewcommand{\thetable}{A.\arabic{table}}
\setlength\LTleft{0pt}
\setlength\LTright{0pt}
\setlength\LTcapwidth{\linewidth}
\begin{longtable}{ |p{1cm}|p{7cm}|p{6,5cm}|}
% \begin{longtable}{ @{\extracolsep{\fill}}|p{1cm}|p{6,5cm}|p{6,5cm}|@{}}
    % \centering
    \caption{Bảng danh sách các dạng biểu thức được hỗ trợ trong mã nguồn}
    % \vspace{0.5cm}
    % \begin{adjustbox}{width=1\textwidth}
    % \small
    % \begin{tabular}{ |p{1cm}|p{6,5cm}|p{6,5cm}|  }
    \\
         \hline
         \textbf{STT} & \textbf{Kiểu biểu thức} & \textbf{Ví dụ minh họa}\\
         \hline
         \endhead
         \hline
         \endfoot
        1 & Biểu thức sử dụng các phép toán với biến nguyên thủy & \makecell[l]{ a = 1 + 2;\\
         a = 3*4/2;\\
        if (a > 1);\\ if(a > 1 + 2); \\
        if (a + 1 > 3*4)} 
        \\
        \hline
        2 & Biểu thức sử dụng các phép toán với tên biến.   & \makecell[l]{ a = b + c;\\ a = b*c;\\
        if (a > b); \\if (a > b+c); \\
        if (a + b ==  c + d); \\
        if (a*b != c/d)}
        \\
        \hline
        3 & Biểu thức sử dụng thuộc tính length của biến kiểu string & \makecell[l]{ s = “abcdef”;\\
        a = s.length;\\
        if (s.length > 10);\\
        if(s.length > a+b); \\
        if (s1.length < s2.length + a);}
        \\
        \hline
        4 & Biểu thức sử dụng phương thức startsWith(), endsWith(), includes() của biến kiểu string & \makecell[l]{ a = s.startsWith(“ABC”); \\
        a = endsWith(“DEF”);\\
        if (s.includes(“XYZ”))}
        \\
        \hline
        5 & Gán một chuỗi cụ thể cho biến kiểu string & a = “abcdef”;\\
        \hline
        6 & Biểu thức sử dụng các phép toán với thuộc tính của tham số đối tượng   & \makecell[l]{a = person.height + person.age;\\
        if (person.height > 180)\\
        if (person.height/person.age <  8)} 
        \\ 
        \hline
        7 & Biểu thức sử dụng các phép toán với các hàm getter của đối tượng & \makecell[l]{s = person.getName();\\
        if (person.getName().\\startsWith(“hoaithu”))} 
        \\
         \hline
        8 & Biểu thức sử dụng các phép toán với phần tử mảng có index cụ thể &\makecell[l]{ a = arr[0] + b; \\
        a[0] = a + b + a[1];} 
        \\
         \hline
        9 & Biểu thức khai báo/gán giá trị là một mảng & a = [1,2,3,4,5,6];\\
         \hline
        10 &Biểu thức khai báo/gán giá trị là một Json object & \makecell[l]{a= \{height: 180, age: 23, \\school:\{name: “UET”\}\}}\\
         \hline
        11 & Biểu thức điều kiện kép bao gồm biều biểu thức điều kiện đơn thỏa mãn các tính chất 1 -> 8 & If (a > b \&\& s.length > 10 \&\& s.startsWith(“ABC”))
        % \end{tabular}
    % \end{adjustbox}
    \label{table:expressions}
\end{longtable}

\begin{longtable}{ |p{1cm}|p{7cm}|p{6,5cm}|}
% \begin{longtable}{ @{\extracolsep{\fill}}|p{1cm}|p{6,5cm}|p{6,5cm}|@{}}
    % \centering
    \caption{Bảng danh sách các dạng hình thức mã nguồn chưa được hỗ trợ}
    % \vspace{0.5cm}
    % \begin{adjustbox}{width=1\textwidth}
    % \small
    % \begin{tabular}{ |p{1cm}|p{6,5cm}|p{6,5cm}|  }
    \\
         \hline
         \textbf{STT} & \textbf{Hình thức mã nguồn} & \textbf{Ví dụ minh họa}\\
         \hline
         \endhead
         \hline
         \endfoot
        1 & Cấu trúc điều khiển vòng lặp & \makecell[l]{let list = [4, 5, 6];\\
        for (let i in list) \{ console.log(i); \}\\
        for (let i of list) \{ console.log(i); \}\\
        for (var i = 0; i < list.length; i++) \{\\
            var num = list[i];\\
            console.log(num);\\
        \}
        } \\
        \hline
        2 & Cấu trúc điều khiển \textit{switch}   & \makecell[l]{ a = b + c;\\ a = b*c;\\
        if (a > b); \\if (a > b+c); \\
        if (a + b ==  c + d); \\
        if (a*b != c/d)}
        \\
        \hline
        3 & Câu điều kiện rút gọn & \makecell[l]{ s = “abcdef”;\\
        a = s.length;\\
        if (s.length > 10);\\
        if(s.length > a+b); \\
        if (s1.length < s2.length + a);}
        \\
        \hline
        4 & Biểu thức thay đổi giá trị rút gọn & \makecell[l]{ a = s.startsWith(“ABC”); \\
        a = endsWith(“DEF”);\\
        if (s.includes(“XYZ”))}
        \\
        \hline
        5 & Chỉ số truy cập phần tử mảng là biểu thức & a = “abcdef”;\\
        \hline
        6 & Arrow Function  & \makecell[l]{a = person.height + person.age;\\
        if (person.height > 180)\\
        if (person.height/person.age <  8)} 
        \\ 
        \hline
        7 & Gọi hàm bên ngoài trả về kiểu dữ liệu là đối tượng & \makecell[l]{s = person.getName();\\
        if (person.getName().\\startsWith(“hoaithu”))} 
        \\
         \hline
        8 & Biến kiểu dữ liệu enum, any, undefined, never, tuple, v.v. &\makecell[l]{ a = arr[0] + b; \\
        a[0] = a + b + a[1];}  \\
        \hline
        v.v. & v.v. & v.v.
        %  \hline
        % 9 & Biểu thức khai báo/gán giá trị là một mảng & a = [1,2,3,4,5,6];\\
        %  \hline
        % 10 &Biểu thức khai báo/gán giá trị là một Json object & \makecell[l]{a= \{height: 180, age: 23, \\school:\{name: “UET”\}\}}\\
        %  \hline
        % 11 & Biểu thức điều kiện kép bao gồm biều biểu thức điều kiện đơn thỏa mãn các tính chất 1 -> 8 & If (a > b \&\& s.length > 10 \&\& s.startsWith(“ABC”))
        % \end{tabular}
    % \end{adjustbox}
    \label{table:expressions_not_handle}
\end{longtable}